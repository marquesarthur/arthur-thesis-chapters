%% The following is a directive for TeXShop to indicate the main file
%%!TEX root = diss.tex

\chapter{Preface}

All of the work presented henceforth was conducted in the Software Practices
Lab at the University of British Columbia, Point Grey campus.
All projects and associated methods were approved by the University of British Columbia's Research
Ethics Board [certificates \#H18-02104 and \#H19-04054].



Parts of the research presented in this dissertation have been previously published.
I list publications in chronological order. Publications
involved other collaborators who have participated in drafting and/or revision of the manuscripts. 


\begin{enumerate}
    \item Part of Chapter~\ref{ch:introduction} was published in the 27th ACM Joint Meeting on European Software Engineering Conference and Symposium on the Foundations of Software Engineering (ESEC/FSE)~\cite{marques2019}:


    \begin{itemize}
        \item A. Marques.
            \textit{Helping Developers Search and Locate Task-relevant Information in Natural Language Documents}. 
            In 27th ACM Joint Meeting on European Software Engineering Conference and Symposium on the Foundations of Software Engineering (ESEC/FSE),
            pages 1168---1171, 2019.
    \end{itemize}        


\item A version of Chapter~\ref{ch:characterizing} was published in the IEEE  International Conference on Software Maintenance and Evolution (ICSME)~\cite{marques2020}:


    \begin{itemize}
        \item A. Marques, Nicholas C. Bradley, and G. C. Murphy.
            \textit{Characterizing Task-Relevant Information in Natural Language Software Artifacts}. 
            In IEEE  International Conference on Software Maintenance and Evolution (ICSME),
            pages 476---487, 2020. 
    \end{itemize}

\item The tool \textit{SEFrame} used in Chapter~\ref{ch:identifying} and discussed in Chapter~\ref{ch:discussion} was published in the  IEEE/ACM 29th International Conference on Program Comprehension (ICPC)~\cite{marques2021}:


    \begin{itemize}
        \item A. Marques, Giovanni Viviani, and Gail C. Murphy.
            \textit{Assessing Semantic Frames to Support Program Comprehension Activities}. 
            In IEEE/ACM 29th International Conference on Program Comprehension (ICPC), 
            pages 13---24, 2021.
    \end{itemize}


\item A version of Chapters~\ref{ch:android-corpus}---\ref{ch:identifying} was published in the IEEE International Conference on Software Analysis, Evolution and Reengineering (SANER)~\cite{marques2022}: 


    \begin{itemize}
        \item A. Marques and Gail C. Murphy. 
            \textit{Evaluating the Use of Semantics for Identifying Task-relevant Textual Information}. 
            In IEEE International Conference on Software Analysis, Evolution and Reengineering (SANER),
            pages 240---251, 2022.
    \end{itemize}

\end{enumerate}



Other contributions from the manuscripts' authors and other collaborators are detailed as follows.
% nick
Nick C. Bradley contributed in
the interview analysis described in Chapter~\ref{ch:characterizing} and published in the second paper.
% giovanni
Giovanni Viviani helped in the analysis, design and implementation of the 
semantic parser used in Chapter~\ref{ch:identifying} and published in the third paper.
% dataset
Alison Li, Katharine Kerr, and Tarc{\'i}sio Teixeira helped 
in the creation of the dataset presented in Chapter~\ref{ch:android-corpus} and published in the fourth paper.
% shaunak
Shaunak Tulshibagwale helped implement some of the 
data gathering scripts for the 
corpora presented in Chapters~\ref{ch:identifying} and~\ref{ch:assisting}.
% nadi and treude
The scripts that gathered data from Stack Overflow are an extension of 
 Nadi and Treude's scripts~\cite{nadi2019Rep}.
% muhammad
Muhammad Abdul-Mageed was a supervisory committee member and helped provide access to the 
Compute Canada research grid; he also provided valuable feedback  at several stages of this work. 
% reid
Reid Holmes was a supervisory committee member and provided valuable feedback at several stages of this work. 
% gail
Gail C. Murphy was the supervisory author on this project
and was involved throughout the project in concept formation, data analysis, and
manuscript composition.

