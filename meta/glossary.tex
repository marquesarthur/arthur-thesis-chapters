% https://mirror.its.dal.ca/ctan/macros/latex/contrib/acronym/acronym.pdf

\chapter*{Glossary}

% use \acrodef to define an acronym, but no listing
\acrodef{UI}{user interface}
\acrodef{UBC}{University of British Columbia}

% The acronym environment will typeset only those acronyms that were
% *actually used* in the course of the document
\begin{acronym}
% My own defined acronyms

\acro{qa}[Q\&A]{question-and-answer}
\acro{ERB}[ERB]{Research Ethics Board}
\acro{UBC}[UBC]{University of British Columbia}
\acro{AnsBot}[AnsBot]{AnswerBot\acroextra{, a tool that automates generation of answer summaries for a developer's task using Stack Overflow}}   
\acro{Krec}[Krec]{Knowledge Recommender\acroextra{, a tool that automates the identification of information fragments in API documentation}}
\acro{Hurried}[Hurried]{Hurried bug report summarizer\acroextra{, a tool that automates the identification of sentences a developer would first read when inspecting bug reports}}
\acro{DS-synthetic}[DS$_{synthetic}$]{Synthetic tasks dataset\acroextra{, comprising six synthetic tasks with annotations from 20 participants of text deemed as relevant in 20 associated artifacts with natural language text}}
\acro{DS-android}[DS$_{Android}$]{Android tasks dataset\acroextra{, comprising 12,401 unique sentences annotated by three developers 
and
originating from artifacts associated to 50 software tasks
drawn from GitHub issues and Stack Overflow posts about Android development}}
\acro{DS-python}[DS$_{Python}$]{Python tasks dataset\acroextra{, containing 28 natural language artifacts where 24 participants indicated text containing information that assisted them in writing a solution for three programming tasks involving well-known Python modules}}
\acro{DS-android-small}[DS-Android$_{small}$]{ \acs{DS-android} 10 tasks sample\acroextra{, a set of 10 tasks randomly sampled from the Android tasks dataset}}
\acro{DS-android-large}[DS-Android$_{large}$]{Android tasks large sample\acroextra{, a set of 30 tasks randomly sampled from the Android tasks dataset}}

\acro{SO}{Stack Overflow\acroextra{, a question and answer website for software developers}}   
\acro{stdv}[stdv]{Standard deviation}
\acro{SDK}{software development kit}

% Sample from thesis
\acro{ANOVA}[ANOVA]{Analysis of Variance\acroextra{, a set of statistical techniques to identify sources of variability between groups}}
\acro{API}{application programming interface}
\acro{CTAN}{\acroextra{The }Common \TeX\ Archive Network}
\acro{DOI}{Document Object Identifier\acroextra{ (see \url{http://doi.org})}}
\acro{GPS}[GPS]{Graduate and Postdoctoral Studies}
\acro{PDF}{Portable Document Format}
\acro{NLP}{Natural Language Processing}
\acro{RCS}[RCS]{Revision control system\acroextra{, a software tool for tracking changes to a set of files}}
\acro{TLX}[TLX]{Task Load Index\acroextra{, an instrument for gauging the subjective mental workload experienced by a human in performing a task}}
\acro{UML}{Unified Modelling Language\acroextra{, a visual language for modelling the structure of software artefacts}}
\acro{URL}{Unique Resource Locator\acroextra{, used to describe a means for obtaining some resource on the world wide web}}
\acro{W3C}[W3C]{\acroextra{the }World Wide Web Consortium\acroextra{, the standards body for web technologies}}
\acro{XML}{Extensible Markup Language}
\acro{IR}{Information Retrieval}
\acro{LDA}{Latent Dirichlet Allocation}
\acro{LSA}{Latent Semantic Analysis}
\acro{SVD}{Singular Value Decomposition}
\acro{LSI}{Latent Semantic Indexing}
\acro{IFT}{Information foraging theory}
\acro{ML}{Machine Learning}
\acro{LSTM}{Long Short-Term Memory}
\acro{RNN}{Recurrent neural network}
\acro{DL}{Deep Learning}
\acro{API}{Application Programming Interface}
\acro{web}[WWW]{World Wide Web}
\acro{CNN}{Convolutional Neural Network}
\acro{VSM}{Vector Space Model}
\acro{BERT}{Bidirectional Encoder Representations from Transformers}
\acro{tool}[TaRTI]{Automatic Task-Relevant Text Identifier\acroextra{, our proof-of-concept semantic-based tool, which uses BERT to automatically identify and show text relevant to an input task in a given web page}}

\end{acronym}


% USAGE:

% \acs{} -- short version
% \acf{} -- full version
% \ac -- ac -- prints the full name (short name) for the  first time the acronym is used












% You can also use \newacro{}{} to only define acronyms
% but without explictly creating a glossary
% 
% \newacro{ANOVA}[ANOVA]{Analysis of Variance\acroextra{, a set of
%   statistical techniques to identify sources of variability between groups.}}
% \newacro{API}[API]{application programming interface}
% \newacro{GOMS}[GOMS]{Goals, Operators, Methods, and Selection\acroextra{,
%   a framework for usability analysis.}}
% \newacro{TLX}[TLX]{Task Load Index\acroextra{, an instrument for gauging
%   the subjective mental workload experienced by a human in performing
%   a task.}}
% \newacro{UI}[UI]{user interface}
% \newacro{UML}[UML]{Unified Modelling Language}
% \newacro{W3C}[W3C]{World Wide Web Consortium}
% \newacro{XML}[XML]{Extensible Markup Language}
