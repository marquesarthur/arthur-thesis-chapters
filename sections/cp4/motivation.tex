

\section{Motivation}
\label{cp4:motivation}


When performing a software task, a software developer does not restrict her work
solely to coding~\cite{Meyer2017}.
Rather, there are many activities that a developer undertakes to complete a software task
as when they:



\begin{itemize}
    \item refer to API documentation or Q\&A websites for API usage purposes~\cite{umarji2008archetypal, Singer1998, robillard2011field};
    \item discuss in mailing lists a possible reusable library to be incorporated into their implementation~\cite{umarji2008archetypal, Bacchelli2012}; or
    \item confirm a system's behaviour referring to past discussions in community forums or in the system's documentation~\cite{Arya2019, Lotufo2012, Singer1998}.
\end{itemize}


A common aspect to these activities is the need to work with unstructured textual data, which comprises 80\% of the overall information created and used in enterprises~\cite{Bavota2014, holzinger2013}.
Due to such prevalence, there has been a large body of studies utilizing various techniques to extract
information from this text so that it can be embedded in
tools to software developers~\cite{Bavota2014, Xu2017, Robillard2015, Lotufo2012}. For instance, Xu et al. propose to mine relevant text from Stack Overflow
to generate answers to developers' technical question~\cite{Xu2017}
while FRAPT identifies key paragraphs explaining API elements in code tutorials~\cite{Jiang2017}.
As other examples, Lotufo et al. proposed a technique to identify sentences a developer would first read when inspecting bug reports~\cite{Lotufo2012} while Nadi and Treude investigated sentences that help a developer decide whether a Stack Overflow post is relevant to her task~\cite{nadi2020}.




Although these studies provide significant contributions, the fact that information to answer a question a developer has is usually located across many artifacts~\cite{Rastkar2013t} is often overlooked.
Hence existing techniques operate in an artifact-centric basis
what can be insufficient to provide to developers all the information needed
to completely and correctly accomplish a software task as several studies suggest that developers use heterogeneous sources to put together the information needed for task completion~\cite{josyula2018, Li2013, rao2020}.




We seek to design techniques that can scale to cover different types of artifacts
so that can be integrated into developers' information-seeking activities regardless of the artifact that a developer browses/inspects.
To design such techniques, we require a corpus with heterogeneous sources that a developer
might use to locate information relevant to her task.
