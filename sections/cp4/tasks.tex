\section{Software Tasks}
\label{cp4:corpus-tasks}


We start corpus creation by identifying software tasks for which a 
developer can benefit
from
foraging information across different artifact types. 
Two common places a software task can be found are:

% ART: move definition to an early chapter
% We consider a software task as a piece of work undertaken by a developer that often has to be finished within a certain time~\cite{2004merriam}.
% \gm{Is there a reference you
% could give to a definition
% of a software task in 
% software? Or can you relate to
% a definition in a previous chapter?}




\begin{itemize}
    \item the description of an issue
    (e.g., a bug or feature request) reported in a issue tracking system; or in
    \item a post in a community forum, development mailing lists, and others.
\end{itemize}

Several studies have used such  sources for software tasks~\cite{Arya2019, baltes2019, nadi2020, Xu2017}. Following the lead
of these studies, we select
GitHub issues and Stack Overflow (SO) posts as sources for potential tasks.

When selecting tasks in GitHub and Stack Overflow, a major challenge arises due to the sheer amount of data available.
Baltes et al.~\cite{baltes2019} argues that even a cursory inspection of a sample set
of Stack Overflow posts shows clear differences in a post's content or structure due to aspects such as programming languages, frameworks, associated technologies, and others.

To ensure the tasks in the corpus we produce
circumvent the heterogeneity of data on GitHub and Stack Overflow, we scope task selection to the \textit{Android} development domain. This decision
restricts task selection to a single programming language (\textit{Java})
while still enabling investigation of a domain that has been
widely discussed by practitioners and researchers alike.
For instance, over 35,000 developers have used Q\&A forums to discuss tasks covering 87\% of the Android API~\cite{parnin2012}
while researchers have investigated how changes to the Android SDK impact its ecosystem and development community~\cite{linares2014, bavota2014b, mcdonnell2013}.



\subsubsection{GitHub tasks}

To select tasks from GitHub, we are guided by studies that use 
stars~\cite{borges2016, borges2018}
as a proxy for a projet's popularity~\cite{Ferreira2016, Xavier2020}.
We selected 15 projects,
ranging from mail clients\footnote{\url{https://github.com/k9mail/k-9}}
to development frameworks\footnote{\url{https://github.com/libgdx/libgdx}},
by filtering the list of top-starred projects in the platform to the ones that contained the \textit{Java} and \textit{Android} tags.





For each of these projects, we randomly select 10 issues as the GitHub tasks of our corpus for a total of 150 distinct issues.
While selecting issues, we took care to check that they had at least one follow-up comment and that the issue title did not contain certain words, e.g., {\small \texttt{test}} or {\small  \texttt{ignore}},
so that our selection ignored issues automatically created by scripts or bots---a common pitfall that researchers must be aware of when mining GitHub~\cite{kalliamvakou2014}.


Figure~\ref{fig:lock-screen-task} shows an example of a GitHub task in our corpus.
Although the expected behaviour is that the app controls should be visible even with the screen locked,  a user reports that the app screen is missing.
A developer addressing this issue might need to review the Android lock task documentation~\cite{apiLockTask}
or refer to examples of applications that use the Android lock screen~\cite{mediumLockApp}.
For the remainder of this chapter, we use the lock screen task as a running example.


\begin{figure}
    \centering
    \includegraphics[width=\textwidth]{cp4/lock-screen-task}
    \caption{Sample GitHub task from our corpus}
    \label{fig:lock-screen-task}
\end{figure}



\subsubsection{Stack Overflow tasks}

%Provided that we restrict our corpus to tasks related to %the Android development domain,
We randomly select 150 Stack Overflow posts from a curated list about Android development~\cite{baltes2020}.
This list was built by Baltes et al. 
by identifying 209,536 unique posts that contained the the \textit{Java} and \textit{Android} tags in the Stack Overflow dump published on June 5, 2018~\cite{baltes2019-rep, SOTorrent2019}.


We consider Stack Overflow posts as software tasks because, to answer a post,
a developer often needs to provide references
supporting their answer~\cite{yazdaninia2021}.
Finding these references in a timely manner and writing the key information that helps a user understand 
their problem and the provided solution encompass many of the activities found in a developer's daily work, e.g., work related browsing, coding, debugging, and reading/writing documentation~\cite{Meyer2017}.
For example, Figure~\ref{fig:webview-task} depicts a task where a developer 
describes her struggles using the Android WebView component.
To answer this question, a developer will not only provide a code snippet but 
also explain key points of the Android Webview API~\cite{apiWebView}
and how they were used in the solution provided to the task.

\begin{figure}
    \centering
    \includegraphics[width=0.85\textwidth]{cp4/webview-task}
    \caption{Sample Stack Overflow task from our corpus}
    \label{fig:webview-task}
\end{figure}