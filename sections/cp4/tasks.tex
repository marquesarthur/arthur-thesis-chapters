\section{Software Tasks}
\label{cp4:corpus-tasks}

We start corpus creation by identifying software tasks 
 which a developer will likely benefit from 
the use of additional information to complete.
We scope task selection to \textit{Android development} 
because the 
Android \acf{SDK} evolves constantly due to 
functionality, security and performance-related improvements~\cite{Li2018android, Mateus2020}.
These improvements impact its development community, requiring them to often  seek information regarding changes in the SDK~\cite{linares2014, bavota2014b, mcdonnell2013}.
For example, over 35,000 developers have used Q\&A forums to discuss tasks covering 87\% of the classes in the Android API~\cite{parnin2012}.


Two common places where Android task can be found are:


\begin{itemize}
    \item the description of an issue
    (e.g., a bug or feature request) reported in an issue tracking system; or in
    \item a post in a community forum, development mailing lists, and others.
\end{itemize}

Several studies have used issue tracking systems and software development communities for software tasks~\cite{Arya2019, baltes2019, nadi2020, Xu2017}
and, following the lead
of these studies, we select GitHub issues and \acf{SO} posts on Android development as 
the two sources for the tasks in our corpus.



\subsubsection{GitHub tasks}

To select tasks from GitHub, we are guided by studies that use 
stars~\cite{borges2016, borges2018}
as a proxy for a project's popularity~\cite{Ferreira2016, Xavier2020}.
We selected 14 projects,
ranging from mail clients\footnote{\url{https://github.com/k9mail/k-9}}
to development frameworks\footnote{\url{https://github.com/libgdx/libgdx}},
by filtering the list of top-starred projects in GitHub to those with the \textit{Java} and \textit{Android} tags.
We then randomly selected 25 distinct issues  originating from these starred projects as the GitHub tasks of our corpus.
While selecting issues, we \rev{balanced the number of issues selected per project} (average of 1.78 issues per project)
\rev{and we ensured that all the issues were resolved}. We also took care to 
check that they had at least one follow-up comment and that the issue title did not contain certain words, e.g., {\small \texttt{test}} or {\small  \texttt{ignore}},
as these words indicate issues  created automatically by scripts or bots---a common pitfall that researchers must be aware of when mining GitHub~\cite{kalliamvakou2014}.


Figure~\ref{fig:lock-screen-task} shows an example of a GitHub task in our corpus.
Although the expected behaviour is that the app controls should be visible even with the screen locked,  a user reports that the app screen is missing.
A developer addressing this issue might need to review the Android lock screen documentation~\cite{apiLockTask}
or refer to examples of applications that use the Android lock screen~\cite{mediumLockApp}.
For the remainder of this chapter, we use the lock screen task as a running example.


\begin{figure}
    \centering
    \includegraphics[width=\textwidth]{cp4/lock-screen-task}
    \caption{Sample GitHub task from our corpus}
    \label{fig:lock-screen-task}
\end{figure}



\subsubsection{Stack Overflow tasks}

%Provided that we restrict our corpus to tasks related to %the Android development domain,


We consider Stack Overflow posts as software tasks because to answer a post,
a developer often needs to provide references
supporting their answer~\cite{yazdaninia2021}.
Finding these references in a timely manner and writing the key information that helps a user understand 
 the provided solution encompasses many of the activities found in a developer's daily work, e.g., work-related browsing, coding, debugging, and reading/writing documentation~\cite{Meyer2017}.
For example, Figure~\ref{fig:webview-task} depicts a task where a developer describes her struggles using the Android WebView component~\cite{apiWebView}.
To answer this question, a developer will not only provide a code snippet, but also explain key points of the Android WebView API
and how they were used in the solution provided to the task, 
as presented in Figure~\ref{fig:webview-task-answer}.


\begin{figure}
    \centering
    \includegraphics[width=0.98\textwidth]{cp4/webview-task}
    \caption{Sample Stack Overflow question}
    \label{fig:webview-task}
\end{figure}



\begin{figure}
    \centering
    \includegraphics[width=0.98\textwidth]{cp4/webview-task-answer}
    \caption{Excerpt of a Stack Overflow answer}
    \label{fig:webview-task-answer}
\end{figure}

We randomly select 25 Stack Overflow posts from a curated list about Android development~\cite{baltes2020}.
This list was built by Baltes et al. 
using the Stack Overflow dump published on June 5, 2018~\cite{baltes2019-rep, SOTorrent2019}
and it contains 209,536 unique posts with the \textit{Java} and \textit{Android} tags.



% When selecting tasks in GitHub and Stack Overflow, a major challenge arises due to the sheer amount of data available.
% Baltes et al.~\cite{baltes2019} argues that even a cursory inspection of a sample set
% of Stack Overflow posts shows clear differences in a post's content or structure due to aspects such as programming languages, frameworks, associated technologies, and others.
% \gm{-It feels like there is a missing sentence about how differences relate
% to the 'sheer amount of data' and why the differences matter.}

% To ensure the tasks in the corpus we produce
% circumvent \gm{-circumvent has a negative connotation - is there a way
% to say this positively?} the heterogeneity of data on GitHub and Stack Overflow, we scope task selection to the \textit{Android} development domain. This decision
% restricts task selection to a single programming language (\textit{Java})
% while still enabling investigation of a domain that has been
% widely discussed by practitioners and researchers alike.
