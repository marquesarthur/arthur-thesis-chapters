
\section{Corpus Summary}
\label{cp4:corpus-summary}

In this chapter, we introduced the need for corpora 
for the development of 
automatic techniques able to identify relevant text
to solve a task in artifacts
pertinent to the task.
Since  no such corpora
existed, we detailed 
a set of structured procedures for its creation. 
The \acs{DS-android} corpus consists of  
12,401 unique sentences
originating from artifacts associated to 50 software tasks
drawn from GitHub issues and Stack Overflow posts about Android development. 
We found that 
three annotators with professional experience indicated that,
out of these 12,401 unique sentences, 
\red{n} of them were relevant to a particular task and that 
annotators agreed on the relevancy of  
only a fraction of the marked sentences (\red{n\%}).
Due to these characteristics, 
we outline how our corpus can be used for evaluation purposes 
using both standard precision and recall metrics as well as using
metrics that equate how many annotators marked a sentence. 
Ultimately, we expect that the \acs{DS-android} corpus
lays the foundation for studies that explore relationships between software tasks
and text within different types of natural language artifacts.



