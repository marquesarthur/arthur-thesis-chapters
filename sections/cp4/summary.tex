
\section{Summary}
\label{cp4:corpus-summary}

In this chapter, we introduced the need for corpora 
for the development of 
automatic techniques able to identify relevant text
to solve a task in artifacts
pertinent to the task.
Since  no such corpora
existed, we detailed 
a set of structured procedures for its creation. 
The \acs{DS-android} corpus consists of  
12,401 unique sentences
originating from artifacts associated with 50 software tasks
drawn from GitHub issues and Stack Overflow posts about Android development. 
We found that 
three annotators with professional experience indicated that,
out of these 12,401 unique sentences, 
1,393 of them were relevant to a particular task and that multiple annotators agreed of the relevance of 29\% of the sentences,
which lead us to provide recommendations on how our data can be used for evaluation purposes.
Ultimately, we expect that the \acs{DS-android} corpus
provides a foundation for studies that explore relationships between software tasks and text found across different types of artifacts that a developer might seek information on and that are pertinent to the developer's task.
 
