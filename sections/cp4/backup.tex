
\subsection{Automatic relevant text detection}
\label{cp4:relevant-text-auto}





We rely on state-of-the-art approaches~\cite{nadi2020, Robillard2015, Lotufo2012, Xu2017} able to automatically identify relevant text for a parcel of the  artifact types in our corpus, namely API documentation, GitHub issues, and Stack Overflow answers.
The text automatically identified by these approaches can serve as a baseline for 
any new technique applied to the same artifact types. 
To establish this baseline we use the text identified by human annotators and report the accuracy of the approaches we make use of.


\subsubsection{Automatic approaches}


To identify approaches applicable to the artifacts in our corpus, we systematically reviewed related work. We searched for approaches based on their availability in existing replication packages and their readiness for use.
We also refrained from using approaches with training procedures (e.g., ~\cite{liu2020} or ~\cite{Treude2016}) because of the challenges related to correctly tuning such supervised approaches~\cite{Chaparro2017, fucci2019}. Based on these criteria, three approaches were selected for the artifact sources in parenthesis:


\begin{itemize}[leftmargin=\parindent, font=\normalfont\itshape]
    \item \texttt{\acs{AnsBot}} (\textit{SO Answers}) uses several features (e.g., information entropy, textual patterns, entity overlap, etc.) to determine that a sentence has useful information to a developer's technical question~\cite{Xu2017}.
    
    \item \texttt{\acs{Krec}} (\textit{API Documentation}) identifies text fragments that reflect ``potentially important text that programmers cannot afford to ignore when using the API''~\cite{Robillard2015}.
    
    \item \texttt{\acs{Hurried}} (\textit{GitHub issues}) identify the most relevant sentences in a bug report based on three factors used to assess a sentence's relevancy (i.e., sentence's prominence in the issue, topic, and its similarity to the task)~\cite{Lotufo2012}.
\end{itemize}


\gm{I think you have to be clearer  that
Misc are included as non-annotated artifacts.}
\art{rephrased}

Due to the amount of variety in the content of miscellaneous sources, 
applying automatic approaches to detect relevant text for these types of artifacts
incurs the risk of using an approach for an artifact that the approach was not designed for.
Therefore, we refrain from using automatic approaches for miscellaneous artifacts.
With the exception of \acs{DS-android-small}, miscellaneous artifacts do not have associated task-relevant text detected and we leave this to future work.



% For the pertinent artifacts of our running example (Figure~\ref{fig:lock-screen-task}), 
% Tables~\ref{tbl:git-example-ansbot} to~\ref{tbl:git-example-hurried}
% illustrate sentences automatically detected by \acs{AnsBot}, \acs{Krec}, and \acs{Hurried}, respectively.
% \gm{Why give these three examples?}

 

% 
\begin{table}[H]
\centering    
\begin{scriptsize}
\begin{threeparttable}
\rowcolors{2}{}{lightgray}
\begin{tabular}{ll}

\hline
\multicolumn{2}{c}{\textit{How to add MediaPlayer controls on lock screen?}} \\
\hline
\hline

1 & \parbox[l][.8cm][c]{10.5cm}{I had the same problem, and well, the solution was simple, do not use any widget, simply use the RemoteControlClientCompat class.} \\
2 & \parbox[l][.8cm][c]{10.5cm}{Here is my lockScreenControls() method code, which I call whenever I want to show this type of control (when plays a song).} \\
3 & \parbox[l][.5cm][c]{10.5cm}{Thank @ianhlake for the good 2 video} \\

\hline


\end{tabular}
\end{threeparttable}
\end{scriptsize}
\caption{Pertinent sentences automatically detected by \acs{AnsBot}}
\label{tbl:git-example-ansbot}
\end{table}
% 
\begin{table}[H]
\centering    
\begin{scriptsize}
\begin{threeparttable}
\rowcolors{2}{}{lightgray}
\begin{tabular}{ll}
    
\hline
\multicolumn{2}{c}{\textit{Lock task mode - Android Developers}} \\
\hline
\hline

1 & \parbox[l][.8cm][c]{10.5cm}{You might use lock task mode if you're developing a kiosk application or a launcher to present a collection of apps.} \\
2 & \parbox[l][.8cm][c]{10.5cm}{To check if the current app is running in lock task mode, use the methods on ActivityManager as shown in the following example:} \\
3 & \parbox[l][.8cm][c]{10.5cm}{You can call KeyguardManager methods to find out if the device is locked and use an Activity lifecycle callback (such as onResume() that's called after unlocking) to start lock task mode.} \\
\hline

\end{tabular}
\end{threeparttable}
\end{scriptsize}
\caption{Pertinent sentences automatically detected by \acs{Krec}}
\label{tbl:git-example-krec}
\end{table}
% 
\begin{table}[H]
\centering    
\begin{scriptsize}
\begin{threeparttable}
\rowcolors{2}{}{lightgray}
\begin{tabular}{ll}

\hline
\multicolumn{2}{c}{\textit{How to add MediaPlayer controls on lock screen?}} \\
\hline
\hline

1 & \parbox[l][.8cm][c]{10.5cm}{I had the same problem, and well, the solution was simple, do not use any widget, simply use the RemoteControlClientCompat class.} \\
2 & \parbox[l][.8cm][c]{10.5cm}{Here is my lockScreenControls() method code, which I call whenever I want to show this type of control (when plays a song).} \\
3 & \parbox[l][.5cm][c]{10.5cm}{Thank @ianhlake for the good 2 video} \\

\hline


\end{tabular}
\end{threeparttable}
\end{scriptsize}
\caption{Pertinent sentences automatically detected by \acs{Hurried}}
\label{tbl:git-example-hurried}
\end{table}







\subsection{Accuracy}
\label{cp4:relevant-text-accuracy}

\gm{You need to get across that in applying
the techniques to this new data you need to
check if the results are similar to what was
reported for the technique authors. Doesn't
this require you to speak to whether the accuracy
is similar to what the authors reported for
the techniques?}

\vspace{3mm}
\art{I'm stuck here because of two potential result outcomes: similar results VS negative results.}



\vspace{3mm}
\gm{W}e apply a set of approaches 
to tasks and artifacts outside the ones where they were originally proposed and evaluated~\cite{nadi2020, Robillard2015, Lotufo2012, Xu2017}.
For instance, \acs{AnsBot}'s design was based on general Java programming tasks~\cite{Xu2017} while the tasks in our corpus comprise Android development and originate both from GitHub and from Stack Overflow. 
Likewise, \acs{Krec} was originally designed using the Java SE documentation~\cite{Robillard2015} 
while API documents in \acs{DS-android} comprise Android development\footnote{\url{https://developer.android.com/docs}}.
Because of such differences, we ask:


\begin{enumerate}[label={},leftmargin=0.7cm]
\item \textit{What is the accuracy of the approaches used to automatically detect text relevant in the task and artifacts in \acs{DS-android}?} 

\end{enumerate}


Answering this question will determine the portion of the text in \acs{DS-android} identified by an approach that 
is indeed relevant (i.e., precision) and how much 
of the relevant text an approach is able to detect (i.e., recall).
In turn, this information will help future research in understanding
how the automatically detected text in our corpus can be used in evaluation of new techniques for the automatic detection of relevant text. 






% To understand precision and recall metrics metrics, Table~\ref{tbl:type-I-II-errors} show all possible outcomes  when evaluating a technique able to automatically detect text relevant to a software task. The \textit{relevant} and \textit{not-relevant} columns represent the text 
% marked (or not) by \textit{at least two} annotators~\cite{Lotufo2012}. Rows represent the text identified by an automatic approach.


% \begin{table}[H]
\centering    
\begin{scriptsize}
\begin{threeparttable}
\begin{tabular}{l|l|l}

\hline

\textbf{}
& \textbf{Relevant}    
& \textbf{Not-relevant} \\

\hline
\hline

\textbf{Identified as relevant} & true positive ($TP$) & false positive ($FP$) \\
\hline
\textbf{Identified as Not-relevant} & false negative ($FN$) & true negative ($TN$) \\
\hline

\end{tabular}
\end{threeparttable}
\end{scriptsize}
\caption{Result outcomes}
\label{tbl:type-I-II-errors}
\end{table}

    



% For a given task $t$ and artifact $a$, $precision$ is the fraction of the sentences identified that are marked as relevant by at least two annotators over the total number of sentences identified, as shown in Equation~\ref{eq:cp4:precision}.


% \begin{equation}
% \label{eq:cp4:precision}    
%     Precision(t, a) = \frac{TP}{TP + FP}
% \end{equation}


% Recall ($recall$) represents how many of all sentences marked by at least two annotators are identified by a technique (Equation~\ref{eq:cp4:recall}).


% \begin{equation}
% \label{eq:cp4:recall}        
%     Recall(t, a) = \frac{TP}{TP + FN}
% \end{equation}






% \vspace{3mm}




% \subsection{Results}
% \textcolor{white}{force ident} % this is just for the chapter outline


% --- Discuss results.\footnote{\red{think which summary tables should I have in the thesis body and which I can move to Appendices}} \vspace{3mm}



% --- Precision~\ref{tbl:ds-small-results-precision}  \vspace{3mm}


% --- Recall~\ref{tbl:ds-small-results-recall} \vspace{3mm}

% --- Likely explanation for the results obtained.

% % When interpreting results, we favor precision instead of recall.
% % A false positives may contribute to a developer abandoning reading of an artifact that would otherwise provide crucial information for her task~\cite{Rastkar2010}.




% \begin{table}[H]
\centering    
\begin{scriptsize}
\begin{threeparttable}
\begin{tabular}{lcccccc}

\hline


\multirow{2.5}{*}{Technique}
& \multicolumn{2}{c}{\textit{$Precision_{n=3}$}}
& \multicolumn{2}{c}{\textit{$Precision_{n=2}$}}
& \multicolumn{2}{c}{\textit{$Precision_{n=1}$}}
\\ \cmidrule(l){2-3} \cmidrule(l){4-5} \cmidrule(l){6-7} 


& \textit{mean}
& \textit{std}
& \textit{mean}
& \textit{std}
& \textit{mean}
& \textit{std}
\\


\hline
\hline

\acs{AnsBot} 
& 0.5 & 0.5 % = 3
& 0.5 & 0.5 % = 2
& 0.5 & 0.5 % = 1
\\

\acs{Krec} 
& 0.5 & 0.5 % = 3
& 0.5 & 0.5 % = 2
& 0.5 & 0.5 % = 1
\\

\acs{Hurried} 
& 0.5 & 0.5 % = 3
& 0.5 & 0.5 % = 2
& 0.5 & 0.5 % = 1
\\

\hline

\end{tabular}
\end{threeparttable}
\end{scriptsize}
\caption{Precision of each technique for the tasks of \acs{DS-android-small}}
\label{tbl:ds-small-results-precision}
\end{table}

    

% \begin{table}[H]
% \centering    
% \begin{scriptsize}
% \begin{threeparttable}
% \begin{tabular}{lcccccccccccc}

% \hline


% \multirow{2.5}{*}{Technique}
% & \multicolumn{10}{c}{\textit{Tasks}} 
% & \multicolumn{2}{c}{\textit{Precision}}
% \\  \cmidrule(l){2-11} \cmidrule(l){12-13} 



% &
% \textit{T1} & \textit{T2} & \textit{T3} & \textit{T4} & \textit{T5}
% & \textit{T6} & \textit{T7} & \textit{T8} & \textit{T9} & \textit{T10}
% & \textit{mean}
% & \textit{std}
% \\


% \hline
% \hline

% \acs{AnsBot} 
% & 0.5 & 0.5 & 0.5 & 0.5 & 0.5
% & 0.5 & 0.5 & 0.5 & 0.5 & 0.5
% & 0.5 % mean
% & 0.5 % std
% \\

% \acs{Krec} 
% & 0.5 & 0.5 & 0.5 & 0.5 & 0.5
% & 0.5 & 0.5 & 0.5 & 0.5 & 0.5
% & 0.5 % mean
% & 0.5 % std
% \\

% \acs{Hurried} 
% & 0.5 & 0.5 & 0.5 & 0.5 & 0.5
% & 0.5 & 0.5 & 0.5 & 0.5 & 0.5
% & 0.5 % mean
% & 0.5 % std
% \\

% \hline

% \end{tabular}
% \end{threeparttable}
% \end{scriptsize}
% \caption{Precision of each technique for the tasks of \acs{DS-android-small}}
% \label{tbl:ds-small-results-precision}
% \end{table}

    

% \begin{table}[H]
\centering    
\begin{scriptsize}
\begin{threeparttable}
\begin{tabular}{lcccccc}

\hline


\multirow{2.5}{*}{Technique}
& \multicolumn{2}{c}{\textit{$Recall_{n=3}$}}
& \multicolumn{2}{c}{\textit{$Recall_{n=2}$}}
& \multicolumn{2}{c}{\textit{$Recall_{n=1}$}}
\\ \cmidrule(l){2-3} \cmidrule(l){4-5} \cmidrule(l){6-7} 


& \textit{mean}
& \textit{std}
& \textit{mean}
& \textit{std}
& \textit{mean}
& \textit{std}
\\


\hline
\hline

\acs{AnsBot} 
& 0.5 & 0.5 % = 3
& 0.5 & 0.5 % = 2
& 0.5 & 0.5 % = 1
\\

\acs{Krec} 
& 0.5 & 0.5 % = 3
& 0.5 & 0.5 % = 2
& 0.5 & 0.5 % = 1
\\

\acs{Hurried} 
& 0.5 & 0.5 % = 3
& 0.5 & 0.5 % = 2
& 0.5 & 0.5 % = 1
\\

\hline

\end{tabular}
\end{threeparttable}
\end{scriptsize}
\caption{Recall of each technique for the tasks of \acs{DS-android-small}}
\label{tbl:ds-small-results-recall}
\end{table}

    


% \subsection{Threats}
% \label{cp4:corpus-threats}

% --- \red{Argue that this is not a replication study but rather establishing thresholds} \vspace{3mm}

% --- Discuss threats \vspace{3mm}



