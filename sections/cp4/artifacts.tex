
\section{Artifact Selection}
\label{cp4:corpus-artifacts}


When selecting artifacts pertinent to a task in our corpus, we simulate everyday practices on how developers search the Web~\cite{rao2020, Xia2017}. That is, we formulate a query for each task and use a Web search engine to retrieve artifacts that are pertinent to that task.


\subsubsection{Artifact sources}

As there are many different sources of artifacts, we restrict artifact selection to well known and studied sources~\cite{Starke2009,Kevic2014, Li2013}, i.e.,
Android and Java SE API documentation, Github issues, Stack Overflow answers; and Web tutorials or blog posts from Java and Android development.



\subsubsection{Query formulation}



We consider a task's title (i.e., SO question or GitHub issue title) as the seed used to search artifacts
using the \texttt{googlesearch} API~\cite{googlesearch}.


Coming up with proper search terms is a critical step of any search~\cite{Haiduc2013}
and, ideally, we should be able to formulate a query with terms able to retrieve the most pertinent artifacts for a software task.
However, studies have shown that developers perform poorly in identifying good search terms~\cite{Starke2009,Kevic2014, Li2013} and thus, using a task's title
as an educated approximation to terms that a developer might use is a common procedure adopted by other studies in the field (e.g.,~\cite{Xu2017} or ~\cite{Silva2019}).







\subsubsection{Search results}


We fetch a maximum of 5 resources per artifact source --- a limitation necessary due to throttling or even blocking mechanisms in the APIs used to get the content of each source considered. 


When selecting results, we exclude any result that does not appear in the Amazon Alexa~\cite{alexa} Web  traffic for Java and Android development in the period from April 2020 to March 2021. 
While applying this filter potentially decreased the number of artifacts per task, it ensured that results were indeed related to Android development. 
For instance, for a task discussing ``\textit{left and right-hand swap}'' 
filtering avoided fetching unrelated resources, such as a Web page on  ``\textit{stock swap}''.
Table~\ref{tbl:googlesearch-example-git} shows one search result per artifact source for the GitHub task introduced in Section~\ref{cp4:corpus-tasks}.


\begin{table}[H]
\centering    
\begin{scriptsize}
\begin{threeparttable}
\rowcolors{2}{}{lightgray}    
\begin{tabular}{l|l}

\hline

\multicolumn{2}{c}{\textit{Saving WebView page to cache}}  \\

\hline
\hline

\multirow{1}{*}{API documentation}
& Managing WebView objects - Android Developers \\
% & WebView - Android Developers \\

\multirow{1}{*}{Github issues}
& WebView Caching contents WebView using the cachePolicy \\
% & Investigate whether removing WebView files on erase is a good idea \\


\multirow{1}{*}{StackOverflow answers}
& Android WebView not loading second page from cache \\
% & Save webview content for offline browsing? \\
\hline

\end{tabular}
\end{threeparttable}
\end{scriptsize}
\caption{Sample of artifacts obtained for a StackOverflow task~\cite{so18607655}}
\label{tbl:googlesearch-example-so}
\end{table}



\begin{table}[H]
\centering    
\begin{scriptsize}
\begin{threeparttable}
\rowcolors{2}{}{lightgray}    
\begin{tabular}{l|l}

\hline

\multicolumn{2}{c}{\textit{No lock screen controls ever}}  \\

\hline
\hline

\multirow{1}{*}{API documentation}
& Lock task mode - Android Developers \\
% & Recents Screen - Android Developers \\

\multirow{1}{*}{Github issues}
& Lock screen controls disappear on Android 11 \\
% & Bug: No lock screen image and controls \\


\multirow{1}{*}{StackOverflow answers}
& How to add MediaPlayer controls on lock screen? \\
% & How to disable home button in Android like lock screen apps do? \\



\multirow{1}{*}{Miscellaneous}
& Create A React Native App - Which works on Lock Screen (Android) \\

\hline


\end{tabular}
\end{threeparttable}
\end{scriptsize}
\caption{Sample of artifacts obtained for a Github task~\cite{git3578} }
\label{tbl:googlesearch-example-git}
\end{table}



\subsubsection{Artifact's content}


Processing an artifact's content into a sequence of individual sentences 
followed procedures analogous to other studies in the field~\cite{Arya2019, nadi2020}.
Given a search result \texttt{URL}, we use \texttt{BeautifulSoup}~\cite{beautifulsoup4},
\texttt{StackAPI}~\cite{StackAPI} and \texttt{PyGithub}~\cite{PyGithub}
to fetch the artifacts' content. Sentences in each paragraph
were identified using the Stanford CoreNLP toolkit~\cite{CoreNLP}.








