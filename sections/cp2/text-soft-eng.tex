\section{Natural Language Text in Software Artifacts}
\label{cp2:text-in-se}




% Natural language  software artifacts are produced and consumed on a continuous basis~\cite{Rastkar2013t}
% and 

Software engineering researchers
have investigated several questions on the nature of the text in software artifacts.
These studies 
have focused on different kinds of artifacts and have deepen
 researchers' knowledge on the richness of information in natural language artifacts,
 serving as a first step towards the design of automatic tools 
for identifying and extracting text from these artifacts~\cite{Arya2019, Maalej2013}.




A significant body of work has 
 proposed taxonomies for the text available 
 in development mailing lists~\cite{Sorbo2015, huang2018automating},
API documents~\cite{Maalej2013}, bug reports~\cite{Arya2019}, and others.
In written development emails, Di Sorbo et al. have found that 
text can be classified according to its purpose (e.g., feature request, solution proposal, etc.)~\cite{Sorbo2015},
suggesting that text under certain categories might be useful for developers performing specific tasks. 
They have sampled 100 emails and identified that text for feature requests 
often contain expressions in the form suggestions
(e.g., `\textit{we should add a new button}') while solution proposals 
are often expressed in the form of attempts (e.g., `\textit{let's try a new method to compute cost}').



Maalej and Robillard have inspected the Java SDK 6 and .NET 4.0. API documentation~\cite{Maalej2013}
and proposed a set of categories (e.g., directives, purpose, rationale, functionality, etc.) that define how 
knowledge is structured in API documentation. 
This taxonomy led Robillard and Chhetri 
to find that directives---or pieces of information that the programmers
cannot afford to ignore, such as instructions on how to efficiently
perform some I/O operation---often 
follow certain linguistic patterns~\cite{Robillard2015},
what made they produce a catalog of patterns that capture directives found in the 
Java SDK 6 documentation.



Rastkar et al. observe that much of the text in a bug report resembles a 
conversation between different subjects~\cite{Rastkar2013} 
while Arya et al. argue that due to this conversational nature, 
a single bug report contains several threads each 
pertaining to some topic, e.g., 
reproducing a bug, discussing potential solutions, 
testing, or reporting progress~\cite{Arya2019}. 
Although the topicality of the content in bug reports 
might encourage the design of tools that automatically identify such
text, researchers have also found high variance in the text of different 
issues within a software project~\cite{Chaparro2016},
what poses significant challenges to approaches that 
gather information from multiple bug reports.



Although the studies described above shed light on many 
aspects of natural language text in software artifacts, 
they do not discuss common properties of the text 
that is relevant to some software task
found in different kinds of artifacts.
In Chapter~\ref{ch:characterizing}
we study text in API documentation, bug reports, and community forums that is considered relevant to a software task.


% A number of studies~\cite{Piorkowski2015, Piorkowski2016, chi2007, Ko2006a} has focused on general aspects on the organization and structure of 
% software engineering artifacts and how these aspects affect how developers find text that assist them complete a software task~\cite{BenCharrada2016, Forward2002, Starke2009, DeGraaf2014}.




% Among the many information foraging studies in software engineering,
%  Forward and Lethbridge describe criteria that developers use to assess the relevance of software engineering documentation. They observe that the content of an artifact, the presence of examples, and the document's structure are common attributes that impact an artifact's relevance~\cite{Forward2002}.
% In another study, Charrada and Mussato investigated how 
% practitioners manage and search for software engineering documents, 
% identifying that data scattered across different sources is 
% often a significant challenge to the information foraging process.



% When attempting to find artifacts pertinent to a task, 
% Starke et al. explain that a software developer often makes a set of hypotheses about a software task,
% translating these hypotheses into multiple search queries that will lead to  likely pertinent artifacts~\cite{Starke2009}. 
% They observe that, instead of systematically inspecting search results,
% developers often skim through them to decide which documents are of relevance. 
% In other studies, Brandt et al. found that developers interleave web foraging, learning and writing code~\cite{Brandt2009a} while de Graaf and colleagues 
% identified that prior knowledge about the structure of a software artifact helps professionals
% to search software documents efficiently and effectively~\cite{DeGraaf2014}.


% This thesis adds to the existing theory by discussing 
% common properties in the text of an artifact which several software developers deemed relevant 
% to certain software tasks.  










% Standard \acf{IR} approaches locate pertinent artifacts
% using metrics to compute the relevance of an artifact to a query posed manually by a developer.
% \acs{IR} uses the terms shared between a search query and an artifact
% to compute the relevance of an artifact (or document).
% Search algorithms weight and compute relevance based on 
% criteria such as how many documents contain a given term or  
% how unique a term is~\cite{Manning2009IR}.



% A number of the techniques commonly employed by software engineering researchers are based on the
% frequency of co-occurrence of words (or phrases) in documents.
% An early example is Maarek and Smadja's use of lexical relations to index
% software libraries~\cite{maarek1989}.
% Since this early use, software engineering
% researchers have continued to leverage advances in
% these approaches, such as when
% Antoniol et al. applied \acf{VSM}~\cite{Salton1975vsm} 
% to construct vector representations 
% for recovering traceability links 
% between code and documentation~\cite{antoniol1999, antoniol2000}
% or when Marcus and Maletic used \acf{LSI}~\cite{deerwester1990LSI}
% to help cluster software components to aid
% program comprehension of a software system~\cite{marcus2003}.



% While IR systems are widely used, researchers 
% observed that developers often find it difficult to identify good search terms~\cite{Kevic2014, Huang2018},
% what significantly impacts the retrieval of relevant documents~\cite{Kevic2014, mills2017}.
% To mitigate  the necessity of a developer coming up with good search terms,
% researchers have used information available in a task to automatically generate search terms~\cite{Kevic2014, Haiduc2013}. 
% Although automatically generating search terms assist the retrieval of pertinent artifacts,
% standard \acs{IR} algorithms will have 
% little success if search terms 
% differ significantly from the text in a software artifact~\cite{Huang2018}.
% These so-called \textit{lexical mismatches}~\cite{Ye2016, silva2019} have led researchers to investigate 
%  semantic-based \acs{IR} approaches, which we discuss in 
% Section~\ref{cp2:artifact-semantics}.




% % 


% \subsection{Contextual Approaches} 


% Contextual approaches use information 
% about a developer's task so that 
% they recommend artifacts pertinent to that task drawing data from some knowledge base.
% % , for example, a tool might need to know what part of the system is a developer currently inspecting. 
% Fishtail~\cite{Sawadsky2011}
% is an example of a tool that uses contextual information. It uses the source code currently being inspected by a developer to find web resources relevant to the developer's task.



% While the web is the most accessible knowledge base, certain contextual approaches consider that 
% collection of artifacts produced and consumed in past software tasks 
%  implicitly
% form knowledge bases that might assist developers performing new tasks~\cite{Cubranic2005}. 
% Hipikat~\cite{Cubranic2005} is a seminal tool that exemplifies this concept in action.
% It automatically tracks the artifacts 
% used by a developer as part of a development tasks
% so that it can recommend these artifacts to 
% any future developers who work on similar tasks~\cite{Cubranic2005}.
% As another example, Deep Intellisense
% finds code changes, filed bugs, and forum discussions 
% mining these artifacts from an organization's shared database~\cite{Holmes2008}.










% In the past decade, question-and-answer (\acs{qa}) web sites such as Stack Overflow\footnote{\url{https://stackoverflow.com/}}, have also
% become a common knowledge source used by contextual approaches~\cite{Ponzanelli2013b, Ponzanelli2014b, Treude2016, delfim2016}.
% For example, {\small PROMPTER} uses contextual information 
% available in a developer's IDE to retrieve Stack Overflow discussions that might be pertinent 
% to the developer's current task. Other tools such as {\small BIKER}~\cite{Huang2018} mine Stack Overflow posts to assist developers in locating API components needed in a programming tasks. 





% This thesis assumes the usage of one or more techniques presented here to locate pertinent
% artifacts from which task-relevant text will be extracted.



