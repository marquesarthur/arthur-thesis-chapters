\section{Finding Pertinent Artifacts}
\label{cp2:foraging-tools}

% \art{I tried  summarize this section, but it still looks lengthy}


Software development often requires knowledge beyond what developers already posses~\cite{Li2013} and thus, 
software developers use different information sources to fill such gaps. 
More than often, a developer will ask if any of their 
peers have the information that they need~\cite{singer2011}. 
However, the fragmented and distributed nature of software development  
might prevent a developer from approaching their peers~\cite{ko2007}.
Due to this and other reasons~\cite{Xia2017, rao2020}, a developer might seek
online web resources for information 
that may assist them in completing the task-at-hand.
This section provides a breadth of approaches that help a developer find such resources.




\subsection{Information Retrieval Approaches} 
\label{cp2:ir-approaches}





Standard \acf{IR} approaches locate pertinent artifacts
using metrics to compute the relevance of an artifact to a query posed manually by a developer.
\acs{IR} uses the terms shared between a search query and an artifact
to compute the relevance of an artifact (or document).
Search algorithms weight and compute relevance based on 
criteria such as how many documents contain a given term or  
how unique a term is~\cite{Manning2009IR}.



A number of the techniques commonly employed by software engineering researchers are based on the
frequency of co-occurrence of words (or phrases) in documents.
An early example is Maarek and Smadja's use of lexical relations to index
software libraries~\cite{maarek1989}.
Since this early use, software engineering
researchers have continued to leverage advances in
these approaches, such as when
Antoniol et al. applied \acf{VSM}~\cite{Salton1975vsm} 
to construct vector representations 
for recovering traceability links 
between code and documentation~\cite{antoniol1999, antoniol2000}
or when Marcus and Maletic used \acf{LSI}~\cite{deerwester1990LSI}
to help cluster software components to aid
program comprehension of a software system~\cite{Marcus2003}.



While IR systems are widely used, researchers 
observed that developers often find it difficult to identify good search terms~\cite{Kevic2014, Huang2018},
what significantly impacts the retrieval of relevant documents~\cite{Kevic2014, mills2017}.
To mitigate  the necessity of a developer coming up with good search terms,
researchers have used information available in a task to automatically generate search terms~\cite{Kevic2014, Haiduc2013}. 
Although automatically generating search terms assist the retrieval of pertinent artifacts,
standard \acs{IR} algorithms will have 
little success if search terms 
differ significantly from the text in a software artifact~\cite{Huang2018}.
These so-called \textit{lexical mismatches}~\cite{Ye2016, silva2019} have led researchers to investigate 
 semantic-based \acs{IR} approaches, which we discuss in 
Section~\ref{cp2:artifact-semantics}.




% 


\subsection{Contextual Approaches} 


Contextual approaches use information 
about a developer's task so that 
they recommend artifacts pertinent to that task drawing data from some knowledge base.
% , for example, a tool might need to know what part of the system is a developer currently inspecting. 
Fishtail~\cite{Sawadsky2011}
is an example of a tool that uses contextual information. It uses the source code currently being inspected by a developer to find web resources relevant to the developer's task.



While the web is the most accessible knowledge base, certain contextual approaches consider that 
collection of artifacts produced and consumed in past software tasks 
 implicitly
form knowledge bases that might assist developers performing new tasks~\cite{Cubranic2005}. 
Hipikat~\cite{Cubranic2005} is a seminal tool that exemplifies this concept in action.
It automatically tracks the artifacts 
used by a developer as part of a development tasks
so that it can recommend these artifacts to 
any future developers who work on similar tasks~\cite{Cubranic2005}.
As another example, Deep Intellisense
finds code changes, filed bugs, and forum discussions 
mining these artifacts from an organization's shared database~\cite{Holmes2008}.










In the past decade, question-and-answer (\acs{qa}) web sites such as Stack Overflow\footnote{\url{https://stackoverflow.com/}}, have also
become a common knowledge source used by contextual approaches~\cite{Ponzanelli2013b, Ponzanelli2014b, Treude2016, delfim2016}.
For example, {\small PROMPTER} uses contextual information 
available in a developer's IDE to retrieve Stack Overflow discussions that might be pertinent 
to the developer's current task. Other tools such as {\small BIKER}~\cite{Huang2018} mine Stack Overflow posts to assist developers in locating API components needed in a programming tasks. 





This thesis assumes the usage of one or more techniques presented here to locate pertinent
artifacts from which task-relevant text will be extracted.



