

\section{Interpreting the Semantics of Text}
\label{cp2:artifact-semantics}


% In Section~\ref{cp2:ir-approaches}




At times, software engineering researchers have argued
that general lexicon techniques 
are insufficient to address text appearing in
software engineering artifacts. 
Arguments on why lexicon-based natural 
language techniques are not applicable are often based
on a need for access to the \textit{meaning}, or semantics, 
of words, phrases or sentences appearing in text~\cite{jurafsky2014speech}.
In this section, 
we present background information on semantics focusing on
its usage in software engineering research.



\subsection{Word Semantics}

Word semantic techniques are mostly rooted on the hypothesis
that similar words appear in similar context~\cite{harris1954distributional}.
Hence, word semantic techniques use statistical models to infer the meaning of words. 


Early word semantic techniques, such as \acf{LSI}~\cite{deerwester1990LSI}, 
have been used by software engineering researchers 
to improve the 
the retrieval of artifacts pertinent to a certain task. 
For example, Marcus and Maletic apply \acf{LSI} to 
recover traceability links between source code and
software documentation~\cite{marcus2003}.
Other techniques, such as \acf{LDA}~\cite{blei2003latent},
assist to group and cluster textual information 
what allowed researchers to identify common topics in developers' blog posts~\cite{Pagano2011}
or facilitated the design of tools that identify duplicated bug reports~\cite{nguyen2012}.








\subsection{Sentence Semantics}







Surprisingly 
most of the deep learning approaches applied to software engineering tasks 
focus on source code and 

natural language text has been very shy....



% Semantic-based approaches  build a meaning by establishing relationships between terms,
% for instance words, based on the context in which these terms appear.
% For example
% the terms `\textit{car}' and `\textit{automobile}' are likely to co-occur in different phrases with terms like `\textit{motor}' and `\textit{wheel}' and thus,
% semantic approaches indicate that these terms are likely similar in meaning~\cite{Bavota2016}.

% As an example of the usage of semantics in software engineering researcher, Marcus and Maletic applied Latent Semantic Indexing (LSI) to help cluster software components to aid
% program comprehension of a software systems~\cite{marcus2003}. 
% Ye et al.'s study on the usage of word embeddings~\cite{Ye2016} was among the first to use the language models in the software engineering domain~\cite{Mikolov2013}. 

% Researchers have also developed approaches
% to interpret the meaning of sentences in
% software engineering documents. 
% Maalej and Robillard have
% developed a knowledge taxonomy~\cite{Maalej2013} using grounded theory to analyze documentation in open-source systems and then, they validated their taxonomy on documentation units sampled from the Java SDK 6 and .NET 4.0.
% Arya et al.'s have determined information types in open-source issue discussions~\cite{Arya2019},  Di Sorbo et al. have created an approach to classify  emails based on a developer's intentions~\cite{Sorbo2015} and
%  Marques et al. have investigated the use of frame semantics~\cite{fillmore1976frame}.
%  The techniques we propose in this paper
%  build on these earlier findings.





% \subsection{Machine Learning Approaches} 


% To bridge lexical gaps, researchers have investigated 
% approaches that leverage the 
% semantic aspects of software artifacts~\cite{Maletic2001, Ye2016}.
% While Section~\ref{cp2:artifact-semantics} provides an overview of semantics 
% in software engineering, this section especifically details how semantics might assist in finding 
% pertinent artifacts. 









%  is an early example of an \acs{IR} method that address \textit{lexical mismatches}.

% These and more advanced approaches 


% or when Nguyen and colleagues
% applied Word2Vec~\cite{mikolov2013word2vec} to support the retrieval of API
% examples~\cite{nguyen2017}.




% While ML approaches are effective in locating pertinent artifacts, a developer still needs to
% manually find the parts within that artifact
% that are relevant to her task~\cite{Cubranic2005}. 







% they have been used 
% for several purposes by software engineering researchers.



% While it is obvious to the human reader that `car' and `automobile' are synonyms and can be used interchangeable,  
% standard \acs{IR}  fail to capture this and other differences\footnote{
%     \textbf{Oxford English dictionary definitions}~\cite{dictionary1989oxford} 
%     \begin{itemize}
%         \item \textbf{synonymy:} the fact of two or more words having the same meaning;
%         \item \textbf{polysemy:} the fact of having more than one meaning.
%     \end{itemize}
% }
% that significantly affected the performance of \acs{IR} systems~\cite{DeLucia2012}.
% To address these issues, researchers have proposed a set of word semantic techniques the use statistical models to infer the meaning of words. 
