

\section{Semantics in Software Engineering}
\label{cp2:artifact-semantics}




% We use the term semantics
% to refer to having access to a meaning for words, 
% phrases or sentences appearing in text.
% Semantic-based approaches  build a meaning by establishing relationships between terms,
% for instance words, based on the context in which these terms appear.
% For example
% the terms `\textit{car}' and `\textit{automobile}' are likely to co-occur in different phrases with terms like `\textit{motor}' and `\textit{wheel}' and thus,
% semantic approaches indicate that these terms are likely similar in meaning~\cite{Bavota2016}.

% As an example of the usage of semantics in software engineering researcher, Marcus and Maletic applied Latent Semantic Indexing (LSI) to help cluster software components to aid
% program comprehension of a software systems~\cite{Marcus2003}. 
% Ye et al.'s study on the usage of word embeddings~\cite{Ye2016} was among the first to use the language models in the software engineering domain~\cite{Mikolov2013}. 

% Researchers have also developed approaches
% to interpret the meaning of sentences in
% software engineering documents. 
% Maalej and Robillard have
% developed a knowledge taxonomy~\cite{Maalej2013} using grounded theory to analyze documentation in open-source systems and then, they validated their taxonomy on documentation units sampled from the Java SDK 6 and .NET 4.0.
% Arya et al.'s have determined information types in open-source issue discussions~\cite{Arya2019},  Di Sorbo et al. have created an approach to classify  emails based on a developer's intentions~\cite{Sorbo2015} and
%  Marques et al. have investigated the use of frame semantics~\cite{fillmore1976frame}.
%  The techniques we propose in this paper
%  build on these earlier findings.





% \subsection{Machine Learning Approaches} 


% To bridge lexical gaps, researchers have investigated 
% approaches that leverage the 
% semantic aspects of software artifacts~\cite{Maletic2001, Ye2016}.
% While Section~\ref{cp2:artifact-semantics} provides an overview of semantics 
% in software engineering, this section especifically details how semantics might assist in finding 
% pertinent artifacts. 









%  is an early example of an \acs{IR} method that address \textit{lexical mismatches}.

% These and more advanced approaches 


% or when Nguyen and colleagues
% applied Word2Vec~\cite{mikolov2013word2vec} to support the retrieval of API
% examples~\cite{nguyen2017}.




% While ML approaches are effective in locating pertinent artifacts, a developer still needs to
% manually find the parts within that artifact
% that are relevant to her task~\cite{Cubranic2005}. 