



\section{Information Foraging}
\label{cp2:information-foraging}



Information foraging theory~\cite{Pirolli1999} explains how a person
has an information need and how they navigate through  
a search space looking for information patches  based on 
a set of cues about the effort and gains that a patch provides to them~\cite{Pirolli1999}.




Foraging happens between patches (e.g., consulting different sources) or within a patch (e.g., over the many sections in a web page) 
For example,
consider an electronic library catalog and a search for `\textit{android notifications}'.
Figure~\ref{fig:library-catalogue} shows two hypothetical results for this search. 
Foraging between patches represents deciding which book to consult 
and \textit{explicit} or \textit{implicit} factors affect judging the relevance of a patch~\cite{saracevic1975}.
An explicit factor might represent how the keywords in the `\textit{subject and genre}' field
directly match one of the keywords used in the person's query. An implicit factor might represent 
a person's background or previous knowledge, such as how 
they might know that the first book targets a more general population and 
that it might not be helpful for a software developer.
As shown in our motivating example (Figure~\ref{fig:android-search-results}),
a similar search could be performed on a web search engine\footnote{e.g., \href{https://www.google.com/}{Google} or \href{https://www.bing.com/}{Microsoft Bing}}
and each web page represents an information patch.




\vspace{3mm}
\begin{figure}[H]
\noindent\fbox{%
\parbox{\textwidth}{%
{\ttfamily% 
    \begin{scriptsize}
        \begin{enumerate}[itemindent=-1em,leftmargin=1em]
            \item \textbf{Android for Dummies} 

            Author: Gookin, Dan 

            Location: Britannia Branch Library 

            LC Call No.: 004.167 A57G6ad 

            Subject and genre: Android (electronic resource), Smartphone, Mobile computing

            % Synopsis: This book will tell you pretty much everything you need to know about your Android smartphone or tablet in an equally friendly manner, because that's the best way to learn how to get the most from your Android

        \end{enumerate}          
    \end{scriptsize}
    }%
}}
\noindent\fbox{%
\parbox{\textwidth}{%
{\ttfamily% 
    \begin{scriptsize}
        \begin{enumerate}[itemindent=-1em,leftmargin=1em]\setcounter{enumi}{1}
            \item \textbf{Learning Android} 

            Author: Gargenta, Marko 

            Location: Champlain Heights Branch Library 

            LC Call No.: 005.44 A5G2L1 

            Subject and genre: Android (electronic resource), Application software (development), Mobile computing

            % Synopsis: Presents an introduction on the fundamentals of Android to create a variety of applications.

        \end{enumerate}
    \end{scriptsize}
    }%
}}
\caption{Sample library catalog with two results associated with the keyword `android' }
\label{fig:library-catalogue}
\end{figure}



We can  similar principles to assessing relevance of information within a patch. 
For instance, let us consider the Android notification Stack Overflow discussion (Figure~\ref{fig:qa-notification-icon}).
Each answer in a Stack Overflow post might represent a patch.
An explicit factor for judging which answer to read first might relate to 
meta-data available in the platform, such as whether an answer is an accepted answer.
In turn, an implicit factor might represent a person's preference for succinct answers.







For both between-patches and within-patches foraging, researchers have investigated 
what affects how a person finds pertinent results and their decision to inspect them.
For example, in electronic library catalogs,
Hildreth observed how keyword searches were used more than any other type of search (e.g., by author or by title)~\cite{hildreth1997}
while other studies observed impatience and near-random search habits in novice librarians~\cite{novotny2004don}.
In contemporary web search engines, researchers have  also 
investigated how individuals formulate queries~\cite{gross2005have, bendersky2012},
how prior knowledge helps them
to more efficiently perform searches~\cite{DeGraaf2014},
and what search results they inspect, going to the lengths
of using eye-tracking technology for this purpose~\cite{Cutrell2007, marcos2015}.
