\section{Information Foraging and Relevance Theory}
\label{cp2:foraging}




To understand how software developers find  information pertinent to their software tasks,
software engineering researchers have extensively studied information foraging~\cite{Pirolli1999}
and relevance theory~\cite{clark2013relevance, saracevic1975, Saracevic2007c, Saracevic2007b}.


\textit{\acf{IFT}}~\cite{Pirolli1999} explains how 
an individual navigates through some search space looking for \textit{information patches} based on 
a set of cues about the effort and gains that a patch provides to them~\cite{Pirolli1999}.
Foraging occurs \textit{between} or \textit{within} patches, i.e., 
searching \textit{which} documents are relevant and \textit{what} within a document is of relevance,
and \textit{relevance theory}~\cite{clark2013relevance, saracevic1975} elaborates the relationships between an information object (i.e., text in an artifact),
some context (i.e., a task), and what properties guide a forager's relevance assessment (i.e., clarity, utility, completeness, etc.)~\cite{Saracevic2007c}. 




Within software development~\cite{Piorkowski2015, Piorkowski2016, chi2007, Ko2006a},
 Forward and Lethbridge describe criteria that developers use to assess the relevance of software engineering documentation, where they observe that the content of an artifact, the presence of examples, and the document's structure are common attributes that impact an artifact's relevance~\cite{Forward2002}.
As another example, Charrada and Mussato studied how 
practitioners manage and search for software engineering documents, 
identifying that data scattered across different sources is 
often a significant challenge to the information foraging process.



Starke et al. explain that a software developer often makes a set of hypotheses about a software task,
translating these hypotheses into multiple search queries that will lead to artifacts likely pertinent to the developer's task~\cite{Starke2009}. 
They observe that, instead of systematically inspecting search results,
developers often skim through search results to decide which documents are of relevance. 
In other studies, Brandt et al. found that developers interleave web foraging, learning and writing code 
to accomplish software tasks~\cite{Brandt2009a} while de Graaf and colleagues 
identified that prior knowledge about the structure of a software artifact helps professionals
to search software documents efficiently and effectively~\cite{DeGraaf2014}.


This thesis adds to the existing theory by discussing 
common properties in the text of an artifact which several software developers deemed relevant 
to certain software tasks.  

