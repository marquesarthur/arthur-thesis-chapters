


\section{Finding Pertinent Artifacts}
\label{cp2:searching}


We start by considering how an individual finds 
artifacts, or documents, which might 
address an information need. 
For this, information foraging theory~\cite{Pirolli1999} explains how a person navigates through  
a search space looking for information patches  based on 
a set of cues about the effort and gains that a patch provides to them.


According to Pirolli and Card~\cite{Pirolli1999}, a forager performs scent-following activities, which represent 
how they navigate a search space until they find a high-yield patch. 
Once in a patch, a forager inspects its content until their information need is addressed or until the amount of new information available depletes, making the forager move to another patch. 


In both instances (navigating between patches and searching within a patch), \textit{explicit} or \textit{implicit} factors affect how a forager judges the relevance of a patch~\cite{saracevic1975}.
For example,
consider an electronic library catalog and a search for `\textit{android notifications}'.
Figure~\ref{fig:library-catalogue} shows two hypothetical results for this search. %to help illustrate explicit and implicit factors
An explicit factor might represent how one of the keywords in the `\textit{subject and genre}' field
directly match one of the keywords used in the person's query. An implicit factor might represent 
a person's background or previous knowledge, such as how 
they might know that the first book targets a more general population and 
that it might not be helpful for a software developer.







A similar search could be performed on a web search engine\footnote{e.g., \href{https://www.google.com/}{Google} or \href{https://www.bing.com/}{Microsoft Bing}}
and, with their prominence, other search problems also gained attention.
Most notably,  Carbonell and Goldstein investigated the relevance and the novelty of search results~\cite{Carbonell1998}.
On the one hand, prioritizing relevance may lead to a scenario where all results contain redundant information, leaving some information needs unanswered. 
On the other hand, it might be difficult for an individual 
to find information that corroborates and consolidates some knowledge if search results are too diverse~\cite{clark2013relevance}.  
This has led researchers to investigate 
algorithms that balance the relevance and the novelty of the results retrieved
by a web search~\cite{najork2001, rafiei2010, vieira2011}
and commercial search engines have since considered how to strike a balance
between relevance and novelty, seeking to provide diverse  
yet relevant results.



% and for both library catalogs or web searches, researchers have investigated 
% what affects how a person finds pertinent results and their decision to inspect them.
% For example, in electronic library catalogs,
% Hildreth observed how keyword searches were used more than any other type of search (e.g., by author or by title)~\cite{hildreth1997}
% while other studies observed impatience and near-random search habits in novice librarians~\cite{novotny2004don}.
% In contemporary web search engines, researchers have  also 
% investigated how individuals formulate queries~\cite{gross2005have, bendersky2012},
% how prior knowledge helps them
% to more efficiently perform searches~\cite{DeGraaf2014},
% and what search results they inspect, going to the lengths
% of using eye-tracking technology for this purpose~\cite{Cutrell2007, marcos2015}.



\vspace{3mm}
\begin{figure}[H]
\noindent\fbox{%
\parbox{\textwidth}{%
{\ttfamily% 
    \begin{scriptsize}
        \begin{enumerate}[itemindent=-1em,leftmargin=1em]
            \item \textbf{Android for Dummies} 

            Author: Gookin, Dan 

            Location: Britannia Branch Library 

            LC Call No.: 004.167 A57G6ad 

            Subject and genre: Android (electronic resource), Smartphone, Mobile computing

            % Synopsis: This book will tell you pretty much everything you need to know about your Android smartphone or tablet in an equally friendly manner, because that's the best way to learn how to get the most from your Android

        \end{enumerate}          
    \end{scriptsize}
    }%
}}
\noindent\fbox{%
\parbox{\textwidth}{%
{\ttfamily% 
    \begin{scriptsize}
        \begin{enumerate}[itemindent=-1em,leftmargin=1em]\setcounter{enumi}{1}
            \item \textbf{Learning Android} 

            Author: Gargenta, Marko 

            Location: Champlain Heights Branch Library 

            LC Call No.: 005.44 A5G2L1 

            Subject and genre: Android (electronic resource), Application software (development), Mobile computing

            % Synopsis: Presents an introduction on the fundamentals of Android to create a variety of applications.

        \end{enumerate}
    \end{scriptsize}
    }%
}}
\caption{Sample library catalog with two results associated with the keyword `android' }
\label{fig:library-catalogue}
\end{figure}








Across information science studies that investigate information foraging, 
a common trend is that the identification of good search terms is as, or even more
important than the search algorithm itself~\cite{Kevic2014}. 
Nonetheless, identifying good search terms is often challenging and 
as a consequence, many searches are unsuccessful or retrieve resources that 
a person loses time inspecting only to find that they are not pertinent to the task at hand~\cite{novotny2004don, Haiduc2013}.
This indicates that individuals usually inspect multiple and potentially different types of artifacts 
before finding relevant information, which further motivates the work presented in this thesis.




Within software engineering research, the concept of an information patch 
is often more fine-grained and it usually relates to source code, where a significant body of work has 
used information foraging theory in the context of software documentation~\cite{Forward2002, DeGraaf2014, Wildemuth2012},
program comprehension~\cite{piorkowski2013, Ko2006a}, and debugging or refactoring activities~\cite{fleming2013, lawrance2010}. 
More recently, researchers have also investigated developers' web search 
behaviour~\cite{Starke2009, Brandt2009a, Xia2017}, concentrating on what developers search for on web search engines. 
We differ from these studies by focusing 
on the content within different kinds of natural language artifacts, i.e., 
the natural language text within them.




% which contrasted how more experienced users were able to better translate an information 
% need into keyword searches that retrieved useful results


