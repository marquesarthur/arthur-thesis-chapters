


\section{Finding Pertinent Artifacts}
\label{cp2:task-approaches}


We start by considering how an individual finds 
artifacts, or documents, that they might deem relevant for 
the task that they are working on. 


For this end, information foraging theory~\cite{Pirolli1999} explains how a person navigates through  
a search space looking for information patches  based on 
a set of cues about the effort and gains that a patch provides to them~\cite{Pirolli1999}.
A patch can represent both search results or the content in an artifact
and several \textit{explicit} or \textit{implicit} factors affect judging the relevance 
of an information patch~\cite{saracevic1975}.
For example,
consider an electronic library catalog and a search for `\textit{android notifications}'.
Figure~\ref{fig:library-catalogue} shows two hypothetical results for this search
to help ilustrate explicit and implicit factors.
An explicit factor might represent how the keywords in the `\textit{subject and genre}' field 
of the entries retrieved 
directly match the keyword used in a search. An implicit factor might represent 
a person's background or previous knowledge, for example 
the person performing this search 
might know that the first book targets a more general population and 
that it might not be helpful for a software developer.





\vspace{3mm}
\begin{figure}[H]
\noindent\fbox{%
\parbox{\textwidth}{%
{\ttfamily% 
    \begin{scriptsize}
        \begin{enumerate}[itemindent=-1em,leftmargin=1em]
            \item \textbf{Android for Dummies} 

            Author: Gookin, Dan 

            Location: Britannia Branch Library 

            LC Call No.: 004.167 A57G6ad 

            Subject and genre: Android (electronic resource), Smartphone, Mobile computing

            % Synopsis: This book will tell you pretty much everything you need to know about your Android smartphone or tablet in an equally friendly manner, because that's the best way to learn how to get the most from your Android

        \end{enumerate}          
    \end{scriptsize}
    }%
}}
\noindent\fbox{%
\parbox{\textwidth}{%
{\ttfamily% 
    \begin{scriptsize}
        \begin{enumerate}[itemindent=-1em,leftmargin=1em]\setcounter{enumi}{1}
            \item \textbf{Learning Android} 

            Author: Gargenta, Marko 

            Location: Champlain Heights Branch Library 

            LC Call No.: 005.44 A5G2L1 

            Subject and genre: Android (electronic resource), Application software (development), Mobile computing

            % Synopsis: Presents an introduction on the fundamentals of Android to create a variety of applications.

        \end{enumerate}
    \end{scriptsize}
    }%
}}
\caption{Sample library catalog with two results associated with the keyword `android' }
\label{fig:library-catalogue}
\end{figure}




A similar search could be performed on a web search engine\footnote{\href{https://www.google.com/}{Google} or \href{https://www.bing.com/}{Microsoft Bing}}
and for both library catalogs or web searches, researchers have investigated 
what affects how a person finds pertinent results and their decision to inspect them.
For example, in electronic library catalogs,
Hildreth observed how keyword searches were used more than any other type of searches (e.g., by author or by title)~\cite{hildreth1997}
while other 
others studies observed impatience and near-random search habits in novice librarians~\cite{novotny2004don}.
In contemporary web search engines, researchers have  
investigated how individuals formulate queries~\cite{gross2005have, bendersky2012},
how prior knowledge helps professionals
to more efficiently and effectively perform searches~\cite{DeGraaf2014},
and what search results they inspect, going to lengths
of using eye-tracking technology for this purpose~\cite{Cutrell2007, marcos2015}.



With the prominence of web search engines, other search problems gained attention.
For example,  Carbonell and Goldstein pioneered the need 
to investigate both the relevance and novelty of search results~\cite{Carbonell1998}.
On one hand, if one returns only relevant resources, it may be the case that all of them contain redundant information 
which might leave some information needs unanswered. 
On the other hand, if the search results are too diverse, it might be difficult for an individual 
to find information that corroborates and consolidates some conclusion~\cite{clark2013relevance}.  
This has led researchers to investigated 
how to balance the relevance and novelty of the results retrieved
by a web search~\cite{najork2001, rafiei2010, vieira2011}. 





Across these general studies, and also in software engineering-specific studies~\cite{Starke2009, Brandt2009a, DeGraaf2014},
a common trend is that the identification of good search terms is as, or even more
important than the search algorithm itself~\cite{Kevic2014}. 
Nonetheless, identifying good search terms is often challenging~\cite{novotny2004don, Haiduc2013} and 
as a consequence, many searches are unsuccessful or retrieve resources that 
a person loses time inspecting only to find that they are not pertinent to the task at hand.


This survey of the state of the art indicates that individuals usually inspect multiple types of artifacts 
to find task-relevant information, motivating the study of generalizable techniques 
for this purpose as proposed in this thesis.



% which contrasted how more experienced users were able to better translate an information 
% need into keyword searches that retrieved useful results


