


\section{Finding Pertinent Artifacts}
\label{cp2:task-approaches}


We start by considering how an individual finds 
artifacts, or documents, that they might deem relevant for 
the task that they are working on. 


Information foraging theory~\cite{Pirolli1999} explains how one would navigate through  
a search space looking for information patches  based on 
a set of cues about the effort and gains that a patch provides to them~\cite{Pirolli1999}.
A patch can represent both search results or the content within a result
and several \textit{explicit} or \textit{implicit} factors affect the judged relevance 
of an information patch~\cite{saracevic1975}.
For example,
consider an electronic library catalog and a search for `\textit{android notifications}'.
Figure~\ref{fig:library-catalogue} shows two hypothetical results for this search
to help ilustrate explicit and implicit factors, i.e., 
an explicit factor might represent how the keywords in the `\textit{subject and genre}' field 
directly match the search keyword whereas an implicit factor might represent how 
the person performing this search 
knows that the first book targets a more general population and 
that it might not be helpful for a software developer.





\vspace{3mm}
\begin{figure}[H]
\noindent\fbox{%
\parbox{\textwidth}{%
{\ttfamily% 
    \begin{scriptsize}
        \begin{enumerate}[itemindent=-1em,leftmargin=1em]
            \item \textbf{Android for Dummies} 

            Author: Gookin, Dan 

            Location: Britannia Branch Library 

            LC Call No.: 004.167 A57G6ad 

            Subject and genre: Android (electronic resource), Smartphone, Mobile computing

            % Synopsis: This book will tell you pretty much everything you need to know about your Android smartphone or tablet in an equally friendly manner, because that's the best way to learn how to get the most from your Android

        \end{enumerate}          
    \end{scriptsize}
    }%
}}
\noindent\fbox{%
\parbox{\textwidth}{%
{\ttfamily% 
    \begin{scriptsize}
        \begin{enumerate}[itemindent=-1em,leftmargin=1em]\setcounter{enumi}{1}
            \item \textbf{Learning Android} 

            Author: Gargenta, Marko 

            Location: Champlain Heights Branch Library 

            LC Call No.: 005.44 A5G2L1 

            Subject and genre: Android (electronic resource), Application software (development), Mobile computing

            % Synopsis: Presents an introduction on the fundamentals of Android to create a variety of applications.

        \end{enumerate}
    \end{scriptsize}
    }%
}}
\caption{Sample library catalog with two results associated with the keyword `android' }
\label{fig:library-catalogue}
\end{figure}




A similar search could be performed on a web search engine\footnote{\href{https://www.google.com/}{Google} or \href{https://www.bing.com/}{Microsoft Bing}}
and for both library catalogs or web searches, researchers have investigated 
what affects how a person finds pertinent results and their decision to inspect them.
For example, in electronic library catalogs,
Hildreth observed how keyword searches were used more than any other type of searches (e.g., by author or by title)~\cite{hildreth1997}
while other 
others studies observed impatience and near-random search habits by novice librarians~\cite{novotny2004don}.
In contemporary web search engines, researchers have  
investigated how individuals formulate queries~\cite{gross2005have, bendersky2012},
how prior knowledge helps professionals
to more efficiently and effectively perform searches~\cite{DeGraaf2014},
and what search results they inspect, going to lengths
of using eye-tracking technology for this purpose~\cite{Cutrell2007, marcos2015}.




Across these general studies, and also in software engineering-specific studies~\cite{Starke2009, Brandt2009a, DeGraaf2014},
a common trend is that the identification of good search terms is as, or even more
important than the search algorithm itself~\cite{Kevic2014}. 
Nonetheless, 
studies have shown how identifying good search terms is challenging~\cite{novotny2004don, Haiduc2013} and 
as a consequence, many searches are unsuccessful or retrieve resources that only upon 
their inspection it is revealed that an artifact is not pertinent to the task at hand.
In summary, individuals usually inspect multiple types of artifacts before
finding obtaining the knowledge that triggered their search. 



% which contrasted how more experienced users were able to better translate an information 
% need into keyword searches that retrieved useful results


