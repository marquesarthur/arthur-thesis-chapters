



\section{Improving Developers' Productivity}
\label{cp2:dev-productivity}


The tools and approaches presented in Sections~\ref{cp2:general-approaches}
and~\ref{cp2:task-approaches}
are examples of studies that help developers in locating 
text that might address an information need.
These studies fit in the bigger context 
of software engineering research 
facilitating or improving the quality of a developer's work~\cite{Kersten2006, Meyer2017, satterfield2020}. 



Software developers engage in many sensemaking and decision-making activities~\cite{sillito2006} and several studies have investigated how to 
provide means to better  
assist developers in performing such activities~\cite{Liu2018Unakite, liu2021, barnett2015}.
For instance, to assist a developer in deciding which web pages to inspect next,
researchers have proposed tools that monitor a developer's search history
and show how similar or not the results of a new web search are in comparison to already visited 
pages~\cite{Ponzanelli2017}.



Other tools focus on assisting a developer make sense of the
code that they might have to change as part of a task. For example, 
Deep Intellisense~\cite{Holmes2008} displays code changes, filed bugs, and forum discussions
mined from an organization's database so that a developer can understand the rationale
behind the code.
Other tools such as CueMeIn~\cite{sun2021} use online resources for this same purpose, 
finding excerpts from web tutorials 
that might contain explanations 
for the classes and methods that a developer inspects. 



Researchers have also been interested in 
making knowledge bases
that developers working on a task can benefit from. 
Hipikat~\cite{Cubranic2005} is a seminal tool that exemplifies this concept in action.
It takes a query explicitly prompted by a developer 
or implicitly based on the code that the developer 
has inspected and  
it recommends artifacts from a project's archive 
that are pertinent to the query.
This archive, or project memory, is produced 
based either on relationships between the artifacts in a software project 
or based on the files changed or accessed as part of a bug fix or feature request~\cite{Cubranic2005}.
Other tools like Strata summarize knowledge produced by developers 
while they navigate on the web so that other developers
can use this knowledge to have a 
 head start when performing similar tasks~\cite{liu2021}.


This dissertation builds upon many of the research procedures outlined 
in these and many other studies in the field. 
For example, in the evaluation of Strata, 
 Liu et al. describe procedures 
considering how a control and tool-assisted group 
complete information-seeking tasks~\cite{liu2021},
which assisted in the design of our experiment for 
evaluating an automated approach to
task-relevant text identification (Chapter~\ref{ch:assisting}).
