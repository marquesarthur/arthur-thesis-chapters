



\section{Improving Developers' Productivity}
\label{cp2:dev-productivity}



The tools and approaches presented in Section~\ref{cp2:automatic-approaches}
are examples of studies that help developers in locating 
information that assists them to complete a software task.
These studies fit in the bigger context 
of software engineering research 
that facilitates or improves the quality of a developer's work~\cite{Kersten2006, Meyer2017, satterfield2020}. 



Researchers have had a long interest in 
making knowledge bases
that developers working on a task can benefit from. 
Hipikat~\cite{Cubranic2005} is a seminal tool that exemplifies this concept in action.
It automatically tracks the artifacts 
used by a developer as part of a development tasks \gm{?}
so that it can recommend these artifacts to 
any future developers who work on similar tasks~\cite{Cubranic2005}.
Other tools like Strata summarize knowledge produced by developers 
while they navigate on the web so that other developers have a 
a head start when performing similar tasks~\cite{liu2021}.



As part of their work, developers engage in many sensemaking and decision making activities~\cite{sillito2006} and several studies have investigated how to 
provide means to better  
assist developers in performing such activities~\cite{Liu2018Unakite, liu2021, barnett2015}.
For example, tools such as Deep Intellisense 
display code changes, filed bugs, and forum discussions 
mined from an organization's database to better assist a developer
understand 
the rationale behind the code they currently work on~\cite{Holmes2008},
and researchers have extended this idea to  
publicly available data, e.g., GitHub issues~\cite{Viviani2019}
or pull requests~\cite{freire2021}. 



We build upon many of the research procedures outlined 
in these and many other studies in the field. 
For example, in the evaluation of Strata 
 Liu et al. describe procedures 
considering how a control and tool-assisted group 
complete information-seeking tasks~\cite{liu2021}
that assisted in the design of our experiment for 
evaluating an automated approach to
task-relevant text identification (Chapter~\ref{ch:assisting}).



