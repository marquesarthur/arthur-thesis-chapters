



\section{Improving Developers' Productivity}
\label{cp2:dev-productivity}




% Software developers are knowledge workers that not only 
% are overloaded with information~\cite{},
% but that also switch between several tasks in 
% the course of a work day.
Researchers have and practitioners  have had a long interest 
in providing tools that improve the quality of a developer's work~\cite{Kersten2006, Meyer2017, satterfield2020}. 


\art{TODO}

% The work presented in this dissertation fit in the umbrella of studies 


% , in this section, we discuss 
% other research in the field, honing in on studies that 
% honing in on studies that help developers working on a particular task. 



% As a result, researchers
% and practitioners have both had a long interest in better understanding how developers work and how their work
% understanding how developers work and how their work could be quantified to optimize productivity and efficiency.
% could be quantified to optimize productivity and efficiency. Researchers have investigated work practices and work


% \art{Use contributions to drive related work}

% \art{I might even talk about Google search somewhere here}


% To understand how software developers find  information useful to their software tasks,
% software engineering researchers have mostly used theories from the information foraging
% field~\cite{Pirolli1999}. 


% \textit{\acf{IFT}}~\cite{Pirolli1999} explains how 
% an individual navigates through some search space looking for \textit{information patches} based on 
% a set of cues about the effort and gains that a patch provides to them~\cite{Pirolli1999}.
% Foraging occurs between or within patches, i.e., 
% searching which documents are relevant and what within a document is of relevance.
% In turn, \textit{relevance theory}~\cite{clark2013relevance, saracevic1975} further elaborates the relationships between an information object (i.e., text in an artifact),
% some context (i.e., a task), and what properties (i.e., clarity, utility, completeness, etc.) guide a forager decision 
% on the relevance of a patch~\cite{Saracevic2007b, Saracevic2007c}. 



% Among the many information foraging studies in software engineering~\cite{Piorkowski2015, Piorkowski2016, chi2007, Ko2006a},
%  Forward and Lethbridge describe criteria that developers use to assess the relevance of software engineering documentation. They observe that the content of an artifact, the presence of examples, and the document's structure are common attributes that impact an artifact's relevance~\cite{Forward2002}.
% In another study, Charrada and Mussato investigated how 
% practitioners manage and search for software engineering documents, 
% identifying that data scattered across different sources is 
% often a significant challenge to the information foraging process~\cite{BenCharrada2016}.



% When attempting to find artifacts pertinent to a task, 
% Starke et al. explain that a software developer often makes a set of hypotheses about a software task,
% translating these hypotheses into multiple search queries that will lead to  likely pertinent artifacts~\cite{Starke2009}. 
% They observe that, instead of systematically inspecting search results,
% developers often skim through them to decide which documents are of relevance. 
% In other studies, Brandt et al. found that developers interleave web foraging, learning and writing code~\cite{Brandt2009a} while de Graaf and colleagues 
% identified that prior knowledge about the structure of a software artifact helps professionals
% to search software documents efficiently and effectively~\cite{DeGraaf2014}.


% This thesis adds to the existing theory by discussing 
% common properties in the text of an artifact which several software developers deemed relevant 
% to certain software tasks.  
