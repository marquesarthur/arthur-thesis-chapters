



\section{Improving Developers' Productivity}
\label{cp2:dev-productivity}


\art{stopped reviewing related-work here}


% The tools and approaches presented in Sections~\ref{cp2:general-approaches}
% and~\ref{cp2:task-approaches}
% are examples of studies that help developers in locating 
% information that assists them in completing a software task.
% These studies fit in the bigger context 
% of software engineering research 
% facilitating or improving the quality of a developer's work~\cite{Kersten2006, Meyer2017, satterfield2020}. 





% As part of their work, developers engage in many sensemaking and decision-making activities~\cite{sillito2006} and several studies have investigated how to 
% provide means to better  
% assist developers in performing such activities~\cite{Liu2018Unakite, liu2021, barnett2015}.
% For example, 
% Ponzanelli et al. proposed a tool, Libra, that monitors the web pages a developer 
% has navigated and uses that to show 
% how similar or not the results of a new web search are in comparison to the already navigated 
% pages, which offers contextual information-seeking support~\cite{Ponzanelli2017}.



% Researchers have also been interested in 
% making knowledge bases
% that developers working on a task can benefit from. 
% In Section~\ref{cp2:task-approaches} 
% we cited Hipikat~\cite{Cubranic2005}, a seminal tool that exemplifies this concept in action.
% Based on the current code being inspected by a developer or based on a query prompted by a developer, 
% it recommends artifacts from a project's archive 
% that are pertinent to the code being inspected.
% This archive, or project memory, is produced 
% based either on relationships between the artifacts in a software project 
% or based on the source code changed in a bug fix or feature request~\cite{Cubranic2005}.
% Other tools like Strata summarize knowledge produced by developers 
% while they navigate on the web so that other developers
% can use this knowledge to have a 
%  head start when performing similar tasks~\cite{liu2021}.




% We build upon many of the research procedures outlined 
% in these and many other studies in the field. 
% For example, in the evaluation of Strata, 
%  Liu et al. describe procedures 
% considering how a control and tool-assisted group 
% complete information-seeking tasks~\cite{liu2021},
% which assisted in the design of our experiment for 
% evaluating an automated approach to
% task-relevant text identification (Chapter~\ref{ch:assisting}).



