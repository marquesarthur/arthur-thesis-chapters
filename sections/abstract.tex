

\chapter*{Abstract}

\begin{small}
The information that a developer seeks to aid in the completion of a task typically exists across different kinds of natural language software artifacts. In the artifacts that a developer consults, only some portions of the text will be useful to a developer's current task. Locating just the portions of text useful to a given task can be time-consuming as the artifacts can include substantial text to peruse organized in different ways depending on the type of artifact. For example, artifacts structured as tutorials might be easier to locate information given their structured headings whereas artifacts consisting of developer conversations might need to be read in detail. 

To aid developers in this activity, given the limited time they have to spend on any given task, researchers have proposed a range of techniques to automate the identification of relevant text. 
However, this prior work is generally constrained to one or only a few types of artifacts.  Integrating such artifact-specific approaches to allow developers to seamlessly search for information across the multitude of artifact types they find relevant to a task is challenging, if not impractical.

In this dissertation, we propose a set of generalizable techniques to aid developers in locating the portion of the text that might be useful for a task. These techniques are based on semantic patterns that arise from the empirical analysis of the text relevant to a task in multiple kinds of artifacts, leading us to propose techniques that incorporate the semantics of words and sentences to automatically identify text likely relevant to a developer's task.

We evaluate the proposed techniques assessing the extent to which they identify text that developers deem relevant in different kinds of artifacts associated with Android development tasks. We then investigate how a tool that embeds the most promising semantic-based technique might assist developers while they perform a task. Results show that semantic-based techniques perform equivalently well across multiple artifact types and that a tool that automatically provides task-relevant text assists developers effectively complete a software development task.
    
\end{small}

