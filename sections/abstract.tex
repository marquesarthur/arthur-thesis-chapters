

\chapter*{Abstract}


\vspace{-8mm}
The information a developer seeks to aid the completion of a task typically exists across different kinds of natural language software artifacts. However, only portions of the text within these artifacts might be useful to a developer's current task and a developer that fails to locate most of the information needed will have an incomplete or incorrect basis from which to perform a task. A problem exacerbated by the fact that developers often consult multiple artifacts of different kinds to understand what is needed for a task.


To aid developers in this activity, prior work has proposed a range of techniques that automate the identification of relevant text. Nonetheless, this support is generally constrained to one or only a few types of artifacts and integrating such artifact-specific approaches to allow developers to seamlessly search for information is challenging, if not impractical.


In this dissertation, we propose a set of generalizable techniques to aid developers in locating the portion of the text that might be useful for a task. We base our design on semantic patterns that arise from the empirical analysis of the text relevant to a task in multiple kinds of artifacts, leading us to propose techniques that incorporate the semantics of words and sentences to automatically identify text likely relevant to a developer's task.


We evaluate the proposed techniques assessing the extent to which they identify text that developers deem relevant in different kinds of artifacts associated with Android development tasks. Then, in a second study, we investigate how a tool that embeds a semantic-based technique might assist developers while they perform a task. Results show that semantic-based techniques perform equivalently well across multiple artifact types and that a tool automatically provides task-text relevant assists developers effectively complete a software development task.


% We investigate possible 
% patterns in the text relevant to software tasks across different types of artifacts
% and whether techniques building on approaches to interpret the meaning, or semantics, of the text help 
% identify relevant text across all kinds of artifacts a developer  may consult.


% across the different types of artifacts that a developer may come across in their daily work



% In this dissertation, 



% Prior work has proposed approaches to automatically identify relevant text for particular kinds of artifacts. Although effective, these techniques rely on assumptions about an artifact's structure or content that prevent applying them across the different types of artifacts that a developer may come across in their daily work.

% In this thesis, we propose techniques building on approaches to interpret the meaning, or semantics, of the text help to overcome these limitations.


% \vfill
% \begin{center}
% \begin{sf}
% \fbox{Revision: \today}
% \end{sf}
% \end{center}





% The abstract is a concise and accurate summary of the scholarly work described in the document. It states the problem, the methods of investigation, and the general conclusions, and should not contain tables, graphs, complex equations, or illustrations. There is a single scholarly abstract for the entire work, and it must not exceed 350 words in length.



% many approaches attempt to automatically extract relevant information from natural language artifacts.
% However, existing approaches are able to identify relevant text only for certain
% types of tasks and artifacts.


% To explore how \rev{these} limitations could be relaxed, \rev{we conducted a controlled experiment in which we asked 20 software developers to examine 20 natural language artifacts consisting of 1,874 sentences and highlight the text they considered relevant to six software development tasks.





%  For example, for a task that requires upgrading to a new version of an API component, a developer may seek information in the API's official documentation, check community discussions about the newer version, and so on. To aid developers in locating the portion of the text that might be useful in these artifacts, prior work has used syntactic properties of the text, and an artifact's meta-data, to automatically identify relevant text for particular kinds of artifacts. Although effective, these techniques rely on assumptions about an artifact's structure or content that prevent applying them across the different types of artifacts that a developer may come across in their daily work. In this paper, we investigate whether techniques building on approaches to interpret the meaning, or semantics, of the text help to overcome these limitations. Particularly, we introduce six semantic-based techniques and evaluate that they can identify up to 58\% of the text that developers deem relevant to Android development tasks. When compared to a state-of-the-art approach, we find that our techniques achieve comparable recall values, identifying 63\% of the small fraction of the task-relevant text of Stack Overflow artifacts, but without the need for artifact-specific information.


% Not 


% Since failing to locate relevant information may lead
% developers to incorrect or incomplete solutions, many approaches attempt to au-
% tomatically extract relevant information from natural language artifacts. However,
% existing approaches are able to identify relevant text only for certain types of tasks
% and artifacts.



%  To aid developers in locating the portion of the text that might be useful in these artifacts,  In this paper, we investigate whether techniques building on approaches to interpret the meaning, or semantics, of the text help to overcome these limitations. Particularly, we introduce six semantic-based techniques and evaluate that they can identify up to 58\% of the text that developers deem relevant to Android development tasks. When compared to a state-of-the-art approach, we find that our techniques achieve comparable recall values, identifying 63\% of the small fraction of the task-relevant text of Stack Overflow artifacts, but without the need for artifact-specific information.
