




\section{Presenting Task-Relevant Text}
\label{cp7:info-viz}




In Chapter~\ref{ch:assisting}, we described how \acs{tool}
highlights the text in a natural language artifact that it identifies 
as relevant to an input task. Our idea to highlight text 
draws from related work which suggests that textual highlights 
are a simple approach to surface the most important information in 
an artifact~\cite{Robillard2015,nadi2020}. 
Although simple, participants shared limitations of 
this strategy:



\smallskip
\begin{footnotesize}
\begin{quote}
``\textit{It was much easier to follow with previously highlighted text.  
    However it would have been nicer to have some sort of side bar/index of highlighted snippets
    where I could know and scroll directly through the highlighted parts of a page.}''
\end{quote}
\end{footnotesize}



\smallskip
\begin{footnotesize}
\begin{quote}
``\textit{There was a resource page which was super long, and I found it very difficult to locate which sentences were highlighted, therefore, making that resource useless to me because I didn't have the motivation to scroll through it to find all highlights. An interface that gathers highlight locations would make a difference.}''
\end{quote}
\end{footnotesize}


% \smallskip
% \begin{footnotesize}
% \begin{quote}
% ``\textit{the highlights were helpful, but not in terms of orienting myself on the page.}''---P17
% \end{quote}
% \end{footnotesize}



 


\smallskip
This feedback made us question other potential ways to present the text identified by \acs{tool}.
For example, we could have followed design principles adopted by Unakite~\cite{Liu2018Unakite}---a tool that collects, organizes and keeps track of information using context menu---to make \acs{tool} display the text identified on its context menu.
In this menu, a user could click on the text identified and navigate to the part of the documentation 
originally containing the text identified.



% To discuss potential ways to visually represent the 
% relevant text identified by \acs{tool}, we take inspiration from Strata~\cite{liu2021} and Libra~\cite{Ponzanelli2017}.
A second approach to organize the text identified could consider 
the semantic meaning of each sentence, as extract through frame semantics or other semantic-based approaches. 
For instance, the frames identified could 
assist a user in comprehending the content of the text automatically identified
without the need to read it. 
Semantic tags could also be used to group the text identified in bubbles or a hierarchical representation 
which, similar to Libra~\cite{Ponzanelli2017}, could help a user 
in deciding what portions of the text they would inspect first. 
For example, a user could first read sentences 
with frames associated with directives on how
to use some API element and then, 
jump to sentences with warnings, known issues,
or limitations about that API. 
Further research is needed to understand how to group and synthesize the text
that is relevant to a task. 




% By providing tags or labels indicating the information conveyed in a relevant sentence 




% Libra's interface provides a bubble chart with bubbles representing 
% web pages obtained from a search result list and axis representing 
% how the content of a web page complements or adds to the content 
% already seen by a developer~\cite{Ponzanelli2017}.
% Following Libra's design, 
% we could have used a bubble chart to present the text identified. 
% A bubble could represent the semantic meaning of a sentence, as extract through frame semantic parsing.
% Its radius could indicate the text's similarity with regards to the developers task
% while the bubble position could represent its semantic
% distance to the other relevant text identified and, by clicking on a bubble, 
% \acs{tool} could direct a developer to the part of the artifact containing 
% the text. This chart could assist both in navigating through the artifact's content 
% and in comprehending the content of the text automatically identified.


% A second approach 



% \begin{figure}[h!]
%   \centering	
%   \includegraphics[width=.7\textwidth]{figs/bubble-representation.pdf}	
%   \caption{Visual representation of task-relevant text. As a user selects a bubble (representing a group of semantically similar task-relevant text), the tool provides the synthesized text for that group. Figure adapted from the representation in ~\cite{Ponzanelli2017}}	
%   \label{fig:bubble-representation}	
% \end{figure}
