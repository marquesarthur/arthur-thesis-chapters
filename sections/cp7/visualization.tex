




\section{Visually Presenting Task-Relevant Text}
\label{cp7:info-viz}




In Chapter~\ref{ch:assisting}, we described how \acs{tool}
highlgihts the text in a natural language artifact that it identifies 
as relevant to an input task. Our idea to highlight text 
draws from related work suggesting that textual highlgihts 
are a simple approach to surface the most important information in 
an artifact~\cite{Robillard2015,nadi2020}. 
Although simple, this approach has limitations that could have been 
addressed if we had considered other methods for presenting the text identified,
as noted by one of the participants in the study:




\smallskip
\begin{footnotesize}
\begin{quote}
``\textit{It was much easier to follow [the task] with previously highlighted text.  
    However it would have been nicer to have some sort of side bar/index of highlighted snippets
    where I could know and scroll directly through the highlighted parts of a page.}''
\end{quote}
\end{footnotesize}


\smallskip
To discuss potential ways to visually represent the 
relevant text identified by \acs{tool}, we take inspiration from Strata~\cite{liu2021} and Libra~\cite{Ponzanelli2017}.
Libra's interface provides a bubble chart with bubbles representing 
web pages obtained from a search result list and axis representing 
how the content of a web page complements or adds to the content 
already seen by a developer~\cite{Ponzanelli2017}.
Following Libra's design, 
we could have used a bubble chart to present the text identified. 
A bubble could represent the semantic meaning of a sentence, as extract through frame semantic parsing.
Its radius could indicate the text's similarity with regards to the developers task
while the bubble position could represent its semantic
distance to the other relevant text identified and, by clicking on a bubble, 
\acs{tool} could direct a developer to the part of the artifact containing 
the text. This chart could assist both in navigating through the artifact's content 
and in comprehending the content of the text automatically identified.


A second approach 



% \begin{figure}[h!]
%   \centering	
%   \includegraphics[width=.7\textwidth]{figs/bubble-representation.pdf}	
%   \caption{Visual representation of task-relevant text. As a user selects a bubble (representing a group of semantically similar task-relevant text), the tool provides the synthesized text for that group. Figure adapted from the representation in ~\cite{Ponzanelli2017}}	
%   \label{fig:bubble-representation}	
% \end{figure}
