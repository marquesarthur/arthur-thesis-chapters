\section{Technique Deployment}
\label{cp7:deployment}

%\paragraph{\textbf{Training Data.}}

% \gcm{I think both training data and costs are
% more about how you would actually deploy
% your technique. I'm a bit confused
% on the training data whether you are suggesting
% that the technique needs to be trained on
% project-specific data?} \art{not project specific data, 
% but it need training samples as the ones in the 
% datasets that we produced}


% make it more explicit, dev ops example is a good idea

% \gcm{On Costs, I'm a bit
% confused as to whether needing a server 
% is just because of current hardware that
% might be eased in the future.} \art{for now, yes}

In Chapter~\ref{ch:identifying}, we have shown that semantic-based approaches identify task-relevant textual information across
different types of artifacts without relying on
assumptions about the nature of an artifact or its meta-data.
However, we
observe that the techniques with the best accuracy require fine-tuning 
a deep learning model
and this could be considered as an impediment to deploying such techniques. 


Fine-tuning requires training data---tasks, natural language artifacts, and text annotated as relevant---and
one potential limitation arises from how the training data might lead the model to identify 
text that is not relevant to certain types of tasks.
For example, with the adoption of Agile practices~\cite{fowler2001agile} and continuous  delivery~\cite{humble2010continuous}, 
there has been a rise in automated approaches for organizing and facilitating 
continuous delivery, which are commonly referred to as \textit{DevOps}~\cite{senapathi2018devops, leite2019ops}.
DevOps tasks and natural language artifacts documenting DevOps tools 
might significantly differ from tasks and artifacts associated with bug fixing or implementing new features
and, despite the fact that we observed that a model fine-tuned with 50 Android tasks 
was able to identify text relevant to Python tasks (Figure~\ref{fig:eval-comparison}), 
it is not clear if a model fine-tuned with certain data is able to identify text relevant to tasks 
in a different context. 
Future research should more deeply investigate the extent to which 
tasks and artifacts affect the task-relevant text identified by a model 
that requires fine-tuning.







A second limitation to deploying the techniques we explored
relates to the current cost associated with deep learning models. 
The neural embeddings and the neural networks we used
 often requires dedicated servers with significant 
memory and high-throughput computational power.
 Due to the high demand for commercial servers with such properties, 
 it might be difficult to deploy our tools for usage a large scale.
We believe that hardware cost might be eased in the future.
% \art{I simplified the cost paragraph, but it might still be better to remove it}



% The need for dedicated servers also means that our proposed techniques and our web browser plug-in cannot run offline. If we consider a standard client-server architecture, such as the one of \acs{tool}, it means that one must send data about the task and artifacts that they are working to a remote server for processing. This might not always be possible due to privacy reasons, i.e., outside the public domain, organizations would not be willing to use our tool and, depending on their size,  organizations might not have the resources needed for in-house solutions. Due to these limitations, the techniques and tools described in this dissertation 
% must be treated as proofs-of-concept and we hope that 
% these limitations might be eased in the future.

