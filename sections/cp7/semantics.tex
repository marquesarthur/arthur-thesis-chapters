

\section{Semantics in Software Development Artifacts}
\label{cp7:semantics}



In Chapter~\ref{ch:characterizing}, we used frame semantics---a general
linguistic approach---to infer the meaning of the sentences
that developers deemed relevant to complete a software task.
As one potential alternative to this approach, 
we could have considered the taxonomies available 
in other studies such as the knowledge types in API documents~\cite{Maalej2013}
or the information types in Open Source GitHub issues~\cite{Arya2019} or 
in development mailing lists~\cite{Sorbo2015}.


We decided to not use such taxonomies because the categories available in these and other studies 
are often based
on a need for access to the meaning of
sentences in the natural language text
and we were interested in assessing the
applicability of generic semantic frame
parsing for this purpose.
To the best of our knowledge, there have been only a few uses of frame
semantics in software engineering research~\cite{jha2017, kundi2017, alhoshan2019using}
and these approaches
have largely focused on text associated
with software requirements, leaving open the
question of applicability of the approach to
text in different kinds of natural language artifacts.





Motivated by our findings on the semantic analysis of the relevant text 
found in API documents, GitHub issues, and \acs{qa} pages, 
we addressed the question of whether semantic
frames can help identify the meaning of
software engineering text
in a study 
orthogonal to this dissertation~\cite{marques2021}. 
In this study, we assessed the applicability of generic semantic frame
parsing to software engineering text
aimed at supporting program
comprehension activities.
First, we assessed how the tool we used in our semantic analysis, i.e., SEMAFOR~\cite{das2014frame},
 applies to text sampled from 1,802 documents drawn from existing datasets~\cite{Arya2019, Xu2017, Maalej2013, Chaparro2017}. 
Based on the results from this analysis, 
we proposed \textit{SEFrame}, a tool that tailors 
frame parsing to natural language text in software engineering artifacts.
We assessed the correctness and robustness of \textit{SEFrame} in a second evaluation where we found that the approach was 
 correct in between 73\% and 74\% of
the cases and that it can parse text from a variety of software artifacts used to support program
comprehension. These results motivated our decision to apply \textit{SEFrame} 
in the design of the techniques we explored in Chapter~\ref{ch:identifying}.
Nonetheless,  
future research could consider more complex and modern 
frame semantic parsing tools (e.g.,~\cite{swayamdipta17, chen2021joint}) to more
accurately classify the semantic information in software engineering
text and other potential applications 
of frame semantics to software engineering problems.
 




