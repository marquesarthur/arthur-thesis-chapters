

\section{Semantics in Software Development Artifacts}
\label{cp7:semantics}


we've shown that semantics 


In Chapter~\ref{ch:characterizing}, we used frame semantics~\cite{fillmore1976frame}
to infer the meaning of the sentences
that developers deemed relevant to  a software task.
\gcm{I'm not sure why you are talking about alternatives?}
As one potential alternative to this approach, 
we could have considered 
taxonomies
available in other software engineering studies. 
For example, the knowledge types in API documents~\cite{Maalej2013}
or the information types in Open Source GitHub issues~\cite{Arya2019} or 
in development mailing lists~\cite{Sorbo2015}.



We decided to not use such taxonomies because the categories available in these and other studies 
are often based
on a need for access to the meaning of
the text
and we were interested in assessing the
applicability of a general linguistic approach for this purpose.
To the best of our knowledge, there have been only a few uses of frame
semantics in software engineering research~\cite{jha2017, kundi2017, alhoshan2019using}
and these approaches
have largely focused on text associated
with software requirements, leaving open the
question of the applicability of the approach to
text in different kinds of natural language artifacts.



We address the question of whether semantic
frames can help identify the meaning of
software engineering text
in a study 
orthogonal to this dissertation~\cite{marques2021}. 
In this study, we assessed the applicability of generic semantic frame
parsing to software engineering text
aimed at supporting program
comprehension activities.
First, we assessed how the tool we used in Chapter~\ref{ch:characterizing}
to analyze the meaning of the text considered relevant (i.e., SEMAFOR~\cite{das2014frame})
 applies to text sampled from 1,802 documents drawn from a set of datasets~\cite{Arya2019, Xu2017, Maalej2013, Chaparro2017}. 
Based on the results from this analysis, 
we proposed \textit{SEFrame}, a tool that tailors 
frame parsing to natural language text in software engineering artifacts.
We assessed the correctness and robustness of \textit{SEFrame} in a second evaluation where we found that the approach was 
 correct in between 73\% and 74\% of
the cases and that it can parse text from a variety of software artifacts used to support program
comprehension. These results motivated our decision to apply \textit{SEFrame} 
in the design of the techniques we explored in Chapter~\ref{ch:identifying}.
Nonetheless,  
future research could consider
investigating other 
frame semantic parsing tools (e.g.,~\cite{swayamdipta17, chen2021joint}) 
to classify the meaning of software engineering text.
 




