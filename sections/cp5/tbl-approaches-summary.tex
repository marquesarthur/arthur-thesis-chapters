
\begin{table}[H]
\caption{Summary of semantic-based approaches for automatically identifying task-relevant text}
\label{tbl:approaches-summary}
\centering    
\begin{scriptsize}
\begin{threeparttable}
\rowcolors{2}{}{lightgray}    
\begin{tabular}{lll}




% \multicolumn{2}{l}{\textbf{Identifier}} &  \textbf{Description} \\
\textbf{Base technique} & \textbf{Filters} &  \textbf{Description} \\

\hline



& \textit{no filter} &
\parbox[l][1.0cm][c] {7cm}{
    Uses the Skip-gram model to identify relevant sentences as the sentences most semantically similar to a task 
}
\\

\texttt{word2vec} & \textit{w/ frame-elements} &
\parbox[l][1.0cm][c] {7cm}{
    Modifies the output of the \texttt{word2vec}  according to whether sentences contain meaningful frame elements
}
\\

    & \textit{w/ frame-associations} &
\parbox[l][1.0cm][c] {7cm}{
    Modifies the output of the \texttt{word2vec}  according to whether sentences contain meaningful task-artifact frame pairs
}
\\

\hline

    & \textit{no filter} &
\parbox[l][1.0cm][c] {7cm}{
    Fine-tunes BERT to predict the sentences that are likely relevant to an input task
}
\\

\texttt{BERT} & \textit{w/ frame-elements} &
\parbox[l][1.0cm][c] {7cm}{
    Modifies the output of the \texttt{BERT}  according to whether sentences contain meaningful frame elements
}
\\

    & \textit{w/ frame-associations} &
\parbox[l][1.0cm][c] {7cm}{
    Modifies the output of the \texttt{BERT}  according to whether the sentences contain meaningful task-artifact frame pairs
}
\\


\hline


\end{tabular}
\end{threeparttable}
\end{scriptsize}
\end{table}

