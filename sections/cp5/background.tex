\clearpage

\section{Background}
\label{cp5:background}


This section describes approaches that software engineering researchers have used to extract information from natural language artifacts, scenarios where they have been applied, and potential limitations of each approach. We present approaches incrementally, from early studies to techniques that build upon the findings
and limitations outlined in these studies.



\subsubsection{Lexical Approaches}


A number of the techniques commonly employed by researchers are based on the
frequency of co-occurrence of words (or phrases) in natural language artifacts. 
An early example is Maarek and Smadja's use of lexical relations to index
software libraries~\cite{maarek1989}. 
Since this early use, software engineering
researchers have continued to leverage advances in
these approaches, such as when 
Maletic and Marcus applied \acf{LSI}~\cite{deerwester1990LSI} to help cluster software components to aid
program comprehension of a software system~\cite{Marcus2003}, or when Lin and colleagues
applied \acf{VSM}~\cite{salton1975vector}
to support tracing software requirements to source code~\cite{Lin2021}.



Researchers have also identified limitations on the applicability of lexical-based techniques~\cite{silva2019, Ye2016, Sorbo2015}. For example, Ye et al. has shown that \acs{VSM} 
fails at retrieving API references that are relevant to Stack Overflow Java questions~\cite{Ye2016} while
Di Sorbo and colleagues observed that a lexicon analysis, like \acs{LDA}~\cite{blei2003latent}, was insufficient to classify emails based on developers' intentions~\cite{Sorbo2015}, what prompted them 
to define rules to
distinguish sentences discussing feature requests, asking for an
opinion, or proposing solutions, amongst others.



\subsubsection{Semantic Approaches}



To address issues that might arise from lexicon-based approaches,
many existing studies~\cite{silva2019, Huang2018, Ye2016, huang2018automating} have used techniques able to infer the meaning, or semantics, of text. 
\red{early examples and studies}



\red{Connect early studies to approaches that have used language models in software engineering,} as  when Nguyen and colleagues
applied Word2Vec~\cite{Mikolov2013} to support the retrieval of API
examples~\cite{nguyen2017} or when Lin et al. used BERT to assist software traceability~\cite{Lin2021}. 


Since our main focus is on techniques able to infer the semantics of text, we further detail core concepts related to these tecchniques. 



\paragraph{\textbf{Language Models}}

Language models represent the semantics of words based on the context in which words appear. They allow a more ``human-like reasoning'' even when words are lexically different, as in inferring that the word \textit{king} and
\textit{queen} refer to genders of \textit{royalty}~\cite{Mikolov2013}.


To infer the meaning of words, a language model exploits Harris' distributional hypothesis~\cite{harris1954distributional}---which states that words that appear in a similar context tend to have similar meanings---and builds vector representations, namely \textit{word embeddings}, for each of the words in a text corpus.
With a significantly large text corpus, the model associates similar vector embeddings to words that are similar in meaning~\cite{Ye2016}. 





\paragraph{\textbf{Skip-gram Model}}
\label{cp5:skip-gram}

One common challenge to language models is that they need to learn word embeddings that are good at predicting the nearby words at a low computational costs, e.g., the time needed to train a model, the model size, etc.
The \textit{Skip-gram} model, proposed by Mikolov et al.~\cite{Mikolov2013}, addresses such challenges using simple yet efficient training procedures. 
That is, the model learns vector representations by \textit{(i)} looking at the $n$ words that preceded and succeed word $w_t$
as positive training examples, and by \textit{(ii)} randomly sampling words that do not appear in the same context of $w_t$ as negative training examples. 
Empirical results have shown that negative sampling allows for an accurate model able to handle noise data and that 
the vector representations provided by the model could be used to improve many natural language processing tasks~\cite{mikolov2013efficient}.


\red{references to papers in SE that have used the model to...}




\paragraph{\textbf{BERT Model}}
\label{cp5:bert}

Context in the Skip-gram model refers to the positive/negative examples used during the model's training procedures; this, however, does not allow the model to disambiguate words based on their surrounding text. In other words, a Skip-gram model will have a single vector representation for a word such as \textit{company} even when it can have different meanings, i.e., a business organization or being with someone. In contrast, 
the \acf{BERT} model, proposed by Google~\cite{Devlin2018Bert}, provides different representations for the same word based on the sentence in which a word appears.
This additional layer allows the model to perform more complex operations, such as word disambiguation \red{ref}.


BERT is initially trained 
on a massive amount of corpora. During training, a percentage of the tokens in a sentence---usually 15\%---are replaced with a special token and the model is optimized to predict these replaced tokens based on contextual information, a procedure embedded into what is called an \textit{attention function} \red{ref}. 
BERT's training procedures first create a base model using a massive text corpora and then, this base model fine-tuned to specific tasks, such as text classification, using a dataset specific to the fine-tuning steps. 
Empirical evaluation has show that BERT obtained state-of-the-art results in several natural language tasks including  question and answering and named entity recognition tasks~\cite{Devlin2018Bert}. 
%  This procedure is often referred to as \textit{transfer learning} \red{ref} where one only needs to provide a dataset for the steps associated with fine-tuning the model. 

\red{BERT usage in requirements eng}~\cite{Lin2021}
~\cite{Araujo2021}




\paragraph{\textbf{Frame Semantics}}
\label{cp5:frame-semantics}


Although word embeddings have provided significant improvements to a diverse set of NLP tasks \red{ref}, they may fail to disambiguate similar sentences with slight differences that alter their meaning. To illustrate that, we utilize an example
from Di Sorbo et al.~\cite{Sorbo2015}:


% 0.82 sim
\begin{itemize}
\item  \textit{we could use a leaky bucket algorithm to limit the band-width}; and
\item \textit{the leaky bucket algorithm fails in limiting the band-width}.
\end{itemize}

Although these messages have different meaning, \acs{VSM} or a Skip-gram model would indicate that the sentences are lexically and semantically similar, what is not sufficient to determine that the former sentence
 is a \textit{solution proposal} while the latter, a \textit{problem discovery}, as discussed by Di Sorbo et al.~\cite{Sorbo2015}. 


A potential reason that explains why these techniques fail to determine the topicality of each sentence is that they capture semantics at the word level. 
An alternative would be to consider techniques in the field of pragmatics~\cite{ariel2008pragmatics, austin1975pragmatics}, i.e., techniques able to infer a sentence's meaning.



We use \textit{frame semantics}~\cite{fillmore1976frame, Baker1998} as a proxy to a sentence's meaning.
Frame semantics allows us to identify \textit{frames},
or key events, that assist in inferring the meaning of a sentence.
Figure~\ref{fig:frame-example} presents the results of a frame
analysis for Di Sorbo's sentences.
The frames of each sentence (in grey) represent a
triggering event and the \textit{frame elements (fe)} (in red) are arguments needed
to understand the event. The enclosing square brackets
mark all lexical units, or words,
associated with either a frame or a frame element.
In the first sentence, the \textit{Using} frame
captures that an \textit{instrument}, the leaky bucket algorithm, is
manipulated to achieve a \textit{purpose}, namely to limit band-width.
In contrast, the second sentence contains a frame that represents the \textit{Success or Failure}
of achieving the \textit{goal} of limiting the band-width. 


\vspace{-0.1in}
\begin{figure}[h]
\begin{footnotesize}
\begingroup
\setlength{\tabcolsep}{10pt} % Default value: 6pt
\renewcommand{\arraystretch}{2} % Default value: 1    
\begin{tabular}{|l|}
    
    \hline

    % \circled{\scriptsize 1} 
    
    % You  need to $\big[$have$\big]$\textsubscript{\ttfamily \hl{\textbf{Being obligated}}} $\big[$implement a \\

    % BroadcastReceiver and declare it in the AndroidManifest.xml$\big]$\textsubscript{\ttfamily  \color{rufous} \textbf{fe:Duty}}  \\
    
    % \circled{\scriptsize 2} 
    
    You  $\big[$need$\big]$\textsubscript{\ttfamily \hl{\textbf{Required event}}}  $\big[$to have implement a BroadcastReceiver \\

    and declare it in the AndroidManifest.xml$\big]$\textsubscript{\ttfamily  \color{rufous} \textbf{fe:Required situation}}  \\

    \hline
\end{tabular}
\endgroup
\end{footnotesize}
\caption{Frame analysis example}
\label{fig:frame-example}
\end{figure}



Although investigating if frame semantics assists discerning the topics in Di Sorbo's sentences requires further empirical evaluation, requirements engineering researchers have used frame semantics to 
\red{Examples of frame semantics in SE}




