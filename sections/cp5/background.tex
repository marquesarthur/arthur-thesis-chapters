\clearpage

\section{Background}
\label{cp5:background}


This section describes approaches that software engineering researchers have used to extract information from natural language artifacts, scenarios where they have been applied, and potential limitations of each approach. We present approaches incrementally, from early studies to techniques that build upon the findings
and limitations outlined in these studies.



\subsubsection{Lexical Approaches}


A number of the techniques commonly employed by researchers are based on the
frequency of co-occurrence of words (or phrases) in natural language artifacts. 
An early example is Maarek and Smadja's use of lexical relations to index
software libraries~\cite{maarek1989}. 
Since this early use, software engineering
researchers have continued to leverage advances in
these approaches, such as when De Lucia et al. applied \acf{VSM}~\cite{DeLucia2012}
to support identification of human labelled source code components~\cite{Lin2021}.



As detailed in the motivation of this chapter, 
researchers have also identified limitations on the applicability of lexical-based techniques~\cite{silva2019, Ye2016, Sorbo2015}. For example, Ye et al. has shown that \acs{VSM} 
fails at retrieving API references that are relevant to Stack Overflow Java questions~\cite{Ye2016} while
Di Sorbo and colleagues observed that a lexicon analysis, like \acs{LDA}~\cite{blei2003latent}, was insufficient to classify emails based on developers' intentions~\cite{Sorbo2015}, what prompted them 
to define rules to
distinguish sentences discussing feature requests, asking for an
opinion, or proposing solutions, amongst others.



\subsubsection{Semantic Approaches}



To address issues that might arise from lexicon-based approaches,
many existing studies~\cite{silva2019, Huang2018, Ye2016, huang2018automating} have used techniques able to infer the meaning, or semantics, of text. 
In such context, semantics is often built by identifying co-occurrence of terms such that one can infer that certain terms are related. For example 
the terms \textit{car} and \textit{automobile} are likely to co-occur in different phrases with terms like \textit{motor} and \textit{wheel}~\cite{Bavota2016}
what can be used to establish relationships or associations between these words.
This concept is used in techniques such as \acf{LSI}~\cite{deerwester1990LSI} or language models~\cite{Mikolov2013} to infer the meaning of the text,
where Maletic and Marcus application of \acs{LSI} to help cluster software components to aid
program comprehension of a software system
is an early example on the usage of semantics by software engineering researchers~\cite{Marcus2003}.


Ever since these early examples, advancements in the \acf{NLP} field have lead software engineering researchers 
to investigate and apply a set of \acs{NLP} techniques to assist software engineering tasks. For instance,
Nguyen and colleagues
applied Word2Vec~\cite{Mikolov2013} to support the retrieval of API
examples~\cite{nguyen2017} while Lin et al. used BERT to assist software traceability~\cite{Lin2021}. 
Since our focus is on techniques able to infer the semantics of text, we further detail core concepts related to these techniques. 



\paragraph{\textbf{Language Models}}

Language models represent the semantics of words based on the context in which words appear. They allow a more ``human-like reasoning'' even when words are lexically different, as in inferring that the word \textit{king} and
\textit{queen} refer to \textit{royalty}~\cite{Mikolov2013}.


To infer the meaning of words, a language model exploits Harris' distributional hypothesis~\cite{harris1954distributional}---which states that words that appear in a similar context tend to have similar meanings---and builds vector representations, namely \textit{word embeddings}, for each of the words in a text corpus.
With a significantly large text corpus, the model associates similar vector embeddings to words that are similar in meaning~\cite{Ye2016}. 





\paragraph{\textbf{Skip-gram Model}}
\label{cp5:skip-gram}

One common challenge to language models is that they need to learn word embeddings that are good at predicting the nearby words at a low computational costs, e.g., the time needed to train a model, the model size, etc.
The \textit{Skip-gram} model, proposed by Mikolov et al.~\cite{Mikolov2013}, addresses such challenges using simple yet efficient training procedures. 
The model learns vector representations by \textit{(i)} looking at the $n$ words that preceed and succeed word $w_t$
as positive training examples, and by \textit{(ii)} randomly sampling words that do not appear in the same context of $w_t$ as negative training examples. 
Empirical results have shown that such procedures allow for an accurate model able to handle noise data and that 
the vector representations provided by the model could be used to improve many natural language processing tasks~\cite{mikolov2013efficient}.


\red{references to papers in SE that have used the model to...}




\paragraph{\textbf{BERT Model}}
\label{cp5:bert}

Context in the Skip-gram model refers to the positive/negative examples used during the model's training procedures; this, however, does not allow the model to disambiguate words based on their surrounding text. In other words, a Skip-gram model will have a single vector representation for a word such as \textit{company} even when it can have different meanings, i.e., a business organization or being with someone. In contrast, 
the \acf{BERT}~\cite{Devlin2018Bert} model provides different representations for the same word based on the sentence in which a word appears.
This additional layer allows the model to perform more complex operations, such as word disambiguation \red{ref}.


BERT is initially trained 
on a massive amount of corpora. During training, a percentage of the tokens in a sentence---usually 15\%---are replaced with a special token and the model is optimized to predict these replaced, or \textit{masked}, tokens. 
The model relies on a mechanism, called \textit{attention} \red{ref}, that correlates and weights all non-replaced tokens in the input so that the model maximizes predictions.
A fully trained BERT model also uses transfer learning  \red{ref}, i.e., 
first one creates a base model using massive text corpora and the general token prediction task. 
Then, this base model is fine-tuned to specific tasks, such as text classification, using a dataset specific to the fine-tuning steps. 
Empirical evaluation has shown that BERT obtained state-of-the-art results in several natural language tasks including question and answering and named entity recognition tasks~\cite{Devlin2018Bert}. 
%  This procedure is often referred to as \textit{transfer learning} \red{ref} where one only needs to provide a dataset for the steps associated with fine-tuning the model. 

Within software engineering, BERT has been applied to requirements traceability, where
Lin et al. has shown that a BERT model can generate trace links between natural language artifacts and programming language artifacts~\cite{Lin2021} when applied to the CodeSearchNet dataset. In turn, Araujo used the model to identify software requirements from app reviews,
showing that BERT achieved state-of-the-art results for this traceability problem~\cite{Araujo2021}.




\paragraph{\textbf{Frame Semantics}}
\label{cp5:frame-semantics}


Although word embeddings have provided significant improvements to a diverse set of NLP tasks \red{ref}, they may fail to disambiguate similar sentences with slight differences that alter their meaning. To illustrate that, we utilize an example
from Di Sorbo et al.~\cite{Sorbo2015}:


% 0.82 sim
\begin{itemize}
\item  \textit{we could use a leaky bucket algorithm to limit the band-width}; and
\item \textit{the leaky bucket algorithm fails in limiting the band-width}.
\end{itemize}

Although these messages have different meanings, \acs{VSM} or a Skip-gram model would indicate that the sentences are lexically and semantically similar, what is not sufficient to determine that the former sentence
 is a \textit{solution proposal} while the latter, a \textit{problem discovery}, as discussed by Di Sorbo et al.~\cite{Sorbo2015}. 
A potential reason that explains why these techniques fail to determine the topicality of each sentence is that they capture semantics at the word level. 
An alternative would be to consider linguistic approaches~\cite{ariel2008pragmatics, austin1975pragmatics} able to infer a sentence's meaning.



In this work, we use \textit{frame semantics}~\cite{fillmore1976frame, Baker1998} as a proxy to a sentence's meaning.
Frame semantics allows us to identify \textit{frames},
or key events, that assist in inferring the meaning of a sentence.
Figure~\ref{fig:frame-example} presents the results of a frame
analysis for Di Sorbo's sentences.
The frames of each sentence (in grey) represent a
triggering event and the \textit{frame elements (fe)} (in red) are arguments needed
to understand the event. The enclosing square brackets
mark all lexical units, or words,
associated with either a frame or a frame element.
In the first sentence, the \textit{Using} frame
captures that an \textit{instrument}, the leaky bucket algorithm, is
manipulated to achieve a \textit{purpose}, namely to limit bandwidth.
In contrast, the second sentence contains a frame that represents the \textit{Success or Failure}
of achieving the \textit{goal} of limiting the bandwidth. 
This example shows that the frames identified in Figure~\ref{fig:frame-example} might assist discerning the topics in Di Sorbo's sentences\footnote{We leave empirical evaluation of this hypothesis as future work.}.


\vspace{-0.1in}
\begin{figure}[h]
\begin{footnotesize}
\begingroup
\setlength{\tabcolsep}{10pt} % Default value: 6pt
\renewcommand{\arraystretch}{2} % Default value: 1    
\begin{tabular}{|l|}
    
    \hline

    % \circled{\scriptsize 1} 
    
    % You  need to $\big[$have$\big]$\textsubscript{\ttfamily \hl{\textbf{Being obligated}}} $\big[$implement a \\

    % BroadcastReceiver and declare it in the AndroidManifest.xml$\big]$\textsubscript{\ttfamily  \color{rufous} \textbf{fe:Duty}}  \\
    
    % \circled{\scriptsize 2} 
    
    You  $\big[$need$\big]$\textsubscript{\ttfamily \hl{\textbf{Required event}}}  $\big[$to have implement a BroadcastReceiver \\

    and declare it in the AndroidManifest.xml$\big]$\textsubscript{\ttfamily  \color{rufous} \textbf{fe:Required situation}}  \\

    \hline
\end{tabular}
\endgroup
\end{footnotesize}
\caption{Frame analysis example}
\label{fig:frame-example}
\end{figure}



Within software engineering, frame semantics have largely been applied to problems related
to requirements engineering.
Among other work, we cite 
Jah and Mahmoud use of frame semantics
to determine which app user reviews are feature requests, which are problem reports and which
are something else~\cite{jha2017},
Kundi and Chitchyan use of frame semantics to assist with requirements
elicitation~\cite{kundi2017}, and Alhoshan et al. technique to 
to identify textual related requirements, which they use to establish traceability links 
over multiple requirement documents~\cite{alhoshan2019using}.


