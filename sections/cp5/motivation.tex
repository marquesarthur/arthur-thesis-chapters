
\section{Problem Statement}
\label{cp5:motivation}





Researchers have long recognized the value of natural language
text, utilizing various approaches to extract
information from this text that can be embedded in
tools for software developers \red{ref}.
These approaches exploit lexical and semantic properties available in the text to determine 
whether it contains information of interest to a developer \red{ref}. 
In this work, information of interest represents text that a developer would consider as relevant to a specific software task. Ideally, we could establish that text in a natural language artifact that matches  
words (or terms) also present in a task description is likely relevant to the developer's task. However, direct matches are often scarce and researchers have observed that:



\medskip
\begin{bluequote}
    \textit{The text  that contains information associated with the solution for a task in a natural language artifact often differs from the text in the software task itself.}
\end{bluequote}
% \smallskip



Due to these differences, automatically determining the text in a natural language artifact that is of interest to a software task is not trivial. 
It requires techniques able to shorten the gap between the two sources, identifying or establishing relationships that would indicate the relevance of the text to the task-at-hand.


As a first step towards addressing this problem, we consider a set of techniques
 building on approaches to interpret the meaning, or semantics, of text.
 For that, we first provide theoretical background for the approaches that we explore. Then, we detail how we use these approaches in the design space of a technique that automatically identifies text relevant to a software task.






