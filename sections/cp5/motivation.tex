\section{Motivation}
\label{cp5:motivation}

\gm{I think you may wish to redirect this section towards
being about the design space, drawing on the related work
chapter. I think leave for now and revist later.}


More than often, the information associated 
with the solution for a software task in a natural language artifact differs from the information provided in the task itself~\cite{silva2019, Ye2016}.
These differences might arise both from:

\begin{itemize}
    \item \textit{lexical gaps}, or the fact that the same semantic meaning can be expressed by different words~\cite{Huang2018}, and from;
    \item \textit{knowledge gaps},
    or the fact that a developer's background may 
    influence how they describe a software task and, consequently,
    the information that developers performing that task search for~\cite{Kevic2014}.
\end{itemize}











To address issues that might arise from such \acs{IR} or lexicon-based approaches, many existing studies~\cite{silva2019, Huang2018, Ye2016} have used 
techniques able to infer the semantics of the text to aid activities such as bug localization or API recommendation. As an example, Nguyen and colleagues applied Word2Vec~\cite{Mikolov2013space} to support the retrieval of API examples~\cite{nguyen2017}.



Following the findings from these studies, we hypothesize that text semantics plays a role in identifying information relevant to a software task. 
For example, in a task that requests updating an Android API component\footnote{\url{https://github.com/robolectric/robolectric/issues/1425}}, a technique
able to access the meaning of the text could indicate that sentences whose meaning reflect updating or usage instructions
are likely relevant to that task.
To investigate this hypothesis, we consider 
word semantics and sentence semantics techniques and we explore: 



\begin{enumerate}[label=\textit{RQ\arabic*}]

    \item \textit{How accurately can word semantics identify text deemed relevant to a task?}
    With this question, we assess the applicability of models that provide semantic representations of words~\cite{Mikolov2013, Devlin2018Bert}  to the identification of  task-relevant text.
    
    
    \item \textit{Do sentence semantics help to identify text relevant to a software task?}
    With this question, we assess if the meaning of a sentence---captured through frame semantics~\cite{fillmore1976frame}---help to locate task-relevant text.
\end{enumerate}


Answering this two questions will allow us to explore the \textit{design space} of an approach that uses word semantics and/or sentence semantics  to automatically identify text in natural language software artifacts that is relevant to a particular task.