
\section{Motivation}
\label{cp5:motivation}



% Performing a task on a software system, like fixing a bug
% or adding a feature, typically requires a developer to consult
% a number of different kinds of artifacts, such
% as API documentation, bug reports, community forums
% and web tutorials.  When consulting these artifacts,
% a developer must identify, from the large amount of text
% in these documents, just the fraction of text relevant
% to the task-at-hand.



% Unfortunately, developers tend to struggle with identifying
% the relevant text and when they are unable to locate all, or most, of it,
% they may produce incomplete or incorrect solutions~\cite{Murphy2005}.
% To aid developers in this situation,
% researchers have proposed various 
% techniques and tools to
% identify likely relevant text.
% For example, 
% \textit{Krec}~\cite{Robillard2015}
% identifies text that a developer cannot afford to ignore when reading an API document. 
% As another example,
% \textit{AnswerBot}~\cite{Xu2017} generates answers for a developer's
% task by identifying text pertinent to that particular task on Stack Overflow.
% Although effective, these techniques target specific
% types of artifacts, limiting their use across the 
% many different kinds of artifacts developers encounter
% daily in their work.


% In Chapter~\ref{ch:introduction}, we introduced 
% the need for more generalizable techniques able to
% identify text relevant to an input task across the different 
% types of artifacts. 

In Chapter~\ref{ch:characterizing},
we observed consistency in the meaning of text of 
different types of artifacts that  
participants deemed relevant to six software tasks.
Guided by these findings and aslo by recent success 
in studies that use techniques that interpret the meaning, or \textit{semantics}, of text
for a variety of development activities, such as
for finding who should fix a bug~\cite{yang2016}, searching for comprehensive code examples~\cite{silva2019}, 
or assessing the quality of information available in bug reports~\cite{chaparro2019}, we ask:



\medskip
\begin{bluequote}
    \textit{to which extent can semantic-based techniques identify task-relevant text across different kinds of software artifacts?}
\end{bluequote}




To investigate this question, we introduce six possible techniques that incorporate the  
semantics of words and sentences to identify textual information likely relevant to a developer's task.






% Researchers have long recognized the value of natural language
% text, utilizing various approaches to extract
% information from this text that can be embedded in
% tools for software developers \red{ref}.
% \gm{Not quite clear if previous sentence is about SE researchers or researchers and
% text in general.}
% These approaches exploit lexical and semantic properties available in the text to determine 
% whether it contains information of interest to a developer \red{ref}. 


% In this work, information of interest represents text that a developer would consider as relevant to a software task assigned to them. Ideally, we would establish that text in natural language artifacts that matches  
% words (or terms) also present in a task description is likely relevant to the developer's task.
% \gm{Why would this method be ideal? Focus on the problem and finding exactly the
% right text rather than how.} However, direct matches are often scarce \red{ref} and several researchers have observed that:




% % \smallskip