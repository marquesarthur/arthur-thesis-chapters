\section{Motivation}
\label{cp5:motivation}


\gm{I think you may wish to redirect this section towards
being about the design space, drawing on the related work
chapter. I think leave for now and revist later.}


More than often, the information associated 
with the solution for a software task in a natural language artifact differs from the information provided in the task itself~\cite{silva2019, Ye2016}.
These differences might arise both from:

\begin{itemize}
    \item \textit{lexical gaps}, or the fact that the same semantic meaning can be expressed by different words~\cite{Huang2018}, and from;
    \item \textit{knowledge gaps},
    or the fact that a developer's background may 
    influence how they describe a software task and, consequently,
    the information that developers performing that task search for~\cite{Kevic2014}.
\end{itemize}





Because of both lexical and knowledge gaps, traditional \acf{IR}~\cite{Manning2009IR} or lexicon-based approaches (e.g.,~\cite{Ponzanelli2015} or ~\cite{Xu2017}) might fail at automatically identifying text that could assist a developer in completing their software task. 
For example, Ye et al. has shown that \acf{VSM}~\cite{salton1975vector} 
 fails at retrieving API references that are relevant to Java questions
posted on Stack Overflow~\cite{Ye2016}.
Similarly,
Di Sorbo and colleagues observed that lexicon analysis, like \acs{LDA}~\cite{blei2003latent}, was insufficient to classify 
emails based on developers' intentions,
what prompted them to define linguistic patterns to classify a sentence's intention~\cite{Sorbo2015}.



To address issues that might arise from such \acs{IR} or lexicon-based approaches, many existing studies~\cite{silva2019, Huang2018, Ye2016, huang2018automating} have used 
techniques able to infer the
meaning, or semantics, of text to aid activities such as bug localization or API recommendation. As an example, Nguyen and colleagues applied word semantic representations~\cite{Mikolov2013space} to improve  retrieval of API examples~\cite{nguyen2017}.
As another example, Huang et al. have also used semantics 
to more accurately classify the intentions of text in development emails~\cite{huang2018automating}.



Guided by these studies, we first consider if a general lexicon-based approach suffices to identify text that is relevant to a software task in a natural language artifact pertinent to that task. If we can show that a general approach, such as \acs{VSM}, is able to accurately identify relevant text across a wide variety of software tasks and artifacts, 
it can be embedded into a tool that assists developers in locating task-relevant information at a low cost~\cite{Rastkar2013}.


However, similar to related work, we may also observe that lexical approaches are not sufficient to address our domain problem. In such case, approaches that infer the meaning of the text can be also suitable.
For example, in a task that requests updating an Android API component\footnote{\url{https://github.com/robolectric/robolectric/issues/1425}}, a technique
able to access the meaning of the text could indicate that sentences whose meaning reflect updating or usage instructions
are likely relevant to that task. Therefore, we consider 
word semantics and sentence semantics techniques and we explore: 



\begin{enumerate}[label=\textit{RQ\arabic*}]

    \item \textit{How accurately can word semantics identify text deemed relevant to a task?}
    With this question, we assess the applicability of models that provide semantic representations of words~\cite{Mikolov2013, Devlin2018Bert}  to the identification of  task-relevant text.
    
    
    \item \textit{Do sentence semantics help to identify text relevant to a software task?}
    With this question, we assess if the meaning of a sentence---captured through frame semantics~\cite{fillmore1976frame}---help to locate task-relevant text.
\end{enumerate}


Answering these questions will allow us to explore a \textit{design space} for building a technique that uses semantic approaches to automatically identify text in natural language software artifacts that is relevant to a particular task.