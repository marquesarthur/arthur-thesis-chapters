\clearpage

\section{Problem Statement}
\label{cp5:motivation}





Researchers have long recognized the value of natural language
text, utilizing various approaches to extract
information from this text that can be embedded in
tools for software developers.
These approaches exploit lexical and semantic properties available in the text to determine 
whether it contains information of interest to a developer \red{ref}. 
In this work, information of interest represents text that a developer would consider as relevant to a specific software task. Ideally, this text could be identified if it shared  
words (or terms) also present in a task description. However, differences between the two sources (i.e., software task and natural language artifacts) pose challenges to establishing relationships between the text within them. Most notably, we observe that:



\medskip
\begin{bluequote}
    \textit{The text  that contains information associated with the solution for a task in a natural language artifact often differs from the text in the software task itself.}
\end{bluequote}
% \smallskip





As a first step towards addressing this problem, we consider a set of techniques
 building on approaches to interpret the meaning, or semantics, of text.
 For that, we first provide theoretical background for the approaches that we explore. Then, we detail how we use these approaches in the design space of a technique that automatically identifies text relevant to a software task.








% For example, researchers have observed \textit{lexical} gaps, i.e., the fact that the same meaning can be expressed by different words; and \textit{knowledge} gaps, or the fact that a developer's background influences how they describe a task and, consequently, what information developers performing that task might seek.




% Hence, identifying the text that is relevant to a software task is not trivial \red{ref}, and we must address the fact that: