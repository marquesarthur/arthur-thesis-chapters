\section{Motivation}
\label{cp5:motivation}


When provided with a set of artifacts that potentially contain information relevant to their task,
software developers must inspect the artifact's content and locate the portion of the text that might assist them in completing their task. 
Some of the artifacts inspected are short enough that a developer can decide if they contain any helpful information at a quick glance.
Some others are lengthy~\cite{Rastkar2013t} and factors such as high time pressure or the need to meet deadlines~\cite{meyer2019}
may lead a developer to quickly skim the document
in order to decide if it is worthy of careful inspection~\cite{Starke2009}.



A developer that 
reads an artifact in this ``hurried'' manner
often jumps through the sections and the sentences in the text~\cite{Brandt2009a, Starke2009}.
When doing so, a series of intrinsic and extrinsic factors~\cite{Freund2015} 
serve as indicators of the relevance of the text to the developer's task
lead the developer to stop at certain portions for scrutiny.
For example, a code element that is mentioned in a task description 
and that also appear in a sentence may lead 
her to inspect the sentence more carefully and is an example of an extrinsic factor (i.e., shared keywords).
The developer's familiarity with the document's structure
may guide her to skip certain parts of an artifact while focusing her attention to others and is 
an example of an intrinsic factor (i.e., the developer's past experience reading similar artifacts).





As Table~\ref{tbl:text-properties-references} shows, many studies have 
provided possible factors that guide the automatic identification of text relevant to a particular task.
For instance, Krec uses the presence of code elements in a sentence to identify 
text that programmers cannot afford to ignore when using an API~\cite{Robillard2015}, 
e.g., ``\textit{Use \hl{getName} to get the logical name of the font}''.
As another example, Sara and Treude argue that relevant text is often found in conditional sentences~\cite{nadi2020}, 
e.g., ``\textit{\hl{If} the permission you need to add isn't listed under the normal permissions, you'll need to deal with Runtime Permissions}''.


\vspace{4mm}        
\begin{table}[H]
\begin{small}
\begin{tabular}{lr}

\hline

\parbox[l][0.5cm][c]{12.5cm}{ \cellcolor{lightgray}
    \textbf{Features}
} \\    

% -----------------------------------------------------------------------------------
\hline




\parbox[l][1.2cm][c]{12.5cm}{ 
\textbf{API word pattern:} relevant text is captured through a set of pattern extracted from annotated data on indispensable knowledge required to use API elements. 
 \cite{nadi2020, Robillard2015} } \\  
    
\parbox[l][1.2cm][c]{12.5cm}{ 
    \textbf{Code element:} relevant text often mentions code elements such as class name or method calls    
 \cite{Robillard2015} } \\ 

\parbox[l][0.6cm][c]{12.5cm}{ 
    \textbf{Conditional sentence:} relevant text is often expressed as a condition in a sentence
 \cite{nadi2020} } \\  

\parbox[l][1.2cm][c]{12.5cm}{ 
    \textbf{Entity overlap:} sentences that have software terms that also appear in a task description are likely relevant 
 \cite{Xu2017, Ponzanelli2015} } \\  

\parbox[l][1.2cm][c]{12.5cm}{ 
    \textbf{Information entropy:} high-entropy sentences often represent unique and useful information.
 \cite{Xu2017, Rastkar2010} } \\  

\parbox[l][1.7cm][c]{12.5cm}{ 
    \textbf{Lexical similarity:} text relevant to a task often contains several of the lexical terms present in the task 
    description, which is identifiable through textual similarity metrics such as 
    term-frequency inverse-document-frequency based cosine similarity.    
 \cite{Lotufo2012, Ponzanelli2015} } \\  

 \parbox[l][0.5cm][c]{12.5cm}{ \red{other references \dots} } \\  


% -----------------------------------------------------------------------------------
\hline

\end{tabular}
\end{small}
\caption{Textual properties used to automatically identify relevant text in related work}
\label{tbl:text-properties-references}
\end{table}





While the properties outlined in Table~\ref{tbl:text-properties-references} rely on 
syntactic properties, we hypothesize that one can determine the relevancy of a sentence based on its semantics. 
More specifically, our study characterizing task-relevant text in the
 \acs{DS-synthetic} corpus has shown that certain frames---identifiable through frame semantics~\cite{fillmore1976frame}---are prominent 
in sentences deemed relevant, such as sentences describing required events, obligations, actions, or how to use API elements to achieve some goal, as observed by the frames identified 
in the sentence about the \texttt{getName} method:

\begin{figure}[H]
\centering
% \smallskip
\begin{footnotesize}
\begingroup
\setlength{\tabcolsep}{12pt} % Default value: 6pt
% \renewcommand{\arraystretch}{2} % Default value: 1    
\begin{tabular}{l}
    
    % \circled{\scriptsize 1}

    $\big[$Use$\big]$\textsubscript{\ttfamily \hl{\textbf{Using}}} getName $\big[$to get the logical name of the font$\big]$\textsubscript{\ttfamily  \color{rufous} \textbf{fe:Goal}}  \\

    % \circled{\scriptsize 2}

    % ... you'll $\big[$need$\big]$\textsubscript{\ttfamily \hl{\textbf{Required event}}} $\big[$ to deal with Runtime Permissions $\big]$\textsubscript{\ttfamily  \color{rufous} \textbf{fe:Required situation}}  \\

\end{tabular}
\endgroup
\end{footnotesize}
% \caption{Example of frames and frame elements}
% \label{fig:frame-example}
\end{figure}
    




Following findings from studies on text comprehension~\cite{Kintsch1978a, Bystrom1995} and how developers forage information in software
artifacts~\cite{Fritz2014, Brandt2009a, DeGraaf2014},
we consider how jumping through sentences in an artifact 
resembles how the PageRank algorithm 
estimates the relevancy of nodes in a graph~\cite{Page1999, Lotufo2012}.
We use this insight to investigate using the algorithm in our domain problem and what syntactic and semantic properties assist in identifying the text relevant to a task in a pertinent natural language software artifact.


