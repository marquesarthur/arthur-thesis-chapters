
\section{Problem Statement}
\label{cp5:motivation}





Researchers have long recognized the value of natural language
text, utilizing various approaches to extract
information from this text that can be embedded in
tools for software developers \red{ref}.
\gm{Not quite clear if previous sentence is about SE researchers or researchers and
text in general.}
These approaches exploit lexical and semantic properties available in the text to determine 
whether it contains information of interest to a developer \red{ref}. 


In this work, information of interest represents text that a developer would consider as relevant to a software task assigned to them. Ideally, we would establish that text in natural language artifacts that matches  
words (or terms) also present in a task description is likely relevant to the developer's task.
\gm{Why would this method be ideal? Focus on the problem and finding exactly the
right text rather than how.} However, direct matches are often scarce \red{ref} and several researchers have observed that:



\medskip
\begin{bluequote}
    \textit{The text in a natural language artifact that contains information associated with the solution for a task often differs from the text in the software task itself.}
\end{bluequote}
% \smallskip



Due to these differences, automatically determining the text in a natural language artifact that is of interest to a software task is not trivial. 
It requires techniques able to shorten the gap between the two sources, identifying or establishing relationships that would indicate the relevance of the text to the task-at-hand.


As a first step towards addressing this problem, we consider a set of techniques
 that builds on approaches to interpret the meaning, or semantics, of text.
We first provide theoretical background for the approaches that we explore and then we detail how we use these approaches in the design space of a technique that automatically identifies text relevant to a software task.






