\section{Motivation}
\label{cp5:motivation}


% obtained via a search~\cite{Li2013} or a recommendation engine~\cite{Cubranic2005, Ponzanelli2017},
When provided with a set of artifacts that potentially contain information relevant to their task,
software developers must inspect the artifact's content and locate the portion of the text that might assist them completing their task. 
Some of the artifacts inspected are short enough that a developer can decide if they contain any helpful information at a quick glance.
Some others, are lengthy~\cite{Rastkar2013t} and, due to factors such as high time pressure or the need to meet deadlines~\cite{meyer2019}, a developer may quickly skim the document
to decide if it is worthy of careful inspection~\cite{Starke2009}.

% instead of carefully inspecting them.


When reading an artifact in this ``hurried'' manner, a developer often jumps through the sections and the sentences in the text~\cite{Brandt2009a, Starke2009},
deciding to stop at certain portions based on a series of intrinsic and extrinsic factors 
that might indicate the relevance of the text to the developer's task~\cite{Freund2015}.
An example of a extrinsic factor is when a sentence mention a code element that also appears in a task description while 
an intrinsic factor is how familiar a developer is with the document's structure, e.g., she may ignore all the content of a GitHub issue except 
the last few comments because the developer implicitly expects to find a comment discussing the adopted solution near the end of the document.



%%% 

\red{Related work}

Many studies have modeled such factors that guide identifying relevant text.

lore ipsum

%%% 


\red{Semantic frames}


lore ipsum






%%%%% ----------------------------- jump to approach ???




% guage artifacts identified as potentially useful for a task, 



% they followed its incoming and outgoing dependencies
% Among the several models (e.g.,~\cite{Kintsch1978a} or~\cite{Fritz2014}) that might explain such reading behaviour,
% Lotufo et al.'s 
% ~\cite{Lotufo2012}
% meet deadlines?



% might not be considered an efficient use of a developer's time.




% and would require .






% developers perceive such attentive reading as 
% efficient use of their time 






% As part of a software task, software developers often seek information on many natural language artifacts, including  
% technical blogs, forum discussions, Q\&A posts, and many others~\cite{Li2013, umarji2008archetypal}.
% These artifacts are found using search engines~\cite{Li2013} or recommendation systems~\cite{Li2013,Ponzanelli2017},









% To help software developers find text within natural lan-
% guage artifacts identified as potentially useful for a task, many studies have provided possible properties that guide identifying relevant text [8].






% Software devs are knowledge workers. They interleave Web foraging, learning, and writing ~\cite{Brandt2009a}




% Li has shown that 
% Our task video analysis reveals that developers heavily
% rely on search engines (such as Google, Baidu, and Bing) to find useful technical blogs, forum discussions, and Q&A posts. 







% In many cases our participants decided very quickly that
% nothing was relevant and tried a different search.~\cite{Starke2009}






% other times a developer must read the report, which
% can be lengthy, involving discussions amongst multiple team members and
% other stakeholders.
% ~\cite{Rastkar2013t}





% Yet it has been often accepted that information needs and information seeking processes depend on worker's tasks~\cite{Bystrom1995}




% the worker performing the task experi-ences gaps in her knowledge and thus information needs which reflect her interpretation of in-formation requirements, her prior experience and knowledge, and her ability to memorize it.



% answers were complete and correct. We found that the search behaviour of software profession-We found that the search behaviour of software profession-
% als is heavily influenced by their prior knowledge about the
% als is heavily influenced by their prior knowledge about the documentation and the software specified in this documen-documentation and the software specified in this documen- tation.
% ~\cite{DeGraaf2014}






% % important for page rank
% When developers found relevant code, they followed its incoming and outgoing dependencies,~\cite{Ko2006}





% Our model describes program understanding as a
% process of searching, relating, and collecting relevant informa-tion, all by forming perceptions of relevance from cues in
% the programming environment.~\cite{Ko2006}








% Help seeking has been considered as a metacognitive skill
% [30][27] for knowing when and how to use particular
% strategies for learning or for problem solving.~\cite{Li2013}




% many studies have provided possible properties that guide identifying relevant text [8].