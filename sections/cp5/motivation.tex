\section{Motivation}
\label{cp5:motivation}


\gm{I think you may wish to redirect this section towards
being about the design space, drawing on the related work
chapter. I think leave for now and revist later.}


More than often, the information associated 
with the solution for a software task in a natural language artifact differs from the information provided in the task itself~\cite{silva2019, Ye2016}.
These differences might arise both from:

\begin{itemize}
    \item \textit{lexical gaps}, or the fact that the same semantic meaning can be expressed by different words~\cite{Huang2018}, and from;
    \item \textit{knowledge gaps},
    or the fact that a developer's background may 
    influence how they describe a software task and, consequently,
    the information that developers performing that task will search for~\cite{Kevic2014}.
\end{itemize}





Due to these differences, traditional \acf{IR}~\cite{Manning2009IR} or lexicon-based approaches (e.g.,~\cite{Ponzanelli2015} or ~\cite{Xu2017}) might fail at automatically identifying information that is relevant to a developer's task. 
For example, Ye et al. has shown that \acf{VSM}~\cite{salton1975vector} 
 fails at retrieving API references that are relevant to Stack Overflow Java questions~\cite{Ye2016}.
Similarly,
Di Sorbo and colleagues observed that lexicon analysis, like \acs{LDA}~\cite{blei2003latent}, was insufficient to classify 
emails based on developers' intentions,
what prompted them to define linguistic patterns to classify a sentence's intention~\cite{Sorbo2015}.



Many existing studies~\cite{silva2019, Huang2018, Ye2016, huang2018automating} have used 
techniques able to infer the
meaning, or semantics, of text to address issues that might arise from such \acs{IR} or lexicon-based approaches.
As an example, Nguyen and colleagues applied word semantic representations~\cite{Mikolov2013space} to improve  retrieval of API examples~\cite{nguyen2017} while Huang et al. have used word semantics 
to more accurately classify the intentions of text in development emails~\cite{huang2018automating}.



Guided by these studies, we first consider if a general lexicon-based approach suffices to identify text that is relevant to a software task in a natural language artifact pertinent to that task. If we can show that a general approach, such as \acs{VSM}, is able to accurately identify relevant text across a wide variety of software tasks and artifacts, 
the approach can be embedded into a tool that assists developers in locating task-relevant information
in a cost-effective way~\cite{Rastkar2013}.


However, we may also observe that lexical approaches are not sufficient, what raises the question of whether approaches that infer the meaning of the text apply to our domain problem.
Thus, we also explore approaches able to infer the semantic meaning of text at the word and sentence level 
and we evaluate how accurately can these approaches identify text deemed relevant to a software task.


\art{If motivation describes a design space, I think that research questions may fit better under evaluation?}


% Exploring lexical and semantic-based approaches will allow us to explore a \textit{design space} for building a technique  to automatically identify text that is relevant to a particular task in natural language software artifacts
% pertinent to that task.


