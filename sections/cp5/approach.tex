\section{Approach}
\label{cp5:approaches}


Our goal is to design an \textit{artifact-agnostic} approach
to identify sentences relevant to a task in an artifact pertinent to that task.
We formulate this goal as an extractive query-based summarization problem~\cite{Goldsteinet1999}.
That is, given a query (i.e., a software task),
and a natural language software artifact as inputs,
identify a number of sentences from the input artifact that
most likely contain information relevant to that task. 
Thus, the output of our approach is a set of sentences,
which can  be also seen as extractive summary.



We provide details about steps towards this goal
in a top-down manner, i.e., we first introduce general concepts
used to produce a summary and then, we describe
the \textit{textual features} from related work and of our own used to identify task-relevant sentences.




\subsection{Summary Generation}


LexRank~\cite{Erkan2004} is the backbone of our approach. 
It is based on the PageRank~\cite{Page1999} algorithm
and, on its most simplistic form, LexRank threats each sentence in an artifact as a node.
Then, the algorithm
establishes edges between all the node pairs using a similarity function, ultimately producing a graph. 
The more edges a node has, more central the node is to the graph and the more likely it is that the sentence 
which that node represents contains key information. 
Thus, we can obtain a summary of the content of a pertinent artifact selecting the $n$ most central nodes of the graph. 


To include information about a task, one can build a bipartite graph.
A first set of nodes represent sentences originating from a task while a second set, sentences in a pertinent artifact.
This time,  a similarity function 
establishes edges only  between the nodes of each one of the sets, i.e., edges from nodes in the artifact to nodes in the task.
By focusing our attention to the nodes (or sentences) with the most edges belonging to the pertinent artifact set, we can produce a summary with information relevant to that task.




At this point, it is noticeable that the similarity function used by the algorithm 
is the stepping stone for determining the summary's output.  
To devise a similarity function that accounts for the richness of data available in the 
text, we base our approach on Ponzanelli el al.'s meta-information model~\cite{Ponzanelli2015}.
This model defines a similarity function that uses a set of properties or features to compute similarity and thus, establish edges in the graph.
The overall similarity value of the model is the normalized sum of individual values obtained for each feature or property.


\art{Provide formulas/formal definitions once we've agreed upon approach/chapter outline}

\subsection{Textual Features}



To devise an artifact-agnostic similarity function, we need a set of features that apply to the different artifact types a developer might seek information as part of their task.
Due to this need, we identify features that do not rely on artifact-specific data in existing related work. 
We also define new features based 
on heuristics derived from annotations from 20 participants of text deemed relevant
for the six tasks of the \acs{DS-synthetic} corpus.
Table~\ref{tbl:approach-textual-features} groups features based on the textual properties that they rely on. Details for each feature are discussed in the remainder of this section.



\art{Expand section detailing features from Table~\ref{tbl:approach-textual-features}. For now, there's a short sentence for the novel features}

\vspace{3mm}
\begin{hangparas}{1em}{1}
    \textbf{Semantic frame patterns:} We have shown that certain semantic frames are more prominent in sentences deemed relevant in the annotated text of \acs{DS-android}.
    Thus, we encode semantic frames in a set of \textit{relevancy patterns}. 
    Guided by Xu et al.~\cite{Xu2017}, we set the value for this property to 1 if a sentence in a pertinent artifact contains at least one relevancy pattern, otherwise 0.
\end{hangparas}


\vspace{3mm}
\begin{hangparas}{1em}{1}
    \textbf{Semantic frame similarity:} The intentionality of a task might assist us determining sentences relevant to that task. 
    This feature computes the cosine similarity between the vectors representing the semantic frames present in the task's title with the frames of each sentence within a pertinent artifact. \art{I need to explore this in \acs{DS-android}}
\end{hangparas}


\vspace{3mm}
\begin{hangparas}{1em}{1}
    \textbf{Word embeddings similarity:} Semantic similarity, as captured via word embeddings, might be an indicator of the relevance of a sentence to a task. With this property, we measure semantic similarity using the cosine similarity between the embedding vectors of a task's title and a sentence within a pertinent artifact. 
    For that, we use the \textit{FastText} model~\cite{bojanowski2017FastText} and the asymmetric semantic similarity function described by Ye et al.~\cite{Ye2016} and used in several other studies~\cite{silva2019, Huang2018, Xu2017}.
\end{hangparas}


\vspace{4mm}        
\begin{table}[H]
\begin{small}
\begin{tabular}{lr}

\hline

\parbox[l][0.5cm][c]{11cm}{
    \textbf{Features}
} & \\    

% -----------------------------------------------------------------------------------
\hline
\parbox[l][0.5cm][c]{11cm}{\cellcolor{lightgray} 
    \textbf{Lexical features:}
} & \cellcolor{lightgray} \\  

\parbox[l][1cm][c]{11cm}{ 
    \textbf{API word pattern:} indicates if a sentence matches a pattern that captures 
     indispensable information. 
} & \cite{nadi2020, Robillard2015} \\  


\parbox[l][0.5cm][c]{11cm}{ 
    \textbf{Conditional sentence:} indicates if a sentence contains a condition.
} & \cite{nadi2020} \\  

\parbox[l][1cm][c]{11cm}{ 
    \textbf{Entity overlap:} indicates if software specific terms mentioned in the task also appear in the sentences within a pertinent artifact.
} & \cite{Xu2017, Ponzanelli2015, Ponzanelli2017} \\  

\parbox[l][1.5cm][c]{11cm}{ 
    \textbf{Information entropy:} indicates the entropy of each sentence within a pertinent artifact. High-entropy sentences often represent unique and useful information.
} & \cite{Xu2017,Rastkar2010} \\  

\parbox[l][1.5cm][c]{11cm}{ 
    \textbf{Lexical similarity of the task's content and the artifact's content:}
    cosine similarity based on term-frequency inverse-document-frequency (\textit{tf-idf})
    between a task's title and each sentence within a pertinent artifact.
} & \cite{Lotufo2012, Ponzanelli2015, Ponzanelli2017} \\  

% -----------------------------------------------------------------------------------

\hline

\parbox[l][0.5cm][c]{11cm}{\cellcolor{lightgray}
    \textbf{Word semantics features}
} & \cellcolor{lightgray} \\ 

\parbox[l][1.5cm][c]{11cm}{ 
    \textbf{Word embeddings similarity:} asymmetric similarity based on the word embedding vector representations of a task's title and each sentence within a pertinent artifact.
} & \cite{Xu2017, silva2019} \\  

% -----------------------------------------------------------------------------------
\hline

\parbox[l][0.5cm][c]{11cm}{\cellcolor{lightgray}
    \textbf{Sentence semantics features}
} & \cellcolor{lightgray} \\ 

\parbox[l][1cm][c]{11cm}{ 
    \textbf{Semantic frame patterns:} indicates if a sentence contains a set of semantic 
    frames commonly observed in sentences deemed relevant.    
} & - \\  

\parbox[l][1cm][c]{11cm}{ 
    \textbf{Semantic frame similarity:} cosine similarity based on the frames that appear 
    in a task's and the frames of each sentence within a pertinent artifact.
} & - \\  

\hline

\parbox[l][0.5cm][c]{11cm}{ 
    \art{Other potential features}
} & \\

\end{tabular}
\end{small}
\caption{Features explored}
\label{tbl:approach-textual-features}
\end{table}



\clearpage

\subsection{Approach Overview}


Putting it all together, our approach is based on the LexRank algorithm.
We build a bipartite graph with textual data originating from a task as well as textual data originating from a pertinent artifact.
Relationships between nodes in the graph are established using a custom similarity function.
This custom function combines lexical properties from previous related work 
and also semantic properties at the word and sentence-level.
Through this function, we can identify a number of prominent nodes (or sentences)
in an artifact that likely contain information relevant to a task.


% \vspace{3mm}
% --- Emphasize properties of our approach.


% ------ Approach is unsupervised and thus, does not require training procedures. 


% ------ Allowing the creation of shorter or longer summaries might be beneficial to accomodate the amount of information sought by newcomers or experienced developers

% ------ Approach can be augmented with artifact-specific meta-data, what might tailor the identification of task relevant information to specific types of artifacts.



