\section{Approach}
\label{cp5:approaches}


We have described how syntactic and semantic factors might assist 
identifying  text relevant to a task. %  in a natural language artifact pertinent to that task
We have also described how developers jump through the artifact's content when they  
need to assess if an artifact is worthy of further inspection. 
Given how these two aspects resemble the \textit{PageRank} algorithm~\cite{Page1999}, 
we choose to investigate its applicability for identifying task-relevant text.


In its most simplistic form, PageRank models the behaviour 
of a ``random surfer'' in the Web~\cite{Page1999}, i.e., 
a user who follows outgoing links in a Web page, but that 
eventually starts their search again in a new page. 
The algorithm represents Web pages as nodes in a graph and it
establishes edges between them using a similarity function.
The more edges a node has, the more central the node is to the graph
and thus, the more relevant is the Web page that that node represents.


In our context, central nodes represent the text in an artifact that most likely 
contain key information for a given task while edges between the nodes, a trail of the information most relevant to the given task.
To identify these nodes, we are guided by studies that 
propose custom functions in place of the algorithm's original one~\cite{Lotufo2012, Ponzanelli2017}. Hence, we define a similarity function (detailed further in this section)
that determines the relevance of text to an input task
according to syntactic and semantic properties derived from the text.


We provide details  our approach in a top-down manner, i.e., we first
provide an overview of the PageRank algorithm via an example and
then, we introduce the \textit{textual features} from related work and of our own used in our custom similarity function.


\subsection{Overview}


Our approach takes a task and an artifact as inputs and outputs 
a probability score indicating the sentence's relevancy for the input task. 
Based on such output, we can select the the top-n sentences with highest probability as the ones most likely to contain information that assists a developer in completing their task. 
Note that, to identify these sentences, we need to not only identify the sentences most central to the artifact, as in a summarization problem~\cite{Rastkar2010,Lotufo2012},
but also weight them against the input task, as in a query-based summarization problem~\cite{Xu2017, Goldsteinet1999}.



Given an input artifact $A$ and a query representing task $t$, we seek to determine the centrality of each sentence $s \in A$ with respect to task $t$.
For that, we build an adjacency matrix representing the weight between nodes $s_i$ and $s_j$ according to some similarity function.
To incorporates information about the task,
we follow Lotufo et al. procedures~\cite{Lotufo2012} and we also weight every sentence $s_i$ against task $t$, resulting in a function  
$f_{sim}(s_i, s_j, t)$. 
PageRank takes the adjency matrix built by this function as an input and outputs the centrality of each node, which in our case indicates how likely the sentence which that node represents 
is to contain information relevant to the input task.



For example, consider an Android task\footnote{\url{https://github.com/Kotlin/anko/issues/135}} where a user describes that a feature that helps layout placement 
is not working as expected.
Table~\ref{tbl:gravity-task}, shows an excerpt of the sentences of a Stack Overflow answer\footnote{\url{https://stackoverflow.com/questions/19025301}}
that might contain information relevant to this task.
The table also shows probability outputs from PageRank when using some arbitrary function $f_{sim}(s_i, s_j, t)$.
If our goal is to identify the most probable sentence that might be relevant to the Android gravity task, 
we would first identify s\textsubscript{4}, which has the highest probability ($p=0.42$).
Likewise, the three most relevant sentences for this example are sentences s\textsubscript{4}, s\textsubscript{5}, and s\textsubscript{3}, respectively.


\vspace{4mm}        
\begin{table}[H]
\begin{scriptsize}
\begin{tabular}{ccl}

\hline

\textit{-} & \textit{task:}  & \parbox[l][.5cm][c]{11cm}{ The gravity is not working on the TextView in some situation. } \\

0.03 & \textit{s\textsubscript{1}} &  \parbox[l][1cm][c] {10cm}{ You set this text view a width of \texttt{wrap\_content} it means, what ever the text is, the view take the size of the text. }\\
0.15 & \textit{s\textsubscript{2}} &  \parbox[l][.5cm][c]{10cm}{  and in the LinearLayout , the default gravity (used here) is `center' }\\
0.16 & \textit{s\textsubscript{3}} &  \parbox[l][.5cm][c]{10cm}{ You are mistaking gravity and \texttt{layout\_gravity}.}\\
0.42 & \textit{s\textsubscript{4}} &  \parbox[l][.5cm][c]{10cm}{ gravity is the way the text will align itself in the TextView. }\\
0.22 & \textit{s\textsubscript{5}} &  \parbox[l][1cm][c]{10cm}{ The TextView being in \texttt{wrap\_content} does nothing, as the TextView is exactly the size of the text. }\\


\hline

\end{tabular}
\end{scriptsize}
\caption{Excerpt of sentences from a Stack Overflow answer for the Android gravity task}
\label{tbl:gravity-task}
\end{table}



% PageRank would return the following probabilities for this matrix $[0.03, 0.15, 0.16, 0.42, 0.22]$.


% \begin{scriptsize}
% \begin{equation}
% \begin{matrix}
%             & s_1     & s_2   & s_3   & s_4   & s_5  \\
%         s_1   & 0     & 0     & 1     & 0     & 1    \\
%         s_2   & 0     & 0     & 0     & 1     & 1    \\
%         s_3   & 0     & 0     & 0     & 1     & 0    \\
%         s_4   & 0     & 1     & 1     & 0     & 1    \\
%         s_5   & 0     & 0     & 0     & 1     & 0     
% \end{matrix}
% \label{eq:adj-matrix}
% \end{equation}
% \end{scriptsize}



\subsection{Textual Features}


At this point, it is noticeable that the similarity function used by the algorithm is the stepping stone for the identification of sentences that contain information that may be relevant to a task.  
Hence, we seek to design a similarity function that accounts for the richness of data available in the text and that applies to different artifact types.



We base our function on prior work that uses lexical and artifact-specific properties
to identify relevant text in certain artifact types~\cite{Ponzanelli2015, Xu2017,Lotufo2012}.
Nonetheless, we do not use artifact-specific data and instead, we explore combining \textit{lexical, word semantics, and sentence semantics} to 
compute similarity, where our overall similarity score is the normalized sum of the values obtained for each of the properties that we explore.



Table~\ref{tbl:approach-textual-features} provides a short description of the properties that we use. 
They originate both from existing related work and also from heuristics that we derive from the text deemed relevant
in the \acs{DS-synthetic} corpus.
Certain properties have binary values (i.e., either $0$ or $1$), such as when the frames from a sentence 
match a relevancy pattern. Other properties range from $0$ to $1.0$, such as
the lexical cosine similarity value between two sentences.
Note that, for binary values, we compute $f_{sim}(s_i, s_j, t)$ for that property
with regards to $s_i$, what will give more weight to that sentence and thus, affect 
its probability, as computed through PageRank.











% \art{Expand section detailing features from Table~\ref{tbl:approach-textual-features}. For now, there's a short sentence for the novel features}


\vspace{4mm}        
\begin{table}[H]
\begin{small}
\begin{tabular}{lr}

\hline

\parbox[l][0.5cm][c]{11cm}{
    \textbf{Features}
} & \\    

% -----------------------------------------------------------------------------------
\hline
\parbox[l][0.5cm][c]{11cm}{\cellcolor{lightgray} 
    \textbf{Lexical features:}
} & \cellcolor{lightgray} \\  

\parbox[l][1cm][c]{11cm}{ 
    \textbf{API word pattern:} indicates if a sentence matches a pattern that captures 
     indispensable information. 
} & \cite{nadi2020, Robillard2015} \\  


\parbox[l][0.5cm][c]{11cm}{ 
    \textbf{Conditional sentence:} indicates if a sentence contains a condition.
} & \cite{nadi2020} \\  

\parbox[l][1cm][c]{11cm}{ 
    \textbf{Entity overlap:} indicates if software specific terms mentioned in the task also appear in the sentences within a pertinent artifact.
} & \cite{Xu2017, Ponzanelli2015, Ponzanelli2017} \\  

\parbox[l][1.5cm][c]{11cm}{ 
    \textbf{Information entropy:} indicates the entropy of each sentence within a pertinent artifact. High-entropy sentences often represent unique and useful information.
} & \cite{Xu2017,Rastkar2010} \\  

\parbox[l][1.5cm][c]{11cm}{ 
    \textbf{Lexical similarity of the task's content and the artifact's content:}
    cosine similarity based on term-frequency inverse-document-frequency (\textit{tf-idf})
    between a task's title and each sentence within a pertinent artifact.
} & \cite{Lotufo2012, Ponzanelli2015, Ponzanelli2017} \\  

% -----------------------------------------------------------------------------------

\hline

\parbox[l][0.5cm][c]{11cm}{\cellcolor{lightgray}
    \textbf{Word semantics features}
} & \cellcolor{lightgray} \\ 

\parbox[l][1.5cm][c]{11cm}{ 
    \textbf{Word embeddings similarity:} asymmetric similarity based on the word embedding vector representations of a task's title and each sentence within a pertinent artifact.
} & \cite{Xu2017, silva2019} \\  

% -----------------------------------------------------------------------------------
\hline

\parbox[l][0.5cm][c]{11cm}{\cellcolor{lightgray}
    \textbf{Sentence semantics features}
} & \cellcolor{lightgray} \\ 

\parbox[l][1cm][c]{11cm}{ 
    \textbf{Semantic frame patterns:} indicates if a sentence contains a set of semantic 
    frames commonly observed in sentences deemed relevant.    
} & - \\  

\parbox[l][1cm][c]{11cm}{ 
    \textbf{Semantic frame similarity:} cosine similarity based on the frames that appear 
    in a task's and the frames of each sentence within a pertinent artifact.
} & - \\  

\hline

\parbox[l][0.5cm][c]{11cm}{ 
    \art{Other potential features}
} & \\

\end{tabular}
\end{small}
\caption{Features explored}
\label{tbl:approach-textual-features}
\end{table}



\art{Evaluating these properties in the \acs{DS-synthetic} corpus, I currently got an average precision of $.75$ and pyramid-precision of $.52$}


\clearpage


% To devise an artifact-agnostic similarity function that we can use in the PageRank algorithm, we need a set of features that apply to the different artifact types a developer might seek information as part of their task.