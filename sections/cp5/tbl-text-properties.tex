\vspace{4mm}        
\begin{table}[H]
\begin{small}
\begin{tabular}{lr}

\hline

\parbox[l][0.5cm][c]{12.5cm}{ \cellcolor{lightgray}
    \textbf{Features}
} \\    

% -----------------------------------------------------------------------------------
\hline




\parbox[l][1.2cm][c]{12.5cm}{ 
\textbf{API word pattern:} relevant text is captured through a set of pattern extracted from annotated data on indispensable knowledge required to use API elements. 
 \cite{nadi2020, Robillard2015} } \\  
    
\parbox[l][1.2cm][c]{12.5cm}{ 
    \textbf{Code element:} relevant text often mentions code elements such as class name or method calls    
 \cite{Robillard2015} } \\ 

\parbox[l][0.6cm][c]{12.5cm}{ 
    \textbf{Conditional sentence:} relevant text is often expressed as a condition in a sentence
 \cite{nadi2020} } \\  

\parbox[l][1.2cm][c]{12.5cm}{ 
    \textbf{Entity overlap:} sentences that have software terms that also appear in a task description are likely relevant 
 \cite{Xu2017, Ponzanelli2015} } \\  

\parbox[l][1.2cm][c]{12.5cm}{ 
    \textbf{Information entropy:} high-entropy sentences often represent unique and useful information.
 \cite{Xu2017, Rastkar2010} } \\  

\parbox[l][1.7cm][c]{12.5cm}{ 
    \textbf{Lexical similarity:} text relevant to a task often contains several of the lexical terms present in the task 
    description, which is identifiable through textual similarity metrics such as 
    term-frequency inverse-document-frequency based cosine similarity.    
 \cite{Lotufo2012, Ponzanelli2015} } \\  

 \parbox[l][0.5cm][c]{12.5cm}{ \red{other references \dots} } \\  


% -----------------------------------------------------------------------------------
\hline

\end{tabular}
\end{small}
\caption{Textual properties used to automatically identify relevant text in related work}
\label{tbl:text-properties-references}
\end{table}

