\subsection{Feature evaluation}


We evaluate how each individual textual property contributes towards determining 
that a sentence in a pertinent artifact is relevant to a task 
using a \red{placeholder} test. 
Informally, the higher the test output value for a feature, the more the observed outcome (i.e., whether a sentence is relevant or not to a task) depends 
on that feature.
We use this knowledge to identify if certain features contribute more or less 
to the observed outcome.



\art{I started reading chi square, but it does not apply to the features explored because some of them are continuous. I may need to detail a simple logistic regression model to annswer this question}


\subsubsection{Results}



Table~\ref{tbl:features-chi-square} gives insight into the predictive power of each feature we use as part of our approach. 



\begin{table}[H]
\centering    
\begin{small}
\begin{tabular}{lcc}

\hline

\textbf{Features} & \textbf{importance} & \textbf{p-value} \\

\hline

\textbf{API word pattern} & 10\% & 0.003 \\

\textbf{Conditional sentence} & 10\% & 0.003 \\

\textbf{Entity overlap} & 10\% & 0.003 \\

\textbf{Information entropy} & 10\% & 0.003 \\

\textbf{Lexical similarity} & 10\% & 0.003 \\

\textbf{Word embeddings similarity} & 10\% & 0.003 \\  

\textbf{Semantic frame patterns} & 10\% & 0.003 \\

\textbf{Semantic frame similarity} & 10\% & 0.003 \\

\hline

\end{tabular}
\end{small}
\caption{Features predictive power}
\label{tbl:features-chi-square}
\end{table}











