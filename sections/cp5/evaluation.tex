\section{Evaluation}
\label{cp5:evaluation}


Evaluation focuses on two main questions:

\begin{enumerate}
    \item How accurately can our approach detect text considered by humans as being relevant to a particular task?
    
    \item When identifying text relevant to a task, are certain features (i.e., lexical, word-semantics, and sentence-semantics) more informative than others?
\end{enumerate}


To answer these questions, we use our approach to produce summaries for 
the tasks and artifacts in the \acs{DS-android} corpus,
comparing the sentences identified in a summary against the 
sentences in the corpus' golden data. 

For this evaluation, the summaries produced that comprise 20\% of the content of an artifact,
which is a value derived from our study on characterizing 
task-relevant information in natural language artifacts~\cite{marques2020}.






\subsection{Accuracy}





Motivated by the fact that the text marked by the annotators in the \acs{DS-android} differs
(Section~\ref{cp4:corpus-relevant-text}),
we follow procedures outlined by Lotufo et al.~\cite{Lotufo2012}
to measure how accurately our approach identifies text relevant to a task considering two evaluation scenarios:


\begin{enumerate}
    \item one where the text marked only by two or more annotators is considered relevant, and;
    
    \item a second where we consider that any of the text marked by the annotators contributes towards task completion.
\end{enumerate}



Standard precision and recall metrics~\cite{Manning2009IR} are suitable metrics for the first scenario. In the second, we use pyramid precision and pyramid recall metrics
to compute accuracy based on how many annotators marked the text~\cite{Nenkova2004, Lotufo2012}. 
To understand these metrics, we refer to the evaluation outcomes of Table~\ref{tbl:type-I-II-errors}. 
The \textit{relevant} and \textit{not-relevant} columns represent the text 
marked (or not) by annotators~\cite{Lotufo2012} for a particular task and a pertinent artifact. Rows represent the text in a summary produced automatically by our approach for the same task and pertinent artifact.


\begin{table}[H]
\centering    
\begin{scriptsize}
\begin{threeparttable}
\begin{tabular}{l|l|l}

\hline

\textbf{}
& \textbf{Relevant}    
& \textbf{Not-relevant} \\

\hline

\textbf{Identified as relevant} & true positive ($TP$) & false positive ($FP$) \\
\hline
\textbf{Identified as Not-relevant} & false negative ($FN$) & true negative ($TN$) \\
\hline

\end{tabular}
\end{threeparttable}
\end{scriptsize}
\caption{Evaluation outcomes}
\label{tbl:type-I-II-errors}
\end{table}

    


\subsubsection{Precision and recall}


For standard precision and recall metrics, the relevant column in Table~\ref{tbl:type-I-II-errors} represents \textit{text marked by two or more annotators}. 
In this scenario, $precision$ is the fraction of the sentences
 in the summary that are relevant over the summary's total number of sentences, as shown in Equation~\ref{eq:cp5:precision}.


\begin{equation}
\label{eq:cp5:precision}    
    Precision = \frac{TP}{TP + FP}
\end{equation}

\vspace{2mm}
Recall represents how many of all marked sentences are present in a summary (Equation~\ref{eq:cp5:recall}).


\begin{equation}
\label{eq:cp5:recall}        
    Recall = \frac{TP}{TP + FN}
\end{equation}




\subsubsection{Pyramid precision and pyramid recall}


For pyramid precision and pyramid recall, the relevant column in Table~\ref{tbl:type-I-II-errors} represents \textit{text marked by any annotator}. 


Pyramid precision considers the ``optimal'' sentences that could be part of a summary, i.e., the sentences marked by most of the annotators, and compares that to the sentences  automatically identified. Let $marked(s)$ be a function that returns the number of annotators who marked the sentence $s$, pyramid precision is defined as 
the summation of $marked(s)$ for the set of sentences in a summary ($s \in S$) over the set of optimal sentences derived from the golden data ($s \in G$).



\begin{equation}
\label{eq:cp5:pyramid-precision}    
    Pyramid Precision =  \frac{\sum_{s \in S} \text{ } marked(s)}{\sum_{s \in G} \text{ } marked(s)}
\end{equation}


% \vspace{3mm}
% As an example consider an artifact with a total of 6 marked sentences,
% where 2 sentences were marked by the three annotators, 3 by two annotators and 1 sentence by a single annotator. 
% When identifying three sentences, the optimal solution is $8$ $(2 \times 3 + 1 \times 2)$.
% Hence, an approach that identifies three sentences maked by one, two and all annotators 
% would have a pyramid precision score of $0.75$ or $\frac{ (1 + 2 + 3)}{(2 \times 3 + 1 \times 2)}$.




\vspace{2mm}
Pyramid recall calculates the fraction of sentences in a summary over 
any of the sentences marked by the annotators. As we updated the set of $TP$ and $FN$ of Table~\ref{tbl:type-I-II-errors}, we can still compute pyramid precision using Equation~\ref{eq:cp5:recall}.





\subsubsection{Results}





\subsection{Feature evaluation}



\subsubsection{Results}



--- We use the \acs{DS-android} corpus to evaluate our proposed techniques  \vspace{3mm}






\subsubsection{Setup}

--- Explain each evaluation scenario, e.g., using only word-level properties, sentence-level properties, etc.



\subsubsection{Metrics}








% --- Since the number of annotators that deemed a sentence as useful to a task 
% in the \acs{DS-android} corpus varies, we consider 
% two main evaluation scenarios:


% % --- Guided by Lotufo et al.'s evaluation of techniques for the 
% % detection of the most relevant sentences in a bug report based~\cite{Lotufo2012},
% % we consider two evaluation scenarios:
% \begin{itemize}
%     \item only text marked by the majority of the annotators is relevant to a task.
    
%     \item text marked by any annotator contains valuable information.
% \end{itemize}
 


% The first is a conservative scenario where text marked by a single annotator can be an indicator of unrelated information, which may not contribute to task completion. 
% We use precision and recall~\cite{Manning2009IR} as metrics for this evaluation scenario. 


% Pyramid 
% precision and pyramid recall metrics, as reported by Lotufo et al.~\cite{Nenkova2004, Lotufo2012}, are suitable metrics for the second evaluation scenario. 
% If more annotators marked a sentence, it is likely that that sentence contains key information for completing a task albeit the metrics do not exclude sentences marked by a single annotator because these sentences provide a glimpse of an individual's perception of relevancy~\cite{Petrosyan2015}. 

% For instance, 
% a sentence describing how to use a certain API may be relevant to a novice developer using that API for the first time while it may be ``common knowledge'' to a more experienced developer.










