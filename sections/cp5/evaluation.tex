\section{Evaluation}
\label{cp5:evaluation}



\textit{How accurately can lexical properties identify text deemed relevant to a task?}
    We assess to what extent can a lexical similarity approach automatically identify text relevant to a software task.
    We use lexical similarity as a baseline since both our card sorting analysis (\red{Section~\ref{aaa}}) and related work~\cite{Ko2006a, Freund2015} has shown
    that developers often use keyword-matching as a simple and quick search strategy to locate task-relevant information in a software artifact.



% \red{begin with research question}


% We use our approach to identify sentences relevant to
% the tasks in the \acs{DS-android} corpus,
% comparing the sentences identified against the 
% sentences in the corpus' golden data. 
% For this evaluation, we set the target number of sentences identified (i.e., length of a summary) to 20\% of the content of an artifact,
% which is a value derived from our study on characterizing 
% task-relevant information in natural language artifacts~\cite{marques2020}.





% \subsection{Accuracy}


% Motivated by the fact that the text marked by the annotators in the \acs{DS-android} differs
% (Section~\ref{cp4:corpus-relevant-text}),
% we follow procedures outlined by Lotufo et al.~\cite{Lotufo2012}
% to measure how accurately our approach identifies text relevant to a task considering two evaluation scenarios:


% \begin{enumerate}
%     \item one where the text marked only by two or more annotators is considered relevant, and;
    
%     \item a second where we consider that any of the text marked by the annotators contributes towards task completion.
% \end{enumerate}

% \red{flip the above discuss this as a classification problem?}


% Standard precision and recall metrics~\cite{Manning2009IR} are suitable metrics for the first scenario. In the second, we use pyramid precision and pyramid recall metrics
% to compute accuracy based on how many annotators marked the text~\cite{Nenkova2004, Lotufo2012}. 
% To understand these metrics, we refer to the evaluation outcomes of Table~\ref{tbl:type-I-II-errors}. 
% The \textit{relevant} and \textit{not-relevant} columns represent the text 
% marked (or not) by annotators~\cite{Lotufo2012} for a particular task and a pertinent artifact. Rows represent the text automatically identified by our approach for the same task and pertinent artifact.


% \begin{table}[H]
\centering    
\begin{scriptsize}
\begin{threeparttable}
\begin{tabular}{l|l|l}

\hline

\textbf{}
& \textbf{Relevant}    
& \textbf{Not-relevant} \\

\hline

\textbf{Identified as relevant} & true positive ($TP$) & false positive ($FP$) \\
\hline
\textbf{Identified as Not-relevant} & false negative ($FN$) & true negative ($TN$) \\
\hline

\end{tabular}
\end{threeparttable}
\end{scriptsize}
\caption{Evaluation outcomes}
\label{tbl:type-I-II-errors}
\end{table}

    


% \subsubsection{Precision and recall}




% For standard precision and recall metrics, the relevant column in Table~\ref{tbl:type-I-II-errors} represents \textit{text marked by two or more annotators}. 
% In this scenario, $precision$ is the fraction of the sentences
%  identified that are relevant over the total number of sentences identified, as shown in Equation~\ref{eq:cp5:precision}.


% \begin{equation}
% \label{eq:cp5:precision}    
%     Precision = \frac{TP}{TP + FP}
% \end{equation}

% \vspace{2mm}
% Recall represents how many of all marked sentences are identified by our approach (Equation~\ref{eq:cp5:recall}).


% \begin{equation}
% \label{eq:cp5:recall}        
%     Recall = \frac{TP}{TP + FN}
% \end{equation}




% \subsubsection{Pyramid precision and pyramid recall}


% For pyramid precision and pyramid recall, the relevant column in Table~\ref{tbl:type-I-II-errors} represents \textit{text marked by any annotator}. 


% Pyramid precision considers the ``optimal'' sentences that could be identified, i.e., the sentences marked by most of the annotators, and compares that to the sentences actually identified. Let $marked(s)$ be a function that returns the number of annotators who marked the sentence $s$, pyramid precision is defined as 
% the summation of $marked(s)$ for the set of sentences identified ($s \in S$) over the set of optimal sentences derived from the golden data ($s \in G$).



% \begin{equation}
% \label{eq:cp5:pyramid-precision}    
%     Pyramid Precision =  \frac{
%         \sum_{s \in S} \text{ } marked(s)
%     }{
%         \sum_{s \in G} \text{ } marked(s)
%     }
% \end{equation}


% % \vspace{3mm}
% % As an example consider an artifact with a total of 6 marked sentences,
% % where 2 sentences were marked by the three annotators, 3 by two annotators and 1 sentence by a single annotator. 
% % When identifying three sentences, the optimal solution is $8$ $(2 \times 3 + 1 \times 2)$.
% % Hence, an approach that identifies three sentences maked by one, two and all annotators 
% % would have a pyramid precision score of $0.75$ or $\frac{ (1 + 2 + 3)}{(2 \times 3 + 1 \times 2)}$.




% \vspace{2mm}
% Pyramid recall calculates the fraction of the sentences identified by our approach over 
% any of the sentences marked by the annotators. Since we updated the set of $TP$ and $FN$ of Table~\ref{tbl:type-I-II-errors}, we can still compute pyramid precision using Equation~\ref{eq:cp5:recall}.




% \subsubsection{Results}


% When interpreting results, we favour precision/pyramid precision instead of recall/pyramid recall.
% Our rationale is that a developer might completely abandon reading of an artifact due to a false positive.
% Thus, even if we are not able to identify all sentences that are indeed relevant, true positives may encourage a developer to inspect a pertinent artifact more thoroughly~\cite{Singer1998, Brandt2009a}.




% Table~\ref{tbl:approach-results-overall} shows the accuracy of our approach 
% over all features as well as per group of features (i.e., lexical, word-semantics, and sentence-semantics).


% \begin{table}[H]
\centering    
\begin{small}
\begin{threeparttable}
\begin{tabular}{lccccc}


\textbf{technique} & 
\textbf{precision} & \textbf{recall} & 
$\Delta$ \textbf{precision} & $\Delta$ \textbf{recall} \\


\hline


\texttt{baseline} &
0.30 & 0.33 &  
0.38 & 0.48 
\\

\texttt{word2vec + association-pairs} &
0.52 & 0.53 &  
0.52 & 0.51
\\

% \texttt{BERT\textsubscript{DS-synthetic}} &
% 0.54 & 0.55 & 0.51 & 
% 0.55 & 0.57 
% \\

\texttt{BERT\textsubscript{DS-android} + frame-elements} &
0.55 & 0.56 & 
0.55 & 0.56 
\\

\hline

\end{tabular}
\end{threeparttable}
\end{small}
\caption{Pyramid precision, pyramid recall comparison}
\label{tbl:approach-results-overall}
\end{table}




% Table~\ref{tbl:approach-results-task} shows evaluation metrics by type of task, i.e., GitHub or StackOverflow.

% \begin{table}[H]
\centering    
\begin{scriptsize}
\begin{threeparttable}
\begin{tabular}{lcccccccc}




& \multicolumn{4}{c}{\textbf{GitHub}}
& \multicolumn{4}{c}{\textbf{Stack Overflow}}
\\ \cmidrule(l){2-5} \cmidrule(l){6-9}    


\textbf{Features} & 
\textbf{P} & \textbf{R} & $\Delta$\textbf{P} &  $\Delta$\textbf{R} &
\textbf{P} & \textbf{R} & $\Delta$\textbf{P} &  $\Delta$\textbf{R} \\


\hline

\textbf{Lexical} &
0.50 & 0.50 & 0.50 & 0.50 &
0.50 & 0.50 & 0.50 & 0.50 
\\

\textbf{Word semantics} &
0.50 & 0.50 & 0.50 & 0.50 &
0.50 & 0.50 & 0.50 & 0.50 
\\

\textbf{Sentence semantics} &
0.50 & 0.50 & 0.50 & 0.50 &
0.50 & 0.50 & 0.50 & 0.50 
\\

\hline
\textbf{All} &
0.50 & 0.50 & 0.50 & 0.50 &
0.50 & 0.50 & 0.50 & 0.50 
\\
\hline

\end{tabular}
\begin{tablenotes}
    \item[P] Precision; 
    \item[R] Recall;
    \item[$\Delta$P] Pyramid precision; 
    \item[$\Delta$R] Pyramid recall;
\end{tablenotes}
\end{threeparttable}
\end{scriptsize}
\caption{Accuracy results per type of task}
\label{tbl:approach-results-task}
\end{table}





% We also investigate if an artifact type influences the observed accuracy. 
% Table~\ref{tbl:approach-results-artifacts} gives insight into the accuracy of our approach for the artifact types in the \acs{DS-android} corpus. Results comprise the set of features 
% that provide the best accuracy results. A breakdown of results per group of features is available under Appendix~\ref{a}.


% We use a Wilcoxon-Mann-Whitney test~\cite{mannWhitneyU} to check if differences in the  results for each 
% type of artifact are statistically significant. Results of the test indicate that... 

% \begin{table}[H]
\centering    
\begin{scriptsize}
\begin{threeparttable}
\begin{tabular}{llcccc}


\textbf{} & \textbf{Technique} &
\textbf{Precision} & \textbf{Recall} & 
\parbox[c][.7cm][c]{1.5cm}{\centering \textbf{Pyramid precision}} & 
\parbox[c][.7cm][c]{1.5cm}{\centering \textbf{Pyramid recall}} \\


\hline

\multirow{ 2}{*}{\textbf{Baseline}}  & VSM &
0.50 & 0.50 & 
0.50 & 0.50 
\\

& ours &
0.50 & 0.50 & 
0.50 & 0.50 
\\

\hline

\multirow{ 2}{*}{\textbf{API documentation}}  & Kreck &
0.50 & 0.50 & 
0.50 & 0.50 
\\

& ours &
0.50 & 0.50 & 
0.50 & 0.50 
\\

\hline

\multirow{ 2}{*}{\textbf{GitHub issues}} & Hurried &
0.50 & 0.50 & 
0.50 & 0.50 
\\

 & Ours &
0.50 & 0.50 & 
0.50 & 0.50 
\\

\hline

\multirow{ 2}{*}{\textbf{SO answers}} & AnsBot &
0.50 & 0.50 & 
0.50 & 0.50 
\\

 & Ours &
0.50 & 0.50 & 
0.50 & 0.50 
\\

\hline

\end{tabular}
\end{threeparttable}
\end{scriptsize}
\caption{Techniques comparison}
\label{tbl:approach-results-artifacts}
\end{table}









% \subsection{Threats to validity}


