


\section{Motivation}
\label{cp6:method}



Our goal is to examine how a tool that embeds a semantic-based technique can affect a developer's work.
This tool could lead the developer to information---in a software artifact---that is useful to the developer's task,
which would both ease the developer's burden of finding task-relevant information~\cite{Robillard2015}
and allow them to focus their time on other activities such as judging how the information found applies to their task.





To be helpful, such a tool must direct a developer's attention to text that assist them to complete a task~\cite{Robillard2015}.
If we can gather data about the text in an artifact considered by humans as relevant to the task at hand, 
we could assess whether the text automatically identified by the tool is \textit{similar} to the text manually identified. 


In case the text is not similar,
there is a chance that the tool directed a developer to the wrong information or that 
the tool actually found complementary information that humans missed, which might impact the \textit{correctness} of a task. 
Therefore, our evaluation could compare the correctness of solutions submitted by participants who perform a task assisted by such a
tool against that of participants who attempted the same tasks without tool support, i.e., a control group.


Regardless of how correct participants' solutions are, there is a chance that the text shown by the tool is not \textit{useful}---either because it is not relevant for the task at hand or because it is unsurprising, i.e., 
a developer finds the text identified by the tool as ``common-knowledge''~\cite{cwalina2008, Robillard2015}. Hence, our experimental design could also take into account qualitative aspects 
that might not be measured with the previous remarks. For example, the \textit{usefulness} of the text identified by the tool. 




