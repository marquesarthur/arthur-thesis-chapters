


\section{Motivation}
\label{cp6:method}



Our goal is to examine how \acs{tool}---a web browser plug-in that 
 identifies
task-relevant text in pertinent
web pages---can affect a developer's work.
Building on work presented earlier in this
thesis, the tool  uses the BERT with no filters semantic-based technique.
By identifying information that is useful to the developer's task,
a developer's burden to find task-relevant information~\cite{Robillard2015}
can be lowered,
allowing them to focus their time on other activities such as judging how the information found applies to their task.


To be helpful, \acs{tool} must direct a developer's attention to text that assists them in completing a task.
If the tool is successful in identifying text useful to the task, we hypothesize that
the developer will produce a correct solution more often than if they had not used the tool.


Even if a developer is more successful
with the tool than without, there is a chance that the text shown by \acs{tool} is not \textit{useful}. \rev{The text is not useful if it does not provide information that helps a developer in completing the task at hand, or }
a developer finds \rev{that} the text identified by the tool is ``common-knowledge''~\cite{cwalina2008, Robillard2015}. Our experimental design incorporates the gathering of qualitative data to assess the usefulness of the text identified.

 

