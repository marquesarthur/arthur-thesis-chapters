


\section{Motivation}
\label{cp6:method}



Our goal is to examine how a tool that 
automatically identifies
task-relevant text in pertinent
documents can affect a developer's work.
Building on work presented earlier in this
thesis, the tool  uses a semantic-based technique.
By identifying information that is useful to the developer's task,
a developer's burden to find task-relevant information~\cite{Robillard2015}
can be lowered,
allowing them to focus their time on other activities such as judging how the information found applies to their task.


To be helpful, the tool must direct a developer's attention to text that assists them to complete a task.
If the tool is successful in identifying text useful to the task, we hypothesize that
the developer will produce a correct solution more often than if they had not used the tool.


%If we can gather data about the text in an artifact considered by humans as relevant to the task at hand, 
%we could assess whether the text automatically identified by the tool is \textit{similar} to the text manually identified. 


%In case the text is not similar,
%there is a chance that the tool directed a %developer to the wrong information or that 
%the tool actually found complementary %information that humans missed, which might %impact the \textit{correctness} of a task. 
%Therefore, our evaluation could compare the correctness of solutions submitted by participants who perform a task assisted by such a
%tool against that of participants who attempted the same tasks without tool support, i.e., a control group.

Even if a developer is more successful
with the tool than without, there is a chance that the text shown by the tool is not \textit{useful}---either because it is not relevant for the task at hand or because it is unsurprising, that is
a developer finds the text identified by the tool as ``common-knowledge''~\cite{cwalina2008, Robillard2015}. Our experimental design incorporates the gathering of qualitative data to assess the usefulness of the text identified.

 




% , namely \textit{BERT} with no filters