
\subsection{Tasks Correctness}
% \section{Does \acs{beskar} lead to more correct solutions?}
\label{cp6:correctness}



When assisted by a tool able to automatically highlight text identified as relevant to a task, we expect that a developer can produce a solution 
that is equally or more correct than the solution of a developer who attempted a task without tool support. 
To explore the correctness of the solutions submitted by the participants in our experiment, we compile their code and run it against a set of test cases 
that assess their correctness.


\subsubsection{Method}


The correctness of a submitted solution is measured by the number of passing test cases
when running that solution against a set of 10 test cases, specific to each task. 
A solution with compile errors has a correctness score of zero.


\smallskip
\begin{small}


\begin{equation}
    Correctness = \frac{ \text{\textit{\# of passing test cases}}}{\text{\textit{\#  of test cases}}}
\end{equation}
\end{small}

\subsubsection{Results}
 