
\subsection{Comparison of manual and automatically identified task-relevant text}
\label{cp6:comparison}


To assist a developer complete a task correctly, a tool that
automatically identifies text pertinent to that task would ideally 
identify text that humans have also considered as relevant.
Procedures from our control task asked participants to identify text they deemed useful. We
 compare this manually provided data against the text automatically identified.



% \paragraph{\textbf{Metrics.}}

\subsubsection{Metrics}

To investigate the overlap between the participants' manual highlights and the 
the automatic highlights identified by the study tool we use \textit{precision}, \textit{recall}~\cite{manning2010IR}, and \textit{pyramid precision}~\cite{Nenkova2004}.
We compute these metrics for each artifact of each task and report their average.


For this analysis, we follow Lotufo et al.'s procedures~\cite{Lotufo2012} and we consider any text marked by any participant as relevant.
We investigate if our tool  automatically identifies text that multiple participants deemed relevant
 via \textit{pyramid precision}. 
Details for each metric are as follows. 

\medskip

Precision measures the fraction of the text automatically identified  that participants deemed relevant (Equation~\ref{eq:precision-cp6}). 

\smallskip
\begin{small}
\begin{equation}
    Precision = \frac{
        \text{\textit{automatic highlights}~} \cap 
        \text{~\textit{manual highlights}}
    }{\text{\textit{automatic highlights}}}
\label{eq:precision-cp6}    
\end{equation}
\end{small}


Recall represents how many of all the manual highlights were identified by the semantic-based technique applied by our tool (Equation~\ref{eq:recall-cp6}). 



\smallskip
\begin{small}
\begin{equation}
    Recall = \frac{
        \text{\textit{automatic highlights}~} \cap 
        \text{~\textit{manual highlights}}
    }{\text{\textit{manual highlights}}}
\label{eq:recall-cp6}    
\end{equation}
\end{small}

\medskip


\textit{Pyramid precision} compares the text automatically identified to an optimal output, i.e., one where---for the same number of sentences---we identify sentences selected by the most number of participants (Equation~\ref{eq:pyramid-precision-cp6}). The more we identify text that more participants indicated as relevant, the higher pyramid precision is.



\smallskip
\begin{small}
\begin{equation}
    \triangle Precision = \frac{
        weight(\text{\textit{automatic highlights}~})
    }{weight(\text{\textit{optimal highlights}})}
\label{eq:pyramid-precision-cp6}    
\end{equation}
\end{small}
    


To illustrate these metrics, consider an artifact with 4 sentences $\{s_1, s_2, s_3, s_4\}$ that have been selected by $\{2, 0, 1, 1\}$ participants, respectively.
% For an output identifying two sentences for this artifact, an optimal solution would identify sentences $\{s_1, s_3\}$. 
Table~\ref{tbl:metrics-example} shows precision, pyramid precision, and recall metrics in a scenario where we output sentences $\{s_2, s_3\}$ as relevant.
    



\begin{table}[h!]
\caption{Example showing how we compute precision, recall and pyramid precision metrics}
\label{tbl:metrics-example}
\centering    
\begin{small}
\begin{threeparttable}
\rowcolors{2}{}{lightgray}
\begin{tabular}{lcc}

\textit{metric} & \textit{formula} & \textit{result} \\ 
\hline

\textit{precision} & \parbox[c][.9cm][c]{4cm}{\centering $\frac{\{s_2, s_3\}~ \cap ~\{s_1, s_3\}}{\{s_2, s_3\}} = \frac{1}{2}$} & 0.5 
\\


\textit{recall}  & \parbox[c][.9cm][c]{4cm}{\centering $\frac{\{s_2, s_3\}~ \cap ~\{s_1, s_3\}}{\{s_1, s_3,  s_4\}} = \frac{1}{3}$}  & 0.33 
\\

$\triangle$ \textit{precision}  & \parbox[c][.9cm][c]{4cm}{\centering $\frac{weight(s_2) + weight(s_3)}{weight(optimal)} = \frac{0 + 1}{3} $}  & 0.33 
\\

\end{tabular}
\end{threeparttable}
\end{small}
\end{table}



% \smallskip
% \begin{small}
% \begin{equation}
% \begin{split}
% \triangle  Precision(s_2, s_3) = ( 0 + 1) \div 3 =  0.33 \\
% \triangle  Precision(s_1, s_2) = ( 2 + 0) \div 3 =  0.66 \\
% \triangle  Precision(s_1, s_3) =  ( 2 + 1) \div 3 =  1.00 \\
% \label{eq:pyramid-precision-cp6} 
% \end{split}   
% \end{equation}
% \end{small}



% Note that pyramid precision is equal or lower than precision. For example, $Precision(s_1, s_3) = Precision(s_3, s_4) = 1.0$, but 
% results for pyramid precision differ, i.e., 
% $\triangle Precision(s_1, s_3) = 1.0$ and $\triangle Precision(s_3, s_4) = 0.66$. This allows us to check if our tool 
% identified the text deemed relevant by most of the participants who inspected an artifact.



% \paragraph{\textbf{Data.}}
\subsubsection{Data}

Participants who indicated what text was relevant to their assigned control task produced a total of 415 highlights with an average of 7 highlights (std $\pm 3$) per artifact inspected.
On average, this comprises 9\% of the entire content of the artifacts in our experiment. 


Some participants also selected code snippets as relevant to a task---a threat that we discuss in Section~\ref{cp6:threats}. 
Code snippets account for 30\% of the highlights produced, but we remove them from our analysis since our semantic-based approach 
operates on text only. For the textual highlights,
Krippendorf's alpha indicates good agreement of what text in an artifact participants deemed relevant ($\alpha = 0.68$)~\cite{Krippendorff1980, passonneau2006}.
We compare these manually produced highlights to the text automatically identified by our tool.



% \subsubsection{Results}


% \paragraph{\textbf{Results.}}
\subsubsection{Results}



Table~\ref{tbl:comparison-task-wise} summarizes the average of precision, pyramid precision, and recall metrics for each of the tasks in the experiment.
Precision scores range from 0.55 to 0.68, while pyramid precision scores range from 0.55 to 0.57, which suggests that our tool failed to identify some of the text that participants deemed the most relevant.



The results in Table~\ref{tbl:comparison-task-wise} corroborate 
the correctness scores detailed in Figure~\ref{fig:correctness-by-task}. For example, 
in the \texttt{distances} task, participants who performed the task with tool support had solutions less correct than participants in the control group.
This was the task with the lowest precision, recall and pyramid precision values. 
In contrast, the task where participants assisted by our tool obtained the best correctness scores, namely \texttt{titanic}, is the one with the best precision, recall and pyramid precision values.





\begin{table}
\caption{Evaluation metrics per artifact type}
\label{tbl:comparison-overall}
\centering    
% \begin{scriptsize}
\begin{threeparttable}
\begin{tabular}{lcc}

  & \textbf{precision} & \textbf{recall}  \\ 
\hline

Distances & 0. & 0.
\\

NYTimes  & 0. & 0. 
\\

Titanic & 0. & 0.
\\

\hline


\textbf{overall} & 0. & 0.
\\

\hline

\end{tabular}
\end{threeparttable}
% \end{scriptsize}
\end{table}




Table~\ref{tbl:comparison-artifact-type-wise} details evaluation metrics artifact-type wise. 
Stack Overflow posts and API documentation have the highest precision scores. For these types of artifacts, pyramid precision indicates that the 
text automatically identified on Stack Overflow was the text that several participants deemed relevant. 
The same does not apply to API documentation, i.e., our tool failed to detect a portion of the text that many participants deemed relevant. 
Miscellaneous web pages were the artifact type with the lowest scores. As we detail in Section~\ref{cp6:usefulness},
participants indicated that the text identified for this type of artifact was the least useful.



\begin{table}
\caption{Evaluation metrics per task type}
\label{tbl:comparison-artifact-type-wise}
\centering    
% \begin{scriptsize}
\begin{threeparttable}
\rowcolors{2}{}{lightgray}
\begin{tabular}{lccc}




& \textbf{precision} & $\triangle$ \textbf{precision} & \textbf{recall} \\ 
\hline

API documentation & 0.65 & 0.55 & 0.59
\\

Stack Overflow posts  & 0.66 & 0.62 & 0.63
\\

Miscellaneous web pages & 0.53 & 0.53 & 0.54
\\


\hline
\end{tabular}
\end{threeparttable}
% \end{scriptsize}
\end{table}






% \art{Discuss how to account for variability in what the participants highlighted}

