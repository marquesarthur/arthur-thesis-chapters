


\bigskip
\begin{table}[h!]
\centering
\caption{List of artifact types per task}
\begin{scriptsize}
\begin{threeparttable}
\rowcolors{2}{}{lightgray}
\begin{tabular}{llc|llc}
\hline
\textbf{Task} & \textbf{Artifacts} & \textbf{\#} & \textbf{Task} & \textbf{Artifacts} & \textbf{\#} \\
\hline
\hline
%
%
\parbox[l][0.5cm][c]{1cm}{Distances}    & API documents             & 2  & \parbox[l][0.5cm][c]{1cm}{NYTimes}      & API documents             & 3 \\
\parbox[l][0.5cm][c]{1cm}{}             & Stack Overflow posts      & 3 & \parbox[l][0.5cm][c]{1cm}{}             & Stack Overflow posts      & 3 \\
\parbox[l][0.5cm][c]{1cm}{}             & Miscellaneous web pages   & 3 & \parbox[l][0.5cm][c]{1cm}{}             & Miscellaneous web pages   & 4 \\
\hline

\parbox[l][0.5cm][c]{1cm}{Titanic}      & API documents             & 4 & \parbox[l][0.5cm][c]{1cm}{Practice\tnote{*}}      & API documents             & 1 \\
\parbox[l][0.5cm][c]{1cm}{}             & Stack Overflow posts      & 3 & \parbox[l][0.5cm][c]{1cm}{}             & Stack Overflow posts      & 2 \\
\parbox[l][0.5cm][c]{1cm}{}             & Miscellaneous web pages   & 3 \\
\hline


\end{tabular}
\begin{tablenotes}
    \item[*] smaller number of artifacts due to it being a practice task;
\end{tablenotes}
\end{threeparttable}
\end{scriptsize}
\label{tbl:python-task-distribution}
\end{table}





% \bigskip
% \begin{table}[h!]
% \centering
% \caption{List of artifacts per task}
% \begin{scriptsize}
% \begin{threeparttable}
% \rowcolors{2}{}{lightgray}
% \begin{tabular}{l|l}
% \hline
% \textbf{Distances} & \textbf{NYTimes} \\
% \hline

% \parbox[l][0.5cm][c]{6cm}{GeoPy's documentation}    & \parbox[l][0.5cm][c]{6cm}{Requests: HTTP for Humans}    \\
% \parbox[l][0.5cm][c]{6cm}{GeoPy release 0.96.2}     & \parbox[l][0.5cm][c]{6cm}{Requests API}    \\
% \parbox[l][0.5cm][c]{6cm}{Locations in Python (Geocoding w/Geopy)}                              & \parbox[l][0.5cm][c]{6cm}{Beautiful Soup Documentation}    \\
% \parbox[l][1cm][c]{6cm}{Obtaining latitude and longitude of multiple locations using Geopy}     & \parbox[l][1cm][c]{6cm}{Tutorial: Web Scraping with Python Using Beautiful Soup}    \\
% \parbox[l][1cm][c]{6cm}{Python module for getting latitude and longitude from the name of a US city}    & \parbox[l][1cm][c]{6cm}{Extracting an attribute value with beautifulsoup}    \\
% \parbox[l][0.5cm][c]{6cm}{Calculate point based on distance and direction}                              & \parbox[l][0.5cm][c]{6cm}{How to find children of nodes using BeautifulSoup}    \\
% \parbox[l][0.5cm][c]{6cm}{Geocoding in Python Using Geopy}                                              & \parbox[l][0.5cm][c]{6cm}{How to find elements by class}    \\
% \parbox[l][0.5cm][c]{6cm}{Python Geopy to find geocode of an Address}   & \parbox[l][1cm][c]{6cm}{How to extract HTTP response body from a Python requests call?}    \\
% \parbox[l][0.5cm][c]{6cm}{}                                             & \parbox[l][0.5cm][c]{6cm}{Beautiful Soup: Build a Web Scraper With Python}    \\
% \parbox[l][0.5cm][c]{6cm}{}                                             & \parbox[l][0.5cm][c]{6cm}{Web Scraping with BeautifulSoup}    \\

% \hline
% \hline
% \textbf{Titanic} & \textbf{Practice\tnote{*}} \\ 
% \hline

% \parbox[l][0.5cm][c]{6cm}{How do I read and write tabular data?}    & \parbox[l][0.5cm][c]{6cm}{Python: dictionaries}    \\
% \parbox[l][0.5cm][c]{6cm}{API reference pandas.core.groupby}        & \parbox[l][1cm][c]{6cm}{How do I merge two dictionaries in a single expression?}    \\
% \parbox[l][0.5cm][c]{6cm}{API reference pandas.DataFrame.sort\_values}    & \parbox[l][1cm][c]{6cm}{Combine two dictionaries with preference to one of them}    \\
% \parbox[l][0.5cm][c]{6cm}{API reference seaborn.barplot}    & \parbox[l][0.5cm][c]{6cm}{}    \\
% \parbox[l][1cm][c]{6cm}{Pandas query(): How to Filter Rows of Pandas Dataframe?}    & \parbox[l][0.5cm][c]{6cm}{}    \\
% \parbox[l][0.5cm][c]{6cm}{How to Make a Seaborn Barplot}    & \parbox[l][0.5cm][c]{6cm}{}    \\
% \parbox[l][1cm][c]{6cm}{Filter Dataframe Rows Based on Column Values in Pandas}    & \parbox[l][0.5cm][c]{6cm}{}    \\
% \parbox[l][1cm][c]{6cm}{How do I select rows from a DataFrame based on column values?}    & \parbox[l][0.5cm][c]{6cm}{}    \\
% \parbox[l][1cm][c]{6cm}{'Could not interpret input' error with Seaborn when plotting groupbys}    & \parbox[l][0.5cm][c]{6cm}{}    \\
% \parbox[l][1cm][c]{6cm}{Difference between "as\_index = False", and "reset\_index()" in pandas groupby}    & \parbox[l][0.5cm][c]{6cm}{}    \\

% \hline

% \end{tabular}
% \begin{tablenotes}
%     \item[*] smaller number of artifacts due to it being a practice task;
% \end{tablenotes}
% \end{threeparttable}
% \end{scriptsize}
% \label{tbl:python-task-artifacts}
% \end{table}



















































