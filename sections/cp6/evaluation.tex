\clearpage

\section{Evaluation}
\label{cp6:evaluation}



The goal of our evaluation is to investigate whether tools  
 that automatically identify text relevant to a task, according to the text's semantics,
 assist developers complete their tasks. This evaluation helps answer: 


\medskip
\begin{bluequote}
    \textit{ Do semantic-based techniques help a 
    developer complete a software development task more more effectively?} 
\end{bluequote}




To answer this question, we report on a controlled experiment were participants 
had to complete three programming tasks when assisted (or not) by a
 tool (\red{Section}~\ref{}) able to identify text relevant to each task 
in a set of natural language artifacts deemed pertinent to that task. 




For each task, we measure \red{effectiveness} using a test cases that 
exercise a participant's submitted solution. 
A \red{complete} solution must pass all test cases of a given task. 
In turn, a \red{correct} solution must pass a subset of these test cases 
which cover the core functionalities needed in a task's solution.).


% We use \textit{\red{TODO}} and \textit{\red{TODO}} metrics~\cite{manning2010IR} to measure \red{TODO}~\cite{Murphy2005}.




\subsection{Experimental Design}

\red{TODO}

\subsubsection{Participants}


3rd and 4th year students in the CS major who are familiar with Python


\subsubsection{Tasks}


3 programming tasks using well-known python core modules:

- pandas

- strings and regular expressions

- date time manipulation


We design each task in such a way that its solution is not obvious. 
That is, a participant cannot simply copy a code snippet from some API documentation, Stack Overflow post, or blog post to solve the task. Rather, we expect 
that the solution for a task integrates distinct method calls in a logic way...


\subsubsection{Procedures}


Online. Using an online judge-like system


\subsubsection{Design}


Within subjcts OR between subjects



% \subsubsection{Computing Task Correctness}



% We compute values for \textit{precision} and \textit{recall} metrics based on the sentences 
% deemed relevant to a task by \textit{at least two} human annotators.
% To minimize risks from a scenario where one of the techniques and their underlying approaches has a peak or
% bottom performance due to training procedures or due to factors beyond our control, 
% we compute results for each technique over 10 distinct executions over the test data, 
% reporting the average of each metric over all runs.



% We compute values for \textit{precision} and \textit{recall} metrics based on the sentences 
% deemed relevant to a task by \textit{at least two} human annotators.
% To minimize risks from a scenario where one of the techniques and their underlying approaches has a peak or
% bottom performance due to training procedures or due to factors beyond our control, 
% we compute results for each technique over 10 distinct executions over the test data, 
% reporting the average of each metric over all runs.


% For a detailed definition of each metric, we refer to the evaluation outcomes in Table~\ref{tbl:type-I-II-errors}, where  columns represent  labels provided by the annotators and rows,
% the text identified as relevant or not by a technique.

% % 

% \medskip
% \begin{table}[H]
\centering    
\begin{scriptsize}
\begin{threeparttable}
\begin{tabular}{l|l|l}

\hline

\textbf{}
& \textbf{Relevant}    
& \textbf{Not-relevant} \\

\hline

\textbf{Identified as relevant} & true positive ($TP$) & false positive ($FP$) \\
\hline
\textbf{Identified as Not-relevant} & false negative ($FN$) & true negative ($TN$) \\
\hline

\end{tabular}
\end{threeparttable}
\end{scriptsize}
\caption{Evaluation outcomes}
\label{tbl:type-I-II-errors}
\end{table}

    



% \paragraph{\textbf{Precision}}

% Precision measures the fraction of the sentences identified that are relevant over the total number of target sentences identified, as shown in Equation~\ref{eq:cp5:precision}.



% \begin{equation}
% \label{eq:cp5:precision}    
%     Precision = \frac{TP}{TP + FP}
% \end{equation}


% \subsubsection{Computing Task Completeness}


% \medskip
% Precision means identifying only text that is relevant, whereas recall means identifying all relevant text.
% Ideally, we would aim for a technique with high precision and high recall. Unfortunately, this is often not possible and we must reach a compromise.
% As described earlier, our goal is to support developers to locate text that might be relevant to their task, 
% and not locating all the relevant text may lead to incomplete or incorrect solutions, thus the reason why we favour recall.






% \subsection{Results}
% \label{cp6:analytic}





% \red{TODO}




% Table~\ref{tbl:techniques-results-answerbot} shows values of precision and recall for  \texttt{AnsBot} and the semantic techniques that we explore. 
% In the table, rows provide details about a specific technique while columns discriminate 
% precision and recall values and which post-processing filters were applied. 
% We also mark results from paired Wilcoxon-Mann-Whitney tests~\cite{mannWhitneyU} 
% that check if the results between each technique and \texttt{AnsBot} are statistically different.




% Results from our comparison to a state-of-the-art approach, namely AnswerBot, 
% indicate that text-based semantic techniques achieve comparable accuracy  
% when identifying relevant information in Stack Overflow artifacts.





% \subsection{Qualitative Evaluation}
% \label{cp6:qualitative}





% \red{TODO}




% Table~\ref{tbl:techniques-results-answerbot} shows values of precision and recall for  \texttt{AnsBot} and the semantic techniques that we explore. 
% In the table, rows provide details about a specific technique while columns discriminate 
% precision and recall values and which post-processing filters were applied. 
% We also mark results from paired Wilcoxon-Mann-Whitney tests~\cite{mannWhitneyU} 
% that check if the results between each technique and \texttt{AnsBot} are statistically different.




% Results from our comparison to a state-of-the-art approach, namely AnswerBot, 
% indicate that text-based semantic techniques achieve comparable accuracy  
% when identifying relevant information in Stack Overflow artifacts.


