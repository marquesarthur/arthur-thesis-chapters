
\subsection{Threats to Validity}
\label{cp6:threats}




Our experiment compares solutions submitted by participants who attempted each task with and without tool support. 
This represents a between groups design~\cite{Lazar2017-cp3} and we discuss threats inherent to it. 



Since we compare results from different participants, our analysis might be subject to substantial 
impact from individual differences~\cite{Lazar2017-cp3}. 
For example, participants who performed a task with tool support may have been more experienced than participants 
who did the same task without tool support what affects correctness scores.
As another example,  participants' skill and background 
influences the text that they indicate as relevant in the control task as well 
what text they perceive as useful in the tool-assisted task. 
We minimized these threats by recruiting participants of varied background and randomly
assigning tasks to each participant.



The tasks in our experiment impact generalizability. 
Although we opted for simple tasks, we ensured they 
modules used in our tasks were representative. 
For example, we found open-source systems\footnote{\url{https://github.com/ArchiveBox/ArchiveBox/issues/18}} using \texttt{BeautifulSoup} 
with function calls similar to the ones needed to complete the \texttt{NYTimes} task.
Nonetheless, there are clear differences in the artifacts one can gather 
based on the domain or programming language of a task~\cite{baltes2020}.
Hence, we consider other domains and a wider range of task 
and artifacts for future work. 



The selection of tasks also affects our conclusions. We opted for Python programming tasks that 
required writing code, which we use to assess correctness. 
As observed by other researchers~\cite{satterfield2020, meyer2020}, developers
work on many different tasks, some of which focus on code~\cite{Meyer2017}
while others on information seeking~\cite{gonccalves2011}, e.g., finding duplicated bug reports or researching visualization libraries to identify the most suitable one~\cite{satterfield2020}.
Had we decided to use information-seeking tasks, participants could have produced a different set of highlights,
perhaps selecting fewer code snippets. 
Given that 
our experiment was completely remote, instructing participants on how to perform information-seeking 
tasks would have been more difficult. Furthermore, objectively judging their correctness 
would also be more strenuous, which would lead to a different experiment with challenges and risks of its own.




The fact that we consider the text marked by any participant as relevant 
also affect our conclusions. 
We refrain from excluding text selected by a few participants from our analysis 
for reasons similar to the ones in our characterization of task-relevant information (Chapter~\ref{ch:characterizing}). That is, the text marked by these participants may still contain valuable information. 
We minimize this threat by reporting both precision and pyramid precision, where we observe that 
our approach failed to detect the text that multiple participants deemed relevant
for some tasks or types of artifacts. 



Concerning the text automatically identified by our tool, we gather usefulness at the artifact level.
Suppose we had gathered usefulness at the sentence level. In that case, we could have used this information 
to further refine our analysis, for example, reporting precision and recall 
at different usefulness levels or computing accuracy based on the participants' input, as done by Xu et al~\cite{Xu2017}. 
However, asking participants to provide feedback at the sentence level would have considerably increased the time we estimated that the experiment would take,
which would impact recruitment. We weighed the benefits and drawbacks of a fine-grained or more coarse-grained 
analysis, and we opted for the latter so that this would not be a barrier to people deciding on 
whether to participate in our experiment.


Chapter~\ref{ch:discussion} futher discusses limitations or improvements to the semantic-based techniques and to our tool.