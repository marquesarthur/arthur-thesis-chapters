
% \section{Is \acs{beskar} useful?}
\section{Result: Usefulness Analysis}
\label{cp6:usefulness}




% caused by individual differences between the participants who performed a task with and without tool support, what can justify a similar level of correctness in the results; or the presence (or lack) of overlap between the manual and automatically identified 
% text. 
% However, this does not imply that the tool helped participants accomplish their task. 


To further explore if the highlights shown by the tool were helpful, we also asked a participant 
to indicate whether the highlights assisted them complete their assigned task. 
In this section, we described results for this analysis. 



\subsection{Method}



\subsection{Results}



% We use a diverging stacked bar chart~\cite{Heiberger2014} to analyze the  Likert scale
% responses on the usefulness of the text automatically identified.


% Usefulness indicates the percentage of responses agreeing or disagreeing with whether sentences
% automatically highlighted assisted a participant in completing a task.


% \smallskip
% \begin{small}

% \begin{equation}
% Usefulness = \frac{
%     \text{\textit{\# of responses at i}}
% }{
%     \text{\textit{total \# of responses}}
% }
% \end{equation}
        

% \begin{equation*}
% i \in \{ 
%     \text{\textit{
%         (strongly) agree, neither agree or disagree, (strongly) disagree
%     }}  
% \}
% \end{equation*}
% \end{small}





% \subsubsection{Written Feedback}

 
% We use qualitative methods to analyze participants' responses to the open-ended questions. 

% \art{Think this through...}
