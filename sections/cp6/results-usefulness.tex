
\clearpage

% \section{Is \acs{beskar} useful?}
\subsection{Usefulness Analysis}
\label{cp6:usefulness}


Having compared the correctness of manual and tool-assisted tasks as well 
as the amount of overlap between text identified as relevant by the participants and 
text automatically identified by \acs{beskar}, we turn to the question of 
whether the highlights shown by the tool were considered helpful. 
For that, we analyze participants' ratings and the feedback that they 
provided at the end of our experiment.



\subsubsection{Method}

% Usefulness indicates the percentage of responses agreeing or disagreeing with whether sentences
% automatically highlighted assisted a participant in completing a task.


% \smallskip
% \begin{small}

% \begin{equation}
% Usefulness = \frac{
%     \text{\textit{\# of responses at i}}
% }{
%     \text{\textit{total \# of responses}}
% }
% \end{equation}

% \textit{where:}

% \begin{equation*}
% i \in \{ 
%     \text{\textit{
%         (strongly) disagree, neither agree or disagree, (strongly) agree
%     }}  
% \}
% \end{equation*}
% \end{small}


\subsubsection{Results}



% We use a diverging stacked bar chart~\cite{Heiberger2014} to analyze the  Likert scale
% responses on the usefulness of the text automatically identified.




