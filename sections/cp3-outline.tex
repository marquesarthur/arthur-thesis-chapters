\setcounter{chapter}{2}


\chapter{Characterizing Task-relevant Text}
\label{ch:characterizing}



In the last chapter,
we described several studies that analyze text in software engineering artifacts
and many approaches that attempt to automatically extract relevant information from
these natural language artifacts.


Although certainly valuable,
we have established that
most of these studies focus on specific kinds of artifacts.
If we seek to design
techniques that automatically determine relevance regardless of an artifact's type,
we must determine
whether
there is consistency in the text
that software developers identify as relevant to a task across
different kinds of artifacts, and if there is consistency, we also ask
what are common relevance cues in the portions of
the text identified as task-relevant?





To answer these questions, 
this chapter presents an
\textit{empirical study} in which we asked 20 participants with
software development experience to identify relevant text within
selected artifacts for six distinct software development tasks.
We analyze the text that participants considered relevant 
to gain insight into properties of the text 
that are indicative of its relevance to a task~\cite{das2014frame, jurafsky2014speech}. 
We also report the participants'
reasoning process for determining relevance,
which we gathered through interviews
that we use to identify
common themes about their approaches~\cite{spencer2009sorting}.



We start by presenting the research questions that guide 
the design of our study (Section~\ref{cp3:method}).
We then detail experimental procedures (Section~\ref{cp3:experiment}) 
and results (Section~\ref{cp3:results}),
concluding the chapter with a summary of our findings (Section~\ref{cp3:summary}).



% \section{Method}
\label{cp3:method}
% 
\section{Experiment}
\label{cp3:experiment}









\begin{landscape}
    \begin{figure}
        \centering
        \includegraphics[width=\dimexpr\linewidth-4\fboxsep-2\fboxrule]{cp3/heatmap}
    \caption{An example of a task description and a depiction of collected data, shown as a heatmap of text that participants highlighted as relevant to the task; the warmer the color, the more participants highlighted the text}
    \label{fig:task-highlights-heatmap}
    \end{figure}
\end{landscape}



\begin{table}
\caption{Tasks overview}
\begin{scriptsize}
\vspace{-1mm}  


\begin{threeparttable}    
\rowcolors{2}{}{lightgray}
\begin{tabular}{ll}
\hline    
\textbf{Task} & \textbf{Description} \\ 
\hline
\hline
Bugzilla & 
\parbox[l][0.9cm][c]{11cm}{Locate information about Bugzilla's REST API custom fields and how they can be included as part of the \texttt{GET /rest/bug} payload}  \\
%
Databases & 
\parbox[l][0.9cm][c]{11cm}{Review Q\&A forums and decide between the adoption of ORM or JDBC for a system's database being migrated from C to Java}  \\ 
%
GPMDPU & 
\parbox[l][0.9cm][c]{11cm}{Locate constraints or limitations about the GPMDPU shortcuts feature in order to work in a patch for this feature}  \\
%
Lucene & 
\parbox[l][0.9cm][c]{11cm}{Review the Lucene documentation and bug reports to understand how it computes similarity scores during indexing, particularly the BM25 score function, such that you can address a change request} \\
%
Networking & 
\parbox[l][0.9cm][c]{11cm}{Review the MDN documentation and Q\&A forums to decide between the adoption of Server-sent events or WebSockets technologies for a notification system being developed using Javascript } \\ 
%
Yargs & 
\parbox[l][1.2cm][c]{11cm}{Review the Yargs documentation and bug reports to check whether the API provides support for parsing mutually exclusive arguments, which is a requirement for a command line tool being developed in Python } \\
\hline
\end{tabular}
\end{threeparttable}    
\end{scriptsize}
\smallskip
\label{tbl:tasks-overview-cp3}
\end{table}








\begin{table}
\caption{Corpus overview}
\begin{scriptsize}
\begin{center}

% \begin{threeparttable} 
\rowcolors{2}{}{lightgray}
\begin{tabular}{l|cccccccc}
\hline    
\textbf{Task} 
&  \textbf{\# participants}
&   \textbf{\# Artifacts} 
&   \textbf{API} 
&   \textbf{Bugs} 
&   \textbf{Q\&A} 
&  \textbf{\# Sentences} \\ 
\hline    
\hline
Bugzilla    & 18 & 3 & 3 & 0 & 0 & 459 \\
Databases   & 17 & 3 & 0 & 0 & 3 & 232 \\
GPMDPU      & 18 & 3 & 0 & 3 & 0 & 291 \\
Lucene      & 15 & 3 & 2 & 1 & 0 & 170 \\     
Networking  & 17 & 5 & 4 & 0 & 1 & 313 \\
Yargs       & 18 & 3 & 1 & 1 & 1 & 409 \\
\hline    
\hline 
\rowcolor{white}
Total       & 20 & 20 & 10 & 5 & 5 & 1874  \\
\hline 
\end{tabular}
% \begin{tablenotes}
% \end{tablenotes}   
% \end{threeparttable} 
\end{center}
\end{scriptsize}
\label{tbl:corpus-stats}
\end{table}




\begin{figure}
\begin{mdframed}[backgroundcolor=gray!15] 
\begin{scriptsize}

\noindent On a scale of 1 to 5, how familiar are you with the task's technologies: \smallskip

\quad \textit{(not at all familiar)} ~$1$ - $2$ - $3$ - $4$ - $5$ ~\textit{(extremely familiar)} 

\noindent\rule{8cm}{0.4pt} \smallskip

\noindent On a scale of 1 to 5, what was the level of difficulty of the task: \smallskip

\quad \textit{(very easy)} ~$1$ - $2$ - $3$ - $4$ - $5$ ~\textit{(very hard)} 

\noindent\rule{8cm}{0.4pt} \smallskip

\noindent Please evaluate the following sentences and mark only correct statements: \smallskip

\noindent $\square$ SSE are bidirectional and it can be used for pushing notifications to the bill-sharing system; \smallskip

\noindent $\square$ SSE works over HTTP with no additional components; \smallskip
    
\noindent $\square$ WebSockets are bidirectional and it can be used for pushing notifications to the bill-sharing system; \smallskip
    
\noindent $\square$ There are no data type limitations for SSE messages; \smallskip
    
\noindent $\square$ There are no data type limitations for WebSockets messages; \smallskip

\noindent\rule{8cm}{0.4pt} \smallskip

\noindent $\square$ WebSockets should be adopted for the notification system; \smallskip

\noindent $\square$ SSE should be adopted for the notification system; \smallskip

\noindent\rule{8cm}{0.4pt} \smallskip

\noindent $\square$ After reviewing the documentation, I don't have enough knowledge to complete this task

\end{scriptsize}
\end{mdframed}
\caption{Questionnaire for the Networking task asking about previous knowledge, expertise and likely solutions for the task}
\label{fig:networking-questions}
\end{figure}

    

% 
\section{Results}
\label{cp3:results}



\subsection{Agreement}
\label{cp3:agreement}



\begin{figure}
    \centering
    \includegraphics[width=.95\textwidth]{cp3/highlights-distribution}
    \caption{Distribution of the number of participants who deem an HU as relevant}
    \label{fig:highlights-distribution}
\end{figure}


\subsection{Textual Analysis}


We examine syntactic and semantic properties 
of the highlighted text so that 
we might identify
common cues to the relevancy of text to a task (\textit{RQ2}).
Since we intend to use these cues 
in the design general technique, 
we examine the text across 
all the kinds of artifacts 
available for all the tasks. 




 
% For instance, consider the sentence:
% ``\textit{you can use include\_fields if you want specific field names}''.
% We can extract the elements dependent upon the the verb `\textit{use}' (i.e., {\small \texttt{[nsubject, use, code\_word]}})
% and, if we observe that these elements co-occur in multiple HUs, we flag the pattern as a cue to the relevancy of the text.





\subsubsection{Does syntactic structure provide cues to the relevancy of text to a task?}
\label{cp3:syntactic-analysis}


For our syntactic analysis, we follow procedures from related work (Section~\ref{cp2:text-in-se}).
We start by inspecting the elements that compose a sentence (i.e., nouns and verbs)
and then, we analyze possible patterns that may arise from the syntactic structure of the text~\cite{Robillard2015},
investigating if the extracted entities co-occur in multiple HUs.



% We start our syntactic analysis by describing noun phrases and verb phrases
% in the relevant text. 
% We then describe our analysis of syntactic patterns.  



Among noun phrases, we observe that 65\% of the HUs contain acronyms or coding elements.
Existing approaches that rely on these elements to identify relevant text (e.g.,~\cite{Robillard2015} or~\cite{Jiang2016b}) would miss the remaining 35\% of the HUs in our corpus.
This value may seem acceptable at first; however there are no guarantees that
the identified 65\% HUs hold all the crucial information for task completion.
As an example, some of the HUs from the mid and topmost tiers 
do not contain obvious code elements that could signal their relevancy,
such as one of the sentences in the \texttt{Bugzilla} task indicates the need for ``\textit{authentication to allow retrieving non-public data}''.
% The lack of authentication implies missing non-public data, which leads to an incomplete solution, 
% and so this sentence was considered relevant by most of the participants in our study
% even when it .





With regards to verb phrases, we observe a substantial
overlap (81\%) with verbs observed in Ko and colleagues linguistic analysis
of bug report titles~\cite{Ko2006}.
The most common verbs in the HUs include conjugations of verbs such as \textit{use}, \textit{get}, \textit{set}, \textit{be}, or \textit{do},
but with the exception of \textit{use}, \textit{get}, and \textit{set}, the
remaining top common verbs are in English stop words lists~\cite{jurafsky2014speech}. As a result, many
\acs{NLP} techniques would discard them as part of their pre-processing
steps~\cite{Bavota2016}.






As for syntactic patterns, we did not observe a large set of patterns for the variety of tasks and artifacts in our experiment,
where the prominent patterns identified (e.g., $\{$\textit{nsubject, do, negation}$\}$)
reflect common constructs of the English language rather than cues that we might explore for the relevancy of text.
There may be multiple explanations for these results, such as
the fact our corpus contains a small number of natural language artifacts.
Due to this reason, we also checked whether patterns from existing related work (i.e.,~\cite{Chaparro2017, Robillard2015}) applied to the text in our HUs, 
but the small number of matching patters 
raises caution on
their generalizability.



\medskip
\begin{bluequote}
    \textit{We did not find prominent syntactic cues to identify
    task-relevant information. Our analysis of highlights demonstrates:
    1) limitations of existing techniques that rely on code
    elements and acronyms, 2) missed
    information that may occur due to the prevalence of verbs that
    appear in English stop word lists,
    and 3) the absence of patterns derived from the syntactic structure of
    the text.}
\end{bluequote}

 



\subsubsection{Do semantics provide cues to the relevancy of text to a task?}
\label{cp3:semantic-analysis}


For our semantic analysis, we
analyze the meaning of the text in the 
 HUs
using frame semantic parsing~\cite{fillmore1976frame, jurafsky2014speech}.
Semantic frames are centered around events, labelled frames, 
which abstract both the event
as well as relationships, entities or participants related to that
event~\cite{fillmore1976frame, Baker1998frame}. 


We explain semantic frames by considering
 two distinct sentences extracted 
from the \texttt{Databases} and the \texttt{Lucence} tasks, respectively.
For each sentence, Figure~\ref{fig:frame-examples} shows an
excerpt of the frame analysis for the sentences. 
The frames of each sentence (in grey) represent a triggering event and the frame elements \textit{(fe)} (in red) are arguments needed to understand the event. The enclosing square brackets mark all lexical units, or words, associated with either a frame or a frame element.
Observing the frame elements captured by the verbs \textit{see} and
\textit{understand}, both sentences have the common meaning of
describing a `\textit{phenomenon}'.
However, the frame elements
that capture the meaning of each verb differ: the former represents a
`\textit{perception of experience}' while the latter represents a
cognizer `\textit{grasp}'ing her knowledge over the 
phenomenon. 





% \begin{figure}[h!]
% \fbox{%
% \parbox{0.49\textwidth}{%
% }}
% \label{sec:frames-lucene}
% \end{figure}


\begin{figure}[h!]
\begin{mdframed}
\begin{scriptsize}
\noindent \textbf{Databases:} \textit{I have yet to see a situation where hibernate was the reason for poor performance in production.}
{\ttfamily% 
\begin{enumerate}[itemindent=0.5em,leftmargin=0.5em]
\item[] $\big[$I$\big]$\textsubscript{\color{rufous} \textbf{fe:perceiver}} have yet to $\big[$see$\big]$\textsubscript{\hl{\textbf{perception of experience}}} 

\item[] $\big[$a situation where hibernate was the reason for \dots$\big]$\textsubscript{\color{rufous} \textbf{fe:phenomenon}}
\end{enumerate}
}%

\smallskip
\noindent \textbf{Lucence:} \textit{As far as I understand that sums up totalTermFreqs for all terms of a field}
{\ttfamily% 
\begin{enumerate}[itemindent=0.5em,leftmargin=0.5em]
\item[] As far as $\big[$I$\big]$\textsubscript{\color{rufous} \textbf{fe:cognizer}} $\big[$understand$\big]$\textsubscript{\hl{\textbf{grasp}}} 

\item[] $\big[$that sums up totalTermFreqs for all terms of a field$\big]$\textsubscript{\color{rufous} \textbf{fe:phenomenon}}
\end{enumerate}
}%
\end{scriptsize}
\end{mdframed}
    \caption{Example of frames and frame elements}
    \label{fig:frame-examples}
\end{figure}

As multiple sentences might have similar meanings,
we analyze whether there are common frame elements 
that provide cues to the relevancy of text.
For this analysis, all frame elements were extracted automatically
using the \textit{SEMAFOR} toolkit~\cite{das2014frame},
where we extract the frames of every HU, resulting in 3,719 frames across
the 602 HUs. Only 346 distinct frames appear across these 3,719 frames
parsed. The proportionally small number of distinct semantic frames
occurring suggests that different HUs share frame elements. 



Table~\ref{tbl:common-frames} details the most frequent frames identified per tier, filtering to show the 
most frequent frames in a tier that have not appeared in lower tiers.
In the top-most tier, the most frequently identifying distinguishing
frames denote the \textit{cause} or \textit{likelihood} of a phenomenon.
These frames are found in sentences that explain a system's behavior, which are often crucial for task completion,
as in a sentence that provides a cause for the loss in performance when using Hibernate:
 \textit{``if you need to process lots of objects for some reason, though it can seriously affect performance}''.
Other common frames \textit{quantify relationships},
as when a sentence describes the minimum elements needed to perform an operation, e.g., \textit{``to create a flag, at least the status and the type\_id or name must be provided}''.




In the middle tier, we observe frames for actions performed by some entity (\textit{intentionally act}) or facts regarding a topic (\textit{statement}).
Other common frames relate to methods or attributes and the result of some operation such as a method call, which may be useful for identifying code related entities.
For instance, this sentence from the \texttt{Bugzilla} tasks contains both the \textit{being returned} and the \textit{fields} frame, 
\textit{``You can specify to only return custom fields by specifying \_custom or the field name in include\_fields''}.


\afterpage{


\begin{landscape}


\begin{table}
\caption{Most common semantic meanings across all the HUs; frames are presented per tier, from the topmost tier to the bottom tier} 
\begin{scriptsize}
\vspace{-1mm}  
\begin{threeparttable}    
\begin{tabular}{|l|lclr|}

\cline{2-5}       
% 
\multicolumn{1}{c|}{} &
\textbf{Frame} &
\textbf{Freq} &
\textbf{Description} & 
\parbox[l][.4cm][c]{9.8cm}{\textbf{Example}}

\\ \hline

&
\cellcolor{lightgray} Likelihood &
\cellcolor{lightgray} 8\% & 
\parbox[l][1.2cm][c]{4.4cm}{\cellcolor{lightgray} Denotes the likelihood of a hypothetical event occurring}  &
\parbox[l][1.2cm][c]{9.8cm}{\cellcolor{lightgray} \texttt{However, it can be overkill for some types of application, and the backend \underline{could be easier} to implement with a protocol such as SSE.}} \\

\multirow{-2}{*}{\rotatebox[origin=l]{90}{\textbf{top tier}}} &
Causation &
8\% & 
\parbox[l][.8cm][c]{4.4cm}{An effect is observable due to a cause} &
\parbox[l][.8cm][c]{9.8cm}{\texttt{By default this is true, \underline{meaning} overlap tokens do not count when computing norms. }} \\

&
\parbox[l][.8cm][c]{1cm}{\cellcolor{lightgray} Relational \\ Quantity} &
\cellcolor{lightgray} 7\% & 
\parbox[l][.8cm][c]{4.4cm}{\cellcolor{lightgray} Denotes a quantifiable relationship between any two dependent entities}  &
\parbox[l][.8cm][c]{9.8cm}{\cellcolor{lightgray} \texttt{It is \underline{much faster} to get to something working with Hibernate \underline{than it is} with JDBC }} \\

\hline
\hline

&
\parbox[l][.8cm][c]{1cm}{Statement} &
12\% & 
\parbox[l][.8cm][c]{4.4cm}{Verbs and nouns that communicate the act of a speaker to address a message } &
\parbox[l][.8cm][c]{9.8cm}{\texttt{Custom fields \underline{are} normally returned by default unless this is added to exclude\_fields }} \\

\multirow{3}{*}{\rotatebox[origin=l]{90}{\textbf{mid tier}}} &
\parbox[l][.8cm][c]{1cm}{\cellcolor{lightgray} Intentionally \\ act} &
\cellcolor{lightgray} 11\% & 
\parbox[l][.8cm][c]{4.4cm}{\cellcolor{lightgray} An act performed by an entity} &
\parbox[l][.8cm][c]{9.8cm}{\cellcolor{lightgray} \texttt{The data field could, of course, have any string data; \underline{it doesn't have to be} JSON. }} \\

&
\parbox[l][.8cm][c]{1cm}{Fields\tnote{\dag}} &
11\% & 
\parbox[l][.8cm][c]{4.4cm}{Denotes mentioning an object attribute or field} &
\parbox[l][.8cm][c]{9.8cm}{\texttt{computeNorm(FieldInvert state) - computes the normalization value
for a \underline{field} }} \\

&
\parbox[l][.8cm][c]{1cm}{\cellcolor{lightgray} Being \\ returned\tnote{\dag}} &
\cellcolor{lightgray} 11\% & 
\parbox[l][.8cm][c]{4.4cm}{\cellcolor{lightgray} Denotes results from a particular operation or method call} &
\parbox[l][.8cm][c]{9.8cm}{\cellcolor{lightgray} \texttt{You need to be aware of this behaviour otherwise \underline{you will get} cryptic errors}} \\

\hline
\hline

\multirow{3}{*}{\rotatebox[origin=r]{90}{\textbf{bottom tier}}} &
Using &
17\% & 
\parbox[l][.8cm][c]{4.4cm}{An agent uses an instrument in order to achieve a purpose} &
\parbox[l][.8cm][c]{9.8cm}{\texttt{By \underline{using JDBC}, resource leaks and data inconsistency happens as work is done by the developer.}} \\

&
\cellcolor{lightgray} Purpose &
\cellcolor{lightgray} 15\% & 
\parbox[l][.8cm][c]{4.4cm}{\cellcolor{lightgray} Denotes a goal or target to be achieved} & 
\parbox[l][.8cm][c]{9.8cm}{\cellcolor{lightgray} \texttt{Object Relational Mapping \underline{empowers the use of} ``Rich Domain Object'' which are Java classes}} \\

&
Capability & 
14\%  & 
\parbox[l][.8cm][c]{4.4cm}{An entity does or does not meet the pre-conditions for some event or action} & 
\parbox[l][.8cm][c]{9.8cm}{\texttt{\underline{Can't} detect anything outside letters, arrows, ctrl, alt and shift}} \\

\hline

\end{tabular}
\begin{tablenotes}
    \item[\dag] \scriptsize Frame name was modified because its semantic meaning is specific to software engineering;    
\end{tablenotes}
\end{threeparttable}    
\end{scriptsize}
\label{tbl:common-frames}
\end{table}
\end{landscape}
}

The bottom tier contains frames that are common to all HUs.
The most frequent frame in the bottom tier has a semantic meaning of \textit{using}.
HUs with this frame are often sentences detailing how to use a method or a
framework to achieve some goal, what might also explain the second most frequently occurring frame, i.e., \textit{purpose}, which denotes an achievable goal. 
These two frames could be used to filter sentences that contain the means
to use a technology or API with certain intention, as this sentence explaining usage scenarios for
WebSocket and Server-Sent Events in the \texttt{Networking} task:
``\textit{One is synchronous and could/would be used for near real-time data xfer, whereas the other is asynchronous and would serve an entirely different purpose}''.




To provide further evidence supporting these findings,
we also  compared the frequency of the frame elements identified in relevant text (i.e., HUs) and non-relevant text.
We perform a Wilcoxon signed-rank test~\cite{wohlin2012} over the distribution of frames in our data and we observe with statistical significance ($p\text{-value} \le 0.05$) 
that certain frames are more prominent in relevant or non-relevant sentences.
Therefore, our analysis on the semantics of relevant text indicates that there exists some common aspect to the different sentences deemed as relevant.






\medskip
\begin{bluequote}
    \textit{There are recurring semantic meanings in relevant sentences,
    suggesting commonalities in their conveyed information
    and indicating that text might be identified through its semantics.}
\end{bluequote}







% Figure~\ref{fig:frame-distribution} provides insight in the distribution of frames across relevant and non-relevant sentences.
% For example, frames that represent a \textit{required} event
% are more prominent in the relevant text
% as found in a sentence in the \texttt{Bugzilla} REST API that describes
% how to circumvent errors due to invalid tokens:
% ``\textit{An error is thrown if you pass an invalid token; you will need to log in again to get a new token}''.
% On the other hand, frames that relate to user discussions and that do not draw conclusions or provide 
% facts about an API or technology, such as \textit{point of dispute} or \textit{reasoning} are often found in non-relevant text,
% as when users discuss community's procedures in the \texttt{GPMDPU} task: ``\textit{Open a new issue following the template so we can have more details on your device}''. 





% While certain frames are not indicative of relevance when found on their own, we also observe that the co-ocurrance of certain frames in a sentence increase the likelihood of the sentence's relevancy.
% For instance, \textit{purpose} and \textit{using} occur almost evenly across relevant and non-relevant text
% while their co-occurrance is more frequent in relevant text. 
% Figure~\ref{fig:frame-co-occurrence} shows the most frequent frames that co-occur and their ratio on relevant and non-relevant text.





% \begin{figure}
% \centering
% \includegraphics[width=.95\textwidth]{cp3/frames-dist-all}
% \caption{Distribuion of semantic frames over the text; the figure depicts the top five frames most commonly observed in relevant and non-relevant sentences, respectively}
% \label{fig:frame-distribution}
% \end{figure}
    


% \begin{figure}
% \centering
% \includegraphics[width=.95\textwidth]{cp3/frames-dist-2-grams}
% \caption{Distribution of co-occurring frames over the text}
% \label{fig:frame-co-occurrence}
% \end{figure}

\subsection{Interview Analysis}

To understand how participants identified relevant text, we analyzed the participants'
interview responses using a card-sorting approach~\cite{spencer2009sorting}.
We created index cards containing the interview questions and the participants' responses.
The cards were sorted into meaningful themes and then grouped into abstract categories.
We refined themes over three iterations of the data.
At each  iteration, we grouped cards containing responses on similar topics, analyzed the emerging theme, and evaluated
whether there was a broader theme that incorporated two or more of the existing themes.
To reduce individual bias, the first and second authors independently annotated a subset of the cards at every iteration.
Disagreements occurred when more than one theme could apply to a sentence.
The annotators discussed and resolved disagreements refining the set of themes in a subsequent iteration, where annotators had substantial agreement ($\kappa=0.82$).





From participants' comments, we identified nine themes that we group under two major categories.
Table~\ref{tbl:themes-categories} details observed categories, themes, and the number of participants who made a comment pertaining to that theme.
We discuss results per category illustrating comments that best exemplify a theme (grey highlight).



% This analysis led us to a final set of nine themes (Table~\ref{tbl:themes-categories}). We discuss these themes (grey highlights) grouping them under two major categories: \textit{Search strategies} and \textit{Search challenges}.







\subsubsection{How do developers locate relevant information?}
\label{cp3:search-strategies}

Developers often use a mix of \hl{\textit{keyword-matching}} and \hl{\textit{skimming}}~\cite{Starke2009, Ko2006a} to find relevant information within an artifact. 
Some participants said they use these search strategies for all types of artifacts, while others said they use knowledge of a document's structure to assist their searches (\hl{\textit{document-guided}}).


% \pquote{}{P12} \medbreak

\smallskip
\begin{footnotesize}
\begin{quote}
    ``\textit{Definitively [my strategy] wasn't always the same. Going through a StackOverflow question, I would obviously read the first response. For API docs, keywords were the go to. Bug reports are hard. Comments are ordered chronologically and the first ones are sometimes not the most relevant ones}''---P12
\end{quote}
\end{footnotesize}


Participants were also aware of the shortcomings of some artifact types and were less willingly to use faster but less accurate strategies like skimming or keyword-matching for these artifacts when the tasks were difficult. 
In these cases, participants mentioned that they used a \hl{\textit{scrutnizing}} strategy:


\smallskip
\begin{footnotesize}
\begin{quote}
    ``\textit{I usually read every comment [in Stackoverflow]. Obviously, they are ranked, but, in general, every single comment could have something important}''---P11
\end{quote}
\end{footnotesize}
    


Regardless of strategy, participants mentioned using 
implicit criteria to decide when to (not) read some text.
Such implicit criteria often relate to information foraging theory and how an individual follows some \hl{\textit{information scent}} for judging the value, or cost of investigating some text according to available cues~\cite{Pirolli1999}, such as the presence of bullet-points, bold text, or the conciseness and readablilty of the text.



\subsubsection{What challenges do developers face while searching for relevant information?}
\label{cp3:search-challenges}


Participants also commented on factors that made assessing the relevancy of text difficult.
The most common challenge was \hl{\textit{missing information}} that other participants deemed relevant.



\smallskip
\begin{footnotesize}
\begin{quote}
    ``\textit{This [highlight] is a valid alternative if you don't want to use the conflicts [method]. I basically didn't see this because of how I was searching}''---P2
\end{quote}
\end{footnotesize}




We consider missing relevant information as a major threat to task completion because it can lead to an incomplete or incorrect solution~\cite{Murphy2005};
usually, participants missed information due to their workflows, as \textit{P15} explains:





\smallskip
\begin{footnotesize}
\begin{quote}
    ``\textit{It's hard to say that I would have picked [those highlights] without being directed to that. I believe that I knew that there were specific questions coming, but even if I was doing it for myself, I would probably skim first and then, when I start to code, it would be a natural thing to get back and dive into the details}''---P15
\end{quote}
\end{footnotesize}



Participants also missed information due to text \hl{\textit{verbosity}}. 
Bloated text makes it harder for developers to locate task-relevant information~\cite{Robillard2011} and is more cognitively demanding to read:


\smallskip
\begin{footnotesize}
\begin{quote}
    ``\textit{When I looked at the API documentation, there was too much text. I was mentally exhausted just looking at it}''---P6
\end{quote}
\end{footnotesize}


Participants noted that their \hl{\textit{familiarity}} with the task domain also affected how they located relevant information in the text.



\smallskip
\begin{footnotesize}
\begin{quote}
    ``\textit{The easiest one was the one about ORM/JDBC [databases] because I was so familiar with these technologies}''---P02
\end{quote}
\end{footnotesize}




Finally, when presented with \hl{\textit{ambiguous or contradictory}} information, participants had to spend time seeking out additional information to resolve the ambiguity.


\smallskip
\begin{footnotesize}
\begin{quote}
    ``\textit{I think it was in this one [highlight]. They said that \texttt{cmd} works in the same way as \texttt{Ctrl}, but I went with the one that says otherwise. [...] it was actually helpful to have two other highlights so I had a bit more reliance on the thing that was mentioned by more users.}''---P19
\end{quote}
\end{footnotesize}



Our analysis on the participants' comments on their search strategies provide further insights into how developers forage for information in software development natural language artifacts. One of our key observations is that  developers use mixed strategies to locate task-relevant information. In doing so, developers often miss some information that might be relevant for task completion.


\medskip
\begin{bluequote}
    \textit{Developers use mixed strategies to locate task-relevant information, often missing some information that might be relevant to complete a task completely and correctly.}
\end{bluequote}


\begin{landscape}
    



\begin{table}
\caption{Key themes relating to developers' assessment of relevancy}
\begin{small}
% \vspace{-3mm}        
\begin{threeparttable}    
\begin{tabular}{|l|llr|}

\cline{2-4}        

\multicolumn{1}{c|}{} 
& \parbox[l][.4cm][c]{3cm}{\textbf{Theme}}
& \textbf{Definition}
& \textbf{\#}

\\ \hline

\multirow{5}{*}{\rotatebox[origin=r]{90}{\textbf{Searching Strategy}}} 
& \parbox[l][1.9cm][c]{3cm}{\cellcolor{lightgray} document-guided}
& \parbox[l][1.9cm][c]{12cm}{\cellcolor{lightgray} Knowledge about the structure of the document influences how a developer locates relevant text in that particular document, e.g., the fact that StackOverflow has accepted answers, or that bug reports might have status changes such as when a user closes a bug marking it as resolved} 
& \cellcolor{lightgray} 15 \\


 
& \parbox[l][.7cm][c]{3cm}{ keyword-matching} 
& \parbox[l][.7cm][c]{12cm}{ The use of keywords and search filters to locate relevant text} 
&  13 \\


& \parbox[l][1.4cm][c]{3cm}{\cellcolor{lightgray} skimming} 
& \parbox[l][1.4cm][c]{12cm}{\cellcolor{lightgray} The act of quickly reading a portion of a document to locate relevant text. It also represents the act of scrolling from a document's sections and deciding to stop at certain points}
& \cellcolor{lightgray} 11 \\

 
& \parbox[l][.7cm][c]{3cm}{ scrutinizing}
& \parbox[l][.7cm][c]{12cm}{ The act of carefully reading a document/section due to some reason}
&  10 \\

 
& \parbox[l][1.1cm][c]{3cm}{\cellcolor{lightgray} summarizing}
& \parbox[l][1.1cm][c]{12cm}{\cellcolor{lightgray} The desire to first have an overview of a document or section's content before deciding which portions are worth investigating}
& \cellcolor{lightgray} 9 \\


\hline
\hline

\multirow{4}{*}{\rotatebox[origin=r]{90}{\textbf{Challenges}}}
& \parbox[l][1.1cm][c]{3cm}{ missed-information} 
& \parbox[l][1.1cm][c]{12cm}{ Acknowledgement that some information deemed relevant was missed due to the developer's search strategy} 
&  11 \\

& \parbox[l][1.1cm][c]{3cm}{\cellcolor{lightgray} familiarity}
& \parbox[l][1.1cm][c]{12cm}{\cellcolor{lightgray} How familiar is a developer with the domain of the task and/or technology and how his expertise affects her foraging process} 
& \cellcolor{lightgray} 10 \\

&  \parbox[l][1.1cm][c]{3cm}{statement-checking} 
& \parbox[l][1.1cm][c]{12cm}{ The text contains facts that are not accurate and information might be ambiguous or contradictory} 
&  7 \\

& \parbox[l][.7cm][c]{3cm}{\cellcolor{lightgray} verbosity} 
& \parbox[l][.7cm][c]{12cm}{\cellcolor{lightgray} The structure of the text is verbose or extensive hindering readability} 
& \cellcolor{lightgray} 5 \\

\hline

\end{tabular}
\begin{tablenotes}\scriptsize
    \item[\#] Indicates number of participants who have quotes related to the theme.
\end{tablenotes}
\end{threeparttable}    
\end{small}
\label{tbl:themes-categories}
\end{table}

\end{landscape}



\subsection{Summary of results}




\subsection{Threats to Validity}
\label{cp3:threats}
% \section{Summary}
\label{cp3:summary}



In this chapter, we address the problem of locating relevant information
to a particular software development task \textit{within} a natural language artifact.
To better understand how relevant information is encoded in natural language artifacts,
we detailed an empirical study investigating what text is perceived as relevant
and whether there is consistency in what 20 participants with software development experience deem relevant
to a particular software development task.


Our study comprised the analysis of a set of 20 artifacts originating from API documentation, bug reports, and Q\&A documents.
We observe that finding task-relevant information in such artifacts requires filtering to less than
a 20\% of the documents' text and that there is substantial variability in what information participants perceive as relevant.
Nonetheless, there are commonalities shared through most of the identified relevant textual pieces.
These commonalities are found in the semantic meanings extracted from the text suggesting that
the semantics of natural language artifacts might 
can be used in automatic approaches for 
automatically identifying task-relevant text.




% \clearpage

% H18-02104

% 	Task Knowledge Extraction