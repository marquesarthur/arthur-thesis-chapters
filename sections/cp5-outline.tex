\setcounter{chapter}{4}
\setcounter{rq}{1}


\chapter{Identifying Task-Relevant Text}
\label{ch:identifying}


% \begin{enumerate}[label=\textit{RQ\arabic*},leftmargin=1.4cm]
%     \setcounter{enumi}{\the\numexpr \arabic{rq}\relax}
    
%     \item \textit{What techniques can we use to automatically detect text relevant to a software developer's task in artifacts pertinent to the task?} 
    
% \end{enumerate}
% \stepcounter{rq}

% \vspace{1mm}


In this chapter, we discuss the design 
of an approach able to automatically
identify text relevant to a particular task across a range of artifact types.
Our approach called \red{placeholder} opts to accurately determine relevant text within an artifact pertinent to a task such that it encourages a developer to carefully read an artifact that could have been otherwise abandoned by a developer in her search for information relevant to her task.


We guide the design of our approach by using data produced earlier in this thesis, namely the \acs{DS-synthetic} and the \acs{DS-android} corpa, and by using existing techniques 
able to extract text properties at the word level and sentence level. 
We begin detailing the techniques
we explore in Section~\ref{cp5:approaches}. 
In section Section~\ref{cp5:evaluation}, 
we make use of the \acs{DS-android} corpus
to evaluate whether the explored techniques, or their combination, 
are able to detect text that human annotators identified 
as relevant to a task. 
In turn, Section~\ref{cp5:summary} summarizes our key findings.


\art{I want to state that ultimately, I'll have a single approach that may leverages one or more techniques --- word level + sentence level properties}


\clearpage

\section{Approaches}
\label{cp5:approaches}

In this section, we detail three approaches to automatically identify text that is relevant to a particular software task.
These approaches encompass lexical similarity, word semantics, and frame semantics.



All approaches take a task and a pertinent artifact as inputs and output the sentences 
that are most likely to contain information that assists a developer in completing their task. 
To determine how many sentences an approach should identify, we consider that 
no more than 20\% of the content in the artifacts in the
 \acs{DS-synthetic} and the \acs{DS-android} corpora are deemed relevant to a task, which, on average, accounts for 8.93 sentences
 and we approximate these values to identifying a target number of 10 sentences per input task-artifact. 
Our decision to output a certain number of sentences regardless of the approach is to have an easy framework for their comparison (Section~\ref{cp5:evaluation}).



\subsection{Lexical Similarity}

As a baseline, we investigate if we sentences that are lexical similar
to a task are more likely to contain information relevant to that task.


We use Vector Space Model (VSM)~\cite{Salton1975vsm} from Information Retrieval~\cite{Manning2009IR}
to compute the lexical similarity between the sentences within a pertinent artifact and a task. 
VSM represents both a task and individual sentences within an artifact as vectors of term weights,
where the weight of a term
can be computed using a Term-Frequency Inverse-Document-Frequency scheme (\textit{tf.idf})~\cite{Manning2009IR}. 
Once we obtain vector representations $t$ and $s$ 
for an input task and an arbitrary artifact sentence, 
their lexical similarity can be computed 
using the cosine similarity between their vectors, as Equation~\ref{eq:lex-sim} shows:



\begin{equation}
    cos(t,s) = \frac{t^Ts}{\|t\| \|s\|}
    \label{eq:lex-sim}
\end{equation}
\smallskip

By ranking the sentences in an artifact according to their similarity scores, i.e., from highest to lowest,
we can  select the top-n sentences as the ones relevant to an input task.

% ------------------------------------------------


\subsection{Word Semantics}


Language models capture words' semantics based on the context in which words appear~\cite{harris1954distributional}.
They allow a more ``human-like reasoning'' even when words are lexically different, which 
motivates investigating whether we can identify task-relevant text by semantically matching the text in a pertinent artifact to the text in a task.



To detail how we use language models to automatically identify task-relevant text,
 we first introduce general concepts\footnote{
    For a in-depth overview of the concepts behind language models, please refer to~\cite{zhang2021deep-learning}.
} and then, 
Sections~\ref{cp5:skip-gram} and~\ref{cp5:bert} respectively
detail how we use a baseline model and 
a state-of-the-art model to automatically identify task-relevant text.





\subsubsection{Background}


% introduce language models
A core concept of a language model is Harris' distributional hypothesis~\cite{harris1954distributional}, which states that words that appear in a similar context tend to have similar meanings.


A language model exploits this hypothesis by building vector representations, namely \textit{word embeddings}, for each of the words in a text corpus.
For that, it requires a significantly large number of sentences so that
the model associates similar vector embeddings to words that are similar in meaning~\cite{Ye2016}. 


% Overview of baseline model
\smallskip
\begin{hangparas}{.0in}{0}
     \textit{ Skip-gram model.} One common challenge to language models is that they need to learn word vector representations that are good at predicting the nearby words at low computational costs, e.g., the time needed to train a model, the model size, etc.
    The \textit{Skip-gram} model~\cite{Mikolov2013}, proposed by Mikolov et al., addresses such challenges using simple yet efficient training procedures. As Figure~\ref{fig:skip-gram-example} illustrates, the model learns vector representations by \textit{(i)} looking at the $n$ words that preceded and succeed word $w_t$
     as positive training examples, and by \textit{(ii)} randomly sampling words that do not appear in the same context as negative training examples. Empirical results have shown that negative sampling allows for an accurate model able to handle noise data and that 
     the vector representations provided by the model could be used to improve many natural language processing tasks~\cite{mikolov2013efficient}.
\end{hangparas}

\begin{figure}[H]
    \centering
    \includegraphics[width=.65\linewidth]{fig/cp5/ye-skip-gram-example}
    \caption{Positive and negative training examples in the Skip-gram model. Figure originally from~\cite{Ye2016}}
    \label{fig:skip-gram-example}
\end{figure}


Using the skip-gram model, one can identify that words $t$ and $s$ are semantically similar 
computing the cosine similarity between their corresponding word embedding representations, i.e., $w_t$ and $w_s$:



\begin{equation}
    cos(w_t,w_s) = \frac{w_t^Tw_s}{\|w_t\| \|w_s\|}
    \label{eq:word-sim}
\end{equation}




% Overview of state-of-the-art model
\medskip
\begin{hangparas}{.0in}{0}
     \textit{BERT model.} Context in the Skip-gram model refers to the positive/negative examples used during the model's training procedures; this, however, does not allow the model to disambiguate words based on their surrounding text. In other words, a Skip-gram model will have a single vector representation for the word \textit{company} even when it can have different meanings, i.e., a business organization or being with someone. In contrast, state-of-the-art models, such as \textit{BERT}~\cite{Devlin2018Bert}, provide different representations for the same word based on the sentence in which a word appears.
    This additional layer allows for more complex operations, such as word disambiguation \red{ref}.
\end{hangparas}



BERT also addresses tasks that need to understand relationships between sentences, which is a task not directly captured by language modeling~\cite{Devlin2018Bert}.
To capture sentence relationships, BERT training procedures consider both next word prediction---as in any language model---and also next sentence prediction, i.e., given a pair of sentences $A$ and $B$, the model 
is trained to predict the likelihood that sentence $B$ succeeds (or not) sentence $A$ (Figure~\ref{fig:BERT}). 


\begin{figure}
    \centering
    \includegraphics[width=.75\linewidth]{fig/cp5/BERT}
    \caption{BERT next sentence prediction training procedures. Figure originally from~\cite{jay-alammar-bert}}
    \label{fig:BERT}
\end{figure}



Since BERT addresses both next word prediction and next sentence prediction, the model can be used for several
word semantics and sentence relationship tasks such as  ours, i.e., determine the relevance of a sentence within an artifact based on a second sentence representing a task description.


% ------------------------------------------------


\subsubsection{Semantic Similarity}
\label{cp5:skip-gram}



Similar to lexical similarity,  we investigate if the sentences with the highest semantical similarity are most likely to contain information relevant to the input task.


To compute the semantic similarity between the sentences within a pertinent artifact and a task,
we use the Skip-gram model~\cite{Mikolov2013} with word embeddings specifically trained for the software engineering domain~\cite{Efstathiou2018}.
Since word embeddings provide vector representations at the word level, we follow Conneau et al.'s guidelines~\cite{conneau2018} 
and compute vector representations for an entire sentence by averaging the sum of the word embeddings in that sentence.


Provided that we have embeddings $w_t$ and $w_s$ for the text 
of an input task and an arbitrary artifact sentence, 
their semantic similarity can also be obtained 
using the cosine similarity. In turn, we can select the top-n sentences
with the highest semantical similarity as the ones likely relevant to an input software task.


% ------------------------------------------------


\subsubsection{Artifact-Task Sentence Relationships}
\label{cp5:bert}


We use the BERT model~\cite{Devlin2018Bert} to establish relationships between artifact and task sentences pairs and determine 
the sentences within an artifact that most likely contain information relevant to the task.


Since BERT requires training procedures, we start with an already pre-trained model, namely BERT\textsubscript{uncased}, and we tune it to  identifying task-relevant text.
As done by several other studies (e.g., ~\cite{Chaparro2017, fucci2019, Petrosyan2015}), we use standard cross-validation techniques to ensure  that no data used for evaluation is also used
during the model's training procedures. More specifically, we use 10-fold cross-validation with 70\%, 20\% and 10\% splits for training, validation and testing. 


The model outputs probability scores indicating the likelihood of a sentence being relevant to an input task.
We select the top-n sentences predicted by the model as relevant.



% ------------------------------------------------


\subsection{Frame Semantics}


\art{I still need to check frames based on the tasks in \acs{DS-synthetic}, so there might be updates to this section. }


Given our analysis of relevant sentences in the \acs{DS-synthetic} corpus, we pose that 
sentences with certain meanings, such as the ones that provide instructions on using an entity to achieve some goal,
are sentences that a developer would first pay attention to when inspecting a software artifact and thus, they are more likely to contain task-relevant information.



We implement this hypothesis as a post-filtering step~\cite{Manning2009IR} applied to the lexical similarity and word semantics approaches.
Given a set of sentences returned by an approach, 
we use the \textit{SEFrame} tool~\cite{marques2021} as a proxy to the sentence's meaning,
checking if the semantic frames obtained by the tool appear in a set of frames drawn from  sentences annotated as relevant in the \acs{DS-synthetic} corpus.





% ------------------------------------------------


\subsection{Approaches Summary}


Table~\ref{tbl:approaches-summary} bundles the approaches that we explore.
The table provides a short identifier for each approach, identifies the research topic that serve as a basis for each approach and provides a short description for them. From now on, we refer to each approach according to their short identifier.



\begin{table}[H]
\centering    
\begin{scriptsize}
\begin{threeparttable}
\rowcolors{2}{}{lightgray}    
\begin{tabular}{lll}

% \hline

% \multicolumn{2}{c}{\textit{No lock screen controls ever}}  \\



\textbf{Identifier} & \textbf{Based on} & \textbf{Description} \\

\hline


% \texttt{baseline} & 
% \parbox[l][1cm][c] {1.5cm}{lexical\\similarity} &
% \parbox[l][1cm][c] {8.5cm}{
%     Uses VSM to identify the top-n sentences most lexically similar to a task description as task-relevant
% }
% \\


\texttt{word2vec} & 
\parbox[l][1cm][c] {1.5cm}{Skip-gram\\model} &
\parbox[l][1cm][c] {8.5cm}{
    Uses the Skip-gram model to identify the top-n sentences most semantically similar to a task description as task-relevant
}
\\

\texttt{BERT} & 
\parbox[l][1cm][c] {1.5cm}{BERT\\model} &
\parbox[l][1cm][c] {8.5cm}{
    Fine-tunes BERT to predict the sentences that are relevant to an input task
}
\\


\texttt{} & 
\parbox[l][0.5cm][c] {1.5cm}{ ... } &
\parbox[l][0.5cm][c] {8.5cm}{ ... }
\\



% \texttt{SEframes} & 
% \parbox[l][1cm][c] {1.5cm}{frame\\semantics} &
% \parbox[l][1cm][c] {8.5cm}{
%     Postprocessing filter used in conjunction with the other approaches to reduce false positives in an approach's output
% }
% \\

\hline


\end{tabular}
\end{threeparttable}
\end{scriptsize}
\caption{Summary of approaches used to automatically identify task-relevant text}
\label{tbl:approaches-summary}
\end{table}


\section{Evaluation}
\label{cp5:evaluation}


Our evaluation focuses 
on determining what portion of the text identified by human annotators the techniques that we explore can automatically identify.
Our experimental setup is as follows.



\subsubsection{Metrics}


Ultimately, the goal of our evaluation is to assist the design of tools that help developers quickly locate the text relevant to their tasks, surfacing or highlighting the identified text in an artifact under inspection~\cite{Robillard2015}.
This means that the text identified by a technique represents a set containing the information to be surfaced and thus, 
we use \textit{precision} and \textit{recall} to measure the accuracy of each technique~\cite{Manning2009IR}.
We use the evaluation outcomes detailed in Table~\ref{tbl:type-I-II-errors} to understand each metric.

\medskip
\begin{table}[H]
\centering    
\begin{scriptsize}
\begin{threeparttable}
\begin{tabular}{l|l|l}

\hline

\textbf{}
& \textbf{Relevant}    
& \textbf{Not-relevant} \\

\hline

\textbf{Identified as relevant} & true positive ($TP$) & false positive ($FP$) \\
\hline
\textbf{Identified as Not-relevant} & false negative ($FN$) & true negative ($TN$) \\
\hline

\end{tabular}
\end{threeparttable}
\end{scriptsize}
\caption{Evaluation outcomes}
\label{tbl:type-I-II-errors}
\end{table}

    



\paragraph{\textbf{Precision}}

Precision measures the fraction of the sentences identified that are relevant over the total number of sentences identified, as shown in Equation~\ref{eq:cp5:precision}.



\begin{equation}
\label{eq:cp5:precision}    
    Precision = \frac{TP}{TP + FP}
\end{equation}


\paragraph{\textbf{Recall}} Recall represents how many of all marked sentences are identified by a technique (Equation~\ref{eq:cp5:recall}).


\begin{equation}
\label{eq:cp5:recall}        
    Recall = \frac{TP}{TP + FN}
\end{equation}



\paragraph{\textbf{Pyramid Precision \& Pyramid Recall}} 

We measure precision and recall considering sentences marked by two or more annotators; this, however ignores the fact that text marked by a single annotator can equally contribute towards task completion~\cite{marques2020}. To address the text identified by a single annotator, we 
follow evaluation procedures outlined by Lotufo et al.~\cite{Lotufo2012} and we
also report pyramid precision and pyramid recall. These metrics are similar to the ones defined in 
Equations~\ref{eq:cp5:precision} and~\ref{eq:cp5:recall} but treat any of the text marked as relevant.






\subsubsection{Baseline}


As done by~\cite{Lin2021} and~\cite{Ye2016}, we use a standard VSM lexical similarity approach as a baseline. Our rationale to use 
lexical similarity is based on the fact that 
both our card sorting analysis (\red{Chapter~\ref{aaa}}) and related work~\cite{Ko2006a, Freund2015} has shown that developers often use keyword-matching as a simple and quick search strategy to locate text that might contain information relevant to their tasks.


Following procedures analogous to the semantic similarity-based technique (Section~\ref{cp5:approach-w2v}), we compute lexical similarity and output the top-n sentences with highest similarity as the ones likely relevant to an input software task.




\subsubsection{Techniques Configuration}


\art{I could move this section to a subsection or a paragraph at the end of each technique in Section~\ref{cp5:approaches}}


\red{TODO: this is true for all approaches not only IR based ones}
Both the \texttt{baseline} and the \texttt{word2vec} techniques have a set number of sentences in the techniques output. To determine how many sentences these approaches should output, we consider that 
no more than 20\% of the content in the corpora are deemed relevant to a task, which, on average, accounts for 8.93 sentences (\red{Chapter~\ref{}}). We approximate these values to identifying a target number of 10 sentences per input task-artifact, a value that will also allow us to compare the accuracy of our approach to previous work with similar experimental setup, e.g., ~\cite{Xu2017} or~\cite{Lotufo2012}.




% For this evaluation, we set the target number of sentences identified (i.e., length of a summary) to 20\% of the content of an artifact,
% which is a value derived from our study on characterizing 
% task-relevant information in natural language artifacts~\cite{marques2020}.




For the BERT technique, we fine-tune the model training it for up to 10 epochs with a \textit{batch size}---the number of samples passed through the model network at once---of 64. Since the sentences in a task and software artifact are fairly short, with an average length of 26 tokens, we also set the model's sequence length to 64. Cross Entropy is our loss function, and the model is trained to minimize it using the Adam optimizer at a learning rate of \textit{1e-5} with an early stopping criterion. As for training data, we explore two configurations:


\boldparagraph{BERT\textsubscript{DS-android}}{
We split the data available in the \acs{DS-android} dataset and use part of it for training. 
As done by other studies (e.g.,~\cite{Chaparro2017, fucci2019, Petrosyan2015}), we use standard cross-validation techniques to ensure  that no data used for evaluation, i.e., testing, is also used
during the model's training procedures, i.e., training and validation. More specifically, we use 10-fold cross-validation with 70\%, 20\% and 10\% splits for training, validation and testing. \red{ref} We refer to this configuration as \texttt{BERT\textsubscript{DS-android}}.
}

\boldparagraph{BERT\textsubscript{DS-synthetic}}{
To study the impacts of training data on BERT, we train the model in a smaller dataset containing six tasks and a total of 1874 sentences, from which 602 of them were deemed relevant by 20 participants with software development experience (\red{Chapter~\ref{aaa}}). The tasks in this dataset were not drawn from open-source systems, but rather created to stimulate the participants' information-seeking behaviour. Due to the synthetic nature of the tasks in this dataset, we refer to this configuration as \texttt{BERT\textsubscript{DS-synthetic}}.
}


\subsection{Results}






\begin{table}[H]
\centering    
\begin{small}
\begin{threeparttable}
\begin{tabular}{lcccc}


\textbf{Technique} & 
\textbf{Precision} & \textbf{Recall} & 
\parbox[c][.9cm][c]{1.5cm}{\centering \textbf{Pyramid precision}} & 
\parbox[c][.9cm][c]{1.5cm}{\centering \textbf{Pyramid recall}} \\


\hline


\textbf{baseline} &
0.50 & 0.50 & 
0.50 & 0.50 
\\

\textbf{word2vec} &
0.50 & 0.50 & 
0.50 & 0.50 
\\

\textbf{BERT\textsubscript{DS-synthetic}} &
0.50 & 0.50 & 
0.50 & 0.50 
\\

\textbf{BERT\textsubscript{DS-android}} &
0.50 & 0.50 & 
0.50 & 0.50 
\\

\hline

\end{tabular}
\end{threeparttable}
\end{small}
\caption{Accuracy results}
\label{tbl:approach-results-overall}
\end{table}








\begin{figure}
    \centering
    \includegraphics[width=0.95\textwidth]{cp5/results_lollipop}
    \caption{Evaluation results for each technique configuration \red{Draft: I will split the figure in 3. One for each metric. The markers will represent the filters}}
    % \label{fig:webview-task}
\end{figure}







% \art{Change from tables to plots. Similar to Sarah and Treude~\cite{nadi2020}, show Venn diagram with how many sentences each approach identifies and overlaps between them}
\art{Consider changing from tables to plots.}

\art{I've got all results without seframe filters. This week, filters will be my main focus}

% \begin{table}[H]
\centering    
\begin{small}
\begin{threeparttable}
\begin{tabular}{lcccc}


\textbf{Technique} & 
\textbf{Precision} & \textbf{Recall} & 
\parbox[c][.9cm][c]{1.5cm}{\centering \textbf{Pyramid precision}} & 
\parbox[c][.9cm][c]{1.5cm}{\centering \textbf{Pyramid recall}} \\


\hline


\textbf{word2vec + meaningful-frames} &
0.50 & 0.50 & 
0.50 & 0.50 
\\

\textbf{word2vec + similar-meaning} &
0.50 & 0.50 & 
0.50 & 0.50 
\\


\textbf{BERT + meaningful-frames} &
0.50 & 0.50 & 
0.50 & 0.50 
\\

\textbf{BERT + meaningful-frames} &
0.50 & 0.50 & 
0.50 & 0.50 
\\

\hline

\end{tabular}
\end{threeparttable}
\end{small}
\caption{\art{this can be merged with the previous table}}
\end{table}




% \begin{table}[H]
\centering    
\begin{scriptsize}
\begin{threeparttable}
\begin{tabular}{lcccc}


\textbf{Artifact type} & 
\textbf{Precision} & \textbf{Recall} & 
\parbox[c][.7cm][c]{1.5cm}{\centering \textbf{Pyramid precision}} & 
\parbox[c][.7cm][c]{1.5cm}{\centering \textbf{Pyramid recall}} \\


\hline

\textbf{API documentation}  &
0.50 & 0.50 & 
0.50 & 0.50 
\\

\textbf{GitHub issues} &
0.50 & 0.50 & 
0.50 & 0.50 
\\

\textbf{SO answers} &
0.50 & 0.50 & 
0.50 & 0.50 
\\

\textbf{Miscellaneous} &
0.50 & 0.50 & 
0.50 & 0.50 
\\
\hline

\end{tabular}
\end{threeparttable}
\end{scriptsize}
\caption{Accuracy results per type of artifact}
\label{tbl:approach-results-artifacts}
\end{table}









\section{Summary}
\label{cp5:summary}











