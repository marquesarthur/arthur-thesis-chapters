\setcounter{chapter}{5}
\setcounter{rq}{1}


\chapter{Evaluating Text Automatically Identified}
\label{ch:assisting}




In a previous chapter, we have compared different semantic-based techniques for automatically identifying text relevant to a software task.
Such a comparison assisted us in determining the most promising approach, but it did not address whether 
the text automatically identified  
by these techniques can assist a software developer while working on a task.



To fill this gap, this chapter presents a controlled experiment that 
investigates how a tool that embeds one of such semantic-based techniques 
might impact how a developer works on a task. 
The experiment 
compares the solutions of developers who attempted three Python programming tasks with and without tool support
and it also reports several aspects associated with the text automatically identified by the tool,
e.g., how useful was the text automatically found and how does it compare to text that humans deem relevant
in the artifacts available in each task.





We start by outlining our experimental design (Section~\ref{cp6:method}) and then 
by detailing experimental procedures (Sections~\ref{cp6:procedures}).
We report results from the experiment (Section~\ref{cp6:results})
as well as threats to the validity (Section~\ref{cp6:threats}).
We conclude this chapter summarizing our key findings (Section~\ref{cp6:summary}).


\clearpage







% We conclude this chapter presenting threats to the validity of our experiment (Section~\ref{cp6:threats}) and 
%  summarizing our key findings (Section~\ref{cp6:summary}).







\section{Methodology}
\label{cp6:method}


To design an experiment
that examines how a tool that embeds a semantic-based technique might impact a developer's work,
we consider three main factors that might impact experimental design.




\paragraph{\textbf{Factors}}



To be helpful, a tool must direct a developer's attention to text that assist them complete a task~\cite{Robillard2015}. If we can gather data about the text in an artifact considered by humans as relevant to the task at hand, we should be able to assess whether the text automatically identified by the tool is \textit{similar} to the text manually identified. 


In case the text is not similar,
there is a chance that the tool directed a developer to the wrong information, or that 
the tool actually found complementary information that humans missed, which might impact the \textit{correctness} of a task. Therefore, our design must be able to compare the correctness of solutions submitted by participants who perform a task using our envisioned tool against that of developers who attempted the same tasks without tool support, i.e., a control group.


Regardless of how correct participants' solutions are, there is a chance that the text shown by the tool is not \textit{useful}---either because it is not relevant for the task at hand or because it is unsurprising, i.e., the information is ``common-knowledge'' to most developers~\cite{cwalina2008, Robillard2015}. Hence, a successful evaluation would 
also gather data that would allow us to discuss the usefulness of the text identified by the tool. 




\paragraph{\textbf{Design}}




Figure~\ref{fig:tool-experiment-procedures} presents an experiment that allows us to address the aforementioned factors. In this experiment, 24 participants with software development background each attempted a
\textit{manual task} and \textit{tool-assisted task} randomly drawn from a list of well-known Python programming tasks.
Alongside the solution submitted for each task, we use the manual task to collect what text participants deemed relevant to the task performed.
In turn, in the tool-assisted task, we also gathered input on the usefulness of the text automatically identified and shown by our tool. 
This design allow us to:




\begin{itemize}
    \item assess the correctness of the tasks performed \textit{with} or \textit{without} tool support;
    \item compare  the text that participants deemed relevant to the text automatically identified
    and shown in the tool-assisted task, and;
    \item discuss the usefulness of the automatically identified text according to the feedback provided by the participants.
\end{itemize}
 





\begin{figure}
\centering
\includegraphics[width=1.05\textwidth]{cp6/tool-experiment-pipeline.pdf}
\caption{Summary of experimental procedures}
\label{fig:tool-experiment-procedures}
\end{figure}





\subsection{Tasks}
\label{cp6:tasks}


We opted for an experiment with tasks that could be completed by participants on their own time and computer.
This decision was motived by the COVID-19 pandemic and challenges related to recruiting participants and conducting an in-person experiment safely\red{~\cite{}}. 
Since participants would follow instructions on their own, we decided to use tasks that are easy to understand and perform in a single experimental session, but that still required a participant  
to seek information in artifacts associated with that task.



Based on these criteria and guided by~\cite{thiselton2019},
Table~\ref{tbl:python-tasks-modules} details the tasks that we have selected. 
These tasks are based on 
Python w3resource\footnote{\url{https://www.w3resource.com/python-exercises/}} tasks
that require usage of at least one module external to the Python core library.
By using external modules, we aim to reduce the likelihood that a participant 
can provide a solution for a task without consulting any of the artifacts (Section~\ref{cp6:coding-environment})
that detail how to use the modules associated with each task. 



While we leave details about the procedures associated with manual and tool-assisted tasks to another section, Figure~\ref{fig:nytimes-task-github} provides an excerpt of the information shown in a task\footnote{Full descriptions are available in the experiment's supplementary material~\red{\cite{}}.}.
For each task, participants had the task description and examples of input and output scenarios at their disposal. A task contained a list of resources that participants could consult 
so that they could write their solution.
Each task also contained a link to an online coding environment (Section~\ref{cp6:coding-environment})
where they could write and test their solution. 


\begin{table}
\centering
\caption{Python tasks}
\begin{scriptsize}
\rowcolors{2}{}{lightgray}
\begin{tabular}{ll}
\hline
\textbf{Task} & \textbf{Description}                                                                                         \\
\hline
\hline
Distances     &
\parbox[l][1.2cm][c]{11cm}{Given a string representing a rendezvou point and a list of suggested picnic addresses
    you must write an algorithm using the \texttt{geopy} module to find the  picnic address closest to the rendezvou point.} \\
%
NYTimes       &
\parbox[l][1cm][c]{11cm}{Given a string representing the url for NY Times Today's,
    write an algorithm using the \texttt{BeautifulSoup} and \texttt{requests} modules to srape all the headlines of that page.}
\\
%
Titanic       &
\parbox[l][1cm][c]{11cm}{Given a string representing a url for the titanic dataset,
    you must write an algorithm using the \texttt{pandas} and \texttt{seaborn} modules to create a barchart of the data.}    \\
\hline
\end{tabular}
\end{scriptsize}
\smallskip
\label{tbl:python-tasks-modules}
\end{table}




\subsection{Artifacts}
\label{cp6:experiment-artifacts}


Each task requires a set of artifacts that a participant could peruse for information that could assist them in writing their solution.
We select artifacts for a task following procedures similar to the ones we used to create the \acs{DS-android} dataset (Chapter~\ref{ch:android-corpus}). 
That is, for each of the tasks in Table~\ref{tbl:python-tasks-modules}, we use the Google search engine to obtain up to ten artifacts that likely contain 
information that could help a participant correctly complete that task. 


Note that our decision to control the artifacts shown per task relates to our need to study if there is overlap between the text that participants manually identify as relevant and the text that our semantic-based tool identifies for these same artifacts. 
Three pilot runs ensured that the artifacts available in each task had sufficient information to complete a task without 
the need of additional resources. 


\subsection{Coding environment}
\label{cp6:coding-environment}



To ensure that participants had the same conditions to perform each task
and also to minimize setup instructions, we used Google Colab\footnote{\url{https://colab.research.google.com/}} as our coding environment. 


Colab provided participants with a code editor with amenities commonly found in modern IDEs, e.g., code completion and syntax highlighting. It also ensured that all the participants 
performed the tasks in the same Python version and it lifted 
burdens that could arise from installing dependencies associated with the external modules used in each of our tasks. 


Figure~\ref{fig:nytimes-task-colab} shows an example of the Colab coding environment. 
First it handled dependencies management and then, 
it presented a class containing a single method with a \texttt{TODO} block where 
participants should write their solution. 
The environment also provided a main function where participants could see the output
of their code. Alternatively, a participant could use test cases to test their solution
against the examples shown in each task description.



\clearpage

\begin{figure}
    \centering
    \includegraphics[width=1\textwidth]{cp6/task-github.pdf}
    \caption{Information shown in a task}
    \label{fig:nytimes-task-github}
\end{figure}



\clearpage

\begin{figure}
    \centering
    \includegraphics[width=\textwidth]{cp6/task-colab.pdf}
    \caption{Colab environment}
    \label{fig:nytimes-task-colab}
\end{figure}



\clearpage



\subsection{Participants}
\label{cp6:participants}


We advertised our study to professionals in our network and to computer science students at the several universities. 
Our target population comprised professionals and third, fourth-year or graduate students.
We expected participants to have experience in Object-Oriented programming languages, and to consult API documentation when performing a programming task. These questions were part of our demographics and we planned to exclude any participant
who did not meet these requirements. No participants were excluded
based on answers to these questions.




We obtained twenty four responses (3 identified female and 21 male) to our study advertisement. 
At the time of the experiment, 10 participants were working as software
developers and 14 were students (11 graduate and 3 undergrad).
We note that the majority of the students (71\%) also have previous professional experience.


On average, participants self-reported 8 years of programming experience ({\small $\pm$} 3.8, ranging from 3 to 17 years).
The majority of the participants (54\%) had between 5 to 10 years of experience in Object-Oriented programming languages,
closely followed by participants (29\%) with  3 to 4 years of experience.



\section{Procedures}
\label{cp6:procedures}



In this section, we detail the instructions provided to the participants so that 
they could perform our experiment. 
First, we present an overview of the whole experiment and then, we detail
instructions specific to the manual and tool-assisted tasks, respectively.


We advertised the experiment via mailing lists. The email disclosed the purpose of the experiment, eligibility criteria, an estimation of the time that one would take to complete the experiment as well as a link 
to the web survey containing the experiment's consent form and tasks. 


Once a participant consented to participate, the survey gathered demographics (Figure~\ref{fig:experiment-demographics}), which we use to filter participants who did not meet our eligibility criteria. That is, we planned to filter data from a 1st or 2nd year student or from a participant with no experience in Object-Oriented programming languages or that did not consult API documentation when performing a programming task. However, no participants were excluded based on demographics.


\begin{figure}
\begin{mdframed}[backgroundcolor=gray!15] 
\begin{scriptsize}

\noindent To witch gender do you identify? 

\medskip

\noindent If you are a student, in which year of the program are you at?  \smallskip

\quad $\square$~$1st$  
\quad $\square$~$2nd$  
\quad $\square$~$3rd$  
\quad $\square$~$4th$  
\quad $\square$~$5th+$ year 
\quad $\square$~\textit{graduate student} 

\medskip

\noindent For how many years have you been developing software?  

\medskip

\noindent For how many years have you been developing software \underline{professionaly}? 

\medskip

\noindent How many years of experience do you have in Object-Oriented programming languages?~\footnote{\scriptsize closed or open intervals notation} \smallskip

\quad $\square$~\textit{no experience} 
\quad $\square$ $(\infty, 1)$
\quad $\square$ $[1, 3)$
\quad $\square$ $[3, 5)$
\quad $\square$ $[5, 10)$
\quad $\square$ $[10, \infty)$

\medskip

\noindent With which frequency do you consult API documentation when performing a programming task?  \smallskip

\quad \textit{(never)} ~$1$ - $2$ - $3$ - $4$ - $5$ ~ ~\textit{(always)} 

\end{scriptsize}
\end{mdframed}
\caption{Background questions asked to a participant}
\label{fig:experiment-demographics}
\end{figure}

    



Next, the online survey gave participants further instructions 
about how to perform each task and it requested them to install \acs{beskar}.
Setup was followed by a short practice task---separate from the experimental tasks---that allowed participants to familiarize themselves with the content of a task, the tool, and the coding environment that we used (Colab). 


Once a participant completed the practice task, the survey randomly assigned to them a \textit{manual} task, which was followed by a randomly assigned \textit{tool-assisted} tasks---different from the manual task. While tasks were randomly assigned, we made sure that an even number of participants attempted a task with and without tool support.
For each task, including the practice tasks, the survey provided to the participants a link 
to the task description (Figure~\ref{fig:nytimes-task-github}) and asked them to submit a solution for the task, i.e., written Python code. 


Once a participant submitted their solutions, the survey
asked them about any additional feeback that they wished to share and 
offered them the opportunity to enter a raffle for one of two iPads 64 GB 
to compensate them for their time, what concluded the experiment.



\subsection{Manual Task}
\label{cp6:procedures-manual}



In the \textit{manual} task, we instrumented \acs{beskar} to allow participants to highlight text that they deemed useful for the task at hand. In this task, 
the survey asked participants to highlight sentences that they deemed useful and that provided information that assisted task completion---instructions similar to the ones used for the creation of the \acs{DS-android} corpus (Section~\ref{cp4:corpus-relevant-text}).



Figure~\ref{fig:artifact-pre-highlight}
gives insight into how participants highlighted sentences that they deemed relevant. 
Whenever a participant inspected one of the artifacts available for their task, 
the survey instructed them to click on the \texttt{highlight} button in the web browser plug-in context menu.  
\acs{beskar} would then instrumented the HTML of the page identifying individual sentences. 
A participant could hover over identified sentences and select them as relevant by clicking on the hovered text.
As an example, Figure~\ref{fig:artifact-pre-highlight} shows a sentence discussing the \texttt{find\_all} method, which 
one of the participants in our experiment deemed 
relevant to the \texttt{NYTimes} task.



\begin{figure}
    \centering
    \includegraphics[width=0.95\textwidth]{cp6/manual-task.pdf}
    \caption{\texttt{BeautifulSoup} web tutorial showing the \acs{beskar} context menu (top-right corner) and a hovered over sentence}
    \label{fig:artifact-pre-highlight}
\end{figure}




\subsection{Tool-assisted Task}
\label{cp6:procedures-tool-assisted}


In the tool-assisted task, \acs{beskar} would automatically highlight text that 
its underlying semantic-based approach identified as relevant to the participant's task.
This is similar to the highlights shown in Figure~\ref{fig:artifact-pre-highlight}, but without the need for any actions by a participant.


For this task, when a participant submitted their solution, the survey asked them to 
rate---in a 5 points Likert scale~\cite{likert1932technique}---how helpful were the highlights shown by the tool.
As Figure~\ref{fig:experiment-rating} shows, participants rated highlights on a per artifact basis. When rating an artifact, the survey allowed participants to revisit the artifact if so they wished. This would open the artifact web page in a new tab so that a participant
could review the the highlights in that artifact.

Although we evaluate the usefulness of all the highlights for a task
aggregating individual responses, we use the ratings per artifact to explore if the highlights shown in a certain type or artifact, e.g., API documents or tutorials, are more or less useful.


\begin{figure}
\begin{mdframed}[backgroundcolor=gray!15] 
\begin{scriptsize}

\noindent \textbf{1.} Indicate whether you agree with the following statement:

\medskip

\quad \textit{The highlights in \textcolor{steelblue}{``How to extract HTTP response body from a Python requests call''} were helpful to} 

\quad \textit{correctly accomplish the task in question.}  \smallskip

\smallskip

\quad \quad \textit{(Strongly disagree)} ~$1$ - $2$ - $3$ - $4$ - $5$ ~\textit{(Strongly agree)} 


\bigskip


\noindent \textbf{2.} Indicate whether you agree with the following statement:

\medskip

\quad \textit{The highlights in \textcolor{steelblue}{``BeautifulSoup tutorial: Scraping web pages with Python''} were helpful to} 

\quad \textit{correctly accomplish the task in question.}  \smallskip

\smallskip

\quad \quad \textit{(Strongly disagree)} ~$1$ - $2$ - $3$ - $4$ - $5$ ~\textit{(Strongly agree)} 

\centering 

...

\end{scriptsize}
\end{mdframed}
\caption{Questions asking a participant to rate the usefulness of the highlights shown in two artifacts; by clicking on the name of an artifact, a participant could revisit the highlights of that artifact}
\label{fig:experiment-rating}
\end{figure}

    



\subsection{Summary of procedures}


Throughout the past sections, we have described experimental procedures 
where participants attempted a \textit{manual} and a \textit{tool-assisted}.
These procedures allowed us to gather:


\begin{enumerate}
\item a participant's submitted solution (written Python code) for each task;
\item text that participants deemed relevant for completing a manual task;
\item the usefulness of the highlights shown in a tool-assisted task; and
\item any additional feedback (written text) that a participant wished to provide.
\end{enumerate}


We use this data to investigate whether 
a tool embedding a semantic-based technique helps developers complete a software task. 


\clearpage




\section{Results}
\label{cp6:results}

We organize results assessing the solutions submitted by the participants for the manual and tool-assisted tasks, 
comparing manual and automatically identified text as well as discussing the usefulness of the highlights shown 
by the study's tool.



\subsection{Tasks Correctness}
\label{cp6:correctness}



When assisted by a tool able to automatically highlight text identified as relevant to a task, we expect that a developer can produce a solution 
that is equally or more correct than the solution of a developer who attempted a task without tool support. 


To compute how correct a participant's solution is, 
we compile their code and run it against a set of 10 test cases that check whether it produces the correct output for each given test input. 
Hence, \textit{correctness} represents the number of passing test cases of a solution (Equation~\ref{eq:cp6-correctness}).
For example, if the solution of a participant passes 7 out of 10 test cases, we would assign a 
correctness score of $7$ to this solution. 
A solution with compile errors has a correctness score of $0$.


\smallskip
\begin{small}


\begin{equation}
    Correctness = \text{\textit{\# of passing test cases}}
    \label{eq:cp6-correctness}
\end{equation}
\end{small}



\smallskip


From all submitted solutions (24 manual and 24 tool-assisted ones), two 
solutions from tool-assisted tasks had compile errors or failed all test cases. 
One of the solutions that failed all test cases
was from a participant who indicated that they decided to not finish their tool-assisted task due to time constraints; the other, from an Exception thrown in a solution that misused the \texttt{geopy} module. We do not ignore these two solutions when reporting results.


Figure~\ref{fig:correctness-overall} and 
show results for the correctness scores of solutions submitted by participants
who performed a task with and without tool support. 
On average both manual and tool-assisted solutions have a correctness score of $7$.
The notable difference between manual and tool-assisted solutions appear in the 1st quartile of the data which suggests that, 
when assisted by our tool, participants had slightly more correct solutions. 
However, comparison of the correctness scores of manual and tool-assisted solutions via a 
Wilcoxon signed-rank test suggests we cannot draw statistically significant conclusions
based on our data.



% (Table~\ref{tbl:correctness-overall})



% \begin{table}
\caption{Descriptive statistics for correctness}
\label{tbl:correctness-overall}
\centering    
% \begin{scriptsize}
\begin{threeparttable}
\begin{tabular}{lcccccc}

& \textbf{min} & \textbf{1st Qu.} 
& \textbf{median} & \textbf{mean}
& \textbf{3rd Qu.} & \textbf{max}
\\ 
\hline

Manual & 2.0 &    4.0   &  7.5   &  7.0   &  9.0   & 10.0
\\

Tool-assisted  &  0.0  & 5.5 &  8.0 &  7.0 &  9.0 & 10.0
\\


\hline

\end{tabular}
\end{threeparttable}
% \end{scriptsize}
\end{table}







\label{fig:correctness-overall}


\begin{figure}[h!]
    \centering
    \includegraphics[width=\textwidth]{cp6/correctness_aggregated.pdf}
    \caption{Comparison of correctness of solutions for manual and tool-assisted tasks}
    \label{fig:correctness-overall}
\end{figure}



% Figure~\ref{fig:correctness-by-task} details results per task.



\begin{figure}[h!]
    \centering
    \includegraphics[width=\textwidth]{cp6/correctness_overall.pdf}
    \caption{Detailed comparison of correctness of solutions for manual and tool-assisted tasks}
    \label{fig:correctness-by-task}
\end{figure}



\subsection{Comparison of the Task-relevant Text Identified}
\label{cp6:comparison}


To assist a developer correctly complete a task, a tool that
automatically identifies task-relevant text in an artifact pertinent to that task would ideally 
identify text that humans have also considered as relevant, which we gather as part of the procedures 
of the control task in our experiment. 



\paragraph{\textbf{Metrics.}}

To investigate the overlap between the participants' manual highlights and the 
the automatic highlights identified by the study tool we use \textit{precision}, \textit{recall}~\cite{manning2010IR}, and \textit{pyramid precision}~\cite{Nenkova2004}.
We compute these metrics for each artifact of each task and report their average.


For this analysis, we consider any text marked by any participant as relevant
and we examine whether 
the text automatically identified represents text that multiple participants deemed relevant
 via \textit{pyramid precision}~\cite{Lotufo2012}.


Precision measures the fraction of the text automatically identified  that participants also deemed relevant (Equation~\ref{eq:precision-cp6}). 

\smallskip
\begin{small}
\begin{equation}
    Precision = \frac{
        \text{\textit{automatic highlights}~} \cap 
        \text{~\textit{manual highlights}}
    }{\text{\textit{automatic highlights}}}
\label{eq:precision-cp6}    
\end{equation}
\end{small}


Recall represents how many of all the manual highlights were identified by the semantic-based technique that we applied (Equation~\ref{eq:recall-cp6}). 



\smallskip
\begin{small}
\begin{equation}
    Recall = \frac{
        \text{\textit{automatic highlights}~} \cap 
        \text{~\textit{manual highlights}}
    }{\text{\textit{manual highlights}}}
\label{eq:recall-cp6}    
\end{equation}
\end{small}

\medskip


\textit{Pyramid precision} compares the text automatically identified to an optimal output, i.e., one where---for the same number of sentences---we identify sentences selected by the most number of participants (Equation~\ref{eq:pyramid-precision-cp6}). The more we identify text that more participants indicated as relevant, the higher pyramid precision is.



\smallskip
\begin{small}
\begin{equation}
    \triangle Precision = \frac{
        weight(\text{\textit{automatic highlights}~})
    }{weight(\text{\textit{optimal highlights}})}
\label{eq:pyramid-precision-cp6}    
\end{equation}
\end{small}
    


To illustrate these metrics, consider an artifact with 4 sentences $\{s_1, s_2, s_3, s_4\}$ that have been selected by $\{2, 0, 1, 1\}$ participants, respectively.
% For an output identifying two sentences for this artifact, an optimal solution would identify sentences $\{s_1, s_3\}$. 
Table~\ref{tbl:metrics-example} shows precision, pyramid precision, and recall metrics in a scenario where we output sentences $\{s_2, s_3\}$ as relevant.
    



\begin{table}[h!]
\caption{Example showing how we compute precision, recall and pyramid precision metrics}
\label{tbl:metrics-example}
\centering    
\begin{small}
\begin{threeparttable}
\rowcolors{2}{}{lightgray}
\begin{tabular}{lcc}

\textit{metric} & \textit{formula} & \textit{result} \\ 
\hline

\textit{precision} & \parbox[c][.9cm][c]{4cm}{\centering $\frac{\{s_2, s_3\}~ \cap ~\{s_1, s_3\}}{\{s_2, s_3\}} = \frac{1}{2}$} & 0.5 
\\


\textit{recall}  & \parbox[c][.9cm][c]{4cm}{\centering $\frac{\{s_2, s_3\}~ \cap ~\{s_1, s_3\}}{\{s_1, s_3,  s_4\}} = \frac{1}{3}$}  & 0.33 
\\

$\triangle$ \textit{precision}  & \parbox[c][.9cm][c]{4cm}{\centering $\frac{weight(s_2) + weight(s_3)}{weight(optimal)} = \frac{0 + 1}{3} $}  & 0.33 
\\

\end{tabular}
\end{threeparttable}
\end{small}
\end{table}



% \smallskip
% \begin{small}
% \begin{equation}
% \begin{split}
% \triangle  Precision(s_2, s_3) = ( 0 + 1) \div 3 =  0.33 \\
% \triangle  Precision(s_1, s_2) = ( 2 + 0) \div 3 =  0.66 \\
% \triangle  Precision(s_1, s_3) =  ( 2 + 1) \div 3 =  1.00 \\
% \label{eq:pyramid-precision-cp6} 
% \end{split}   
% \end{equation}
% \end{small}



Note that pyramid precision is usually equal or lower than precision. For example, $Precision(s_1, s_3) = Precision(s_3, s_4) = 1.0$, but 
results for pyramid precision differ, i.e., 
$\triangle Precision(s_1, s_3) = 1.0$ and $\triangle Precision(s_3, s_4) = 0.66$. This allows us to check if our tool 
identified the text deemed relevant by most of the participants who inspected an artifact.



\paragraph{\textbf{Data.}}

Participants who indicated what text was relevant to their assigned control task produced a total of 415 highlights with an average of 7 highlights (std $\pm 3$) per artifact inspected.
On average, this comprises 9\% of the content of each of the artifacts in our experiment. 


Some participants also selected code snippets as relevant to a task---a threat that we discuss in Section~\ref{cp6:threats}. 
Code snippets account for 30\% of the highlights produced, but we remove them from our analysis since our semantic-based approach 
operates on text only. For the textual highlights,
Krippendorf's alpha indicates good agreement of what text in an artifact participants deemed relevant (or not) ($\alpha = 0.68$)~\cite{Krippendorff1980, passonneau2006}.
We compare these manually produced highlights to the text automatically identified by our tool.



% \subsubsection{Results}


\paragraph{\textbf{Results.}}




Table~\ref{tbl:comparison-task-wise} summarizes the average of precision, pyramid precision, and recall metrics for each of the tasks in the experiment.
For the tasks in our study, we observe that precision scores range from 0.55 to 0.68 and that 
the lower pyramid precision scores, ranging from 0.55 to 0.57, suggest that our tool failed to identify some of the text that participants deemed the most relevant.


\begin{table}
\caption{Evaluation metrics per artifact type}
\label{tbl:comparison-overall}
\centering    
% \begin{scriptsize}
\begin{threeparttable}
\begin{tabular}{lcc}

  & \textbf{precision} & \textbf{recall}  \\ 
\hline

Distances & 0. & 0.
\\

NYTimes  & 0. & 0. 
\\

Titanic & 0. & 0.
\\

\hline


\textbf{overall} & 0. & 0.
\\

\hline

\end{tabular}
\end{threeparttable}
% \end{scriptsize}
\end{table}


These metrics are not surprising and they corroborate 
the correctness scores detailed in Figure~\ref{fig:correctness-by-task}. For example, 
in the \texttt{distances} task, participants who had tool assistance  task had less correct solutions than participants in the control group.
This was the task with the lowest precision, recall and pyramid precision values. 
In contrast, the task were participants assisted by our tool obtained the best correctness scores, namely \texttt{titanic}, is the one with the best precision, recall and pyramid precision values.








Table~\ref{tbl:comparison-artifact-type-wise} details evaluation metrics artifact-type wise. 
Stack Overflow posts and API documentation have the highest precision scores. For these types of artifacts, pyramid precision indicates that the 
text automatically identified on Stack Overflow was the text that several participants deemed relevant. 
Miscellaneous web pages were the artifact type with the lowest scores. 

\art{I'm stuck here}


We discuss this and other differences in our analysis of the usefulness of the text 
automatically identified (Section~\ref{cp6:usefulness}).



\begin{table}
\caption{Evaluation metrics per type of artifact}
\label{tbl:comparison-artifact-type-wise}
\centering    
% \begin{scriptsize}
\begin{threeparttable}
\rowcolors{2}{}{lightgray}
\begin{tabular}{lccc}




& \textbf{precision} & $\triangle$ \textbf{precision} & \textbf{recall} \\ 
\hline

API documentation & 0.65 & 0.55 & 0.59
\\

Stack Overflow posts  & 0.66 & 0.62 & 0.63
\\

Miscellaneous web pages & 0.53 & 0.53 & 0.54
\\


\hline
\end{tabular}
\end{threeparttable}
% \end{scriptsize}
\end{table}

%cp 5 - API -   p   .52     r   .55
%cp 6 - API -   p   .65     r   .59

%cp 5 - SO -    p   .58     r   .63
%cp 6 - SO -    p   .66     r   .63

%cp 5 - Misc -   p   .52     r  .56
%cp 6 - Misc -   p   .53     r  .54

%cp 5 - ALL -   p   .54     r   .58
%cp 6 - ALL -   p   .60     r   .56

% improvements in precision but slightly lower recall





% \art{Discuss how to account for variability in what the participants highlighted}




\subsection{Usefulness Analysis}
\label{cp6:usefulness}



Having compared the correctness of manual and tool-assisted tasks,
we turn to the question of 
whether the highlights shown by the tool were considered helpful. 
For that, we analyze participants' ratings and the feedback that they 
provided at the end of our experiment.


\subsubsection{Metrics}

To investigate the usefulness of the highlights shown by our tool, we asked participants to indicate on a 5-point Likert scale whether the highlights
of each artifact were helpful to correctly accomplish their assigned task (Figure~\ref{fig:experiment-rating}). We aggregate individual responses to measure how useful the tool was in assisting developers complete each task in our experiment, plotting responses using a diverging stacked bar chart~\cite{spence2001info-viz}.



\subsubsection{Data}


Participants produced a total of 197 ratings representing the usefulness of the highlights in the artifacts that they inspected.
On average, we collected 65 responses per task and 7 responses per artifact.  
These values do not match the exact number of participants and artifacts in our experiment since some participants did skip this part of 
the survey for their own reasons.


We also obtained written feedback from 19 out of the 24 participants, divided between feedback on the tasks (24 data points)
or on the experiment itself (15 data points). We use this feedback to quote scenarios
that support our observations.



\subsubsection{Results}


From all the ratings collected on the  usefulness of the text automatically identified and shown by our tool, 40\% of them agreed that the highlights were useful, 
25\% neither agreed nor disagreed on their usefulness, and 35\% indicated that they were \textit{not} useful.
Similarly to how we presented results on correctness, we analyze usefulness ratings on a per-task basis.  



Figure~\ref{fig:usefulness-by-task} shows participants' ratings aggregated for each task. 
Participants indicated that the highlights shown for the \texttt{titanic} task were the most useful. 
Ratings for this task support our observations on the correctness of the solutions produced. 
Notably, one participant indicated that highlights for this task assisted them in identifying essential function arguments needed to use the \texttt{pandas} group by function.
In contrast, participants indicated that the highlights for the \texttt{distances} task were in its majority not useful. 
For example, one participant pointed out that, in the \texttt{geopy} API documentation, there were no highlights in 
a section where they would have expected otherwise.




The mixed results for the \texttt{NYTimes} task might explain the lack of differences in the correctness scores observed for this task. 
Although anecdotal, one interesting feedback for this task was from a participant who described that 
their experience with the \texttt{BeautifulSoup} module influenced their negative ratings---``\textit{I have experience with BeautifulSoup and webscraping, and so my [negative] ratings of the usefulness of the highlights may have been influenced by that}''.



\afterpage{
\begin{figure}[h!]
    \centering
    \includegraphics[width=.85\textwidth]{cp6/usefulness_overall.pdf}
    \caption{Diverging stacked plot of the usefulness of the text automatically identified for each task}
    \label{fig:usefulness-by-task}
\end{figure}


\begin{figure}[h!]
    \centering
    \includegraphics[width=.85\textwidth]{cp6/usefulness_per_type.pdf}
    \caption{Diverging stacked plot of the usefulness of the text automatically identified for each type of artifact}
    \label{fig:usefulness-by-artifact-type}
\end{figure}
}


Surprisingly, when we look at the ratings per type of artifact (Figure~\ref{fig:usefulness-by-artifact-type}), API documents had the most useful highlights.
For this type of artifact, we observe that positive ratings originate from artifacts that follow a \textit{`how to'} format,
which is not conventional for API documents~\cite{robillard2011field, arya2020}. The highlights on Stack Overflow artifacts were also perceived as useful
and participants expressed familiarity with this type of artifact, where the highlights helped them determine parts of the page to ignore---``\textit{the highlights [on Stack Overflow] helped me quickly determine which parts of the page to ignore}''.



Miscellaneous web pages were the artifact where participants disagreed the most on the usefulness of the text automatically identified.
Due to their `\textit{tutorial}' format, these artifacts are lengthy and some participants expressed that they would like more direct explanations about how to use the modules of each task---``\textit{I have used these libraries before to some extent, so I don't need a narrative around purpose or procedure}''. 

















% These artifacts follow a `\textit{tutorial}' format might be more appropriate for novices 



% ``\textit{I have used these libraries before to some extent so I don't need a narrative around purpose or procedure}''



% 




% ``\textit{When doing the second task, with the highlighted references, I was able to move much quicker. I quickly glanced at each resource, reading just the highlights to determine how valuable that resource was. The highlights allowed me to focus on the most relevant resources, gathering the necessary information to complete the task. }''









% \medskip
% \begin{small}
% \begin{bluequote}
%     ``\textit{Some of the highlights were useful when there were clear special meanings to particular function arguments like as\_index=True}''
% \end{bluequote}
% \end{small}




% \medskip
% \begin{small}
% \begin{bluequote}
%     ``\textit{I also went to the `documentation $>$ distance', but surprisingly nothing was highlighted}''
% \end{bluequote}
% \end{small}


\subsection{Threats to Validity}
\label{cp6:threats}




Our experiment compares solutions submitted by participants who attempted each task with and without tool support. 
This represents a between groups design~\cite{Lazar2017} and we discuss threats inherent to it. 



Since we compare results from different participants, our analysis might be subject to substantial 
impact from individual differences~\cite{Lazar2017}. 
For example, participants who performed a task with tool support may have been more experienced than participants 
who did the same task without tool support what affects correctness scores.
As another example,  participants' skill and background 
influences the text that they indicate as relevant in the control task as well 
what text they perceive as useful in the tool-assisted task. 
We minimized these threats by recruiting participants of varied background and randomly
assigning tasks to each participant.



The tasks in our experiment impact generalizability. 
Although we opted for simple tasks, we ensured they 
modules used in our tasks were representative. 
For example, we found open-source systems\footnote{\url{https://github.com/ArchiveBox/ArchiveBox/issues/18}} using \texttt{BeautifulSoup} 
with function calls similar to the ones needed to complete the \texttt{NYTimes} task.
Nonetheless, there are clear differences in the artifacts one can gather 
based on the domain or programming language of a task~\cite{baltes2020}.
Hence, we consider other domains and a wider range of task 
and artifacts for future work. 



The selection of tasks also affects our conclusions. We opted for Python programming tasks that 
required writing code, which we use to assess correctness. 
As observed by other researchers~\cite{satterfield2020, meyer2020}, developers
work on many different tasks, some of which focus on code~\cite{Meyer2017}
while others on information seeking~\cite{gonccalves2011}, e.g., finding duplicated bug reports or researching visualization libraries to identify the most suitable one~\cite{satterfield2020}.
Had we decided to use information-seeking tasks, participants could have produced a different set of highlights,
perhaps selecting fewer code snippets. 
Given that 
our experiment was completely remote, instructing participants on how to perform information-seeking 
tasks would have been more difficult. Furthermore, objectively judging their correctness 
would also be more strenuous, which would lead to a different experiment with challenges and risks of its own.




The fact that we consider the text marked by any participant as relevant 
also affects our conclusions. 
We refrain from excluding text selected by a few participants from our analysis 
for reasons similar to the ones in our characterization of task-relevant information (Chapter~\ref{ch:characterizing}). That is, the text marked by these participants might still contain valuable information. 
We minimize this threat by reporting both precision and pyramid precision, where we observe that 
our approach failed to detect the text that multiple participants deemed relevant
for some tasks or types of artifacts. 



Concerning the text automatically identified by our tool, we gather usefulness at the artifact level.
Suppose we had gathered usefulness at the sentence level. In that case, we could have used this information 
to further refine our analysis, for example, reporting precision and recall 
at different usefulness levels or computing accuracy based on the participants' input, as done by Xu et al~\cite{Xu2017}. 
However, asking participants to provide feedback at the sentence level would have considerably increased the time we estimated that the experiment would take,
which would impact recruitment. We weighed the benefits and drawbacks of a fine-grained or more coarse-grained 
analysis, and we opted for the latter so that this would not be a barrier to people deciding on 
whether to participate in our experiment.


Chapter~\ref{ch:discussion} futher discusses limitations or improvements to the semantic-based techniques and to our tool.
\clearpage

\section{Summary}
\label{cp6:summary}



\red{TODO}




% \art{I'm trying to set results in a manner similar to how we did in the ICPC paper---present an overview of what we evaluate in the section, describe the method of evaluation and then, the results.}


% \clearpage

\section{Summary}
\label{cp6:summary}



\red{TODO}

