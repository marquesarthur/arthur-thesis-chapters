\setcounter{chapter}{5}
\setcounter{rq}{1}


\chapter{Evaluating Text Automatically Identified}
\label{ch:assisting}


% In early chapters of this thesis, we have presented a research method to create  a corpus with tasks and annotated data representing text---in multiple types of artifacts---that human annotators deemed relevant for solving Android development tasks. We 
% used this corpus to design and evaluate techniques that use semantics to identify part of the text that human 
% annotators deemed relevant for the tasks and artifacts in such corpa.


In this chapter, we build upon the research outlined  early on in this thesis to
present a \textit{controlled experiment}~\cite{Lazar2017-cp2} that focuses on how  a
software developer working on a task can benefit from a tool that automatically identifies text 
relevant to that task.



On the one hand, 
the experiment (Figure~\ref{fig:tool-experiment-procedures})
adapts corpus creation procedures from Chapter~\ref{ch:android-corpus}
to show how 16 participants with software development background identify 
text---in multiple types of artifacts---relevant for solving three Python programming tasks while they performed these tasks. % \textit{without} any tool support 
On the other hand, 
it investigates if participants can more correctly complete these tasks 
when using a tool 
that  automatically highlights 
task-relevant text using one of the semantic-based techniques 
from Chapter~\ref{ch:identifying}. 


In summary, this allow us to investigate whether:

\medskip
\begin{bluequote}
    \textit{text potentially relevant to a task and automatically identified  by a tool using a semantic-based technique assist a software developer working on that task.}
\end{bluequote}



\begin{figure}
    \centering
    \includegraphics[width=1.05\textwidth]{cp6/tool-experiment-pipeline.pdf}
    \caption{Summary of experimental procedures}
    \label{fig:tool-experiment-procedures}
\end{figure}




To show how this experiment allows such a investigation, we begin by detailing
experimental procedures (Sections~\ref{cp6:design} to~\ref{cp6:procedures}).
Then, we detail how we analyze the data obtained from the experiment by:

\newpage

\begin{itemize}
    \item assessing the correctness of the tasks performed \textit{with} or \textit{without} tool support (Section~\ref{cp6:correctness});
    \item comparing  the text that participants deemed relevant to the text automatically identified
    and shown in the tool-assisted task (Section~\ref{cp6:comparison}), and;
    \item discussing the usefulness of the automatically identified text according to the feedback provided by the participants (Section~\ref{cp6:usefulness}).
\end{itemize}
 
We conclude this chapter presenting threats to the validity of our experiment (Section~\ref{cp6:threats}) and 
 summarizing our key findings (Section~\ref{cp6:summary}).


 
\section{Design}
\label{cp6:design}


% To understand how a tool embedding a semantic-based technique might impact a developer's work,
% we are guided by three main factors.


To design our experiment, we consider three main aspects that assist us examine how a tool that embeds a semantic-based technique might impact a developer's work.



\subsubsection{Requirements}



To be helpful, a tool must direct a developer's attention to \textit{prominent} text that assist them complete a task~\cite{Robillard2015}. If we can gather data about the text in an artifact considered by humans as relevant to the task at hand, we should be able to assess whether the text automatically identified by the tool is similar to the text manually identified. 


In case the text is not similar,
there is a chance that the tool directed a developer to the wrong information, or that 
the tool actually found complementary information that humans missed, which might impact the \textit{correctness} of a task. Therefore, our design must be able to compare the correctness of solutions submitted by participants who perform a task using our envisioned tool against that of developers who attempted the same tasks without tool support, i.e., a control group.


Regardless of how correct participants' solutions are, there is a chance that the text shown by the tool is not \textit{useful}---either because it is not relevant for the task at hand or because it is unsurprising, i.e., the information is ``common-knowledge'' to most developers~\cite{cwalina2008, Robillard2015}. Hence, a successful evaluation would 
ideally
gather qualitative data on the usefulness of the text identified by the tool. 



\subsubsection{Experiment}




Figure~\ref{fig:tool-experiment-procedures} presents an experiment that allows us to address the aforementioned requirements. In this experiment, 24 participants with software development background each attempted a
\textit{manual task} and \textit{tool-assisted task} randomly drawn from a list of well-known Python programming tasks.
Alongside the solution submitted for each task, we use the manual task to collect what text participants deemed relevant to the task performed.
In turn, in the tool-assisted task, we also gathered input on the usefulness of the highlights show. 
This design allow us to:




\begin{itemize}
    \item assess the correctness of the tasks performed \textit{with} or \textit{without} tool support;
    \item compare  the text that participants deemed relevant to the text automatically identified
    and shown in the tool-assisted task, and;
    \item discuss the usefulness of the automatically identified text according to the feedback provided by the participants.
\end{itemize}
 





\begin{figure}
\centering
\includegraphics[width=1.05\textwidth]{cp6/tool-experiment-pipeline.pdf}
\caption{Summary of experimental procedures}
\label{fig:tool-experiment-procedures}
\end{figure}





We detail experimental procedures and results in the reminder of this chapter.





 
% On their own time and computer\footnote{Literature define this as an \textit{offline} experiment~\cite{wohlin2012, DeLucia2012}}, we expect that a participant in out experiment attempts two tasks. 
% In a first task, i.e., \textit{manual task}, a participant indicates 
% what text they deem relevant to the task at hand while 
% they consult a 
% In a second task, i.e., \textit{tool-assisted task},
% a participant should write their solution consulting a similar list of curated artifacts, but
% assisted by a tool that highlights the text identified as potentially relevant to their task; for this task, we also gathered feedback on the usefulness of the highlights shown.

 

% With this experiment we evaluate different factors that can help us answer whether 
% a tool embedding a semantic-based technique helps developers complete a software task. 
 
\section{Participants}
\label{cp6:participants}

\art{review this section and include demographics}

We advertised our study to both developers in our professional network and to computer science students at the several universities. 
This provides for breadth of experience where both novice and expert developers attempt the tasks in our experiment. 


With regards to computer science students, we sent the study only to third and fourth-year students to ensure that they had the necessary background required to perform the study's taks.
At this point in the curriculum, students should be familiar with Python and they should be able to come up with a solution 
for a software task when provided with artifacts containing information for that task.


To compensate participants for their time, we offered them the opportunity to enter a raffle for one of two iPads 64 GB.



 
\section{Tasks}
\label{cp6:tasks}



\subsection{Tasks}
\label{sec:experiment-tasks}



We expect the experiment to be executed offline, i.e., the experiment has self-contained instructions that allow a participant to perform their assigned tasks without any supervision.
\gm{Why not just say the participants
completed the tasks on their own time?}
This requires tasks that are easy to understand and perform in a single experimental session. At the same time, our main hypothesis requires tasks for which a developer will likely benefit from the use of additional information to complete.


These criteria lead us to select Python w3resource tasks\footnote{\url{https://www.w3resource.com/python-exercises/}}
that required usage of at least one module external to the Python core modules~\cite{thiselton2019}.
By using an external module, we aim to reduce the likelihood that a participant 
can provide a solution for a task without consulting any of the artifacts accompanying that task. 






Table~\ref{tbl:python-tasks-modules} details the tasks that we have selected. 
We chose these tasks due to how they focus on modules widely adopted both by open source systems and by private companies.
For example, the \texttt{NYTimes} task involves using a HTML parser, namely \texttt{BeautifulSoup},
which Reddit---a social news aggregator with approximately 430 million monthly users---uses 
to parse urls and identify images shown on its posts' headlines~\cite{bs4-reddit}. 
Similarly, the \texttt{Titanic} task encompasses using \texttt{Pandas}, a data analysis and manipulation module
used in many open source data science projects~\cite{ma2017, shrestha2020}.





\begin{table}
\centering
\caption{Python tasks}
\begin{scriptsize}
\rowcolors{2}{}{lightgray}
\begin{tabular}{ll}
\hline
\textbf{Task} & \textbf{Description}                                                                                         \\
\hline
\hline
Distances     &
\parbox[l][1.2cm][c]{11cm}{Given a string representing a rendezvou point and a list of suggested picnic addresses
    you must write an algorithm using the \texttt{geopy} module to find the  picnic address closest to the rendezvou point.} \\
%
NYTimes       &
\parbox[l][1cm][c]{11cm}{Given a string representing the url for NY Times Today's,
    write an algorithm using the \texttt{BeautifulSoup} and \texttt{requests} modules to srape all the headlines of that page.}
\\
%
Titanic       &
\parbox[l][1cm][c]{11cm}{Given a string representing a url for the titanic dataset,
    you must write an algorithm using the \texttt{pandas} and \texttt{seaborn} modules to create a barchart of the data.}    \\
\hline
\end{tabular}
\end{scriptsize}
\smallskip
\label{tbl:python-tasks-modules}
\end{table}









\subsection{Artifacts}
\label{sec:experiment-artifacts}


Each task requires a set of artifacts that a participant could peruse for information that could assist them complete the task.
We select artifacts for a task following procedures similar to the ones we used to create the \acs{DS-android} dataset (Chapter~\ref{ch:android-corpus}). 
For each of the tasks in Table~\ref{tbl:python-tasks-modules}, we use the Google search engine to obtain up to ten artifacts that likely contain relevant
information for that task. 





\subsection{Colab}
\label{cp6:environment}


To ensure that participants had the same conditions to perform each task
and also to minimize setup instructions, we used Google Colab\footnote{\url{https://colab.research.google.com/}} as our coding environment. 
% By using Colab, we also expect setup instructions to be minimal 
% so that we allow a participant to focus on the task and experimental procedures.



Figure~\ref{fig:nytimes-task-colab} shows an example of our online environment for the \texttt{NYTimes} task.
At the left-hand side, participants had the task description and examples of input and output scenarios as well as a list of resources associated with that task. 
By clicking on the coding environment link, participants were redirected to Colab (right-hand side),
where they had a code editor with amenities commonly found in modern IDEs, such as code completion and syntax highlighting. 



Through Colab, a participant could compile their solution and test it against the examples shown alongside the task description.
Upon testing, the system would display full details about the test cases, e.g., the test's input, which assertion failed, and why. 




\clearpage

\begin{landscape}
\begin{figure}
    \centering
    \includegraphics[width=1.5\textwidth]{cp6/task-colab.pdf}
    \caption{NYTimes task and Colab environment.}
    \label{fig:nytimes-task-colab}
\end{figure}
\end{landscape}

 \section{Tool}
\label{cp6:tool}


To assist a participant locate text containing information potentially helpful to their tool-supported task, we have developed a proof-of-concept web-browser plug-in 
that highlights text that a semantic-based technique 
identifies as relevant in the artifacts inspected by a participant.



The tool---\acs{beskar} (BERT task-relevant text identifier)---applies 
one of the most promising semantic-based techniques that we have previously explored and evaluated,
i.e., the BERT technique (Chapter~\ref{ch:identifying}),
to identify 10 sentences that are likely relevant to an input task. 
As Figure~\ref{fig:tool-output} shows,
\acs{beskar} highlights the text it automatically identified as relevant 
to a task in a web page under inspection
so that a developer can more easily locate such text.


As a proof-of-concept, we do not focus on the tool's performance.
Therefore, we pre-cached the output 
for the text identified by \acs{beskar} in each task and artifact 
in our experiment. 




\begin{figure}
    \centering
    \includegraphics[width=0.85\textwidth]{cp6/tool-highlights.png}
    \caption{Example of a highlight (in yellow) automatically identified by \acs{beskar}
    in a web tutorial pertinent to the NYTimes task}
    \label{fig:tool-output}
\end{figure}


 
\section{Procedures}
\label{cp6:procedures}



In this section, we detail the instructions provided to the participants so that 
they could perform our experiment. 
First, we present an overview of the whole experiment and then, we detail
instructions specific to the manual and tool-assisted tasks, respectively.


We advertised the experiment via mailing lists. The email disclosed the purpose of the experiment, eligibility criteria, an estimation of the time that one would take to complete the experiment as well as a link 
to the web survey containing the experiment's consent form and tasks. 


Once a participant consented to participate, the survey gathered demographics (Figure~\ref{fig:experiment-demographics}), which we use to filter participants who did not meet our eligibility criteria. That is, we planned to filter data from a 1st or 2nd year student or from a participant with no experience in Object-Oriented programming languages or that did not consult API documentation when performing a programming task. However, no participants were excluded based on demographics.


\begin{figure}
\begin{mdframed}[backgroundcolor=gray!15] 
\begin{scriptsize}

\noindent To witch gender do you identify? 

\medskip

\noindent If you are a student, in which year of the program are you at?  \smallskip

\quad $\square$~$1st$  
\quad $\square$~$2nd$  
\quad $\square$~$3rd$  
\quad $\square$~$4th$  
\quad $\square$~$5th+$ year 
\quad $\square$~\textit{graduate student} 

\medskip

\noindent For how many years have you been developing software?  

\medskip

\noindent For how many years have you been developing software \underline{professionaly}? 

\medskip

\noindent How many years of experience do you have in Object-Oriented programming languages?~\footnote{\scriptsize closed or open intervals notation} \smallskip

\quad $\square$~\textit{no experience} 
\quad $\square$ $(\infty, 1)$
\quad $\square$ $[1, 3)$
\quad $\square$ $[3, 5)$
\quad $\square$ $[5, 10)$
\quad $\square$ $[10, \infty)$

\medskip

\noindent With which frequency do you consult API documentation when performing a programming task?  \smallskip

\quad \textit{(never)} ~$1$ - $2$ - $3$ - $4$ - $5$ ~ ~\textit{(always)} 

\end{scriptsize}
\end{mdframed}
\caption{Background questions asked to a participant}
\label{fig:experiment-demographics}
\end{figure}

    



Next, the online survey gave participants further instructions 
about how to perform each task and it requested them to install \acs{beskar}.
Setup was followed by a short practice task---separate from the experimental tasks---that allowed participants to familiarize themselves with the content of a task, the tool, and the coding environment that we used (Colab). 


Once a participant completed the practice task, the survey randomly assigned to them a \textit{manual} task, which was followed by a randomly assigned \textit{tool-assisted} tasks---different from the manual task. While tasks were randomly assigned, we made sure that an even number of participants attempted a task with and without tool support.
For each task, including the practice tasks, the survey provided to the participants a link 
to the task description (Figure~\ref{fig:nytimes-task-github}) and asked them to submit a solution for the task, i.e., written Python code. 


Once a participant submitted their solutions, the survey
asked them about any additional feeback that they wished to share and 
offered them the opportunity to enter a raffle for one of two iPads 64 GB 
to compensate them for their time, what concluded the experiment.



\subsection{Manual Task}
\label{cp6:procedures-manual}



In the \textit{manual} task, we instrumented \acs{beskar} to allow participants to highlight text that they deemed useful for the task at hand. In this task, 
the survey asked participants to highlight sentences that they deemed useful and that provided information that assisted task completion---instructions similar to the ones used for the creation of the \acs{DS-android} corpus (Section~\ref{cp4:corpus-relevant-text}).



Figure~\ref{fig:artifact-pre-highlight}
gives insight into how participants highlighted sentences that they deemed relevant. 
Whenever a participant inspected one of the artifacts available for their task, 
the survey instructed them to click on the \texttt{highlight} button in the web browser plug-in context menu.  
\acs{beskar} would then instrumented the HTML of the page identifying individual sentences. 
A participant could hover over identified sentences and select them as relevant by clicking on the hovered text.
As an example, Figure~\ref{fig:artifact-pre-highlight} shows a sentence discussing the \texttt{find\_all} method, which 
one of the participants in our experiment deemed 
relevant to the \texttt{NYTimes} task.



\begin{figure}
    \centering
    \includegraphics[width=0.95\textwidth]{cp6/manual-task.pdf}
    \caption{\texttt{BeautifulSoup} web tutorial showing the \acs{beskar} context menu (top-right corner) and a hovered over sentence}
    \label{fig:artifact-pre-highlight}
\end{figure}




\subsection{Tool-assisted Task}
\label{cp6:procedures-tool-assisted}


In the tool-assisted task, \acs{beskar} would automatically highlight text that 
its underlying semantic-based approach identified as relevant to the participant's task.
This is similar to the highlights shown in Figure~\ref{fig:artifact-pre-highlight}, but without the need for any actions by a participant.


For this task, when a participant submitted their solution, the survey asked them to 
rate---in a 5 points Likert scale~\cite{likert1932technique}---how helpful were the highlights shown by the tool.
As Figure~\ref{fig:experiment-rating} shows, participants rated highlights on a per artifact basis. When rating an artifact, the survey allowed participants to revisit the artifact if so they wished. This would open the artifact web page in a new tab so that a participant
could review the the highlights in that artifact.

Although we evaluate the usefulness of all the highlights for a task
aggregating individual responses, we use the ratings per artifact to explore if the highlights shown in a certain type or artifact, e.g., API documents or tutorials, are more or less useful.


\begin{figure}
\begin{mdframed}[backgroundcolor=gray!15] 
\begin{scriptsize}

\noindent \textbf{1.} Indicate whether you agree with the following statement:

\medskip

\quad \textit{The highlights in \textcolor{steelblue}{``How to extract HTTP response body from a Python requests call''} were helpful to} 

\quad \textit{correctly accomplish the task in question.}  \smallskip

\smallskip

\quad \quad \textit{(Strongly disagree)} ~$1$ - $2$ - $3$ - $4$ - $5$ ~\textit{(Strongly agree)} 


\bigskip


\noindent \textbf{2.} Indicate whether you agree with the following statement:

\medskip

\quad \textit{The highlights in \textcolor{steelblue}{``BeautifulSoup tutorial: Scraping web pages with Python''} were helpful to} 

\quad \textit{correctly accomplish the task in question.}  \smallskip

\smallskip

\quad \quad \textit{(Strongly disagree)} ~$1$ - $2$ - $3$ - $4$ - $5$ ~\textit{(Strongly agree)} 

\centering 

...

\end{scriptsize}
\end{mdframed}
\caption{Questions asking a participant to rate the usefulness of the highlights shown in two artifacts; by clicking on the name of an artifact, a participant could revisit the highlights of that artifact}
\label{fig:experiment-rating}
\end{figure}

    



\subsection{Summary of procedures}


Throughout the past sections, we have described experimental procedures 
where participants attempted a \textit{manual} and a \textit{tool-assisted}.
These procedures allowed us to gather:


\begin{enumerate}
\item a participant's submitted solution (written Python code) for each task;
\item text that participants deemed relevant for completing a manual task;
\item the usefulness of the highlights shown in a tool-assisted task; and
\item any additional feedback (written text) that a participant wished to provide.
\end{enumerate}


We use this data to investigate whether 
a tool embedding a semantic-based technique helps developers complete a software task. 


\clearpage




\art{I am not sure about the titles for the sections below}

 
\section{Tasks Correctness}
\label{cp6:correctness}


 
The correctness of a submitted solution is measured by the number of passing test cases
when running that solution against a set of 10 test cases. 
A solution with compile errors has a correctness score of zero.


\smallskip
\begin{small}


\begin{equation}
    Correctness = \frac{ \text{\textit{\# of passing test cases}}}{\text{\textit{\#  of test cases}}}
\end{equation}
\end{small}
 
% \section{Does \acs{beskar} identify the right text?}
\section{Result: Task-relevant Text Identified}
\label{cp6:comparison}




\subsection{Method}



\subsection{Results}


We compare the participants' manual highlights  against the ones automatically identified by our semantic-based technique. 
For that, we use \textit{precision} and \textit{recall} metrics. 




Precision measures the fraction of the automatically identified text that was  considered relevant
by the participants.

\smallskip
\begin{small}


\begin{equation}
    Precision = \frac{
        \text{\textit{automatic highlights}~} \cap 
        \text{~\textit{manual highlights}}
    }{\text{\textit{automatic highlights}}}
\end{equation}
\end{small}


Recall represents how many of all the manual highlights were identified by the semantic-based technique that we applied.

\smallskip
\begin{small}

\begin{equation}
    Recall = \frac{
        \text{\textit{automatic highlights}~} \cap 
        \text{~\textit{manual highlights}}
    }{\text{\textit{manual highlights}}}
\end{equation}
\end{small}

\medskip
\art{Argue which metric matters the most here}


\art{Discuss how to account for variability in what the participants highlighted}


 
\section{Tool Usefulness}
\label{cp6:usefulness}


We use a diverging stacked bar chart~\cite{Heiberger2014} to analyze the  Likert scale
responses on the usefulness of the text automatically identified.


Usefulness indicates the percentage of responses agreeing or disagreeing with whether sentences
automatically highlighted assisted a participant in completing a task.


\smallskip
\begin{small}

\begin{equation}
Usefulness = \frac{
    \text{\textit{\# of responses at i}}
}{
    \text{\textit{total \# of responses}}
}
\end{equation}
        

\begin{equation*}
i \in \{ 
    \text{\textit{
        (strongly) agree, neither agree or disagree, (strongly) disagree
    }}  
\}
\end{equation*}
\end{small}





\subsubsection{Written Feedback}

 
We use qualitative methods to analyze participants' responses to the open-ended questions. 

\art{Think this through...}

 
\subsection{Threats to Validity}
\label{cp6:threats}




Our experiment compares solutions submitted by participants who attempted each task with and without tool support. 
This represents a between groups design~\cite{Lazar2017} and we discuss threats inherent to it. 



Since we compare results from different participants, our analysis might be subject to substantial 
impact from individual differences~\cite{Lazar2017}. 
For example, participants who performed a task with tool support may have been more experienced than participants 
who did the same task without tool support what affects correctness scores.
As another example,  participants' skill and background 
influences the text that they indicate as relevant in the control task as well 
what text they perceive as useful in the tool-assisted task. 
We minimized these threats by recruiting participants of varied background and randomly
assigning tasks to each participant.



The tasks in our experiment impact generalizability. 
Although we opted for simple tasks, we ensured they 
modules used in our tasks were representative. 
For example, we found open-source systems\footnote{\url{https://github.com/ArchiveBox/ArchiveBox/issues/18}} using \texttt{BeautifulSoup} 
with function calls similar to the ones needed to complete the \texttt{NYTimes} task.
Nonetheless, there are clear differences in the artifacts one can gather 
based on the domain or programming language of a task~\cite{baltes2020}.
Hence, we consider other domains and a wider range of task 
and artifacts for future work. 



The selection of tasks also affects our conclusions. We opted for Python programming tasks that 
required writing code, which we use to assess correctness. 
As observed by other researchers~\cite{satterfield2020, meyer2020}, developers
work on many different tasks, some of which focus on code~\cite{Meyer2017}
while others on information seeking~\cite{gonccalves2011}, e.g., finding duplicated bug reports or researching visualization libraries to identify the most suitable one~\cite{satterfield2020}.
Had we decided to use information-seeking tasks, participants could have produced a different set of highlights,
perhaps selecting fewer code snippets. 
Given that 
our experiment was completely remote, instructing participants on how to perform information-seeking 
tasks would have been more difficult. Furthermore, objectively judging their correctness 
would also be more strenuous, which would lead to a different experiment with challenges and risks of its own.




The fact that we consider the text marked by any participant as relevant 
also affects our conclusions. 
We refrain from excluding text selected by a few participants from our analysis 
for reasons similar to the ones in our characterization of task-relevant information (Chapter~\ref{ch:characterizing}). That is, the text marked by these participants might still contain valuable information. 
We minimize this threat by reporting both precision and pyramid precision, where we observe that 
our approach failed to detect the text that multiple participants deemed relevant
for some tasks or types of artifacts. 



Concerning the text automatically identified by our tool, we gather usefulness at the artifact level.
Suppose we had gathered usefulness at the sentence level. In that case, we could have used this information 
to further refine our analysis, for example, reporting precision and recall 
at different usefulness levels or computing accuracy based on the participants' input, as done by Xu et al~\cite{Xu2017}. 
However, asking participants to provide feedback at the sentence level would have considerably increased the time we estimated that the experiment would take,
which would impact recruitment. We weighed the benefits and drawbacks of a fine-grained or more coarse-grained 
analysis, and we opted for the latter so that this would not be a barrier to people deciding on 
whether to participate in our experiment.


Chapter~\ref{ch:discussion} futher discusses limitations or improvements to the semantic-based techniques and to our tool.
 \clearpage

\section{Summary}
\label{cp6:summary}



\red{TODO}
 



% to the tool identified text
% as well as from the analysis of the participants' perception on the tool's usefulness.


% i.e., did they find the text automatically identified by our tool useful to their working task?



% This experiment allows us to investigate whether \textit{the text automatically identified by the tool is similar to text that developers would deem relevant}.





















% A participant first attempted one randomly assigned tasks where, while performing the task, they indicated what text---in a set of artifacts that we made available to the participant---they deemed relevant to the task at hand. Then, in another randomly assigned task, we investigate whether the text identified
% automatically by our tool changes how a participant completed a task. 





% attempted these tasks. 
% \gm{A bit more description of
% the setup might be needed.}
% We publicly share experimental materials and results to help future research in the field~\cite{experiment_material}.


% We describe experimental procedures (Section~\ref{cp6:procedures}) before
% presenting the results of our experiment (Section~\ref{cp6:results}). Section~\ref{cp6:summary} summarizes our key findings.







% Early in this thesis, we have shown research methods for 
% bringing together tasks from an existing
% well-known domain, Android development, with relevant text
% for solving that task from multiple artifact types.






% This chapter presents a \textit{controlled experiment}
% that builds upon the research methods presented earlier in this thesis.
% We detail three tasks related to well-known Python modules
% and we describe 
% how \red{15} developers with varying background 
% identify relevant text for solving these task
% in multiple artifact types.












% complete software tasks when assisted, or not, by a tool that 
% embeds one of our semantic-based techniques. 
% \gm{'evaluate how' is a bit unspecified
% as 'how' isn't usually evaluated -
% maybe restate as 'investigate how tools that ... impact how ...'?}




% assists a software developer while working on a task.
% We built upon the research procedures used to create the \acs{DS-android} corpus and we report how d identify relevant text in artifacts pertinent to three tasks from to well-known Python modules. This experiment also allows us to 


% Through 




% To what extent the text automatically identified by a tool that embeds a semantic-based technique assists developers 
% complete a task




% \art{find how to connect this}
% This experiment provides complementary data to our early evaluation 
% on whether automatic approaches can identify text that humans deem useful to a software task. 
% To this end, we use three software tasks related to w and
% report results from \red{15} participants with software development background 


