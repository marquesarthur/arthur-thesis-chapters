\setcounter{chapter}{5}
\setcounter{rq}{1}


\chapter{Evaluating Text Automatically Identified}
\label{ch:assisting}



% Evaluating text that automatic techniques identify as relevant to a software task means assessing 
% whether this text assists software developers complete the task at hand. 
% Text that automatic techniques identify as relevant to a software task is intended 
% to assist software developers complete the said task.



% While Chapter~\ref{ch:android-corpus} describes a corpus with text that human annotators considered 
%  relevant to a number of software tasks and Chapter~\ref{ch:identifying} explores 
% automatic techniques able to identify such text, 
% they do not address whether text automatically identified 
% assists software developers complete their software task.
% To fill this gap, this chapter evaluates one of the semantic techniques introduced early in this thesis 
% through a user study. 

% \art{This is too much success or fail... not the message I want to carry forward}


% In this study, \red{n} software developers attempt two programming tasks associated with 
%  well-known Python APIs. Developers 



% With this study, we intend to provide data strengthening or discounting 
% the usefulness of text 


% through the lens of software developers

% how text identified by a semantic based technique assists 



% on the strengths and short-comings 
% on how a semantic based technique helps developers on identifying information useful to their task 
% and how such text compare against text that humans deem relevant to the task at hand. 




% At early chapters, we have described a corpus with text that human annotators considered 
% relevant to a number of software tasks (Chapter~\ref{ch:android-corpus}). 
% We have also shown that semantic based techniques are able to automatically identify such text (Chapter~\ref{ch:identifying}).
% We now turn our attention towards the question of whether a semantic based technique assists developers complete their software task.



% To answer this question, this chapter describes a user study where \red{n} software developers perform programming tasks associated with 
%  well-known Python APIs. In a first task, we asked a developer to indicate text that contained information useful to complete her task. In a second task, a tool implementing one of our semantic based techniques 
 
 
%   and the 
%  developer's perception on the usefulness of text automatically identified by one of our semantic based techniques in a second task. We use this data to quantitatively and quantitatively evaluate the role of automatic techniques 
%  in assisting software developers identify information useful to their software tasks. 









% To provide evidence supporting or refuting this proposition, we now describe how 

% To aid developers in identifying the fraction of the text relevant to a software task

% explored the design space of a set of  techniques,



\art{I'm trying to phrase it in a more exploratory and complimentary way to the previous study rather than an affirmative study that just says ``yes it does help'' or ``no it doens't''}



In a previous chapter, we have shown that semantic-based techniques automatically identify part of the text relevant to a task in an artifact associated with that task.
Such evaluation provides empirical data on the role of semantics in the identification of task-relevant information.
Nonetheless, it does not address whether the text identified as relevant indeed assists software developers in completing a software task.



In this chapter, we seek to provide further data supporting, or refuting, the role of semantics 
in identifying text that humans deem relevant to a software task through a \textit{user study}. 
We present three software tasks related to well-known Python modules and we show 
results from \red{20} software developers who performed two tasks each. 
This study provides quantitative data comparing the text that developers manually indicated as relevant 
while they performed a first randomly assigned task to
the text automatically identified by the best semantic-based technique for 
that same task. It also details qualitative data by reporting the developers' perception on 
the usefulness of text automatically identified in each of the artifacts associated with aq second randomly assigned task. 


We start by describing the experimental procedures of our user study (Section~\ref{cp6:procedures}) and then by presenting the results of this study  (Section~\ref{cp6:results}). Section~\ref{cp6:summary} summarizes our key findings.


% manually identified as relevant to one randomly assigned tasks while they performed the task in question. We also gather a developer's perception on the usefulness of text that the  identifies in a second randomly assigned task.








% Text that automatic techniques identify as relevant to a software task is intended 
% to assist software developers complete the said task


%  in a human annotated corpus, it is possible that the text that annotators indicate as relevant and the text that developers performing the same task deem relevant does not match. 


% because developers are unable to introspect reliably on their work practices.


%  in a corpus where human annotators indicated the portion of the text that they deemed relevant to a number of software tasks


% we scale...




% it is possible that what people
% say they do in response to survey questions bears no relationship to what they actually do,
% because they are
 

% when assisted or not by one

% semantic techniques




% we have shown how the text identified by six semantic techniques
% compare against text that humans deem relevant to a software task (Chapter~\ref{ch:identifying})


% In Chapter~\ref{ch:android-corpus} we present the creation of a corpus 
% with software tasks and text, originating from different kinds of artifacts,
% containing information potentially useful to humans performing the task.
% We use this corpus in the design and quantitative assessment of possible techniques able to automatically identify such text.













% Assessing whether text identified by an automatic technique assists developers 
% complete software tasks requires 



% A developer can more effectively complete a software development task when automatically provided with text relevant to their task extracted from pertinent natural language artifacts and represented in a concise and thorough way.  

% \textbf{Thesis statement:} \textit{
%     Information relevant to a software development task can be automatically extracted from natural 
%     language software artifacts by interpreting the meaning, or semantics, of the text in these artifacts.
%     This task-relevant text assists a developer in effectively completing their software development task.
% }



% Reviewing \textbf{thesis statement} to show chapter motivation.

% \medskip
% % \textbf{Thesis statement:} 
% \textit{
%     ``A developer can more effectively complete a software development task when provided
%     with text that automatic techniques determine as relevant to the developer's task 
%     according to the meaning, or semantics, of the text.''    
% }

% \medskip


% Chapter needs to show that a semantic-based technique from previous chapter assists a 
% developer effectively complete their task. Must define \textit{effectively}: 

% \begin{itemize}
%     \item correct: the solution for the task meets some correctness criteria, e.g., it meets expectations from a unit test
%     \item complete: the solution meets some completeness criteria, e.g., it passes `\textit{all}' expected unit tests
% \end{itemize}


% This leads to a `within subjects' or `between subjects' evaluation where we assess whether 
% a developer's solution is more correct or complete
%  when assisted by a tool that embeds one semantic-based technique.

% \begin{itemize}
% \item independent variable: task done with or without tool
% \item dependent variable(s): completeness and correctness of the task
% \end{itemize}


\clearpage


% \includepdf[pages=-]{fig/cp6/ideas.pdf}


% \clearpage

\subsubsection{Brainstorm - A}

% \begin{small}
\begin{description}[leftmargin=\parindent, font=\normalfont\itshape]

    
    \item[procedures:] given a software task from a Stack Overflow post or a GitHub issue, write a short plan with instructions to solve the task

    
    \item[follow-up:] given user comments extracted from the Stack Overflow post or the GitHub issue, evaluate their correctness (whether the comment is correct or not)

    \item[tasks:] 5 random SO tasks and 5 random github tasks from Android
    
    \item[participants:] experienced developers
    
    \item[design:] between subjects. Independent variable is presence (or not) of the technique that highlights text that might assist in the task
    
    \item[measurement-1:] Likert scale on participant's confidence about the correctness of their answer
    
    \item[measurement-2:] If the assessment provided by a participant for each statement matches the ones of an oracle. To create the oracle:
    
        -- True statements are picked from the comment history on SO or GitHub
        
        -- False statements are created by picking a comment and modifying it
        
    \item[tooling:] fixed set of artifacts associated with each task. Highlights in each artifact are cached
    
    \item[pros:] easier to run online

        -- might be doable on MTurk, which increases participant pool

    \item[cons:]  if tasks are just to evaluate truthness, participants are more likely to just search for the bits that answer each statement

        -- participants do not get to perform the software task itself
\end{description}
% \end{small}

% \clearpage

\subsubsection{Brainstorm - B}

% \begin{small}
\begin{description}[leftmargin=\parindent, font=\normalfont\itshape]

    
    \item[procedures:] given a development task \& a maintenance task, provide a solution to the task

    
    \item[tasks:] 1 development task and 1 maintenance task. Each solvable in at most a 1 hour session
    
    -- development task: same as in~\cite{Ponzanelli2014b}---create a program that, given an URL of a webpage, an email address, and a subject for the email
    converts the HTML page into a PDF. 
    
    \textit{[this can be futher divided into smaller tasks, e.g. 1 task to download some url as a PDF and another to convert it into a PDF]}
    
    -- maintenance task: adapted from~\cite{Brandt2009a}---modify a chat room web application from HTTP to websockets \textit{[this can be futher divided into smaller tasks]}
    
    
    \item[participants:] experienced developers. Awarded an Amazon gift card as compensation
    
    \item[design:] within subjects. Independent variable is presence (or not) of the technique that highlights text that might assist in the task
    
    \item[measurement-1:] Likert scale on participant's confidence about the correctness of their answer
    
    \item[measurement-2:] Unit tests to assess the solution provided by a participant 
    
    \item[tooling:] some pseudo search engine that provides a set of URLs for each task. Highlights for each task-url are cached
    
    \item[pros:] more realistic and tasks were validated in related work

        -- unit tests give a better sense on how correct is a participant's solution

    \item[cons:]  harder to recruit participants

        -- participants do not get to perform the software task itself
\end{description}
% \end{small}


\clearpage

\section{Evaluation}
\label{cp6:evaluation}



The goal of our evaluation is to investigate whether tools  
 that automatically identify text relevant to a task, according to the text's semantics,
 assist developers complete their tasks. This evaluation helps answer: 


\medskip
\begin{bluequote}
    \textit{ Do semantic-based techniques help a 
    developer complete a software development task more more effectively?} 
\end{bluequote}




To answer this question, we report on a controlled experiment were participants 
had to complete three programming tasks when assisted (or not) by a
 tool (\red{Section}~\ref{}) able to identify text relevant to each task 
in a set of natural language artifacts deemed pertinent to that task. 




For each task, we measure \red{effectiveness} using a test cases that 
exercise a participant's submitted solution. 
A \red{complete} solution must pass all test cases of a given task. 
In turn, a \red{correct} solution must pass a subset of these test cases 
which cover the core functionalities needed in a task's solution.).


% We use \textit{\red{TODO}} and \textit{\red{TODO}} metrics~\cite{manning2010IR} to measure \red{TODO}~\cite{Murphy2005}.




\subsection{Experimental Design}

\red{TODO}

\subsubsection{Participants}


3rd and 4th year students in the CS major who are familiar with Python


\subsubsection{Tasks}


3 programming tasks using well-known python core modules:

- pandas

- strings and regular expressions

- date time manipulation


We design each task in such a way that its solution is not obvious. 
That is, a participant cannot simply copy a code snippet from some API documentation, Stack Overflow post, or blog post to solve the task. Rather, we expect 
that the solution for a task integrates distinct method calls in a logic way...


\subsubsection{Procedures}


Online. Using an online judge-like system


\subsubsection{Design}


Within subjcts OR between subjects



% \subsubsection{Computing Task Correctness}



% We compute values for \textit{precision} and \textit{recall} metrics based on the sentences 
% deemed relevant to a task by \textit{at least two} human annotators.
% To minimize risks from a scenario where one of the techniques and their underlying approaches has a peak or
% bottom performance due to training procedures or due to factors beyond our control, 
% we compute results for each technique over 10 distinct executions over the test data, 
% reporting the average of each metric over all runs.



% We compute values for \textit{precision} and \textit{recall} metrics based on the sentences 
% deemed relevant to a task by \textit{at least two} human annotators.
% To minimize risks from a scenario where one of the techniques and their underlying approaches has a peak or
% bottom performance due to training procedures or due to factors beyond our control, 
% we compute results for each technique over 10 distinct executions over the test data, 
% reporting the average of each metric over all runs.


% For a detailed definition of each metric, we refer to the evaluation outcomes in Table~\ref{tbl:type-I-II-errors}, where  columns represent  labels provided by the annotators and rows,
% the text identified as relevant or not by a technique.

% % 

% \medskip
% \begin{table}[H]
\centering    
\begin{scriptsize}
\begin{threeparttable}
\begin{tabular}{l|l|l}

\hline

\textbf{}
& \textbf{Relevant}    
& \textbf{Not-relevant} \\

\hline

\textbf{Identified as relevant} & true positive ($TP$) & false positive ($FP$) \\
\hline
\textbf{Identified as Not-relevant} & false negative ($FN$) & true negative ($TN$) \\
\hline

\end{tabular}
\end{threeparttable}
\end{scriptsize}
\caption{Evaluation outcomes}
\label{tbl:type-I-II-errors}
\end{table}

    



% \paragraph{\textbf{Precision}}

% Precision measures the fraction of the sentences identified that are relevant over the total number of target sentences identified, as shown in Equation~\ref{eq:cp5:precision}.



% \begin{equation}
% \label{eq:cp5:precision}    
%     Precision = \frac{TP}{TP + FP}
% \end{equation}


% \subsubsection{Computing Task Completeness}


% \medskip
% Precision means identifying only text that is relevant, whereas recall means identifying all relevant text.
% Ideally, we would aim for a technique with high precision and high recall. Unfortunately, this is often not possible and we must reach a compromise.
% As described earlier, our goal is to support developers to locate text that might be relevant to their task, 
% and not locating all the relevant text may lead to incomplete or incorrect solutions, thus the reason why we favour recall.






% \subsection{Results}
% \label{cp6:analytic}





% \red{TODO}




% Table~\ref{tbl:techniques-results-answerbot} shows values of precision and recall for  \texttt{AnsBot} and the semantic techniques that we explore. 
% In the table, rows provide details about a specific technique while columns discriminate 
% precision and recall values and which post-processing filters were applied. 
% We also mark results from paired Wilcoxon-Mann-Whitney tests~\cite{mannWhitneyU} 
% that check if the results between each technique and \texttt{AnsBot} are statistically different.




% Results from our comparison to a state-of-the-art approach, namely AnswerBot, 
% indicate that text-based semantic techniques achieve comparable accuracy  
% when identifying relevant information in Stack Overflow artifacts.





% \subsection{Qualitative Evaluation}
% \label{cp6:qualitative}





% \red{TODO}




% Table~\ref{tbl:techniques-results-answerbot} shows values of precision and recall for  \texttt{AnsBot} and the semantic techniques that we explore. 
% In the table, rows provide details about a specific technique while columns discriminate 
% precision and recall values and which post-processing filters were applied. 
% We also mark results from paired Wilcoxon-Mann-Whitney tests~\cite{mannWhitneyU} 
% that check if the results between each technique and \texttt{AnsBot} are statistically different.




% Results from our comparison to a state-of-the-art approach, namely AnswerBot, 
% indicate that text-based semantic techniques achieve comparable accuracy  
% when identifying relevant information in Stack Overflow artifacts.







% \section{Summary}
\label{cp5:summary}



