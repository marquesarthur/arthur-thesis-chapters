\setcounter{chapter}{1}


\chapter{Related Work}
\label{ch:related-work}








Performing a task on a software system, like fixing a bug
or adding a feature typically requires a developer to consult
a number of different kinds of artifacts, such
as API documentation, bug reports, community forums
and web tutorials. 
We have provided an overview on how these software
artifacts contain structured and unstructured parts, 
with both code elements and natural language text
as well as challenges associated with finding task-relevant information 
in these artifacts (Section~\ref{cp1:example})
and many researchers in the 
software engineering field have studied
how to extract information from such natural language 
artifacts to assist developers performing a task~\cite{Holmes2008, Cubranic2005, Bavota2016}. 
In Section~\ref{cp2:text-in-se}, we provide 
a categorization of studies that investigate 
several aspects about these artifacts 
and how software developers make use of them.
In Section~\ref{cp2:automatic-approaches},
we outline techniques that
assist developers in finding information in 
certain natural language artifacts by 
automatically identifying text of interest 
to a particular task.



% these artifacts have been 
% the focus of

% Developers produce such natural language artifacts on a 
% continuous basis~\cite{Rastkar2013t} 
% and there has been a steep growth in the number 
% of natural language artifacts available~\cite{Bavota2016, umarji2008archetypal}.




%  software engineering researchers to both
% investigate 
% properties of the text 
% appearing in these artifacts~\cite{Maalej2013, Sorbo2015}
% and design approaches that 
% can be embedded in
% tools that mine the textual data available to assist software developer performing some task~\cite{Holmes2008, Cubranic2005, Bavota2016}.




Given how 
natural language artifacts have become intrinsically
tied to software development~\cite{liu2021, watson2022, umarji2008archetypal},
most of the techniques that extract information from these artifacts, including the work presented in this thesis, 
are examples of applications that help developers in performing 
software development activities~\cite{Meyer2017}. 
Other applications have exploited textual data
to provide guidance to developers reading software documents~\cite{Treude2016,  robillard2017} 
or  to  explain solutions for programming tasks found in development community forums~\cite{silva2019}.
We discuss these and other applications in Section~\ref{cp2:dev-productivity}.


% mcburney2014


\section{Natural Language Software Artifacts}
\label{cp2:text-in-se}




Software engineering researchers
have investigated several questions on the nature of the text in software artifacts.
These studies have focused on the analysis of different kinds of artifacts and have deepened researchers' knowledge of the type of information available in them,
 which is often a first step towards the design of automated tools 
for identifying and extracting text from these artifacts~\cite{Arya2019, Maalej2013}.
In this section, we outline 
different types of investigations considered,
presenting seminal studies for each category.


\art{unsure if I need to have subsections here}

\subsubsection{Textual Analysis} 

A significant body of work has 
focused on the \textit{syntactic analysis} of the natural language text 
in software artifacts. An early example
is Ko and colleagues' analysis of bug report titles~\cite{Ko2006},
where they identify regularities in the text of nearly 200,000 bug report titles,
discussing how such regularities can 
assist in the automatic identification 
of information in bug reports. 
This study was followed by many others where 
software engineering researchers have 
observed other regularities both in bug reports~\cite{Rastkar2010, Chaparro2016}
and also in other types of artifacts, such as 
when Robillard  and Chhetri inspected 1,000 API elements in the Java SDK 6 
reference documentation~\cite{Maalej2013} or when 
Nadi and Treude examined 20 Stack Overflow posts~\cite{nadi2020}.
The syntactic analysis in these and other studies (e.g.,~\cite{chaparro2019} or~\cite{lin2019})
shows that regular expressions or 
linguistic patterns 
might be used to automatically extract 
relevant information from the text  
in the artifacts studied, as we explain in Section~\ref{cp2:pattern-matching}.

Although certainly valuable, these studies have focused on the analysis of 
text in a single type of artifact. In contrast, this thesis investigates 
text relevant to a task across different artifact types.





\subsubsection{Structural Analysis}

Several other studies have 
investigated structured data, or \textit{meta-data}, available in
natural language software artifacts~\cite{Ponzanelli2015}. In Section~\ref{cp1:example},
we have presented an example of structured data in a Stack Overflow post, 
i.e., the number of votes an answer has and whether it is the accepted answer.
Among other examples, we mention 
code blocks in API documentation or web tutorials 
and also stack traces or other fields included in a bug report~\cite{Davies2014, Breu2010},
which, for example, have been used to 
extract complementary 
information that assists a developer in understanding a bug report~\cite{bettenburg2008}.


Considering how several techniques and tools make use of an artifact's meta-data, 
researchers have also
investigated the frequency with which
meta-data is available~\cite{Davies2014, bettenburg2008makes, uddin2015} 
as well as how often developers update it~\cite{ahmad2018, dig2006, shi2011}.
Findings from these studies led 
software engineering researchers to discuss the promises and perils of 
using an artifact's meta-data~\cite{kalliamvakou2014, ahmad2018}.
For example, an accepted answer on Stack Overflow 
might not be the correct answer~\cite{wang2018}
and 
valuable information exists in elements not
associated with any meta-data~\cite{zhang2019so},
which we discussed as part of our 
scenario about finding information about Android notifications (Section~\ref{cp1:example}).


To better understand the impact of meta-data in finding information 
relevant to a task, Chapter~\ref{ch:identifying} 
examines how using (or not) meta-data available on 
Stack Overflow affects the automatic identification 
of task-relevant text.





\subsubsection{Semantic Analysis} 

Information within natural language artifacts 
serve different purposes and several other studies have
 proposed \textit{taxonomies} classifying the type of information
 available in the text of
 different kinds of natural language software artifacts.
For example, Di Sorbo et al. have found that 
text in development mailing lists can be classified according to the developers' intentions (e.g., feature request, solution proposal, etc.)~\cite{Sorbo2015}.
They have sampled 100 emails and identified that text for feature requests 
often contain expressions in the form of suggestions
(e.g., `\textit{we should add a new button}'), whereas solution proposals 
are often expressed in the form of attempts (e.g., `\textit{let's try a new method to compute cost}'),
suggesting that one can automate this classification to assist developers in finding 
certain type of information.
Other examples include Maalej and Robillard's taxonomy of patterns of knowledge in API documentation~\cite{Maalej2013}
or Arya et al. analysis of information types in Open Source issues~\cite{Arya2019}.



The taxonomies or categorizations presented in these studies are often based
on a need for access to the meaning of
sentences in the natural language text~\cite{berners2001, calero2006, witte2007}.
However, they originate from strenuous manual analysis 
of hundreds of artifacts and the information types identified 
might not extend to different kinds of artifacts.
In Chapter~\ref{ch:characterizing}, 
we use a general linguistic approach~\cite{fillmore1976frame} to
examine the meaning of the task-relevant text in 
API documents, GitHub issues and Stack Overflow posts.


\section{Automatic Text Identification Approaches}
\label{cp2:text-approaches}



Information useful to a software task can be buried in irrelevant text or attached to 
non-intuitive blocks of text, making it difficult to discover~\cite{Robillard2015}.
In this section, we detail tools and approaches from related work
that seek to assist developers in 
identifying information within the natural language
text of software artifacts.





\subsection{Pattern Matching Approaches}
\label{cp2:pattern-matching}


Pattern matching approaches rely on regular expressions describing a sequence of tokens that represent
 a relevant text fragment~\cite{Bavota2016}. Tokens can either represent words or linguistic elements 
extracted using \acf{NLP}.
    
    
As examples  of pattern matching approaches,  {\small DeMIBuD}~\cite{Chaparro2017}
 and Knowledge Recommender (Krec)~\cite{Maalej2013, Robillard2015} are tools that detect relevant sentences in bug reports and API documentation, respectively. 
These tools derive a set of patterns from annotated data and use them as part of heuristics 
that identify relevant text. Krec  uses  361 unique patterns
to 
detect relevant sentences mentioning a 
code element (e.g., a class or method name) in API documentation~\cite{Robillard2015}.
In a similar manner, {\small DeMIBuD} uses a set of 154 discourse patterns to detect sentences 
relevant to understanding a bugs observed or expected behaviour and steps to reproduce it,
which are essential to bug triaging tasks.




In Stack Overflow posts,
Nadi and Treude~\cite{nadi2020} have both applied the original set of patterns from Krec~\cite{Robillard2015} 
and proposed heuristics that rely on the conditional clauses (i.e., sentences with the word `\textit{if}')
to identify text that help a developer 
decide whether they want to carefully inspect a Stack Overflow posts or skip it. 



Although the heuristics and regular expressions used in the aforementioned studies 
are often light-weight and effective~\cite{Bavota2016, Maalej2013}, 
pattern matching approaches are specific to certain kinds of domain and 
types of artifact~\cite{fucci2019}, what limits
using them in a general approach.





\subsection{Summarization Approaches}
\label{cp2:summarization}



Extractive text summarization techniques are used in natural language artifacts in software engineering to
produce a summary of the artifact's content. 
A summary represents key information that may help a developer complete their task~\cite{Bavota2016}.
There are summarization techniques based on both supervised and unsupervised learning~\cite{moreno2017}
and one can summarize the entire content of an artifact
or content specific input query, as in \textit{query-based} summarization~\cite{Huang2018, Goldsteinet1999}.




A number of summarization approaches target bug reports and GitHub issues, largely
focusing on identifying key information within these artifacts. 
Rastkar and colleagues~\cite{Rastkar2010} use a supervised learning approach to summarize the content 
of bug reports showing that conversational features used to summarize emails~\cite{Murray2008}
can also be applied to bug reports while
Lotufo et al.~\cite{Lotufo2012} proposed an unsupervised summarization approach 
that automates the identification of sentences that a developer would first read when
a inspecting bug report.



While many of the approaches described above
largely rely on  lexical aspects in text, researchers have also made use
of structured textual information in the artifacts~\cite{Ponzanelli2015, Treude2016, chen2016}. 
For example, Ponzanelli et al. 
proposed a summarization technique that mixes natural language text and structured data 
available on Stack Overflow
to produce more accurate summaries for Stack Overflow answers~\cite{Ponzanelli2015}. 
As another example, DeepSum~\cite{Li2018} pre-processes a bug report dividing sentences 
containing software elements, the reporter of the bug, and any other sentences 
in the bug report to produce summaries containing more diverse information.




A smaller number of summarization approaches have focused on
producing task specific summaries~\cite{Xu2017, silva2019}.
These approaches pose the problem of finding task-relevant text 
as a query-based extractive summarization problem and
tools such as AnswerBot~\cite{Xu2017}
identify relevant text in Stack Overflow posts 
based on 
the content of the text, how similar that content is with regards to a input query (i.e., task)
and the structured data available on each of the answers in a Stack Overflow post 
(i.e., number of votes or whether an answer is the accepted answer).
As the state-of-the-art, Chapter~\ref{ch:identifying}
compares AnswerBot to the techniques that we explore in this thesis.



\subsection{Machine Learning Approaches}
\label{cp2:machine-learning}


\acf{ML} approaches take the text of a natural language software artifact and identify 
the sentences likely relevant to a particular software task using \textit{supervised} or 
\textit{unsupervised learning} methods~\cite{zhang2005machine}.



Supervised learning approaches use a set of features and labeled data
 to train classifiers with the goal of identifying sentences relevant to 
 certain software activity.
We have already presented supervised approaches that use text summarization (\textit{i.e.,}~\cite{Rastkar2010})
and there are also approaches that identify relevant 
parts of software tutorials~\cite{Jiang2016b}
or API documents~\cite{fucci2019, Maalej2013}.
Despite their value, 
the cost and effort of hiring skilled workers to produce 
labeled data in software engineering artifacts 
has been a major limitation 
to the usage of supervised learning 
methods~\cite{Arpteg2018}.





Unsupervised learning approaches do not require labelled data and determine 
relevant sentences according to properties inferred from the data. 
DeepSum~\cite{Li2018} and Lotufo et al.'s~\cite{Lotufo2012} techniques are examples of 
unsupervised approaches in the scope of text summarization. 



Other unsupervised approaches 
(\textit{i.e.,} {\small FRAPT}~\cite{Jiang2017} or HoliRank~\cite{Ponzanelli2015, Ponzanelli2017})
are mostly based around variations of the PageRank~\cite{Page1999} or LexRank~\cite{Erkan2004} algorithms. 
These algorithms represent all the text in an artifact as a graph.
Then, they establish relationships (\textit{i.e.,} weighted edges in the graph) 
between different sentences (\textit{i.e.,} nodes in the graph) 
and select the nodes with highest weights as the most relevant ones.
A crucial step in building the graph is in the definition of 
how to establish  relationships between nodes.
Early approaches~\cite{Lotufo2012, Jiang2017} 
use \ac{VSM}~\cite{Salton1975vsm} 
for this purpose while more modern ones~\cite{Huang2018, silva2019}
use different word embeddings~\cite{Mikolov2013, bojanowski2017FastText},
which we detail in Section~\ref{cp2:deep-learning}.
To identify key sentences using these techniques, one must 
assume that certain elements will be cross-referenced 
or mentioned multiple types in a document, 
what is only common in certain types of software artifacts, e.g., mailing lists~\cite{Bacchelli2012}.








\subsection{Deep Learning Approaches}
\label{cp2:deep-learning}



One substantial challenge of standard \acf{ML}
approaches is that researchers must engineer which 
features or properties of the text to use~\cite{ferreira2021}.
For example, Rastkar et al. uses conversational features in 
the text of a bug report to assist in determining which sentences 
to include in the bug report's summary~\cite{Rastkar2010}
while Petrosyan and colleagues use 
linguistic and structural properties 
in the text of API documents to identify key text 
explaining API elements~\cite{Petrosyan2015}.
Given the specificity of such features, 
researchers have questioned the generalizability
of standard \acs{ML} approaches~\cite{Xiao2018, fucci2019}.



In contrast to the human-engineered features,
\acf{DL} approaches allow the automatic extraction of features 
from training data through a series of mathematical transformations~\cite{Deng2018, zhang2021deep}.
Deep learning has led to groundbreaking advancements in many 
research areas (e.g., machine translation~\cite{lopez2008translation}) 
and, given its wide range of applications, 
we focus
on its usage in natural language text appearing in software engineering artifacts~\cite{ferreira2021, li2018deep, watson2022}.









Software engineering researchers have identified that the text 
in software task 
often differs from the text in the artifacts that are related to that tasks~\cite{Huang2018}. 
These so-called \textit{lexical mismatches}~\cite{Ye2016} 
 make it difficult to identify information of interest 
to a task and a number of studies have used \acs{DL}
neural embeddings~\cite{Mikolov2013} to bridge lexical gaps between the text of different software artifacts. 


Neural, or word, embeddings produce vector representations in a continuous space,
where words with similar meanings are typically close in the vector space model~\cite{harris1954distributional, mikolov2013efficient}. 
Their usage has allowed researchers to improve 
the identification API elements pertinent to a programming task~\cite{Ye2016} 
or to more accurately assess the quality of the content in bug reports~\cite{chaparro2019}.
Word embeddings have become a common way 
to compare the semantic similarity of the text~\cite{mihalcea2006},
being applied in query-based summarization techniques such as 
AnswerBot~\cite{Xu2017}
or PageRank-based approaches such as HoliRank~\cite{Ponzanelli2017}.



Many other \acs{DL} studies in software engineering~\cite{ferreira2021,li2018deep, watson2022}
use neural network architectures 
in a variety of software engineering tasks, including
code comprehension~\cite{allamanis2015, mi2018}, community forum analysis~\cite{Lin2018, wang2019}, or requirements traceability~\cite{chen2019, guo2017}.
DeepSum~\cite{Li2018}, which we described earlier, is an example of a 
summarization approach that uses an encoder-decoder architecture~\cite{cho2014-encoder-decoder}  to identify 
sentences of interest in bug reports. As another example,
Xi et al.~\cite{Xia2017} use a \ac{CNN}~\cite{krizhevsky2012CNN} 
to identify sentences in GitHub issues discussing topics such as 
problem discovery, solution proposal, or feature requests.



Few studies in software engineering have considered more modern neural networks~\cite{watson2022}
able to establish relationships not only between words but also sentence pairs (e.g., \acs{BERT}~\cite{Devlin2018Bert}). 
These models might assist in determining 
implicit relationships between the text in a task and the text in a relevant sentence within a software artifact.
As such, Chapter~\ref{ch:identifying} describes how we use \acs{BERT} for automatically 
identifying text relevant to a software task.





\section{Improving Developers' Productivity}
\label{cp2:dev-productivity}



The tools and approaches presented in Sections~\ref{cp2:general-approaches}
and~\ref{cp2:task-approaches}
are examples of studies that help developers in locating 
information that assists them in completing a software task.
These studies fit in the bigger context 
of software engineering research 
facilitating or improving the quality of a developer's work~\cite{Kersten2006, Meyer2017, satterfield2020}. 





As part of their work, developers engage in many sensemaking and decision-making activities~\cite{sillito2006} and several studies have investigated how to 
provide means to better  
assist developers in performing such activities~\cite{Liu2018Unakite, liu2021, barnett2015}.
For example, 
Ponzanelli et al. proposed a tool, Libra, that monitors the web pages a developer 
has navigated and uses that to show 
how similar or not the results of a new web search are in comparison to the already navigated 
pages, which offers contextual information-seeking support~\cite{Ponzanelli2017}.



Researchers have also been interested in 
making knowledge bases
that developers working on a task can benefit from. 
In Section~\ref{cp2:task-approaches} 
we cited Hipikat~\cite{Cubranic2005}, a seminal tool that exemplifies this concept in action.
Based on the current code being inspected by a developer or based on a query prompted by a developer, 
it recommends artifacts from a project's archive 
that are pertinent to the code being inspected.
This archive, or project memory, is produced 
based either on relationships between the artifacts in a software project 
or based on the source code changed in a bug fix or feature request~\cite{Cubranic2005}.
Other tools like Strata summarize knowledge produced by developers 
while they navigate on the web so that other developers
can use this knowledge to have a 
 head start when performing similar tasks~\cite{liu2021}.




We build upon many of the research procedures outlined 
in these and many other studies in the field. 
For example, in the evaluation of Strata, 
 Liu et al. describe procedures 
considering how a control and tool-assisted group 
complete information-seeking tasks~\cite{liu2021},
which assisted in the design of our experiment for 
evaluating an automated approach to
task-relevant text identification (Chapter~\ref{ch:assisting}).





