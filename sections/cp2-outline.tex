\setcounter{chapter}{1}


\chapter{Related Work}
\label{ch:related-work}








Performing a task on a software system, like fixing a bug
or adding a feature typically requires a developer to consult
a number of different kinds of artifacts, such
as API documentation, bug reports, community forums
and web tutorials. 
Developers produce such natural language artifacts on a 
continuous basis~\cite{Rastkar2013t} 
and locating the documents of interest for a task
is not trivial. 
To better understand this activity, 
many software engineering researchers 
have studied  a developer's information foraging process~\cite{Pirolli1999}
providing theories on how a developer searches for pertinent information
and proposing tools to help this activity.
In Section~\ref{cp2:foraging},
we provide a general overview of background information on 
information foraging theory and 
then, in Section~\ref{cp2:foraging-tools}, we detail tools and approaches that assist developers in finding artifacts pertinent to their task.



Once a developer locates pertinent
natural language artifacts, 
they must still identify \textit{within} these documents 
the fraction of text relevant to the task at hand.
To assist this activity, software engineering researchers have made use of the natural language text appearing in different kinds of artifacts in many different ways.
Many of the approaches that
use this text rely on lexical and syntactic properties of the text. 
In Section~\ref{cp2:artifact-approaches}, we review various studies that have been taken using these approaches,
honing in on studies that help identify text relevant to a task-at-hand.
More recently, researchers have also started to use semantic approaches; 
in Section~\ref{cp2:artifact-semantics} we 
provide an overview of these approaches to provide a background of
how semantics can be useful to support software development activities.



\clearpage



\section{Information Foraging and Relevance Theory}
\label{cp2:foraging}



Information foraging~\cite{Pirolli1999}
and relevance theory~\cite{clark2013relevance, saracevic1975, Saracevic2007c, Saracevic2007b} 
provide the theoretical background of this thesis.


\textit{\acf{IFT}} explains how 
an individual navigates through some search space looking for \textit{information patches} based on 
a set of cues about the effort and gains that a patch provides to them~\cite{Pirolli1999}.
Foraging occurs \textit{between} or \textit{within} patches, i.e., 
searching \textit{which} documents are relevant and \textit{what} within a document is of relevance,
and \textit{relevance theory}~\cite{clark2013relevance, saracevic1975} elaborates the relationships between an information object (i.e., text in an artifact),
some context (i.e., a task), and what properties guide a forager's relevance assessment (i.e., clarity, utility, completeness, etc.)~\cite{Saracevic2007c}. 




To understand how software developers find  information pertinent to their software tasks,
software engineering researchers have extensively studied information foraging in the context of software development~\cite{Piorkowski2015, Piorkowski2016, chi2007, Ko2006a}.
As one example,
 Forward and Lethbridge describe criteria that developers use to assess the relevance of software engineering documentation, where they observe that the content of an artifact, the presence of examples, and the document's structure are common attributes that impact an artifact's relevance~\cite{Forward2002}.
As another example, Charrada and Mussato studied how 
practitioners manage and search for software engineering documents, 
identifying that data scattered across different sources is 
often a significant challenge to the information foraging process.



In the context of programming tasks, Starke et al. explain that a software developer often makes a set of hypotheses about a software task,
translating these hypotheses into multiple search queries that will lead to artifacts likely pertinent to the developer's task~\cite{Starke2009}. 
They observe that, instead of systematically inspecting search results,
developers often skim through search results to decide which documents are of relevance. 
In other studies, Brandt et al. found that developers interleave web foraging, learning and writing code 
to accomplish software tasks~\cite{Brandt2009a} while de Graaf and colleagues 
identified that prior knowledge about the structure of a software artifact helps professionals
to search software documents efficiently and effectively~\cite{DeGraaf2014}.


This thesis adds to the existing theory by discussing 
common properties in the text of an artifact, which several software developers deemed relevant 
to certain software tasks.  


\section{Finding Pertinent Artifacts}
\label{cp2:foraging-tools}

% \art{I tried  summarize this section, but it still looks lengthy}


Software development often requires knowledge beyond what developers already posses~\cite{Li2013} and thus, 
software developers use different information sources to fill such gaps. 
More than often, a developer will ask if any of their 
peers have the information that they need~\cite{singer2011}. 
However, the fragmented and distributed nature of software development  
might prevent a developer from approaching their peers~\cite{ko2007}.
Due to this and other reasons~\cite{Xia2017, rao2020}, a developer might seek
online web resources for information 
that may assist them in completing the task-at-hand.
This section provides a breadth of approaches that help a developer find such resources.




\subsection{Information Retrieval Approaches} 
\label{cp2:ir-approaches}





Standard \acf{IR} approaches locate pertinent artifacts
using metrics to compute the relevance of an artifact to a query posed manually by a developer.
\acs{IR} uses the terms shared between a search query and an artifact
to compute the relevance of an artifact (or document).
Search algorithms weight and compute relevance based on 
criteria such as how many documents contain a given term or  
how unique a term is~\cite{Manning2009IR}.



A number of the techniques commonly employed by software engineering researchers are based on the
frequency of co-occurrence of words (or phrases) in documents.
An early example is Maarek and Smadja's use of lexical relations to index
software libraries~\cite{maarek1989}.
Since this early use, software engineering
researchers have continued to leverage advances in
these approaches, such as when
Antoniol et al. applied \acf{VSM}~\cite{Salton1975vsm} 
to construct vector representations 
for recovering traceability links 
between code and documentation~\cite{antoniol1999, antoniol2000}
or when Marcus and Maletic used \acf{LSI}~\cite{deerwester1990LSI}
to help cluster software components to aid
program comprehension of a software system~\cite{Marcus2003}.



While IR systems are widely used, researchers 
observed that developers often find it difficult to identify good search terms~\cite{Kevic2014, Huang2018},
what significantly impacts the retrieval of relevant documents~\cite{Kevic2014, mills2017}.
To mitigate  the necessity of a developer coming up with good search terms,
researchers have used information available in a task to automatically generate search terms~\cite{Kevic2014, Haiduc2013}. 
Although automatically generating search terms assist the retrieval of pertinent artifacts,
standard \acs{IR} algorithms will have 
little success if search terms 
differ significantly from the text in a software artifact~\cite{Huang2018}.
These so-called \textit{lexical mismatches}~\cite{Ye2016, silva2019} have led researchers to investigate 
 semantic-based \acs{IR} approaches, which we discuss in 
Section~\ref{cp2:artifact-semantics}.




% 


\subsection{Contextual Approaches} 


Contextual approaches use information 
about a developer's task so that 
they recommend artifacts pertinent to that task drawing data from some knowledge base.
% , for example, a tool might need to know what part of the system is a developer currently inspecting. 
Fishtail~\cite{Sawadsky2011}
is an example of a tool that uses contextual information. It uses the source code currently being inspected by a developer to find web resources relevant to the developer's task.



While the web is the most accessible knowledge base, certain contextual approaches consider that 
collection of artifacts produced and consumed in past software tasks 
 implicitly
form knowledge bases that might assist developers performing new tasks~\cite{Cubranic2005}. 
Hipikat~\cite{Cubranic2005} is a seminal tool that exemplifies this concept in action.
It automatically tracks the artifacts 
used by a developer as part of a development tasks
so that it can recommend these artifacts to 
any future developers who work on similar tasks~\cite{Cubranic2005}.
As another example, Deep Intellisense
finds code changes, filed bugs, and forum discussions 
mining these artifacts from an organization's shared database~\cite{Holmes2008}.










In the past decade, question-and-answer (\acs{qa}) web sites such as Stack Overflow\footnote{\url{https://stackoverflow.com/}}, have also
become a common knowledge source used by contextual approaches~\cite{Ponzanelli2013b, Ponzanelli2014b, Treude2016, delfim2016}.
For example, {\small PROMPTER} uses contextual information 
available in a developer's IDE to retrieve Stack Overflow discussions that might be pertinent 
to the developer's current task. Other tools such as {\small BIKER}~\cite{Huang2018} mine Stack Overflow posts to assist developers in locating API components needed in a programming tasks. 





This thesis assumes the usage of one or more techniques presented here to locate pertinent
artifacts from which task-relevant text will be extracted.




\section{Identifying Relevant Information within an Artifact}
\label{cp2:artifact-approaches}


\section{On Interpreting the Semantics of Text in Software Artifacts}
\label{cp2:text-semantics}


% In Section~\ref{cp2:ir-approaches}


\art{Think how can I describe without diving into details}



At times, software engineering researchers have argued
that general lexicon techniques 
are insufficient to address text appearing in
software engineering artifacts. 
Arguments on why lexicon-based natural 
language techniques are not applicable are often based
on a need for access to the \textit{meaning}, or semantics, 
of words, phrases or sentences appearing in the text~\cite{jurafsky2014speech}.
In this section, 
we present background information on semantics focusing on
its usage in software engineering research.



\subsection{Word Semantics}

Word semantic techniques are mostly rooted on the hypothesis
that similar words appear in similar context~\cite{harris1954distributional}.
This hypothesis gave origin to a series of
\textit{distributional semantic models}~\cite{Ye2016} that aim to infer the meaning of words.
% In this section, we present prominent models used by software engineering researchers.



Distributional semantic models have been used by software engineering researchers 
to improve the 
the retrieval of artifacts pertinent to a certain task. 
Early models, such as \acf{LSI}~\cite{deerwester1990LSI}, 
have been used to, for example, recover traceability links between source code and
software documentation~\cite{marcus2003}.  
\acs{LSI} takes a initial word representation (i.e., a term by document matrix) and applies \acf{SVD}~\cite{klema1980SVD}
to reduce the dimensionality of this matrix, what causes 
words with similar meaning have the same final representation.


Other word semantic models have assisted software engineering researchers in clustering semantically similar artifacts~\cite{zhang2014, layman2016}. For that, researchers have mostly used
\acf{LDA}~\cite{blei2003latent}---a model that assumes that words used in a similar context often pertain to the same subject to produce topics clustering sentences or documents containing semantically related words---to
identify common themes in developers' blog posts~\cite{Pagano2011} or to design tools that identify duplicated bug reports~\cite{nguyen2012, Thung2014}.


% ---among its many applications~\cite{zhang2014, layman2016}---


% Marcus and Maletic apply \acs{LSI} to 

% In software engineering, \acs{LSI} has been widely used to assist requirements traceability~\cite{lucia2007, hayes2006, gethers2011}.

Despite their significant contributions, early models created word vector representations
by counting the frequency or co-occurrence of words, what is substantially inefficient for large corpora~\cite{Ye2016}.
This and other challenges have been lifted by advancements in the fields of \acf{ML} and \acf{DL}~\cite{ferreira2021, li2018deep}, which walked hand-to-hand with improvements in computational power and the amount of memory available in modern computer architectures~\cite{sharafi2015}.


\acs{DL} models built with \textit{neural, or word, embeddings}~\cite{Mikolov2013} 
are of particular interest to this thesis. 
Neural embeddings produce vector representations in a continuous space 
and researchers have shown that they 

% in a cost-effective manner~\cite{mikolov2013efficient}.


% , Ye2016, Efstathiou2018











\subsection{Sentence Semantics}







Surprisingly 
most of the deep learning approaches applied to software engineering tasks 
focus on source code and 

natural language text has been very shy....



% Semantic-based approaches  build a meaning by establishing relationships between terms,
% for instance words, based on the context in which these terms appear.
% For example
% the terms `\textit{car}' and `\textit{automobile}' are likely to co-occur in different phrases with terms like `\textit{motor}' and `\textit{wheel}' and thus,
% semantic approaches indicate that these terms are likely similar in meaning~\cite{Bavota2016}.

% As an example of the usage of semantics in software engineering researcher, Marcus and Maletic applied Latent Semantic Indexing (LSI) to help cluster software components to aid
% program comprehension of a software systems~\cite{marcus2003}. 
% Ye et al.'s study on the usage of word embeddings~\cite{Ye2016} was among the first to use the language models in the software engineering domain~\cite{Mikolov2013}. 

% Researchers have also developed approaches
% to interpret the meaning of sentences in
% software engineering documents. 
% Maalej and Robillard have
% developed a knowledge taxonomy~\cite{Maalej2013} using grounded theory to analyze documentation in open-source systems and then, they validated their taxonomy on documentation units sampled from the Java SDK 6 and .NET 4.0.
% Arya et al.'s have determined information types in open-source issue discussions~\cite{Arya2019},  Di Sorbo et al. have created an approach to classify  emails based on a developer's intentions~\cite{Sorbo2015} and
%  Marques et al. have investigated the use of frame semantics~\cite{fillmore1976frame}.
%  The techniques we propose in this paper
%  build on these earlier findings.





% \subsection{Machine Learning Approaches} 


% To bridge lexical gaps, researchers have investigated 
% approaches that leverage the 
% semantic aspects of software artifacts~\cite{Maletic2001, Ye2016}.
% While Section~\ref{cp2:artifact-semantics} provides an overview of semantics 
% in software engineering, this section especifically details how semantics might assist in finding 
% pertinent artifacts. 









%  is an early example of an \acs{IR} method that address \textit{lexical mismatches}.

% These and more advanced approaches 


% or when Nguyen and colleagues
% applied Word2Vec~\cite{mikolov2013word2vec} to support the retrieval of API
% examples~\cite{nguyen2017}.




% While ML approaches are effective in locating pertinent artifacts, a developer still needs to
% manually find the parts within that artifact
% that are relevant to her task~\cite{Cubranic2005}. 







% they have been used 
% for several purposes by software engineering researchers.



% While it is obvious to the human reader that `car' and `automobile' are synonyms and can be used interchangeable,  
% standard \acs{IR}  fail to capture this and other differences\footnote{
%     \textbf{Oxford English dictionary definitions}~\cite{dictionary1989oxford} 
%     \begin{itemize}
%         \item \textbf{synonymy:} the fact of two or more words having the same meaning;
%         \item \textbf{polysemy:} the fact of having more than one meaning.
%     \end{itemize}
% }
% that significantly affected the performance of \acs{IR} systems~\cite{DeLucia2012}.
% To address these issues, researchers have proposed a set of word semantic techniques the use statistical models to infer the meaning of words. 
