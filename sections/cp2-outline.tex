\setcounter{chapter}{1}


\chapter{Related Work}
\label{ch:related-work}








Performing a task on a software system, like fixing a bug
or adding a feature typically requires a developer to consult
a number of different kinds of artifacts, such
as API documentation, bug reports, community forums
and web tutorials. 
Developers produce such natural language artifacts on a 
continuous basis~\cite{Rastkar2013t} 
and there has been a steep growth in the number 
of natural language artifacts available~\cite{Bavota2016, aa}.


The prevalence of online web resources and how their usage has become intrinsically tied to software development~\cite{} has led 
 software engineering researchers to 
investigate 
properties of the text 
appearing in natural language software artifacts~\cite{aa}
and also to design approaches that 
can be embedded in
tools that mine the textual data available in these artifacts~\cite{aa}.
In Sections~\ref{cp2:text-properties} and~\ref{cp2:text-approaches}
we  provide background information on the significant body 
of work that investigates properties of the text in software engineering 
artifacts and the techniques that automatically identify text of interest 
to a particular task, respectively.







Most of the techniques that extract information from natural 
language artifacts, including the work presented in this thesis, 
are examples of applications that assist developers in performing 
software tasks. 
Other applications have exploited textual data
to produce better software documentation~\cite{Treude2016, mcburney2014, robillard2017} 
or  succinct  explanations of solutions for programming tasks~\cite{silva2019}.
We discuss these and other applications in Section~\ref{cp2:dev-productivity}.



% More recently, researchers have also started to use semantic approaches; 
% in Section~\ref{cp2:artifact-semantics} we 
% provide an overview of these approaches to provide a background of
% how semantics can be useful to support software development activities.


% \art{the work presented here fits in a bigger context in how developer's work}


% \section{Natural Language Software Artifacts}
\label{cp2:text-in-se}


\art{Lexical Approaches}

\art{Syntactic Approaches}

\art{Meta-data Approaches}

\art{Semantic Approaches}

\art{Lexical Approaches}


Software engineering researchers
have investigated several questions on the nature of the text in software artifacts.
These studies have focused on the analysis of different kinds of artifacts and have deepened researchers' knowledge of the type of information available in them,
 which is often a first step towards the design of automated tools 
for identifying and extracting text from these artifacts~\cite{Arya2019, Maalej2013}.
In this section, we outline 
different types of investigations considered,
presenting seminal studies for each category.


\art{unsure if I need to have subsections here}

\subsubsection{Textual Analysis} 

A significant body of work has 
focused on the \textit{syntactic analysis} of the natural language text 
in software artifacts. An early example
is Ko and colleagues' analysis of bug report titles~\cite{Ko2006},
where they identify regularities in the text of nearly 200,000 bug report titles,
discussing how such regularities can 
assist in the automatic identification 
of information in bug reports. 
This study was followed by many others where 
software engineering researchers have 
observed other regularities both in bug reports~\cite{Rastkar2010, Chaparro2016}
and also in other types of artifacts, such as 
when Robillard  and Chhetri inspected 1,000 API elements in the Java SDK 6 
reference documentation~\cite{Maalej2013} or when 
Nadi and Treude examined 20 Stack Overflow posts~\cite{nadi2020}.
The syntactic analysis in these and other studies (e.g.,~\cite{chaparro2019} or~\cite{lin2019})
shows that regular expressions or 
linguistic patterns 
might be used to automatically extract 
relevant information from the text  
in the artifacts studied, as we explain in Section~\ref{cp2:pattern-matching}.

Although certainly valuable, these studies have focused on the analysis of 
text in a single type of artifact. In contrast, this thesis investigates 
text relevant to a task across different artifact types.





\subsubsection{Structural Analysis}

Several other studies have 
investigated structured data, or \textit{meta-data}, available in
natural language software artifacts~\cite{Ponzanelli2015}. In Section~\ref{cp1:example},
we have presented an example of structured data in a Stack Overflow post, 
i.e., the number of votes an answer has and whether it is the accepted answer.
Among other examples, we mention 
code blocks in API documentation or web tutorials 
and also stack traces or other fields included in a bug report~\cite{Davies2014, Breu2010},
which, for example, have been used to 
extract complementary 
information that assists a developer in understanding a bug report~\cite{bettenburg2008}.


Considering how several techniques and tools make use of an artifact's meta-data, 
researchers have also
investigated the frequency with which
meta-data is available~\cite{Davies2014, bettenburg2008makes, uddin2015} 
as well as how often developers update it~\cite{ahmad2018, dig2006, shi2011}.
Findings from these studies led 
software engineering researchers to discuss the promises and perils of 
using an artifact's meta-data~\cite{kalliamvakou2014, ahmad2018}.
For example, an accepted answer on Stack Overflow 
might not be the correct answer~\cite{wang2018}
and 
valuable information exists in elements not
associated with any meta-data~\cite{zhang2019so},
which we discussed as part of our 
scenario about finding information about Android notifications (Section~\ref{cp1:example}).


To better understand the impact of meta-data in finding information 
relevant to a task, Chapter~\ref{ch:identifying} 
examines how using (or not) meta-data available on 
Stack Overflow affects the automatic identification 
of task-relevant text.





\subsubsection{Semantic Analysis} 

Information within natural language artifacts 
serve different purposes and several other studies have
 proposed \textit{taxonomies} classifying the type of information
 available in the text of
 different kinds of natural language software artifacts.
For example, Di Sorbo et al. have found that 
text in development mailing lists can be classified according to the developers' intentions (e.g., feature request, solution proposal, etc.)~\cite{Sorbo2015}.
They have sampled 100 emails and identified that text for feature requests 
often contain expressions in the form of suggestions
(e.g., `\textit{we should add a new button}'), whereas solution proposals 
are often expressed in the form of attempts (e.g., `\textit{let's try a new method to compute cost}'),
suggesting that one can automate this classification to assist developers in finding 
certain type of information.
Other examples include Maalej and Robillard's taxonomy of patterns of knowledge in API documentation~\cite{Maalej2013}
or Arya et al. analysis of information types in Open Source issues~\cite{Arya2019}.



The taxonomies or categorizations presented in these studies are often based
on a need for access to the meaning of
sentences in the natural language text~\cite{berners2001, calero2006, witte2007}.
However, they originate from strenuous manual analysis 
of hundreds of artifacts and the information types identified 
might not extend to different kinds of artifacts.
In Chapter~\ref{ch:characterizing}, 
we use a general linguistic approach~\cite{fillmore1976frame} to
examine the meaning of the task-relevant text in 
API documents, GitHub issues and Stack Overflow posts.



% 


\section{On the Properties of Software Engineering Text}
\label{cp2:text-properties}



Information useful to a software task can be buried in irrelevant text or attached to 
non-intuitive blocks of text, making it difficult to discover~\cite{Robillard2015}.
Researchers have long recognized the value of assisting developers in 
identifying information of relevance in this natural language
text.
In this section, we detail tools and approaches from related work.




\subsection{Pattern Matching Approaches}


Pattern matching approaches rely on regular expressions describing a sequence of tokens that represent a relevant text fragment~\cite{Bavota2016}. Tokens can either represent words or linguistic elements 
extracted using \acf{NLP}.
    
    
As examples  of pattern matching approaches,  {\small DeMIBuD}~\cite{Chaparro2017} and Knowledge Recommender (Krec)~\cite{Maalej2013, Robillard2015} are tools that detect relevant sentences in bug reports and API documentation, respectively. 
These tools derive a set of patterns from annotated data and use them as part of heuristics 
that identify relevant text. Krec assumes that any relevant sentence mentions a 
code element (e.g., a class or method name) and it uses  361 unique patterns derived from the Java 6 SE API documentation to 
detect sentences that a developer must pay attention to when reading an API document~\cite{Robillard2015}.
In a similar manner, {\small DeMIBuD} uses a set of 154 discourse patterns to detect sentences 
relevant to understanding a bugs observed or expected behaviour and steps to reproduce it.





In  Stack Overflow posts, Nadi and Treude  extend the original set of patterns from Krec~\cite{Robillard2015} with two heuristics that ultimately aim to identify sentences that help a developer decide whether they want to read the post or skip over it~\cite{nadi2020}. 
Their heuristics expect that sentences express conditions about a subject and 
they find that no single heuristic is able to identify all of the sentences 
that humans indicated as useful to \texttt{json} manipulation tasks. 







\subsection{Summarization Approaches}



Extractive text summarization techniques are used in natural language artifacts in software engineering to
produce a summary of the artifact's content. These summaries aim to represent key information that may help a developer complete their task~\cite{Bavota2016}.
There are summarization techniques based on both supervised and unsupervised learning~\cite{moreno2017}
and one can summarize the entire content of an artifact
or content that relates to a specific input query, as in query-based summarization~\cite{Huang2018, Goldsteinet1999}.
We collectively discuss these techniques in this section. 






A number of summarization approaches target bug reports and GitHub issues, largely
focusing on key information within these artifacts. 
Rastkar and colleagues~\cite{Rastkar2010} use a supervised learning approach to summarize the content of bug reports showing that conversational features used to summarize emails~\cite{Murray2008}
can also be applied to bug reports.
Lotufo et al.~\cite{Lotufo2012} proposed an unsupervised summarization approach 
that automates the identification of sentences that a developer would first read when inspecting bug reports. While their approach outperforms Rastkar's, they discuss the need to generate more diverse summaries containing 
a bug's reported problem, possible workarounds, or discussed solution~\cite{Lotufo2012}.


% \paragraph{\textbf{PageRank-based Approaches.}}


Other summarization approaches are mostly based around variations of the PageRank~\cite{Page1999} or LexRank~\cite{Erkan2004} algorithms. 
These algorithms represent all the text in an artifact as a graph.
Then, they establish relationships (\textit{i.e.,} weighted edges in the graph) between different sentences (\textit{i.e.,} nodes in the graph) and select the nodes with highest weights as the most relevant ones.
A crucial step in building the algorithm's graph is in the definition of how to establish  relationship between nodes.
Early approaches~\cite{Lotufo2012, Jiang2017} 
use \ac{VSM}~\cite{Salton1975vsm} representations to compute how similar two sentences are
if the similarity scores between the vector representations of these sentences is above some threshold. 
With the emergence new \acs{NLP} techniques, 
tools such as AnswerBot~\cite{Huang2018} or {\small CROKAGE}~\cite{silva2019}
use different word embeddings~\cite{Mikolov2013, bojanowski2017FastText} to build their graphs.





% \paragraph{\textbf{Structured Data.}} 


While many of the approaches described above
largely rely on  lexical aspects in text, researchers have also made
use
of structured textual information in the artifacts~\cite{Ponzanelli2015, Treude2016, chen2016}. 
For example, recognizing the value of structured data available on Stack Overflow, Ponzanelli et al. 
proposed a summarization technique that mixes natural language text and structured data to produce more accurate summaries for Stack Overflow answers~\cite{Ponzanelli2015}. 
As another example, DeepSum~\cite{Li2018} pre-processes a bug report dividing sentences containing software elements, the reporter of the bug, and any other sentences in the bug report to produce summaries containing more diverse information.


% \paragraph{\textbf{Query-based Summarization.}} 


A smaller number of approaches have focused
on the problem that is the focus of this
thesis, namely identifying text in artifacts
related to a specific task-at-hand.
These approaches often apply to Stack Overflow 
as with AnswerBot's relevant and salient text selection algorithm~\cite{Xu2017}. AnswerBot poses the problem of finding task-relevant text 
as a query-based extractive summarization problem and it identifies relevant text based on 
the content of the text, how similar that content is with regards to a task and the structured data available on each of the answers in a Stack Overflow post (i.e., number of votes or whether an answer is the accepted answer).


\subsection{Machine Learning Approaches}


\acf{ML} approaches take the text of a natural language software artifact and identify 
the sentences likely relevant to a particular software task using \textit{unsupervised} or 
\textit{supervised learning} methods~\cite{zhang2005machine}.



Supervised learning approaches use a set of features and labeled data (which often requires significant manual effort) to train classifiers in detecting relevant sentences.
We have already presented supervised approaches that use text summarization (\textit{i.e.,}~\cite{Rastkar2010})
and there are also approaches that classify a tutorial fragment as relevant or not~\cite{Jiang2016b}.
While effective, supervised approaches may be outperformed by more light-weight approaches such as 
pattern matching~\cite{Bavota2016}.
As an example, Chaparro    
and colleagues 
compared their pattern matching approach ({\small DeMIBuD}) to a supervised version of {\small DeMIBuD} finding that the 
ML approach did not provide significant gains in the classification of relevant text in bug reports~\cite{Chaparro2017}.



Unsupervised learning approaches do not require labelled data and determine relevant sentences according to properties inferred from the data. DeepSum~\cite{Li2018} and Lotufo et al.'s~\cite{Lotufo2012} techniques are examples of unsupervised approaches in the scope of text summarization. Other unsupervised approaches (\textit{i.e.,} {\small FRAPT}~\cite{Jiang2017} or HoliRank~\cite{Ponzanelli2015, Ponzanelli2017}) are mostly based around variations of the PageRank~\cite{Page1999} or LexRank~\cite{Erkan2004} algorithms explained early in this thesis. 




The techniques that we have presented
are specific to one kind of artifact and use properties of the kind of
artifact. In contrast, the techniques we
investigate in this dissertation (Chapter~\ref{ch:identifying}) seek to identify 
task-relevant text across different artifact types.
\section{Automatic Approaches for Identifying Text in Natural Language Software Artifacts}
\label{cp2:text-approaches}



Information useful to a software task can be buried in irrelevant text or attached to 
non-intuitive blocks of text, making it difficult to discover~\cite{Robillard2015}.
Researchers have long recognized the value of assisting developers in 
identifying information of relevance in this natural language
text.
In this section, we detail tools and approaches from related work.




\subsection{Pattern Matching Approaches}
\label{cp2:pattern-matching}


Pattern matching approaches rely on regular expressions describing a sequence of tokens that represent
 a relevant text fragment~\cite{Bavota2016}. Tokens can either represent words or linguistic elements 
extracted using \acf{NLP}.
    
    
As examples  of pattern matching approaches,  {\small DeMIBuD}~\cite{Chaparro2017}
 and Knowledge Recommender (Krec)~\cite{Maalej2013, Robillard2015} are tools that detect relevant sentences in bug reports and API documentation, respectively. 
These tools derive a set of patterns from annotated data and use them as part of heuristics 
that identify relevant text. Krec assumes that any relevant sentence mentions a 
code element (e.g., a class or method name) and it uses  361 unique patterns
to 
detect relevant sentences in API documentation~\cite{Robillard2015}.
In a similar manner, {\small DeMIBuD} uses a set of 154 discourse patterns to detect sentences 
relevant to understanding a bugs observed or expected behaviour and steps to reproduce it,
which are essential to bug triaging tasks.




In Stack Overflow posts,
Nadi and Treude~\cite{nadi2020} have both applied the original set of patterns from Krec~\cite{Robillard2015} 
and proposed heuristics that rely on the conditionality of the text
to identify sentences that help a developer 
decide whether they want to carefully inspect a Stack Overflow posts or skip over it. 



Although the heuristics and regular expressions used in the aforementioned studies 
are often light-weight and effective~\cite{Bavota2016, Maalej2013}, 
pattern matching approaches are often specific to one kind of domain and 
type of artifact~\cite{fucci2019}. 





\subsection{Summarization Approaches}
\label{cp2:summarization}



Extractive text summarization techniques are used in natural language artifacts in software engineering to
produce a summary of the artifact's content. These summaries aim to represent key information that may help a developer complete their task~\cite{Bavota2016}.
There are summarization techniques based on both supervised and unsupervised learning~\cite{moreno2017}
and one can summarize the entire content of an artifact
or content specific input query, as in query-based summarization~\cite{Huang2018, Goldsteinet1999}.




A number of summarization approaches target bug reports and GitHub issues, largely
focusing on key information within these artifacts. 
Rastkar and colleagues~\cite{Rastkar2010} use a supervised learning approach to summarize the content 
of bug reports showing that conversational features used to summarize emails~\cite{Murray2008}
can also be applied to bug reports while
Lotufo et al.~\cite{Lotufo2012} proposed an unsupervised summarization approach 
that automates the identification of sentences that a developer would first read when
a inspecting bug report.



While many of the approaches described above
largely rely on  lexical aspects in text, researchers have also made use
of structured textual information in the artifacts~\cite{Ponzanelli2015, Treude2016, chen2016}. 
For example, Ponzanelli et al. 
proposed a summarization technique that mixes natural language text and structured data 
available on Stack Overflow
to produce more accurate summaries for Stack Overflow answers~\cite{Ponzanelli2015}. 
As another example, DeepSum~\cite{Li2018} pre-processes a bug report dividing sentences 
containing software elements, the reporter of the bug, and any other sentences 
in the bug report to produce summaries containing more diverse information.




A smaller number of summarization approaches have focused
producing task specific summaries.
These approaches pose the problem of finding task-relevant text 
as a query-based extractive summarization problem and
tools such as AnswerBot~\cite{Xu2017}
identify relevant text in Stack Overflow posts 
based on 
the content of the text, how similar that content is with regards to a input query (i.e., task)
and the structured data available on each of the answers in a Stack Overflow post 
(i.e., number of votes or whether an answer is the accepted answer).
Chapter~\ref{ch:identifying}
compares AnswerBot to the techniques that we explore in this thesis.



\subsection{Machine Learning Approaches}
\label{cp2:machine-learning}


\acf{ML} approaches take the text of a natural language software artifact and identify 
the sentences likely relevant to a particular software task using \textit{unsupervised} or 
\textit{supervised learning} methods~\cite{zhang2005machine}.



Supervised learning approaches use a set of features and labeled data
 to train classifiers with the goal of identifying sentences relevant to 
 certain software activity.
We have already presented supervised approaches that use text summarization (\textit{i.e.,}~\cite{Rastkar2010})
and there are also approaches that identify relevant 
parts of software tutorials~\cite{Jiang2016b}
or API documents~\cite{fucci2019, Maalej2013}
and despite their value, 
the cost and effort of hiring skilled workers to produce 
labeled data in software engineering artifacts 
has been a major limitation 
to the usage of supervised learning 
methods in software engineering~\cite{aa}.





Unsupervised learning approaches do not require labelled data and determine 
relevant sentences according to properties inferred from the data. 
DeepSum~\cite{Li2018} and Lotufo et al.'s~\cite{Lotufo2012} techniques are examples of 
unsupervised approaches in the scope of text summarization. Other unsupervised approaches 
(\textit{i.e.,} {\small FRAPT}~\cite{Jiang2017} or HoliRank~\cite{Ponzanelli2015, Ponzanelli2017})
are mostly based around variations of the PageRank~\cite{Page1999} or LexRank~\cite{Erkan2004} algorithms. 
These algorithms represent all the text in an artifact as a graph.
Then, they establish relationships (\textit{i.e.,} weighted edges in the graph) 
between different sentences (\textit{i.e.,} nodes in the graph) 
and select the nodes with highest weights as the most relevant ones.
A crucial step in building the graph is in the definition of 
how to establish  relationships between nodes.
Early approaches~\cite{Lotufo2012, Jiang2017} 
use \ac{VSM}~\cite{Salton1975vsm} 
for this purpose while more modern ones~\cite{Huang2018, silva2019}
use different word embeddings~\cite{Mikolov2013, bojanowski2017FastText},
which we detail in Section~\ref{cp2:deep-learning}.
\red{TODO}










\subsection{Deep Learning Approaches}
\label{cp2:deep-learning}



One substantial challenge of standard \acf{ML}
approaches is that researchers must engineer which 
features or properties of the text to use~\cite{ferreira2021}.
For example, Rastkar et al. uses conversational features in 
the text of a bug report to assist in determining which sentences 
to include in the bug summary~\cite{Rastkar2010}
while Petrosyan and colleagues use 
linguistic and structural properties 
in the text of API documents to determine text 
explaining API elements~\cite{Petrosyan2015}
and given the specificity of such features, 
researchers often question the generalizability
of standard \acs{ML} approaches~\cite{Xiao2018, fucci2019}.



In contrast to the human engineered features,
\acf{DL} approaches allow the automatic extraction of features 
from textual data through a series of mathematical transformations~\cite{Deng2018, zhang2021deep}.
Deep learning has lead to groundbreaking advancements in many 
research areas (e.g., machine translation~\cite{lopez2008translation}) 
and, given its wide range of applications, this section
focuses on its usage in natural language text appearing in software engineering artifacts~\cite{ferreira2021, li2018deep, sharafi2015}.



% and an in-depth explanation of the field is beyond 
% our scope. Therefore, we present \acf{DL} 
% concepts honing in on its applications research.



% At some cost~\cite{strubell2020}, a \acs{DL} neural-network 
% can derive which properties of the text 
% most accurately assist in determining the outcome of some classification 
% task 



% \acf{DL}, which walked hand-to-hand with improvements in computational power and the amount of memory available in modern computer architectures~\cite{}.






\acs{DL} approaches are of particular interest 
software engineering researchers 
since they assist in identifying hidden patterns 
in the natural language text available,
what has ushered in advancements in software engineering areas 
including


~\cite{sharafi2015}




~\cite{sharafi2015} 



can gather diverse corpora, neural networks 








At times, software engineering researchers have argued
that general lexicon techniques 
are insufficient to address text appearing in
software engineering artifacts. 
% Arguments on why lexicon-based natural 
% language techniques are not applicable are often based
% on a need for access to the \textit{meaning}, or semantics, 
% of words, phrases or sentences appearing in the text~\cite{jurafsky2014speech}.
% In this section, 
% we present background information on semantics focusing on
% its usage in software engineering research.



% \subsection{Word Semantics}

Word semantic techniques are mostly rooted on the hypothesis
that similar words appear in similar context~\cite{harris1954distributional}.
This hypothesis gave origin to a series of
\textit{distributional semantic models}~\cite{Ye2016} that aim to infer the meaning of words.
% In this section, we present prominent models used by software engineering researchers.



Distributional semantic models have been used by software engineering researchers 
to improve the 
the retrieval of artifacts pertinent to a certain task. 
Early models, such as \acf{LSI}~\cite{deerwester1990LSI}, 
have been used to, for example, recover traceability links between source code and
software documentation~\cite{marcus2003}.  
\acs{LSI} takes a initial word representation (i.e., a term by document matrix) and applies \acf{SVD}~\cite{klema1980SVD}
to reduce the dimensionality of this matrix, what causes 
words with similar meaning have the same final representation.


Other word semantic models have assisted software engineering researchers in clustering semantically similar artifacts~\cite{zhang2014, layman2016}. For that, researchers have mostly used
\acf{LDA}~\cite{blei2003latent}---a model that assumes that words used in a similar context often pertain to the same subject to produce topics clustering sentences or documents containing semantically related words---to
identify common themes in developers' blog posts~\cite{Pagano2011} or to design tools that identify duplicated bug reports~\cite{nguyen2012, Thung2014}.


% ---among its many applications~\cite{zhang2014, layman2016}---


% Marcus and Maletic apply \acs{LSI} to 

% In software engineering, \acs{LSI} has been widely used to assist requirements traceability~\cite{lucia2007, hayes2006, gethers2011}.

Despite their significant contributions, early models created word vector representations
by counting the frequency or co-occurrence of words, what is substantially inefficient for large corpora~\cite{Ye2016}.
This and other challenges have been lifted by advancements in the fields of \acf{ML} and \acf{DL}~\cite{ferreira2021, li2018deep}, which walked hand-to-hand with improvements in computational power and the amount of memory available in modern computer architectures~\cite{sharafi2015}.


\acs{DL} models built with \textit{neural, or word, embeddings}~\cite{Mikolov2013} 
are of particular interest to this thesis. 
Neural embeddings produce vector representations in a continuous space 
and researchers have shown that they 





% 

\section{Interpreting the Semantics of Text}
\label{cp2:artifact-semantics}


% In Section~\ref{cp2:ir-approaches}




At times, software engineering researchers have argued
that general lexicon techniques 
are insufficient to address text appearing in
software engineering artifacts. 
Arguments on why lexicon-based natural 
language techniques are not applicable are often based
on a need for access to the \textit{meaning}, or semantics, 
of words, phrases or sentences appearing in text~\cite{jurafsky2014speech}.
In this section, 
we present background information on semantics focusing on
its usage in software engineering research.



\subsection{Word Semantics}

Word semantic techniques are mostly rooted on the hypothesis
that similar words appear in similar context~\cite{harris1954distributional}.
Hence, word semantic techniques use statistical models to infer the meaning of words. 


Early word semantic techniques, such as \acf{LSI}~\cite{deerwester1990LSI}, 
have been used by software engineering researchers 
to improve the 
the retrieval of artifacts pertinent to a certain task. 
For example, Marcus and Maletic apply \acf{LSI} to 
recover traceability links between source code and
software documentation~\cite{marcus2003}.
Other techniques, such as \acf{LDA}~\cite{blei2003latent},
assist to group and cluster textual information 
what allowed researchers to identify common topics in developers' blog posts~\cite{Pagano2011}
or facilitated the design of tools that identify duplicated bug reports~\cite{nguyen2012}.








\subsection{Sentence Semantics}







Surprisingly 
most of the deep learning approaches applied to software engineering tasks 
focus on source code and 

natural language text has been very shy....



% Semantic-based approaches  build a meaning by establishing relationships between terms,
% for instance words, based on the context in which these terms appear.
% For example
% the terms `\textit{car}' and `\textit{automobile}' are likely to co-occur in different phrases with terms like `\textit{motor}' and `\textit{wheel}' and thus,
% semantic approaches indicate that these terms are likely similar in meaning~\cite{Bavota2016}.

% As an example of the usage of semantics in software engineering researcher, Marcus and Maletic applied Latent Semantic Indexing (LSI) to help cluster software components to aid
% program comprehension of a software systems~\cite{marcus2003}. 
% Ye et al.'s study on the usage of word embeddings~\cite{Ye2016} was among the first to use the language models in the software engineering domain~\cite{Mikolov2013}. 

% Researchers have also developed approaches
% to interpret the meaning of sentences in
% software engineering documents. 
% Maalej and Robillard have
% developed a knowledge taxonomy~\cite{Maalej2013} using grounded theory to analyze documentation in open-source systems and then, they validated their taxonomy on documentation units sampled from the Java SDK 6 and .NET 4.0.
% Arya et al.'s have determined information types in open-source issue discussions~\cite{Arya2019},  Di Sorbo et al. have created an approach to classify  emails based on a developer's intentions~\cite{Sorbo2015} and
%  Marques et al. have investigated the use of frame semantics~\cite{fillmore1976frame}.
%  The techniques we propose in this paper
%  build on these earlier findings.





% \subsection{Machine Learning Approaches} 


% To bridge lexical gaps, researchers have investigated 
% approaches that leverage the 
% semantic aspects of software artifacts~\cite{Maletic2001, Ye2016}.
% While Section~\ref{cp2:artifact-semantics} provides an overview of semantics 
% in software engineering, this section especifically details how semantics might assist in finding 
% pertinent artifacts. 









%  is an early example of an \acs{IR} method that address \textit{lexical mismatches}.

% These and more advanced approaches 


% or when Nguyen and colleagues
% applied Word2Vec~\cite{mikolov2013word2vec} to support the retrieval of API
% examples~\cite{nguyen2017}.




% While ML approaches are effective in locating pertinent artifacts, a developer still needs to
% manually find the parts within that artifact
% that are relevant to her task~\cite{Cubranic2005}. 







% they have been used 
% for several purposes by software engineering researchers.



% While it is obvious to the human reader that `car' and `automobile' are synonyms and can be used interchangeable,  
% standard \acs{IR}  fail to capture this and other differences\footnote{
%     \textbf{Oxford English dictionary definitions}~\cite{dictionary1989oxford} 
%     \begin{itemize}
%         \item \textbf{synonymy:} the fact of two or more words having the same meaning;
%         \item \textbf{polysemy:} the fact of having more than one meaning.
%     \end{itemize}
% }
% that significantly affected the performance of \acs{IR} systems~\cite{DeLucia2012}.
% To address these issues, researchers have proposed a set of word semantic techniques the use statistical models to infer the meaning of words. 


% 

\art{All sections follow the same structure: a section presents related work and then it finishes with a statement  on how the thesis relates to the work presented.
I tried to be more up-front, but I did not want to abruptly end each section. }


\section{On Improving Developers' Productivity}
\label{cp2:dev-productivity}

\art{Use contributions to drive related work}

\art{I might even talk about Google search somewhere here}


To understand how software developers find  information useful to their software tasks,
software engineering researchers have mostly used theories from the information foraging
field~\cite{Pirolli1999}. 


\textit{\acf{IFT}}~\cite{Pirolli1999} explains how 
an individual navigates through some search space looking for \textit{information patches} based on 
a set of cues about the effort and gains that a patch provides to them~\cite{Pirolli1999}.
Foraging occurs between or within patches, i.e., 
searching which documents are relevant and what within a document is of relevance.
In turn, \textit{relevance theory}~\cite{clark2013relevance, saracevic1975} further elaborates the relationships between an information object (i.e., text in an artifact),
some context (i.e., a task), and what properties (i.e., clarity, utility, completeness, etc.) guide a forager decision 
on the relevance of a patch~\cite{Saracevic2007b, Saracevic2007c}. 



Among the many information foraging studies in software engineering~\cite{Piorkowski2015, Piorkowski2016, chi2007, Ko2006a},
 Forward and Lethbridge describe criteria that developers use to assess the relevance of software engineering documentation. They observe that the content of an artifact, the presence of examples, and the document's structure are common attributes that impact an artifact's relevance~\cite{Forward2002}.
In another study, Charrada and Mussato investigated how 
practitioners manage and search for software engineering documents, 
identifying that data scattered across different sources is 
often a significant challenge to the information foraging process~\cite{BenCharrada2016}.



When attempting to find artifacts pertinent to a task, 
Starke et al. explain that a software developer often makes a set of hypotheses about a software task,
translating these hypotheses into multiple search queries that will lead to  likely pertinent artifacts~\cite{Starke2009}. 
They observe that, instead of systematically inspecting search results,
developers often skim through them to decide which documents are of relevance. 
In other studies, Brandt et al. found that developers interleave web foraging, learning and writing code~\cite{Brandt2009a} while de Graaf and colleagues 
identified that prior knowledge about the structure of a software artifact helps professionals
to search software documents efficiently and effectively~\cite{DeGraaf2014}.


This thesis adds to the existing theory by discussing 
common properties in the text of an artifact which several software developers deemed relevant 
to certain software tasks.  



% \section{Finding Pertinent Artifacts}
\label{cp2:foraging-tools}

% \art{I tried  summarize this section, but it still looks lengthy}


Software development often requires knowledge beyond what developers already posses~\cite{Li2013} and thus, 
software developers use different information sources to fill such gaps. 
More than often, a developer will ask if any of their 
peers have the information that they need~\cite{singer2011}. 
However, the fragmented and distributed nature of software development  
might prevent a developer from approaching their peers~\cite{ko2007}.
Due to this and other reasons~\cite{Xia2017, rao2020}, a developer might seek
online web resources for information 
that may assist them in completing the task-at-hand.
This section provides a breadth of approaches that help a developer find such resources.




\subsection{Information Retrieval Approaches} 
\label{cp2:ir-approaches}





Standard \acf{IR} approaches locate pertinent artifacts
using metrics to compute the relevance of an artifact to a query posed manually by a developer.
\acs{IR} uses the terms shared between a search query and an artifact
to compute the relevance of an artifact (or document).
Search algorithms weight and compute relevance based on 
criteria such as how many documents contain a given term or  
how unique a term is~\cite{Manning2009IR}.



A number of the techniques commonly employed by software engineering researchers are based on the
frequency of co-occurrence of words (or phrases) in documents.
An early example is Maarek and Smadja's use of lexical relations to index
software libraries~\cite{maarek1989}.
Since this early use, software engineering
researchers have continued to leverage advances in
these approaches, such as when
Antoniol et al. applied \acf{VSM}~\cite{Salton1975vsm} 
to construct vector representations 
for recovering traceability links 
between code and documentation~\cite{antoniol1999, antoniol2000}
or when Marcus and Maletic used \acf{LSI}~\cite{deerwester1990LSI}
to help cluster software components to aid
program comprehension of a software system~\cite{marcus2003}.



While IR systems are widely used, researchers 
observed that developers often find it difficult to identify good search terms~\cite{Kevic2014, Huang2018},
what significantly impacts the retrieval of relevant documents~\cite{Kevic2014, mills2017}.
To mitigate  the necessity of a developer coming up with good search terms,
researchers have used information available in a task to automatically generate search terms~\cite{Kevic2014, Haiduc2013}. 
Although automatically generating search terms assist the retrieval of pertinent artifacts,
standard \acs{IR} algorithms will have 
little success if search terms 
differ significantly from the text in a software artifact~\cite{Huang2018}.
These so-called \textit{lexical mismatches}~\cite{Ye2016, silva2019} have led researchers to investigate 
 semantic-based \acs{IR} approaches, which we discuss in 
Section~\ref{cp2:artifact-semantics}.




% 


\subsection{Contextual Approaches} 


Contextual approaches use information 
about a developer's task so that 
they recommend artifacts pertinent to that task drawing data from some knowledge base.
% , for example, a tool might need to know what part of the system is a developer currently inspecting. 
Fishtail~\cite{Sawadsky2011}
is an example of a tool that uses contextual information. It uses the source code currently being inspected by a developer to find web resources relevant to the developer's task.



While the web is the most accessible knowledge base, certain contextual approaches consider that 
collection of artifacts produced and consumed in past software tasks 
 implicitly
form knowledge bases that might assist developers performing new tasks~\cite{Cubranic2005}. 
Hipikat~\cite{Cubranic2005} is a seminal tool that exemplifies this concept in action.
It automatically tracks the artifacts 
used by a developer as part of a development tasks
so that it can recommend these artifacts to 
any future developers who work on similar tasks~\cite{Cubranic2005}.
As another example, Deep Intellisense
finds code changes, filed bugs, and forum discussions 
mining these artifacts from an organization's shared database~\cite{Holmes2008}.










In the past decade, question-and-answer (\acs{qa}) web sites such as Stack Overflow\footnote{\url{https://stackoverflow.com/}}, have also
become a common knowledge source used by contextual approaches~\cite{Ponzanelli2013b, Ponzanelli2014b, Treude2016, delfim2016}.
For example, {\small PROMPTER} uses contextual information 
available in a developer's IDE to retrieve Stack Overflow discussions that might be pertinent 
to the developer's current task. Other tools such as {\small BIKER}~\cite{Huang2018} mine Stack Overflow posts to assist developers in locating API components needed in a programming tasks. 





This thesis assumes the usage of one or more techniques presented here to locate pertinent
artifacts from which task-relevant text will be extracted.




% \section{Identifying Relevant Information within an Artifact}
\label{cp2:artifact-approaches}
% 

\section{Interpreting the Semantics of Text}
\label{cp2:artifact-semantics}


% In Section~\ref{cp2:ir-approaches}




At times, software engineering researchers have argued
that general lexicon techniques 
are insufficient to address text appearing in
software engineering artifacts. 
Arguments on why lexicon-based natural 
language techniques are not applicable are often based
on a need for access to the \textit{meaning}, or semantics, 
of words, phrases or sentences appearing in text~\cite{jurafsky2014speech}.
In this section, 
we present background information on semantics focusing on
its usage in software engineering research.



\subsection{Word Semantics}

Word semantic techniques are mostly rooted on the hypothesis
that similar words appear in similar context~\cite{harris1954distributional}.
Hence, word semantic techniques use statistical models to infer the meaning of words. 


Early word semantic techniques, such as \acf{LSI}~\cite{deerwester1990LSI}, 
have been used by software engineering researchers 
to improve the 
the retrieval of artifacts pertinent to a certain task. 
For example, Marcus and Maletic apply \acf{LSI} to 
recover traceability links between source code and
software documentation~\cite{marcus2003}.
Other techniques, such as \acf{LDA}~\cite{blei2003latent},
assist to group and cluster textual information 
what allowed researchers to identify common topics in developers' blog posts~\cite{Pagano2011}
or facilitated the design of tools that identify duplicated bug reports~\cite{nguyen2012}.








\subsection{Sentence Semantics}







Surprisingly 
most of the deep learning approaches applied to software engineering tasks 
focus on source code and 

natural language text has been very shy....



% Semantic-based approaches  build a meaning by establishing relationships between terms,
% for instance words, based on the context in which these terms appear.
% For example
% the terms `\textit{car}' and `\textit{automobile}' are likely to co-occur in different phrases with terms like `\textit{motor}' and `\textit{wheel}' and thus,
% semantic approaches indicate that these terms are likely similar in meaning~\cite{Bavota2016}.

% As an example of the usage of semantics in software engineering researcher, Marcus and Maletic applied Latent Semantic Indexing (LSI) to help cluster software components to aid
% program comprehension of a software systems~\cite{marcus2003}. 
% Ye et al.'s study on the usage of word embeddings~\cite{Ye2016} was among the first to use the language models in the software engineering domain~\cite{Mikolov2013}. 

% Researchers have also developed approaches
% to interpret the meaning of sentences in
% software engineering documents. 
% Maalej and Robillard have
% developed a knowledge taxonomy~\cite{Maalej2013} using grounded theory to analyze documentation in open-source systems and then, they validated their taxonomy on documentation units sampled from the Java SDK 6 and .NET 4.0.
% Arya et al.'s have determined information types in open-source issue discussions~\cite{Arya2019},  Di Sorbo et al. have created an approach to classify  emails based on a developer's intentions~\cite{Sorbo2015} and
%  Marques et al. have investigated the use of frame semantics~\cite{fillmore1976frame}.
%  The techniques we propose in this paper
%  build on these earlier findings.





% \subsection{Machine Learning Approaches} 


% To bridge lexical gaps, researchers have investigated 
% approaches that leverage the 
% semantic aspects of software artifacts~\cite{Maletic2001, Ye2016}.
% While Section~\ref{cp2:artifact-semantics} provides an overview of semantics 
% in software engineering, this section especifically details how semantics might assist in finding 
% pertinent artifacts. 









%  is an early example of an \acs{IR} method that address \textit{lexical mismatches}.

% These and more advanced approaches 


% or when Nguyen and colleagues
% applied Word2Vec~\cite{mikolov2013word2vec} to support the retrieval of API
% examples~\cite{nguyen2017}.




% While ML approaches are effective in locating pertinent artifacts, a developer still needs to
% manually find the parts within that artifact
% that are relevant to her task~\cite{Cubranic2005}. 







% they have been used 
% for several purposes by software engineering researchers.



% While it is obvious to the human reader that `car' and `automobile' are synonyms and can be used interchangeable,  
% standard \acs{IR}  fail to capture this and other differences\footnote{
%     \textbf{Oxford English dictionary definitions}~\cite{dictionary1989oxford} 
%     \begin{itemize}
%         \item \textbf{synonymy:} the fact of two or more words having the same meaning;
%         \item \textbf{polysemy:} the fact of having more than one meaning.
%     \end{itemize}
% }
% that significantly affected the performance of \acs{IR} systems~\cite{DeLucia2012}.
% To address these issues, researchers have proposed a set of word semantic techniques the use statistical models to infer the meaning of words. 
