\setcounter{chapter}{1}


\chapter{Related Work}
\label{ch:related-work}

\art{make related work more in line with ICSME.}

Information foraging~\cite{Pirolli1999}, or the process through which an individual 
obtains knowledge that assists them accomplish a task often experiences paradigm shifts. 
At times, 
an information need would be directed to a library clerk who would suggest books for the person's search~\cite{saracevic1975}.
Then, when the \acf{web}~\cite{berners1994web} became popular,
individuals started to rely on search engines 
to seek information~\cite{Page1999} now digitally stored in web pages.
Searching for pertinent artifacts, be them books or web pages, 
is one of the first steps to address an information need,
which we detail in Section~\ref{cp2:searching}.


In possession of a pertinent artifact, careful 
inspection of its content might lead a person 
to find the information that they sought. 
However, due to some of the challenges
discussed in the previous chapter, researchers from several 
disciplines have considered means to help 
in this activity. 
In Section~\ref{cp2:general-approaches}, 
we detail general approaches that mine information
from natural language text. 
Then, in Section~\ref{cp2:task-approaches}, 
we hone in on approaches that
automatically identify text 
containing information likely relevant to an input software task.




% Most of the techniques that extract information from natural language software artifacts
% are examples of applications that help developers in performing 
% software development activities~\cite{Meyer2017}. 
Given how 
natural language artifacts have become intrinsically
tied to software development~\cite{umarji2008archetypal},
we conclude this chapter with an overview of 
other applications that make use of textual data
to help developers
across many of the tasks in their daily work (Section~\ref{cp2:dev-productivity}).



% In the last chapter, we introduced some of the challenges
% associated with finding
% information relevant to a developer's task.
% In this chapter, we provide background information on 
% a number of empirical studies that give insight into
% the data available in natural language software artifacts 
% and we also present 
% automated tools that extract text from these artifacts.
% In Section~\ref{cp2:text-in-se}, we begin by providing an  
% overview of the types of textual analysis in the literature
% then, in Sections~\ref{cp2:general-approaches}
% and~\ref{cp2:task-approaches}, we
% detail general textual approaches 
% and task-specific approaches
% that identify and extract text from natural language software artifacts, respectively.
% First, we provide an overview on the kinds of approaches 
% used to extract data from natural language software artifacts.
% Then, we detail how these approaches 
% have been used for automating the identification of text 
% likely relevant to a software task~\cite{a}.







\section{Finding Pertinent Artifacts}
\label{cp2:searching}


We start by considering how an individual finds 
artifacts, or documents, which might 
address an information need. 
For this, information foraging theory~\cite{Pirolli1999} explains how a person navigates through  
a search space looking for information patches  based on 
a set of cues about the effort and gains that a patch provides to them.


According to Pirolli and Card~\cite{Pirolli1999}, a forager performs scent-following activities, which represent 
how they navigate a search space until they find a high-yield patch. 
Once in a patch, a forager inspects its content until their information need is addressed or until the amount of new information available depletes, making the forager move to another patch. 


In both instances (navigating between patches and searching within a patch), \textit{explicit} or \textit{implicit} factors affect how a forager judges the relevance of a patch~\cite{saracevic1975}.
For example,
consider an electronic library catalog and a search for `\textit{android notifications}'.
Figure~\ref{fig:library-catalogue} shows two hypothetical results for this search. %to help illustrate explicit and implicit factors
An explicit factor might represent how one of the keywords in the `\textit{subject and genre}' field
directly match one of the keywords used in the person's query. An implicit factor might represent 
a person's background or previous knowledge, such as how 
they might know that the first book targets a more general population and 
that it might not be helpful for a software developer.







A similar search could be performed on a web search engine\footnote{e.g., \href{https://www.google.com/}{Google} or \href{https://www.bing.com/}{Microsoft Bing}}
and, with their prominence, other search problems also gained attention.
Most notably,  Carbonell and Goldstein investigated the relevance and the novelty of search results~\cite{Carbonell1998}.
On the one hand, prioritizing relevance may lead to a scenario where all results contain redundant information, leaving some information needs unanswered. 
On the other hand, it might be difficult for an individual 
to find information that corroborates and consolidates some knowledge if search results are too diverse~\cite{clark2013relevance}.  
This has led researchers to investigated 
algorithms that balance the relevance and the novelty of the results retrieved
by a web search~\cite{najork2001, rafiei2010, vieira2011}
and commercial search engines have since considered how to strike a balance
between relevance and novelty, seeking to provide diverse  
yet relevant results.



% and for both library catalogs or web searches, researchers have investigated 
% what affects how a person finds pertinent results and their decision to inspect them.
% For example, in electronic library catalogs,
% Hildreth observed how keyword searches were used more than any other type of search (e.g., by author or by title)~\cite{hildreth1997}
% while other studies observed impatience and near-random search habits in novice librarians~\cite{novotny2004don}.
% In contemporary web search engines, researchers have  also 
% investigated how individuals formulate queries~\cite{gross2005have, bendersky2012},
% how prior knowledge helps them
% to more efficiently perform searches~\cite{DeGraaf2014},
% and what search results they inspect, going to the lengths
% of using eye-tracking technology for this purpose~\cite{Cutrell2007, marcos2015}.



\vspace{3mm}
\begin{figure}[H]
\noindent\fbox{%
\parbox{\textwidth}{%
{\ttfamily% 
    \begin{scriptsize}
        \begin{enumerate}[itemindent=-1em,leftmargin=1em]
            \item \textbf{Android for Dummies} 

            Author: Gookin, Dan 

            Location: Britannia Branch Library 

            LC Call No.: 004.167 A57G6ad 

            Subject and genre: Android (electronic resource), Smartphone, Mobile computing

            % Synopsis: This book will tell you pretty much everything you need to know about your Android smartphone or tablet in an equally friendly manner, because that's the best way to learn how to get the most from your Android

        \end{enumerate}          
    \end{scriptsize}
    }%
}}
\noindent\fbox{%
\parbox{\textwidth}{%
{\ttfamily% 
    \begin{scriptsize}
        \begin{enumerate}[itemindent=-1em,leftmargin=1em]\setcounter{enumi}{1}
            \item \textbf{Learning Android} 

            Author: Gargenta, Marko 

            Location: Champlain Heights Branch Library 

            LC Call No.: 005.44 A5G2L1 

            Subject and genre: Android (electronic resource), Application software (development), Mobile computing

            % Synopsis: Presents an introduction on the fundamentals of Android to create a variety of applications.

        \end{enumerate}
    \end{scriptsize}
    }%
}}
\caption{Sample library catalog with two results associated with the keyword `android' }
\label{fig:library-catalogue}
\end{figure}








Across information science studies that investigate information foraging, 
a common trend is that the identification of good search terms is as, or even more
important than the search algorithm itself~\cite{Kevic2014}. 
Nonetheless, identifying good search terms is often challenging and 
as a consequence, many searches are unsuccessful or retrieve resources that 
a person loses time inspecting only to find that they are not pertinent to the task at hand~\cite{novotny2004don, Haiduc2013}.
This indicates that individuals usually inspect multiple and potentially different types of artifacts 
before finding relevant information, what further motivates the work presented in this thesis.




Within software engineering research, the concept of an information patch 
is often more fine-grained and it usually relates to source code, where a significant body of work has 
used information foraging theory in the context of software documentation~\cite{Forward2002, DeGraaf2014, Wildemuth2012},
program comprehension~\cite{piorkowski2013, Ko2006a}, and debugging or refactoring activities~\cite{fleming2013, lawrance2010}. 
More recently, researchers have also investigated developers' web search 
behaviour~\cite{Starke2009, Brandt2009a, Xia2017}, concentrating on what developers search for on web search engines. 
We differ from these studies by focusing 
on the content within different kinds of natural language artifacts, i.e., 
the natural language text within them.




% which contrasted how more experienced users were able to better translate an information 
% need into keyword searches that retrieved useful results





\section{Textual Approaches in Natural Language Software Artifacts}
\label{cp2:general-approaches}


This section provides a general overview of 
background information on automated textual approaches
applied to natural language software artifacts. 




\subsection{Pattern Matching Approaches}
\label{cp2:pattern-matching}


Regularities in the terms and in the syntactic structure of the 
might be automatically identified via pattern matching.
Pattern matching approaches use regular expressions describing a sequence of tokens that represent
the text to be identified~\cite{Bavota2016}. 
We describe two tools, Krec~\cite{Robillard2015}  and DeMIBuD~\cite{Chaparro2017}, that
illustrate how software engineering 
researchers use pattern matching in conjunction with the lexical or linguistic elements 
in the text to automatically identify 
text useful to certain software development activities.
    


Knowledge Recommender (Krec)~\cite{Robillard2015} 
is an example of a
 tool that uses lexical patterns to 
 automatically detect relevant text in  API documentation. 
Krec's premise is that relevant sentences contain a code element, such as a method or class signature.
These code elements are identifiable via regular expressions 
and Krec contains a catalog of 361 unique patterns 
that identify threats and directives on how to use some API element.
For example, Krec uses the pattern {\small \textit{$\{$may}, \textit{efficient}, \textit{code element regex$\}$}} 
to identify sentences giving instructions about an efficient way to 
perform some operation. 



{\small DeMIBuD} is a linguistic-based approach that 
automatically detect sentences discussing steps to reproduce 
a bug or the bug's expected behaviour~\cite{Chaparro2017}.
It uses a set of 154 discourse patterns
derived from nearly 3,000 bug reports 
to identify such sentences. 
For example, the pattern 
{\small \textit{$\{$subject}, \textit{should/shall (not)}, \textit{complement$\}$}}
captures common ways with which developers describe a system's expected behaviour
and empirical assessment of the patterns used by the tool has shown that it 
detects sentences of interest in bug reports with high accuracy.






Although the heuristics and regular expressions used in these and other studies~\cite{nadi2020, Maalej2013}
are lightweight and effective~\cite{Bavota2016}, 
pattern-matching approaches 
are often specific to certain kinds of domains and types of artifact~\cite{fucci2019}, 
limiting their use in the design of a generalizable technique, which is the main goal of this thesis.







\subsection{Machine Learning Approaches}
\label{cp2:machine-learning}


Regularities in the text, or in an artifact's meta-data, can also be 
engineered into features that \acf{ML} 
approaches can leverage to automatically identify and classify
text useful to certain software development activities. 


% where
% one can train a classifier
% to predict whether a sentence belongs (or not)
% to a certain class. 


Researchers pose the problem of identifying text 
in a natural language artifact 
as a binary classification problem. 
That is, the use of a number of features 
to predict whether the text is (or not)
relevant to some context~\cite{}.
In software engineering, 
binary classifiers have been used for,
for example, classify text that describes steps to reproduce a bug~\cite{Chaparro2016} or 
classify text that explains a certain API element, as when 
Petrosyan et al. used 
 sentence-level features
and meta-data features in a classifier 
that 
identifies explanations about an API element  in a web tutorial~\cite{Petrosyan2015}.




Other classification problems predict which class, out of many, some input text belongs to. 
This type of classification, often referred to as multinomial or multi-class classification, 
is of particular interest if 
we consider the taxonomies described in Section~\ref{cp2:text-in-se}.
For example, Arya et al. identified 16 categories of  information available
in open source GitHub issues (e.g., workarounds, solution discussion, task progress, etc.)~\cite{Arya2019}
and they proposed a multinomial classifier 
to automatically identify such categories.
% so that a developer could quickly find information pertaining a certain category. 








Although valuable, the cost and effort of hiring skilled workers to produce 
the labeled data for these and other supervised learning approaches
has been a major limitation 
to the usage of supervised learning 
methods in software engineering research~\cite{Arpteg2018, ferreira2021}. As an alternative,
researchers have also explored 
 unsupervised learning methods---\acs{ML} techniques that do not required training data---for the automatic 
identification of 
key information in natural language artifacts~\cite{}.





A common application of unsupervised learning in software engineering
considers the automatic generation of text summaries.
Most often, automatic summaries are produced 
using extractive techniques that select a subset of 
the sentences of an artifact which will compose the summary~\cite{a}.
Among other natural language artifacts,
extractive summarization techniques
have been applied to Stack Overflow posts~\cite{a}, coding tutorials~\cite{a},
or bug reports, as
when Lotufo et al. 
considered the kinds of sentences a developer would find relevant 
to understand a bug report when pressed with time~\cite{Lotufo2012},
% i.e., sentences with topics frequently discussed, sentences assessed by other sentences and 
% sentences that focus on the topics in the bug report's title and description,
and proposed an unsupervised summarization approach 
based on the PageRank algorithm~\cite{Page1999}
to identify these sentences. 




A second set of unsupervised methods focus on clustering data.
These techniques identify 
subsets in the data that have similar properties or features 
and techniques such as \acf{LDA}~\cite{blei2003latent}  have been use both to 
bootstrap the categorization of information in 
natural language artifacts and as part of tools that identify 
portions of the text in an artifact pertinent to some software activity. 
As an example of the former, 
 Allamanis and Sutton
applied \acs{LDA}
to gain insight into the types of questions 
asked on Stack Overflow~\cite{Allamanis2013}.
For the latter, tools such as FRAPT
use \acs{LDA} to identity topics in a web tutorial
and then extract sentences explaining API elements from each of the topics identified~\cite{Jiang2017}.



% Despite the significant 
% contributions brought by using machine learning methods, 
One substantial challenge inherent the supervised and 
unsupervised \acs{ML} approaches that we discussed
is that researchers must engineer which 
features their \acs{ML} technique will use~\cite{ferreira2021}.
Given the specificity and cost of engineering such features, 
\acs{ML} approaches have limitations similar to pattern 
matching approaches when we consider their use across 
different kinds of artifacts.




% For example, Rastkar et al. uses conversational features in 
% the text of a bug report to assist in determining which sentences 
% to include in the bug report's summary~\cite{Rastkar2010}
% while Petrosyan and colleagues use 
% linguistic and structural properties 
% in the text of API documents to identify key text 
% explaining API elements~\cite{Petrosyan2015}.




\subsection{Deep Learning Approaches}
\label{cp2:deep-learning}





In contrast to the human-engineered features,
\acf{DL} approaches allow the automatic extraction of features 
from training data~\cite{Deng2018, zhang2021deep},
which makes 
deep learning an interesting 
approach to
uncover regularities in the text of a natural language software artifact
that might not obvious or easily identified
by software engineering researchers.



% Deep learning has led to groundbreaking advancements in many 
% research areas (e.g., machine translation~\cite{lopez2008translation}) 
% and, 

Given the wide range of applications that both propose \acs{DL} models~\cite{} and that use 
these models for a certain purpose (e.g., machine translation~\cite{lopez2008translation}), 
we focus
on \acs{DL} applications in the 
software engineering domain~\cite{ferreira2021, li2018deep, watson2022}.
First we discuss neural embeddings and then, we present 
neural network models 
for the same range of problems 
that machine learning approaches apply to.






A common application of \acs{DL} in software engineering is the usage of neural, or word, embeddings~\cite{Mikolov2013}
for information retrieval purposes. 
Neural, or word, embeddings produce vector representations in a continuous space,
where words with similar meanings are typically close in the vector space model~\cite{harris1954distributional, mikolov2013efficient}. 
Researchers have found that word
embeddings mitigate lexical mismatches in the text found across different 
natural language software artifacts,
using them as a way to compare the semantic similarity of the text~\cite{mihalcea2006}.
Their use has improved many information retrieval task,
as shown by Ye et al.'s evaluation of word embeddings
for bug localization tasks~\cite{Ye2016}
or Huang and colleagues' study on 
the usage of word embeddings for API recommendation tasks~\cite{Huang2018}. 


% Due to how word embeddings assist the in identification 
% of similar text, they have also been widely used in unsupervised text 
% summarization approaches~\cite{Ponzanelli2017, Xu2017, Jiang2017}.
% \red{For example, AnswerBot's relevant and salient text selection algorithm~\cite{Xu2017}
% uses word embeddings to find which sentences in a Stack Overflow answer 
% are most similar to a question posed by a developer 
% and that should be included in the summary that the tool produces 
% to answer the developer's question. }





Provided that \acs{DL} address challenges originating from feature engineering, 
many other \acs{DL} studies in software engineering~\cite{ferreira2021,li2018deep, watson2022}
use neural network architectures 
in binary or multinomial classifiers as well as in extractive text summarization.
For example, Fucci et al. used a 
recurrent neural network (\acs{RNN}) with 
\acf{LSTM}~\cite{aa}
to identify the types of information available in 
API documentation~\cite{fucci2019} or when Li et al. used a stepped auto-encoder~\cite{aa}
to produced more accurate and diverse summaries 
for bug reports~\cite{li2018deep}.
State-of-the-art architectures such as \acf{BERT}
have been used for requirements traceability~\cite{Araujo2021}.
Nonetheless, these and other applications of deep learning 
to natural language software artifacts target specific
types or artifact and evaluating their 
usage across different types of artifacts 
is outside the scope of the studies we surveyed.



In Chapter~\ref{ch:identifying}, we consider 
how we can use 
word embeddings and \acs{BERT}-based \acs{DL}
model 
to automatically identify task-relevant text
in different kinds of natural language artifacts.

%  (BERT~\cite{Devlin2018Bert})





\section{Automated Approaches to Task-Relevant Text Identification}
\label{cp2:task-approaches}



In this section, we detail textual approaches that automate the identification 
of text likely relevant to a developer's task. These approaches 
use many of the techniques discussed in Section~\ref{cp2:general-approaches}
in conjunction with the text in a developer's task
 or the code in a developer's workspace
as potential ways to 
to identify task specific information.
A seminal tool that exemplifies 
how these sources assist in locating task specific information is
Hipikat~\cite{Cubranic2005}.
It takes a query explicitly prompted by a developer 
or implicitly based on the code that the developer 
has inspect and  
it recommends artifacts from a project's archive 
that are pertinent to the query.


While Hipikat makes recommendations at the artifact level, 
the tool's authors acknowledged that 
this approach could be fine-tuned to identify 
parts of an artifact that are relevant~\cite{Cubranic2005}
and other studies have since built upon this core idea. 
For example, Deep Intellisense~\cite{Holmes2008} 
uses pattern matching to 
identify 
artifacts containing text that 
would assist a developer
better understand the code that they inspect
while 
CueMeIn~\cite{sun2021} takes the code 
a developer is currently editing and it
uses a \acs{DL} neural network to find 
excerpts from web tutorials 
that might contain explanations 
for the classes and methods 
present in the method being edited. 





With regards to the content in a developer's task, as captured through a feature 
request or a bug report in an issue tracking system (e.g., the notification task in Figure~\ref{fig:fig:android-notifications-task}),
a second potential way to find text relevant to the task at hand is through query-based summarization~\cite{Goldsteinet1999}.
Query-based summarization allows 
producing a summary tailored to a particular query
and a small number of approaches have focused on
producing task-specific summaries~\cite{Xu2017, silva2019, liu2019qapi}.
For example, AnswerBot~\cite{Xu2017}, a state-of-the-art summarization tool, 
uses word embeddings to find which sentences in a Stack Overflow answer 
are most similar to a question  representing a developer's task.
The tool uses semantic similarity in conjunction with 
Stack Overflow's meta-data to decide which text should it include
in the summary that answers the developer's question. 







The possible techniques that we explore in Chapter~\ref{ch:identifying} 
build upon these concepts. We assume that 
a set of artifacts pertinent to a task 
is available and we use the content of a developer's task 
to identify text relevant to that task
in these pertinent artifacts.





% and researchers have extended this idea to  
% publicly available data, e.g., GitHub issues~\cite{Viviani2019}
% or pull requests~\cite{freire2021}. 






% Even though most of the summarization approaches that apply to natural language software artifacts 
% summarize an artifact on its whole~\cite{Rastkar2010, Murray2008, Lotufo2012, Ponzanelli2015},


% Summarization approaches often favour the diversity of information to be included in a summary~\cite{Carbonell1998,li2018deep}
% and thus, a summary might not contain all of the information needed 
% to correctly complete a software task.
% For example, if several relevant sentences discuss different yet similar content about 
% a particular technology, it is less likely that a summary will contain all of these sentences. 
% In contrast, the approaches that we explore in Chapter~\ref{ch:identifying}
% seek to identify the text that is most relevant to a task regardless 
% of its diversity.





% 



\section{Improving Developers' Productivity}
\label{cp2:dev-productivity}



The tools and approaches presented in Sections~\ref{cp2:general-approaches}
and~\ref{cp2:task-approaches}
are examples of studies that help developers in locating 
information that assists them in completing a software task.
These studies fit in the bigger context 
of software engineering research 
facilitating or improving the quality of a developer's work~\cite{Kersten2006, Meyer2017, satterfield2020}. 





As part of their work, developers engage in many sensemaking and decision-making activities~\cite{sillito2006} and several studies have investigated how to 
provide means to better  
assist developers in performing such activities~\cite{Liu2018Unakite, liu2021, barnett2015}.
For example, 
Ponzanelli et al. proposed a tool, Libra, that monitors the web pages a developer 
has navigated and uses that to show 
how similar or not the results of a new web search are in comparison to the already navigated 
pages, which offers contextual information-seeking support~\cite{Ponzanelli2017}.



Researchers have also been interested in 
making knowledge bases
that developers working on a task can benefit from. 
In Section~\ref{cp2:task-approaches} 
we cited Hipikat~\cite{Cubranic2005}, a seminal tool that exemplifies this concept in action.
Based on the current code being inspected by a developer or based on a query prompted by a developer, 
it recommends artifacts from a project's archive 
that are pertinent to the code being inspected.
This archive, or project memory, is produced 
based either on relationships between the artifacts in a software project 
or based on the source code changed in a bug fix or feature request~\cite{Cubranic2005}.
Other tools like Strata summarize knowledge produced by developers 
while they navigate on the web so that other developers
can use this knowledge to have a 
 head start when performing similar tasks~\cite{liu2021}.




We build upon many of the research procedures outlined 
in these and many other studies in the field. 
For example, in the evaluation of Strata, 
 Liu et al. describe procedures 
considering how a control and tool-assisted group 
complete information-seeking tasks~\cite{liu2021},
which assisted in the design of our experiment for 
evaluating an automated approach to
task-relevant text identification (Chapter~\ref{ch:assisting}).





% 



\section{Textual Analysis in Natural Language Software Artifacts}
\label{cp2:text-in-se}



In this section, we detail empirical studies investigating 
textual data in natural language software artifacts. 
Such textual data is  rich in semantic information~\cite{dekhtyar2004} 
and we provide and overview of seminal work 
investigating lexical, syntactic, and semantic aspects of the text.




Initial insights about the text of a natural language artifact might be learned focusing on 
the characters or tokens in the text.
This lexical analysis~\cite{jurafsky2014speech} might be used to identify common or unique words 
used in an artifact as well as words that often appear together in the text.
In software engineering research, Bacchelli et al. used lexical analysis to study how developers describe class elements
in nearly 80,000 emails of an Open Source System~\cite{bacchelli2009}.
Their analysis suggested that lexical terms could be 
used as a mean of automatically linking information in development emails to 
the project's source code.







Syntactic analysis concerns the  elements in the text 
and the grammatical relationships between them~\cite{jurafsky2014speech}. 
Part-of-speech tagging and dependency parsing are common applications of syntactic analysis,
where they identify the nouns, verbs, pronouns, and other elements in the text 
and how these elements relate, respectively. 
As an example of a syntactic analysis study in natural language software artifacts, 
Ko and colleagues identified regularities in the syntactic structure of the text 
of nearly 200,000 bug report titles~\cite{Ko2006}, suggesting that these 
regularities could be used for the automatic identification 
of information in bug reports. 






Sentences with similar terms or similar syntactic structure might convey different information
and researchers have also investigated the meaning of the text in natural language software artifacts. 
These studies consider the semantic analysis of the text and 
software engineering researchers have proposed \textit{taxonomies} to explain the type of information 
available in certain kinds of natural language artifacts~\cite{Maalej2013, Arya2019}. 
For example, Di Sorbo et al. have found that 
text in development mailing lists can be classified according to the developers' intentions (e.g., feature request, solution proposal, etc.)~\cite{Sorbo2015},
suggesting that one can automate this classification to assist developers in finding 
certain types of information.
% Other examples include Maalej and Robillard's taxonomy of patterns of knowledge in API documentation
% or Arya et al. analysis of information types in Open Source issues~\cite{Arya2019}.




Most of the textual analysis in related literature focus on the text of a specific 
type of artifact. In contrast, our work 
characterizes common semantic cues in the text 
of different kinds of artifact that developers 
deemed relevant to a software task (Chapter~\ref{ch:characterizing}).





%  For example, Figure~\ref{fig:syntactic-analysis} shows the
%  Stanford CoreNLP~\cite{CoreNLP} syntactic analysis for a sentence 
%  about the file system:  `\textit{you can use io.StringIO}'.



% \medskip
% \begin{figure}[h!]
%     \centering
%     \includegraphics[width=0.40\textwidth]{cp2/textual-analysis.png}
%     \caption{Example of syntactic analysis}
%     \label{fig:syntactic-analysis}
% \end{figure}






% This section details empirical studies analyzing different aspects associated with 
%  natural language software artifacts.






% Early studies on textual analysis in software engineering often focused on program comprehension 
% and the text in the source code~\cite{Woodfield1981, maletic2002},
% but 
% Researchers have long recognized 
% that natural language artifacts 
% are rich in semantic information
% and that 
% better understanding these artifacts
% would lead to improved software~\cite{dekhtyar2004}.






% These studies have helped
% software engineering researchers make more informed decisions 
% on the design of automated tools 
% that identify and extract text from these artifacts---detailed further in this chapter. 




% The empirical analysis provided by these studies 
% has 
% deepened software engineering researchers' knowledge of the type of information available 
% in natural language artifacts 






% Bird at al. used it to investigate social aspects in development mailing lists~\cite{bird2006}. 
% They inspected nearly 102,000 messages on the Apache HTTP Server development lists
%  to find that character editing algorithms~\cite{levenshtein1966}, 
%  helped in unmasking developers' aliases, which helped investigating email 
%  exchange patterns of the key contributors of the project.


% .
% They have sampled 100 emails and identified that text for feature requests 
% often contain expressions in the form of suggestions
% (e.g., `\textit{we should add a new button}'), whereas solution proposals 
% are often expressed in the form of attempts (e.g., `\textit{let's try a new method to compute cost}')





% With regards to meta-data 




% \clearpage


% Several other studies have 
% investigated the meta-data available in
% natural language software artifacts. 
% % In Section~\ref{cp1:example},
% % we have presented an example of meta-data in a Stack Overflow post,
% %  other examples include fields in a bug report~\cite{Davies2014, Breu2010},
% % or tags and labels in GitHub projects~\cite{prana2019}.
% These studies often focus on 
% the frequency with which
% meta-data is available~\cite{Davies2014, bettenburg2008makes, uddin2015},
%  how up-to-date it is~\cite{ahmad2018, dig2006, shi2011}, 
%  and the perils of relying on meta-data for information extraction purposes.
% For example, Zhang et al. has shown that 
% certain comments on Stack Overflow are equally or more informative 
% that the information found in an accepted answer~\cite{zhang2019so}.


% has led software engineering researchers to 
% make more informed decisions on when to use meta-data 
% and how it can assist in the extraction of useful information from natural language 
% artifacts.
% For example, Wang et al. 


% found that 

% the number of votes 
% an answer has on Stack Overflow is 
% equally or more important than 


% ~\cite{wang2018}



% ~\cite{wang2018}









% Findings from these studies led 
% software engineering researchers to discuss the promises and perils of 
% using an artifact's meta-data~\cite{kalliamvakou2014, ahmad2018}.
% For example, an accepted answer on Stack Overflow 
% might not be the correct answer~\cite{wang2018}
% and 
% valuable information exists in elements not
% associated with any meta-data~\cite{zhang2019so},
% which we discussed as part of our 
% scenario about finding information about Android notifications (Section~\ref{cp1:example}).


% 


\section{Automating the Identification of Relevant Text}
\label{cp2:general-approaches}







To help an individual in finding the text that is potentially useful for a task,
many studies have provided possible properties in a natural language artifact
that guide the automatic identification of relevant text~\cite{Bavota2016}.
In this section, we provide background information on general 
approaches used for such a purpose. 
% Then, in Section~\ref{cp2:task-approaches}, 
% we 






% These studies focus on specific tasks such as finding relevant passages in API documentation~\cite{Maalej2013,
% Petrosyan2015, Robillard2015}, learning about API types in code tutorials~\cite{Jiang2016b, Jiang2017},
% or detecting sentences that discuss a bug's expected behaviour~\cite{Chaparro2017}.
% These approaches respectively use the presence of code words, HTML anchor links, or whether a sentence contains
% a modal verb as properties to automatically identify relevant text.
% Given how quickly developers progress to using new kinds of technology to
% record pertinent information (e.g., slack~\cite{Storey2017, Lin2016}),
% it may be difficult to scale such artifact-centric approaches to cover the range of
% artifacts that could be returned from a search.
% A second issue arises from the fact that coding procedures used to determine relevancy
% in these studies do not consider disagreements~\cite{Stol2015}.
% That is, there are many criteria in how individuals
% assess relevance~\cite{Barry1994, Barry1998, Freund2015}
% and there are no guarantees that
% developers will use the same properties to determine text relevant to their tasks~\cite{Freund2013, Freund2015}.
% In contrast, our study does not pre-assume any relevance cues, and instead,
% we leave the decision of what text is perceived as relevant to a particular task to
% the participants in our experiment.


% \section{Automatic Text Identification Approaches}
\label{cp2:text-approaches}



Information useful to a software task can be buried in irrelevant text or attached to 
non-intuitive blocks of text, making it difficult to discover~\cite{Robillard2015}.
In this section, we detail tools and approaches from related work
that seek to assist developers in 
identifying information within the natural language
text of software artifacts.





\subsection{Pattern Matching Approaches}
\label{cp2:pattern-matching}


Pattern matching approaches rely on regular expressions describing a sequence of tokens that represent
 a relevant text fragment~\cite{Bavota2016}. Tokens can either represent words or linguistic elements 
extracted using \acf{NLP}.
    
    
As examples  of pattern matching approaches,  {\small DeMIBuD}~\cite{Chaparro2017}
 and Knowledge Recommender (Krec)~\cite{Maalej2013, Robillard2015} are tools that detect relevant sentences in bug reports and API documentation, respectively. 
These tools derive a set of patterns from annotated data and use them as part of heuristics 
that identify relevant text. Krec  uses  361 unique patterns
to 
detect relevant sentences mentioning a 
code element (e.g., a class or method name) in API documentation~\cite{Robillard2015}.
In a similar manner, {\small DeMIBuD} uses a set of 154 discourse patterns to detect sentences 
relevant to understanding a bugs observed or expected behaviour and steps to reproduce it,
which are essential to bug triaging tasks.




In Stack Overflow posts,
Nadi and Treude~\cite{nadi2020} have both applied the original set of patterns from Krec~\cite{Robillard2015} 
and proposed heuristics that rely on the conditional clauses (i.e., sentences with the word `\textit{if}')
to identify text that help a developer 
decide whether they want to carefully inspect a Stack Overflow posts or skip it. 



Although the heuristics and regular expressions used in the aforementioned studies 
are often light-weight and effective~\cite{Bavota2016, Maalej2013}, 
pattern matching approaches are specific to certain kinds of domain and 
types of artifact~\cite{fucci2019}, what limits
using them in a general approach.





\subsection{Summarization Approaches}
\label{cp2:summarization}



Extractive text summarization techniques are used in natural language artifacts in software engineering to
produce a summary of the artifact's content. 
A summary represents key information that may help a developer complete their task~\cite{Bavota2016}.
There are summarization techniques based on both supervised and unsupervised learning~\cite{moreno2017}
and one can summarize the entire content of an artifact
or content specific input query, as in \textit{query-based} summarization~\cite{Huang2018, Goldsteinet1999}.




A number of summarization approaches target bug reports and GitHub issues, largely
focusing on identifying key information within these artifacts. 
Rastkar and colleagues~\cite{Rastkar2010} use a supervised learning approach to summarize the content 
of bug reports showing that conversational features used to summarize emails~\cite{Murray2008}
can also be applied to bug reports while
Lotufo et al.~\cite{Lotufo2012} proposed an unsupervised summarization approach 
that automates the identification of sentences that a developer would first read when
a inspecting bug report.



While many of the approaches described above
largely rely on  lexical aspects in text, researchers have also made use
of structured textual information in the artifacts~\cite{Ponzanelli2015, Treude2016, chen2016}. 
For example, Ponzanelli et al. 
proposed a summarization technique that mixes natural language text and structured data 
available on Stack Overflow
to produce more accurate summaries for Stack Overflow answers~\cite{Ponzanelli2015}. 
As another example, DeepSum~\cite{Li2018} pre-processes a bug report dividing sentences 
containing software elements, the reporter of the bug, and any other sentences 
in the bug report to produce summaries containing more diverse information.




A smaller number of summarization approaches have focused on
producing task specific summaries~\cite{Xu2017, silva2019}.
These approaches pose the problem of finding task-relevant text 
as a query-based extractive summarization problem and
tools such as AnswerBot~\cite{Xu2017}
identify relevant text in Stack Overflow posts 
based on 
the content of the text, how similar that content is with regards to a input query (i.e., task)
and the structured data available on each of the answers in a Stack Overflow post 
(i.e., number of votes or whether an answer is the accepted answer).
As the state-of-the-art, Chapter~\ref{ch:identifying}
compares AnswerBot to the techniques that we explore in this thesis.



\subsection{Machine Learning Approaches}
\label{cp2:machine-learning}


\acf{ML} approaches take the text of a natural language software artifact and identify 
the sentences likely relevant to a particular software task using \textit{supervised} or 
\textit{unsupervised learning} methods~\cite{zhang2005machine}.



Supervised learning approaches use a set of features and labeled data
 to train classifiers with the goal of identifying sentences relevant to 
 certain software activity.
We have already presented supervised approaches that use text summarization (\textit{i.e.,}~\cite{Rastkar2010})
and there are also approaches that identify relevant 
parts of software tutorials~\cite{Jiang2016b}
or API documents~\cite{fucci2019, Maalej2013}.
Despite their value, 
the cost and effort of hiring skilled workers to produce 
labeled data in software engineering artifacts 
has been a major limitation 
to the usage of supervised learning 
methods~\cite{Arpteg2018}.





Unsupervised learning approaches do not require labelled data and determine 
relevant sentences according to properties inferred from the data. 
DeepSum~\cite{Li2018} and Lotufo et al.'s~\cite{Lotufo2012} techniques are examples of 
unsupervised approaches in the scope of text summarization. 



Other unsupervised approaches 
(\textit{i.e.,} {\small FRAPT}~\cite{Jiang2017} or HoliRank~\cite{Ponzanelli2015, Ponzanelli2017})
are mostly based around variations of the PageRank~\cite{Page1999} or LexRank~\cite{Erkan2004} algorithms. 
These algorithms represent all the text in an artifact as a graph.
Then, they establish relationships (\textit{i.e.,} weighted edges in the graph) 
between different sentences (\textit{i.e.,} nodes in the graph) 
and select the nodes with highest weights as the most relevant ones.
A crucial step in building the graph is in the definition of 
how to establish  relationships between nodes.
Early approaches~\cite{Lotufo2012, Jiang2017} 
use \ac{VSM}~\cite{Salton1975vsm} 
for this purpose while more modern ones~\cite{Huang2018, silva2019}
use different word embeddings~\cite{Mikolov2013, bojanowski2017FastText},
which we detail in Section~\ref{cp2:deep-learning}.
To identify key sentences using these techniques, one must 
assume that certain elements will be cross-referenced 
or mentioned multiple types in a document, 
what is only common in certain types of software artifacts, e.g., mailing lists~\cite{Bacchelli2012}.








\subsection{Deep Learning Approaches}
\label{cp2:deep-learning}



One substantial challenge of standard \acf{ML}
approaches is that researchers must engineer which 
features or properties of the text to use~\cite{ferreira2021}.
For example, Rastkar et al. uses conversational features in 
the text of a bug report to assist in determining which sentences 
to include in the bug report's summary~\cite{Rastkar2010}
while Petrosyan and colleagues use 
linguistic and structural properties 
in the text of API documents to identify key text 
explaining API elements~\cite{Petrosyan2015}.
Given the specificity of such features, 
researchers have questioned the generalizability
of standard \acs{ML} approaches~\cite{Xiao2018, fucci2019}.



In contrast to the human-engineered features,
\acf{DL} approaches allow the automatic extraction of features 
from training data through a series of mathematical transformations~\cite{Deng2018, zhang2021deep}.
Deep learning has led to groundbreaking advancements in many 
research areas (e.g., machine translation~\cite{lopez2008translation}) 
and, given its wide range of applications, 
we focus
on its usage in natural language text appearing in software engineering artifacts~\cite{ferreira2021, li2018deep, watson2022}.









Software engineering researchers have identified that the text 
in software task 
often differs from the text in the artifacts that are related to that tasks~\cite{Huang2018}. 
These so-called \textit{lexical mismatches}~\cite{Ye2016} 
 make it difficult to identify information of interest 
to a task and a number of studies have used \acs{DL}
neural embeddings~\cite{Mikolov2013} to bridge lexical gaps between the text of different software artifacts. 


Neural, or word, embeddings produce vector representations in a continuous space,
where words with similar meanings are typically close in the vector space model~\cite{harris1954distributional, mikolov2013efficient}. 
Their usage has allowed researchers to improve 
the identification API elements pertinent to a programming task~\cite{Ye2016} 
or to more accurately assess the quality of the content in bug reports~\cite{chaparro2019}.
Word embeddings have become a common way 
to compare the semantic similarity of the text~\cite{mihalcea2006},
being applied in query-based summarization techniques such as 
AnswerBot~\cite{Xu2017}
or PageRank-based approaches such as HoliRank~\cite{Ponzanelli2017}.



Many other \acs{DL} studies in software engineering~\cite{ferreira2021,li2018deep, watson2022}
use neural network architectures 
in a variety of software engineering tasks, including
code comprehension~\cite{allamanis2015, mi2018}, community forum analysis~\cite{Lin2018, wang2019}, or requirements traceability~\cite{chen2019, guo2017}.
DeepSum~\cite{Li2018}, which we described earlier, is an example of a 
summarization approach that uses an encoder-decoder architecture~\cite{cho2014-encoder-decoder}  to identify 
sentences of interest in bug reports. As another example,
Xi et al.~\cite{Xia2017} use a \ac{CNN}~\cite{krizhevsky2012CNN} 
to identify sentences in GitHub issues discussing topics such as 
problem discovery, solution proposal, or feature requests.



Few studies in software engineering have considered more modern neural networks~\cite{watson2022}
able to establish relationships not only between words but also sentence pairs (e.g., \acs{BERT}~\cite{Devlin2018Bert}). 
These models might assist in determining 
implicit relationships between the text in a task and the text in a relevant sentence within a software artifact.
As such, Chapter~\ref{ch:identifying} describes how we use \acs{BERT} for automatically 
identifying text relevant to a software task.


