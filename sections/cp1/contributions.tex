

\section{Contributions}
\label{cp1:contributions}

This thesis makes the following contributions, 
which are organized 
in three main categories
(i.e., empirical studies, techniques, and corpora):



% ---------------------------------------------------------------------------------------

\paragraph{\textbf{Empirical Studies.}} 

It reports on two empirical studies for supporting a developer's discovery of task-relevant information:

\begin{itemize}

    \item A first study characterizes how task-relevant text appears across a variety of natural language artifacts, including  API documentation, Q\&A websites, and
    bug reports that are pertinent to six software tasks; 

    \item A second study addresses whether 
    an automated approach to task-relevant text identification assists a software developer while they work on a task; this study provides empirical evidence on the role of tools that assist developers in locating 
    text relevant to a particular task.
\end{itemize}


% ---------------------------------------------------------------------------------------


\paragraph{\textbf{Techniques.}} 

It introduces six possible techniques that incorporate the semantics of words and sentences for identifying task-relevant text in software artifacts, where:
    
\begin{itemize}
    
    \item it reports on the precision and recall of semantic-based techniques for the identification of text that human annotators deemed relevant to natural language artifacts and tasks associated with Android development; and

    \item it shows that semantic approaches have recall comparable to a state-of-the-art approach
    tailored to one kind of artifact, i.e., Stack Overflow.
\end{itemize}

% ---------------------------------------------------------------------------------------

\paragraph{\textbf{Datasets.}} 

It presents three different datasets that can be used for replication purposes and future research in the field, namely:
    
\begin{itemize}
    \item \acs{DS-synthetic} provides a unique corpus of 20 natural language artifacts associated
    with natural language artifacts that include annotations from 20 participants of text deemed relevant to six software tasks; this dataset was produced as part of our study on characterizing task-relevant text;
    
    \item \acs{DS-android} is a dataset with 50 Android tasks and associated text artifacts 
    that we produced to evaluate the semantic-based techniques that we explore for the automatic identification of  task-relevant text;

    \item \acs{DS-python} is a by-product of our empirical evaluation of the impact of tools that automatically identify text relevant to a task;
    it contains three tasks related to well-known Python libraries, and it includes annotations from 24 participants that indicated what text assisted them in writing each task's solution.
\end{itemize}