

\section{Contributions}
\label{cp1:contributions}

This thesis makes the following contributions to the field of software engineering:




\begin{itemize}

    \item It details a formative study that characterizes how task-relevant text appears across a variety of natural language artifacts, including  API documentation, Q\&A websites, and
    bug reports that are pertinent to six software tasks; 

    \item It introduces six possible techniques that building on approaches to interpret the meaning, or semantics, of text
    to automatically identify task-relevant text across different kinds of software artifacts, where:

    \begin{itemize}
        
        \item We report how accurately these techniques identify text that human annotators considered relevant to natural language artifacts associated with Android development tasks; 

        \item We show that semantic approaches have accuracy comparable to a state-of-the-art approach
        tailored to one kind of artifact~\cite{Xu2017}, i.e., Stack Overflow.
    \end{itemize}



    \item It presents an empirical experiment that provides initial evidence on how  
    an automated approach to task-relevant text identification assists a software developer while they work on a task.
\end{itemize}





% \art{make it chronological: a formative study, three techniques and then, and evaluative study}





% ---------------------------------------------------------------------------------------

% \paragraph{\textbf{Empirical Studies.}} 

% It reports on two empirical studies for supporting a developer's discovery of task-relevant information:




% ---------------------------------------------------------------------------------------


% \paragraph{\textbf{Techniques.}} 

% It introduces six possible techniques that incorporate the semantics of words and sentences for identifying task-relevant text in software artifacts, where:
    


% ---------------------------------------------------------------------------------------

\paragraph{\textbf{Datasets.}} This work of dissertation also contributes with three different datasets ({\small \textit{DS}}) that can be used for replication purposes and future research in the field:
    
\begin{itemize}
    \item \acs{DS-synthetic} provides a unique corpus of 20 natural language artifacts associated
    with natural language artifacts that include annotations from 20 participants of text deemed relevant to 
    the tasks in our study on characterizing task-relevant text;
    
    \item \acs{DS-android} is a dataset with 50 Android tasks and associated text artifacts 
    that we produced to evaluate the semantic-based techniques that we assess for the automatic identification of task-relevant text;

    \item \acs{DS-python} is a by-product of our empirical evaluation on the impact of tools that automatically identify task-relevant text;
    it contains three Python tasks and it includes annotations from 24 participants that indicated what text assisted them in writing each task's solution.
\end{itemize}