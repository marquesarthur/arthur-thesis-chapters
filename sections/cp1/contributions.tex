

\section{Contributions}
\label{cp1:contributions}

This thesis makes the following contributions to the field of software engineering:




\begin{itemize}

    \item It details a formative study that characterizes task-relevant text across a variety of natural language artifacts, including  API documentation, Q\&A websites, and
    bug reports that are pertinent to six software tasks; 

    \item It introduces six possible techniques that build upon approaches that interpret the meaning, or semantics, of text
    to automatically identify task-relevant text across different kinds of software artifacts;


    \item It shows how the most promising semantic-based approaches that we have explored have accuracy comparable to a state-of-the-art approach
    tailored to one kind of artifact~\cite{Xu2017}, i.e., Stack Overflow;

    \item It presents an empirical experiment that provides initial evidence on how  
    a semantic-based tool, \acs{tool}, assists a software developer in completing a software task. 
\end{itemize}





This dissertation also contributes with three different datasets ({\small \textit{DS}}) that can be used for replication purposes and future research in the field:
    
\begin{itemize}
    \item \acs{DS-synthetic} provides a unique corpus of 20 natural language artifacts that include annotations from 20 participants of text deemed relevant to 
    the six tasks in our study on characterizing task-relevant text;
    
    \item \acs{DS-android} is a dataset with 50 Android tasks     
    and associated natural language artifacts 
    with annotations from three developers of the text relevant to these tasks;

    \item \acs{DS-python} contains three Python tasks and it includes annotations from 24 participants that indicated which
    text assisted them in writing each task's solution.
\end{itemize}