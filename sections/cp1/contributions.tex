

\section{Contributions}
\label{cp1:contributions}


The work presented in this thesis 
aims to improve how developers locate task-relevant information and thus, assist them in making more informed decisions related to their task.
The thesis makes the following contributions to the field of software engineering:







% \begin{itemize}
%     \item It . This work:
%     \begin{itemize}
    
   
    
%         \item demonstrates that some information \textit{within} the artifacts is key to the tasks and is considered relevant by many participants; and
    
%         \item reports on the semantic characteristics of relevant text, identifying semantic parsing as a promising approach for recognizing information relevant to a task.
    
%     \end{itemize}

%     % ---------------------------------------------------------------------------------------
%     % ---------------------------------------------------------------------------------------

%     % ---------------------------------------------------------------------------------------
%     % ---------------------------------------------------------------------------------------

%     \item

%     % ---------------------------------------------------------------------------------------
%     % ---------------------------------------------------------------------------------------
     
    
% \end{itemize}


\paragraph{\textbf{Empirical Studies.}} 

It reports on two empirical studies for supporting a developer's discovery of task-relevant information:

\begin{itemize}

    \item A first study characterizes how task-relevant text appears across a variety of natural language artifacts pertinent to a set of software tasks; this study demonstrates consistency in the text 
    considered key to completing six information seeking tasks, and identifies 
    promising semantic-based approaches that can be used to identify task-relevant information in natural language artifacts; 

    \item A second study addresses whether 
    the text automatically identified by semantic-based techniques can assist a software developer while they work on a task; this study provides initial evidence that such tools affect how developers complete a task 
    based on the text that it automatically identifies.
\end{itemize}





\paragraph{\textbf{Techniques.}} 

It introduces six possible techniques that incorporate the semantics of words and sentences for identifying task-relevant text in software artifacts, where:
    
\begin{itemize}
    
    \item it reports on the precision and recall of semantic-based techniques for the identification of text that human annotators deemed relevant to tasks associated with Android development; and
    \item it shows that semantic approaches have recall comparable to state-of-the-art approaches
    tailored specifically to certain kinds of artifacts.
\end{itemize}


\paragraph{\textbf{Datasets.}} 

It presents three different datasets that can be used for replication purposes and future research in the field, namely:
    
\begin{itemize}
    \item \acs{DS-synthetic} provides a unique corpus of 20 natural language artifacts associated
    with open-source projects that includes annotations from 20 participants of text deemed as relevant to six information-seeking tasks;
    
    \item \acs{DS-android} provides 50 Android tasks and associated textually-based artifacts
    that three annotators inspected indicating which text in the artifacts they deemed pertinent to these tasks;

    \item \acs{DS-python} contains three tasks related to well-known Python libraries with artifacts
    that includes annotations from 24 participants that indicated what text assisted them in writing a solution for these coding tasks.
\end{itemize}