

% \subsection{Automatically Identifying Task-relevant Text}


% Provided that there is consistency in the text deemed relevant to a software task, we investigate whether techniques building on approaches to interpreting the meaning of the text help overcome limitations found in artifact-specific approaches, e.g., assumptions on an artifact's structure or content~\red{\cite{}}.



% We introduce six semantic-based techniques that incorporate the semantics of words and sentences, evaluating the extent to which they can identify task-relevant text 
% across a number of Android development tasks and various associated natural language artifacts.
% The first two use word embeddings to identify likely relevant text via semantic similarity~\cite{Mikolov2013} and via a  neural network~\red{\cite{}}.
% Sentence-level techniques are filters that we apply (or not) on top of the word embedding techniques generating four additional techniques.
% These filters use frame semantics~\cite{fillmore1976frame} to obtain entities (frames) that summarize a sentence's meaning.
% To determine a sentence's relevance, the first sentence-level technique uses frames extracted solely from the text of an artifact, while the second combines these frames with the ones obtained in a task.



% \subsubsection{Approach}



% \subsubsection{Evaluation}



% \subsubsection{Novelty}