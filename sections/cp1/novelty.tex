\section{State of the art}
\label{cp1:novelty}


Researchers have long recognized the value of 
assisting developers in locating information in the natural language artifacts sought as part of a software task.
They have proposed many tools and approaches 
that combine \acf{IR}, \acf{NLP} and \acf{ML} techniques,
often using  syntactic properties of the text, and an artifact's meta-data,
to aid developers in locating the portion of the text that
might be useful within these artifacts.



% The majority of these approaches focus on specific activities (e.g., bug triaging~\cite{Chaparro2017} or learning about an API~\cite{Jiang2017}) and apply 
% only to certain kinds of artifact, such as bug reports~\cite{Rastkar2010, Lotufo2012}, \acs{qa} web pages~\cite{Xu2017, silva2019}, or API documentation~\cite{fucci2019}.
% In the paragraphs that follow, we detail how these approaches  to automatically identify what is relevant.



As examples of techniques available in the literature, Nadi and Treude consider 
that relevant information in \acs{qa} 
web pages is often found in text with
conditional clauses (i.e., sentences with the word `\textit{if}')~\cite{nadi2020}
while Robillard and Chhetri assume that relevant 
text in API documents mention a code element such as a class name or method signature~\cite{Robillard2015},
where they use such assumptions 
in techniques that identify this text automatically.
These pre-define rules would fail to identify the relevant text in the artifacts of our running example
(Figure~\ref{fig:android-notifications-task})
both because these syntactic rules 
do not take a developer's task into account 
and because these rules were derived from one type of artifact,
implying that there are no guarantees that 
they extend to the different types of artifact
that a developer might use.

% for determining relevant text in these 
% and other techniques~\cite{chaparro2019, fucci2019}




Recognizing that relevance is often context-specific~\cite{Bavota2016}, i.e., information needs are associated with details of the task 
or characteristics of the developer performing it~\cite{Robillard2015}, many other software engineering researchers 
have proposed approaches that 
automatically identify information based on a specific context~\cite{Ye2016, silva2019, Xu2017}.
For example, Xu et al. identify relevant text in Stack Overflow posts 
based on how similar the text within an answer is with regards to a input query, which represents the developer's programming task.
Although this assists in the automatic identification of context-specific information, 
the use of structural data by this and other techniques~\cite{silva2019, Li2018}
limits their adoption since it is evidently
not possible to apply such techniques
across different types of artifact  that lack such meta-data~\cite{Bavota2016, arnaoudova2015}.
% The usage of structural data might also exclude identifying relevant information found 
% in other parts of an artifact. 
For example, Xu et al.'s approach
only applies to one out of the three artifacts 
from the developer's search 
about android notifications (Figure~\ref{fig:android-search-results})
and for the artifact that it applies to, i.e., the Stack Overflow post in Figure~\ref{fig:qa-notification-icon}, 
the approach would not identify relevant text in answers other than the accepted answer,
what further illustrates some of the drawbacks on the usage of meta-data 
to guide the automatic identification of task-relevant text.


% . 
% For instance, within the same type of artifact, bug reports,
% vendors might store data differently (e.g., Jira vs GitHub)
% and this might hinder  out-of-the-box 
% usage of these techniques~\cite{Bavota2016};
  




% Our overview of the state of the art, further detailed in Chapter~\ref{ch:related-work},
The discussion above  evidences limitations of existing techniques 
in assisting developers in locating useful text across different kinds of artifacts.
In summary:

\begin{enumerate}
    \item it shows the need to investigate rules for the relevance of text 
    considering how such text is found across different types of artifacts;
    \item it shows the need for context-specific approaches; and 
    \item it asks how to relax limitations of existing
    artifact-specific approaches so that we can design 
    more generalizable approaches for the automatic identification
    of text relevant to a developer's task.
\end{enumerate}


