\section{State of the art}
\label{cp1:novelty}

\gcm{We need to talk about these section.
It still isn't sufficiently convincing to
show the novelty of this thesis.}


To understand the state of the art
in assisting developers in locating relevant information in the natural language artifacts, 
we survey some of the artifacts studied and 
the approaches proposed to
automate the extraction of text from these artifacts.
Although we do not claim that the list of artifacts and techniques presented hereafter is complete,
they comprise common artifact types sought by developers~\cite{umarji2008archetypal, Li2013}
and many of the techniques used to parse the natural language text in them~\cite{arnaoudova2015}.







In API documentation, Robillard and Chhetri investigated how to automatically identify 
sentences key to using an API element~\cite{Robillard2015}.
Their approach uses regular expressions to match the text in this kind of artifact 
to a set of patterns derived from the Java SDK 6 documentation.
More complex approaches, i.e.,  \acf{ML} and \acf{DL},
were also used by Fucci et al.
with the purpose of automatically identifying sentences with 
directives, concepts, examples, and other types of information included in API documents~\cite{fucci2019}.



In development mailing lists, Panichella et al. 
proposed an automatic approach leveraging \acf{IR} to extract 
text with useful information that assists in understanding code elements inspected by a developer~\cite{panichella2012}
while Di Sorbo et al. use linguistic patterns---extracted via \acf{NLP}---to
automatically categorize the content of development emails, what might 
assist developers in finding text specific to some category, e.g., 
finding text about the solution for a bug or text discussing potential 
new features for a system~\cite{Sorbo2015}.



In bug reports, researchers have mostly applied text summarization
as a means of identifying text that a developer would first read when inspecting a bug report, 
which might assist developers in finding useful information for their task;
and both \acs{ML} and 
\acs{DL} techniques have been investigated for this purpose, 
e.g.,~\cite{Lotufo2012} or~\cite{li2018deep}.



In community forums and \acs{qa} pages, researchers have explored both lexical 
and syntactic approaches for the automatic identification of sentences that would 
help a developer quickly decide if the content in these artifacts is relevant to 
a developer's task~\cite{nadi2020}. Other approaches have also considered extracting information relevant to a developer's 
programming task combining query-based summarization 
and meta-data available on Stack Overflow~\cite{Xu2017, silva2019}.



Although effective, these and other techniques target specific
types of artifacts and, given 
how quickly developers progress to using new kinds of technology to
record pertinent information,
it may be difficult to apply such artifact-centric approaches to cover the range of
artifacts sought by a developer as part of a software task
and, for the techniques that do consider multiple types of artifacts (e.g.,~\cite{Ponzanelli2017}),
automatically identifying text likely relevant to a developer's task 
falls outside of their scope.


