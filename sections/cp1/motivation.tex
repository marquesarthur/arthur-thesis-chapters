

When performing software tasks in large and complex software systems, software developers typically consult a number of different kinds of artifacts likely \textit{relevant} to their task. For example, 
an Android developer might consult official Android API guides, which  
spam more than 30 web pages
covering topics such as images and graphics, data files, background tasks, and many others, or; seek question-and-answer forums for functionality, security and performance-related information 
that might assist them complete their task~\cite{parnin2012}.




Much critical information in these and other types of artifacts that developers seek information on 
contain data in the form of unstructured text~\cite{Bavota2016} and 
a developer must read the text to find the information that is relevant to their task.
However, the sheer amount of information \textit{within} these natural language artifacts may prevent a developer from comprehensively assessing what is needed to complete their task correctly and completely~\cite{Murphy2005}.
Finding relevant information can be a time-consuming
and cognitively frustrating process~\cite{Begel2008,
robillard2011field}.  Just within one kind of document, API
documentation, studies have shown that it can take 15 minutes or more
of a developer's highly constrained time to identify 
information needed to perform a particular software task~\cite{endrikat2014, Meyer2017}.




As tasks become more complex~\cite{Pirolli2007, Bystrom1995}, a developer often has to forage
for more artifacts---locating and combining information to understand what is needed for the task-at-hand~\cite{Piorkowski2016}. If no tool support is provided, much of the process of discovering 
task-relevant information falls on the developer's shoulders and 
a developer that fails to locate all, or most, of the information needed
 will have an incomplete or incorrect basis from which to perform that task~\cite{}.







% Provided that software developers often require knowledge beyond their own to perform software tasks~\cite{Ponzanelli-thesis}, it is reasonable to assume that they will consult 






% For example, when fixing a bug, a developer might 
% confirm the system's behavior referring to past bug reports or the system's requirements~\cite{Lotufo2012, Singer1998}. Likewise, a developer might consult API documents, development mailing lists, or Q\&A forums to better understand how to incorporate a new library into their 
% software system~\cite{umarji2008archetypal,robillard2011field}. 



% Typically, the artifacts sought by a developer and that likely contain information \textit{relevant}
% to their task are available as web resources, where developers formulate queries 
% to locate the artifacts pertinent to the task-at-hand.
% Much of the crucial information in these pertinent artifacts 
% is found in the form of natural language text. However, not all of the content of these artifacts
% might be relevant to the developer's task
% and they have to comprehensively assess what within an artifact is helpful to the 
% task that they are currently performing.    










% need to find 
% additional information that assists them perform their tasks~\cite{}.
% For example, when fixing a bug, a developer might 
% confirm the system's behavior referring to past bug reports or the system's requirements~\cite{}.
% Likewise, a developer might 
% refer to 
% API documentation or community forums 
% to better understand how to incorporate a new library into their software system~\cite{}.



% \art{need some statement about the scope and how often these documents are updated}


% Much critical information in these artifacts is 
% found in the form of natural language text~\cite{}
% and a developer must comprehensively assess 
% which text fragments contain information helpful to their current task~\cite{}.
% Some artifacts are short enough that a developer determines whether they are helpful to the task-at-hand at a quick glance~\cite{}.
% Some others are lengthy~\cite{Rastkar2013t} and 
% a developer faces the 
%  burden  of having to sit through large amounts of irrelevant text to locate the pieces relevant to their task~\cite{Robillard2015}. For example, \art{example about Android API documents}
 
 


% As tasks become more complex~\cite{Pirolli2007, Bystrom1995}, a developer often has to forage
% for more artifacts --
% locating and combining information to understand what is needed for the task-at-hand~\cite{Piorkowski2016}.







% To produce a solution for a task, a software developer typically engages in 
% a variety of information seeking-activities that extend beyond the system's source code. 
% For example, 
% a software developer might refer to 


 






% Software development is a knowledge-intensive
% activity~\cite{}, software developers often 


%  what is a time-consuming
% and cognitively frustrating process~\cite{Begel2008,
% robillard2011field}. For example, just within one kind of artifact, API
% documentation, studies have shown that it can take 15 minutes or more
% of a developer's highly constrained time to identify needed
% information~\cite{endrikat2014, Meyer2017}


% where a developer
% that fails to locate locate all, or most, of the information that is relevant to their task
% may produce incomplete or incorrect solutions~\cite{}.







% As much of the process of locating relevant text within an artifact and establishing relationships
% between the many relevant text fragments falls on the developer's shoulders,
% we argue that developers could benefit from approaches and tools that assist them in
% locating fine-grained task-related information pieces allowing them to effectively discover all the necessary information for a particular task.





% Failing to locate task-relevant text in these quick search episodes can 
% cause a developer to abandon an artifact that could otherwise contain information  helpful to her task~\cite{Brandt2009a, Starke2009}.
% Hence, several studies have proposed automatic approaches to assist this activity. 
% These approaches leverage textual cues as well as an artifact's meta-data and, ultimately, they have the goal of surfacing 
% the most important information in a document so that a developer can quickly 
% locate it.









% As much of the process of locating relevant text within an artifact and establishing relationships
% between the many relevant text fragments falls on the developer's shoulders,





% As tasks become more complex~\cite{Pirolli2007, Bystrom1995}, a developer often consults 
% multiple artifacts---locating and combining information to understand what is needed for their task~\cite{Piorkowski2016}.

 


% If no tool support is provided, a developer has to read large amounts of irrelevant text to filter just those parts that are relevant to her
% and she also has to establish relationships between the relevant information pieces on her own.
% A developer that fails to locate all the necessary information and establish all the appropriate relationships will have an incomplete or incorrect basis from which to complete a task.


 
 
%  needed to complete their task correctly and completely~\cite{Murphy2005}.
% Most notably, a developer has to filter large amounts
% of irrelevant text to locate the parts that are relevant to her task~\cite{Piorkowski2016}.


% identify, from the large amount of text
% in these documents, just the fraction of text relevant
% to the task-at-hand. 










% the vast amount of information available in a software project 
% plays an important role in how software developers
% For example, 


% asking team mates or consulting web artifacts 







% Software developers produce and consume information that assist them 
% complete a software development task.
% For example, 


% % The scale and complexity of modern software systems 
% % often require developers



% For example, a developer might refer to 
% API documentation or community forums 
% to understand how to incorporate a library into their software system~\cite{} or 
%  confirm a system's behavior referring to past bug reports or the system's requirements~\cite{}.




 



% Many of these artifacts contain data in the form of unstructured text~\cite{Bavota2016}.





% and they are continuously evolving due to 


% the sheer amount of information in 
% these natural language artifacts may prevent a developer from 
% comprehensively assessing what is relevant for their task~\cite{Murphy2005}.





%  to complete their task correctly and completely





% In their daily work, software developers typically engage in a series of information seeking activities 












% Much critical information about software is captured in artifacts other than . 




% Software developers typically rely in the information available in these artifacts 












% Software development is an information intensive activity.

% Far too often, developers ask how to use certain 


% Much critical information about software is captured in artifacts other than source code.
% Examples ...










% a software developer performing a software task does not 


% on the information available
% in such artifacts 


% The amount information available in such artifacts plays an important role in the way developers perform a task.
% Ponzanelli-thesis



% Introduce the context

% What is the problem: findinf info for  a task


% Why is it challenging




% Should/do I need an example?







% When provided with a set of artifacts that potentially contain information relevant to their task,
% software developers must inspect the artifact's content and locate the portion of the text that might be relevant to the task-at-hand. 
% Some of the artifacts inspected are short enough that a developer can find if they contain any helpful information at a quick glance.
% Some others are lengthy~\cite{Rastkar2013t} and factors such
% as high time pressure or
% the need to meet deadlines~\cite{meyer2019}
% may lead a developer to quickly skim the document
% in an attempt to find any of the text that is relevant to their task~\cite{Starke2009},
% which Robillard and Chhetri summarize as the information \textit{filtering problem}:

% \smallskip
% \begin{bluequote}
%     \textit{
% \end{bluequote}



% Failing to locate task-relevant text in these quick search episodes can 
% cause a developer to abandon an artifact that could otherwise contain information  helpful to her task~\cite{Brandt2009a, Starke2009}.
% Hence, several studies have proposed automatic approaches to assist this activity. 
% These approaches leverage textual cues as well as an artifact's meta-data and, ultimately, they have the goal of surfacing 
% the most important information in a document so that a developer can quickly 
% locate it.

