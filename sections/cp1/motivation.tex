\section{Motivation}
\label{cp5:motivation}


When provided with a set of artifacts that potentially contain information relevant to their task,
software developers must inspect the artifact's content and locate the portion of the text that might be relevant to the task-at-hand. 
Some of the artifacts inspected are short enough that a developer can find if they contain any helpful information at a quick glance.
Some others are lengthy~\cite{Rastkar2013t} and factors such
as high time pressure or
the need to meet deadlines~\cite{meyer2019}
may lead a developer to quickly skim the document
in an attempt to find any of the text that is relevant to their task~\cite{Starke2009},
which Robillard and Chhetri summarize as the information \textit{filtering problem}:

\smallskip
\begin{bluequote}
    \textit{The burden  of having to sit through large amounts of irrelevant text, e.g., because of legacy information, boilerplate text, or because the text is intended for a reader other than the developer who is currently inspecting the artifact, to locate the pieces relevant to the developer's task}~\cite{Robillard2015}
\end{bluequote}



Failing to locate task-relevant text in these quick search episodes can 
cause a developer to abandon an artifact that could otherwise contain information  helpful to her task~\cite{Brandt2009a, Starke2009}.
Hence, several studies have proposed automatic approaches to assist this activity. 
These approaches leverage textual cues as well as an artifact's meta-data and, ultimately, they have the goal of surfacing 
the most important information in a document so that a developer can quickly 
locate it.

