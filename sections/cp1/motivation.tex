

When performing software tasks in large and complex software systems, software developers typically consult a number of different kinds of artifacts that assist them in their work~\red{\cite{}}. For example, 
when developing an Android application, a developer might consult official Android API documents and guidelines~\red{\cite{}} or she might consult question-and-answer forums for functionality, security and performance-related topics~\cite{parnin2012}.




Much critical information in this and other non-source code types of artifacts 
% that developers seek information on 
contain data in the form of unstructured text~\cite{Bavota2016} and 
a developer must read the text to find the information that is \textit{relevant} to the task being performed.
However, the sheer amount of information \textit{within} these natural language artifacts may prevent a developer from comprehensively assessing what is useful to their task~\cite{Murphy2005}.  Just within one kind of document, API
documentation, studies have shown that it can take 15 minutes or more
of a developer's highly constrained time to identify 
information needed to perform a particular software task~\cite{endrikat2014, Meyer2017}.


Therefore, finding information that assists a developer complete a task can be a time-consuming
and cognitively frustrating process~\cite{Begel2008,
robillard2011field}.
If no tool support is provided, much of this information foraging process~\red{\cite{}} falls on the developer's shoulders~\red{\cite{}}. A developer that fails to locate all, or most, of the information needed
 will have an incomplete or incorrect basis from which to perform that task~\red{\cite{}}.

%  Therefore, researchers have long recognized the need to 


% needed 

% As tasks become more complex~\cite{Pirolli2007, Bystrom1995}, a developer might also need to forage
% for more artifacts---locating and combining information to understand what is needed for the task-at-hand~\cite{Piorkowski2016}.





Researchers have long recognized the need to assist a developer's discovery of information,
utilizing a number of
\textit{artifact-centric} techniques
to extract
information that can be embedded in
tools for software developers. 
A number of the techniques commonly employed by researchers utilize pattern matching (or regular expressions)~\cite{Maalej2013, Bavota2014, Chaparro2017}   
to check for specific sequences of tokens often present in text relevant to certain tasks. For example, \textit{Krec}~\cite{Robillard2015} utilizes pattern matching to identify text that a developer cannot afford to ignore when reading an API document.
Other techniques that researchers have explored rely on extractive text summarization~\cite{Rastkar2010, Lotufo2012, Murray2008}. 
A summary can be interpreted as the most relevant information in a software artifact 
and tools such as \textit{DeepSum}~\cite{Li2018} summarize the content of bug reports to determine 
the most relevant information regarding the reported problem, solutions discussed, and the bug's resolution. 





Although effective, this and other techniques~\red{\cite{}} typically
target specific kinds of task and a single type of artifact, limiting their use across the
many different kinds of artifacts developers encounter
daily in their work~\red{\cite{}}.
If one technique could use the developer's task
to identify relevant text across the various kinds
of artifacts a developer encounters
it would be possible for a
developer to quickly access information needed 
to complete their task correctly and completely.



