

\section{Thesis Statement}
\label{cp1:thesis}





The scenario presented earlier in this chapter demonstrates
the
 many kinds of 
artifacts a developer may consult to help complete
 a task and the challenges a developer faces in identifying text relevant to the task at hand  within these 
artifacts.
Existing techniques that support a developer
in the automatic identification of relevant
text are constrained to working on one or only a 
few types of artifacts. This thesis aims to
overcome this constraint. We posit that:
% \gcm{This is getting closer. I wonder if it needs to
% be clearer that a technique is needed that
% can evolve with artifact types and that one
% technique is easier to deploy/install than many.}

\bigskip
\begin{bluequote}
    \textit{A developer can effectively complete a software development task when automatically provided with text relevant to their task extracted from pertinent natural language artifacts
    by a generalizable technique.}
\end{bluequote}
\medskip



The design of a generalizable technique might be
more possible if the text relevant to
a task follows similar patterns
 across different types of artifacts~\cite{Kintsch1978a}.
Hence, we ask what are common properties, if any, in the text deemed relevant 
to a software task  found across different types of artifacts?
To investigate this question, we performed a formative study that 
examines  text that twenty developers deemed relevant in artifacts 
of different types associated with six software tasks.
By \textit{characterizing} task-relevant text found in different kinds of artifacts,
this study complements and adds to previous research that has
examined text relevant to particular tasks and one kind of artifact~\cite{Ko2006, Rastkar2010, Chaparro2017, Robillard2015}.
Notably, our analysis of natural language text  in bug
reports, API documents and \ac{qa} websites 
inspected as part of this study 
 show consistency in the meaning, or \textit{semantics}, of the
 text deemed relevant to a task by a developer, suggesting that 
semantics might assist in the automatic identification of
task-relevant text and in the design of a more generalizable technique.




Approaches that interpret the meaning of the text have been successfully used for a variety of development activities,
such as for finding who should fix a bug~\cite{yang2016}, searching for comprehensive code examples~\cite{silva2019}, or assessing the quality of information available in bug reports~\cite{chaparro2019}.
Nonetheless, we are not aware of the usage of the meaning of text in techniques 
that assist developers in discovering task-relevant text over different types of natural language artifacts.
Hence, this thesis describes the investigation of a design space
of six possible techniques that incorporate the semantics of words~\cite{Mikolov2013, Devlin2018Bert}
and sentences~\cite{fillmore1976frame, marques2021}
to automatically identify text likely relevant to a developer's task.
An empirical assessment of these techniques 
on a corpus of tasks
and artifacts reveals that semantic-based techniques
achieve recall comparable to a state-of-the-art technique aimed at one type of artifact~\cite{Xu2017}
while supporting
multiple artifact types.






With our formative study, we found consistency in the text  that is relevant to a task across different kinds of artifacts.
With our empirical assessment, we found that a semantic-based techniques can identify such text
automatically across many artifact types.
However, these studies do not address whether a semantic-based approach can \textit{assist} a software developer while they work on a task.
Hence, we introduce \acs{tool}\footnote{
    Automatic \underline{Ta}sk-\underline{R}elevant \underline{T}ext \underline{I}dentifier
}, a web browser plug-in that 
uses the most promising semantic-based technique identified in our empirical assessment
to automatically identify and highlight text relevant to a developer's task .
We evaluate \acs{tool} with a controlled experiment that investigates the presence (or absence) of benefits that are brought by using the tool.
We report how  participants 
performed two Python programming tasks when 
 assisted or not by \acs{tool},
where experimental results indicate that participants found the text identified automatically
by the tool 
useful in two out of the three tasks of the experiment.
Furthermore, tool support also led to more correct solutions 
in one of the tasks in the experiment, providing thus
initial evidence on the role of semantic-based tools 
for supporting a developer's discovery of task-relevant information
across different natural language artifacts.
