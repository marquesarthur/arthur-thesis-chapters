

\section{Thesis}
\label{cp1:thesis}



We seek to support a developer's discovery of information potentially relevant to their task. 
We posit that:

\medskip
\begin{bluequote}
    \textit{A developer can more effectively complete a software development task when automatically provided with text relevant to their task extracted from pertinent natural language artifacts.}
\end{bluequote}
\medskip




To validate this thesis statement, 
we must consider whether there is consistency to 
the text 
considered key to completing a task
in the different types of pertinent natural language artifacts.
If there is consistency, we can use properties inherent 
to the portions of the text that contain this key information 
to design techniques that automatically identify the relevant text
regardless of an artifact's type, evaluating the impact of tools that use such techniques to assist developers in their work. 
The approach this work of dissertation takes to validate each of these steps is as follows.



\paragraph{\textbf{Characterizing Task-relevant Text.}} 


We start by investigating the relevance of text in natural language software artifacts
through a controlled experiment.
Based on the text that participants considered relevant in the artifacts of six software tasks,
we find that 
locating task-relevant information in bug
reports, API documents and \acf{qa} web sites require filtering
to less than a 20\% of an artifact's text.
% In 
% the portions of the text 
% considered key to completing a task. 
We also 
observe consistency in the meaning, or \textit{semantics}, of the
 text considered relevant
 suggesting that the semantics of natural language artifacts might 
 be key to the design of automatic approaches that detect relevant information.


This study contributes to the body of work that examines possible properties for identifying text relevant to a developer's task (e.g.,~\cite{Forward2002, Jiang2016b, Robillard2015, Bavota2016}).
% The novelty of our work lies in 
% applying a general
% linguistic approach, namely frame semantics~\cite{fillmore1976frame, Baker1998}, 
% to study the meaning of text relevant to software tasks.
A second contribution arises from analyzing
common strategies that developers adopt to locate relevant text
in natural language software artifacts 
and from discussing
factors that make assessing the relevancy of text difficult,
what further motivates this work of dissertation.






\paragraph{\textbf{Identifying Task-relevant Text.}} 


Since we identify consistency in the meaning of the text deemed relevant to a task, 
in the second part of this thesis, we investigate
techniques that build upon semantic approaches 
for automatically identifying text relevant to a particular task in artifacts pertinent to that task.


We introduce six possible techniques that incorporate the semantics of words~\cite{Mikolov2013, Devlin2018Bert}
and sentences~\cite{fillmore1976frame, marques2021}
to automatically identify text likely relevant to a developer's task.
We compare these techniques to a artifact-specific technique, AnswerBot~\cite{Xu2017},
where we evaluate the text that each of them identify
 over different types of artifacts
associated with fifty Android development tasks 
for which human annotators identified task-relevant text.
Evaluation results show that semantic-based techniques
achieve recall comparable to the state-of-the-art, but without the need for artifact-specific data,
and that some of our techniques perform equivalently well across
multiple artifact types, identifying up to 58\%
of the text 
deemed relevant to these Android development tasks.


Semantic-based approaches have been successfully used for a variety of development activities, such as
for finding who should fix a bug~\cite{yang2016}, searching for comprehensive code examples~\cite{silva2019},
or assessing the quality of information available in bug reports~\cite{chaparro2019}.
This thesis complements these and other studies in the field
by investigating how semantic approaches apply across different artifact types
commonly sought by developers in their daily work.




\paragraph{\textbf{Evaluating an Automated Approach to Task-relevant Text Identification.}} 




In the last part of my work, I return to my thesis statement
and examine the impact of tools that use semantic-based techniques to assist developers in 
completing a task. 



I present a controlled experiment where participants had to 
perform two Python programming when assisted (or not) by a tool that automatically identifies task-relevant text 
in the set of artifacts available for each task. 
With this experiment, I compare the correctness of the solutions of each task 
performed by participants with and without tool assistance
as well as the perceived usefulness of the text automatically identified and shown by the tool. 
Results indicate that participants found the text automatically identified
useful in two out of the three tasks of the experiment.
In addition, tool support also led to more correct solutions 
in one of the task in the study.





This experiment strengthen software engineering research on the 
role of semantic-based tools for supporting
automatic text identification~\cite{nadi2020, Xu2017,Lotufo2012}
and we conclude this thesis by discussing implications of our findings and 
directions for future work.




