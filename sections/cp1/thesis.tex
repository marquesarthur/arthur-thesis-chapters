

\section{Thesis}
\label{cp1:thesis}



This thesis 
aims to support a developer in locating information of potential relevance to their task. I posit that:

\medskip
\begin{bluequote}
    A developer can more effectively complete a software development task when provided
    with text that an automatic technique determines as relevant to the developer's task.
    This technique applies across different kinds
    of artifacts a developer might encounter, thus supporting their discovery of task-relevant information.
\end{bluequote}





To validate this thesis statement, 
we must consider whether there is consistency to 
the text 
considered key to completing a task
in the different types of natural language artifacts pertinent to that task.
If there is consistency, we can use properties inherent 
to the portions of the text that contain this key information 
to design techniques that automatically identify text relevant to a task 
regardless of an artifact's type.
Thereupon, evaluating the impact of tools that use such techniques to assist developers in completing software tasks. 
The approach this work of dissertation takes for each of these steps is detailed as follows.



\paragraph{\textbf{Characterizing Task-relevant Text.}} 


We start by investigating the relevance of text in natural language software artifacts
pertinent to six distinct software tasks.
We report a controlled experiment in which we asked software developers to 
indicate the text they considered relevant to each of the tasks we explored and their associated artifacts.
We find that 
locating task-relevant information in bug
reports, API documents and \acf{qa} web sites require filtering
to less than a 20\% of an artifact's text.
In 
the portions of the text 
considered key to completing a task. We also 
observe consistency in the meaning, or \textit{semantics}, of the
task-relevant text.


This experiment contributes to the body of work that examines possible properties for identifying text relevant to a developer's task (e.g.,~\cite{Forward2002, Jiang2016b, Robillard2015, Bavota2016}).
The novelty of our work lies in 
applying a general
linguistic approach, namely frame semantics~\cite{fillmore1976frame, Baker1998}, 
to study the meaning of task-relevant text in 20 natural language software artifacts.
A second contribution arises from analyzing
common strategies that developers adopt to locate relevant text 
and from discussing
factors that make assessing the relevancy of text difficult,
what further motivates this work of dissertation.






\paragraph{\textbf{Identifying Task-relevant Text.}} 


Since we identify consistency in the meaning of the text deemed relevant to a task, 
in the second part of this thesis, we investigate
techniques that build upon semantic approaches 
for automatically identifying task-relevant text.


We introduce six semantic-based techniques that incorporate the semantics of words~\cite{Mikolov2013, Devlin2018Bert}
and sentences~\cite{fillmore1976frame, marques2021}
to automatically identify text likely relevant to a developer's task.
We compare the techniques we explore to a  state-of-the-art artifact-specific technique, AnswerBot~\cite{Xu2017},
 evaluating them over different types of artifacts
associated with 50 Android development tasks 
for which human annotators identified task-relevant text.
Evaluation results show that semantic-based techniques
achieve recall comparable to AnswerBot, but without the need for artifact-specific data,
and that some of our techniques perform equivalently well across
multiple artifact types, identifying up to 58\%
of the text 
deemed relevant to these Android development tasks.


Semantic-based approaches have been successfully used for a variety of development activities, such as
for finding who should fix a bug~\cite{yang2016}, searching for comprehensive code examples~\cite{silva2019},
or assessing the quality of information available in bug reports~\cite{chaparro2019}.
This thesis complements these and other studies in the field
by investigating how semantic approaches apply across different artifact types
commonly sought by developers in their daily work.




\paragraph{\textbf{Evaluating an Automated Approach to Task-relevant Text Identification.}} 




In the last part of my work, I return to my thesis statement
and examine the impact of tools that use semantic-based techniques to assist developers in 
completing a task. 



I present a controlled experiment where 24 participants had to 
perform two Python programming tasks each, where one of these tasks 
was assisted by a tool that automatically identifies task-relevant text 
in the set of artifacts available to each task. 
With this experiment, I compare the correctness of the solutions of each task 
performed by participants with and without tool assistance 
when they consulted the exact same set of artifacts, which included 
API documents, Stack Overflow posts, and web tutorials. 
Results indicate that participants found the text automatically identified
and shown by the tool were useful in two out of the three tasks of the experiment.
In addition, tool support also led to more correct solutions 
in one of the task in the study.





This experiment strengthen other findings on the 
role of semantic-based tools for supporting
automatic text identification~\cite{nadi2020, Xu2017,Lotufo2012}.
We conclude this thesis by discussing implications of our findings and 
directions for future work.




