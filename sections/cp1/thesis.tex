

\section{Thesis}
\label{cp1:thesis}


My thesis is that 
a developer can more effectively complete a software development task when  provided
with text  determined as relevant 
by an automatic technique that applies to 
different natural language artifacts pertinent to the developer's task.
  







% according to the meaning, or \textit{semantics}, of the text. \art{review needs to put related work in place}



An important aspect to consider
is whether there is consistency in 
information considered \emph{key} to completing a task.
If there is consistency, we can use properties inherent 
to the portions of the text that contain this key information 
to design automatic techniques able to identify such text.
Thereupon, evaluating the impact of tools that use such techniques to assist developers in completing software tasks.




To investigate relevance of text to tasks,
this thesis presents 
a controlled experiment in which we asked 20 software developers to examine 20 natural language artifacts
and indicate  the text they considered relevant to six software tasks.
Results from this experiment indicate consistency in 
the information considered key to completing a tasks, as captured by the meaning, or \textit{semantics}, of the
relevant text.
This experiment contributes to the body of work that examine possible properties for identifying text relevant to a developer's task (e.g.,~\cite{Forward2002, Jiang2016b, Robillard2015, Bavota2016}).
The novelty of our work arises from 
applying a general
linguistic approach, namely frame semantics~\cite{fillmore1976frame, Baker1998}, to study the meaning of relevant text found in 
API documents, Q\&A web sites and bug reports,
and from 
informing common strategies that developers adopt to locate relevant text 
and factors that make assessing the relevancy of text difficult.




% whether the cues used to determine relevant text in certain tasks and artifacts 
%  also apply to different tasks and artifacts. 




% the meaning, or \textit{semantics},




% an approach that captures the key meaning of sentences, suggesting that frame semantics can be used in the future to automatically identify task-relevant information in natural language artifacts.






% . 
% Guided by recent success  using
% techniques that interpret the meaning, or \textit{semantics}, of text
% for a variety of development activities, I posit that there is consistency in the meaning of text 



% that might aid in designing such automated technique.


% such as
% for finding who should fix a bug~\cite{yang2016}, searching for comprehensive code examples~\cite{silva2019}, 
% or assessing the quality of information available in bug reports~\cite{chaparro2019}, 












% To design such a technique, it is important to consider 
% whether there is consistency in the text
% that software developers identify as relevant 
% in distinct artifacts pertinent to particular software tasks.
% If there is consistency,
% common factors that
% developers use---regardless of an artifact's type---could provide valuable
% insights for the design of automatic techniques~\cite{Bavota2016, Walters2014}
% making it possible to embed such techniques into tools that assist 
% a developer's discovery of task-relevant information.





% Semantic approaches have  been successfully used 
% to assist developers in a variety of development activities, 
% such as finding who should fix a bug~\cite{yang2016},
% searching for comprehensive code examples~\cite{silva2019},
% or assessing the quality of information available in bug reports~\cite{chaparro2019} 
% and thus, I posit that 




% Hence, I posit that semantics applies for identifying text containing information
% relevant to a developer task in a more general manner. 







% \smallskip
% Semantic approaches have been 








% To validate this thesis statement, I start by investigating whether there is consistency in the meaning of the text within various natural language artifacts that software developers identify as relevant to a task.
% This investigation provides the basis for determining common semantic cues that developers use to locate task-relevant text regardless of an artifact's type. 
% As a result, we can design automatic techniques that use such semantic cues to identify text relevant to a developer's task.
% Thereupon, evaluating the impact of tools that use such techniques to assist developers in completing software tasks.




% My thesis is that interpreting the meaning, or \textit{semantics}, 
% of text poses a more generalizable approach for identifying information
% relevant to a developer's task in a variety of natural language
% software artifacts pertinent to that task.