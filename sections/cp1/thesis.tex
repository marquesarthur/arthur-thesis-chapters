

\section{Thesis}
\label{cp1:thesis}


My thesis is that:




\medskip
\begin{bluequote}
    \textit{A developer can more effectively complete a software development task when  provided
    with text  determined as relevant 
    by an automatic technique that applies to 
    different types of natural language artifacts pertinent to the developer's task.}
\end{bluequote}




To validate this thesis statement, 
we must consider whether there is consistency to 
the text considered key to completing a task (Part I).
If there is consistency, we can use properties inherent 
to the portions of the text that contain this key information 
to design our automatic techniques that apply to 
different artifact types (Part II).
Thereupon, evaluating the impact of tools that use such techniques to assist developers in completing software tasks (Part III). 






\paragraph{\textbf{Part I.}} 


In the first part of this thesis, we investigate the relevance of text in natural language software artifacts
to six distinct software tasks.
We report a controlled experiment in which we asked 20 software developers to examine 20 artifacts
and indicate  the text they considered relevant to each of the tasks we explored.
Results from this experiment indicate that 
finding task-relevant information in bug
reports, API documents and \acf{qa} web sites require filtering
to less than a 20\% of an artifact's text.
We also observe consistency in 
the portions of the text 
considered key to completing a tasks, as captured by the meaning, or \textit{semantics}, of the
 text.


This experiment contributes to the body of work that examine possible properties for identifying text relevant to a developer's task (e.g.,~\cite{Forward2002, Jiang2016b, Robillard2015, Bavota2016}).
The novelty of our work lies in 
applying a general
linguistic approach, namely frame semantics~\cite{fillmore1976frame, Baker1998}, to study the meaning of task-relevant text found in 
API documents, Q\&A web sites and bug reports.
A second contribution arises from a qualitative analysis of
the common strategies that developers adopt to locate relevant text 
and factors that make assessing the relevancy of text difficult,
what further motivates the need for tools that assist such process.






\paragraph{\textbf{Part II.}} 


In the second part of this thesis, we investigate
techniques building on approaches to interpreting the meaning of the text 
for automatically identifying task-relevant text.
We introduce six semantic-based techniques that incorporate the semantics of words~\cite{Mikolov2013} and sentences to automatically identify text likely relevant to a developer's task.
We compare our proposed techniques to a artifact-specific state-of-the-art technique, AnswerBot~\cite{Xu2017},
and we evaluate them over different types of artifacts
associated with 50 software tasks about Android development
for which human annotators identified pertinent text.


Evaluation results show that semantic-based techniques
achieve recall comparable to AnswerBot, but without the need for artifact-specific data,
and that some of our proposed techniques perform equivalently well across
multiple artifact types, identifying up to 58\%
of the text 
in API documentation, Stack Overflow answers,
GitHub issue discussions and other web pages
deemed relevant to these Android development tasks.


% This thesis complements existing work
% by investigating six semantic-based techniques 
% to identify task-relevant text across different artifact types.
Semantic-based approaches have been successfully used for a variety of development activities, such as
for finding who should fix a bug~\cite{yang2016}, searching for comprehensive code examples~\cite{silva2019},
or assessing the quality of information available in bug reports~\cite{chaparro2019}.
We compliment these and other studies in the identification of task-relevant information
by investigating how such semantic approaches apply across different artifact types.




\paragraph{\textbf{Part III.}} 



% whether the cues used to determine relevant text in certain tasks and artifacts 
%  also apply to different tasks and artifacts. 




% the meaning, or \textit{semantics},




% an approach that captures the key meaning of sentences, suggesting that frame semantics can be used in the future to automatically identify task-relevant information in natural language artifacts.






% . 
% Guided by recent success  using
% techniques that interpret the meaning, or \textit{semantics}, of text
% for a variety of development activities, I posit that there is consistency in the meaning of text 



% that might aid in designing such automated technique.


% such as
% for finding who should fix a bug~\cite{yang2016}, searching for comprehensive code examples~\cite{silva2019}, 
% or assessing the quality of information available in bug reports~\cite{chaparro2019}, 












% To design such a technique, it is important to consider 
% whether there is consistency in the text
% that software developers identify as relevant 
% in distinct artifacts pertinent to particular software tasks.
% If there is consistency,
% common factors that
% developers use---regardless of an artifact's type---could provide valuable
% insights for the design of automatic techniques~\cite{Bavota2016, Walters2014}
% making it possible to embed such techniques into tools that assist 
% a developer's discovery of task-relevant information.





% and thus, I posit that 




% Hence, I posit that semantics applies for identifying text containing information
% relevant to a developer task in a more general manner. 







% \smallskip
% Semantic approaches have been 








% To validate this thesis statement, I start by investigating whether there is consistency in the meaning of the text within various natural language artifacts that software developers identify as relevant to a task.
% This investigation provides the basis for determining common semantic cues that developers use to locate task-relevant text regardless of an artifact's type. 
% As a result, we can design automatic techniques that use such semantic cues to identify text relevant to a developer's task.
% Thereupon, evaluating the impact of tools that use such techniques to assist developers in completing software tasks.




% My thesis is that interpreting the meaning, or \textit{semantics}, 
% of text poses a more generalizable approach for identifying information
% relevant to a developer's task in a variety of natural language
% software artifacts pertinent to that task.