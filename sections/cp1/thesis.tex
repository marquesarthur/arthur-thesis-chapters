

\section{Thesis}
\label{cp1:thesis}


% Developers produce such natural language artifacts on a 
% continuous basis~\cite{Rastkar2013t} 
% and locating the documents of interest for a task
% is not trivial~\cite{Starke2009}. 
% To better understand this activity, 
% many software engineering researchers 
% have studied  a developer's information foraging process~\cite{Pirolli1999}
% providing theories on how a developer searches for pertinent information
% and proposing tools to help this activity.



As there is an underlying structure to many software development tasks~\cite{Murphy2005},
we hypothesize that one can use information about a task to assist in 
the design of \textit{artifact-agnostic} techniques able to  
automatically identify text relevant to a developer's task~\cite{Starke2009, Bavota2016}. We posit that:


\bigskip
\begin{bluequote}
    % \textit{A developer can more effectively complete a software development task when automatically provided with text relevant to their task extracted from pertinent natural language artifacts by an artifact-agnostic technique.}
    \textit{A technique that applies to different kinds of natural language artifacts associated with software development assists a developer in correctly completing a software task by 
    automatically providing them with text relevant to their task.  }
\end{bluequote}
\medskip



To investigate this statement and to design a more generalizable technique, we ask what are common properties, if any, in the text deemed relevant to a software task? This question has been the focus of many software engineering studies~\cite{Piorkowski2015, Piorkowski2016, chi2007, Ko2006a} that have explained qualitative aspects that 
guide a developer's decision on the relevance of natural language text~\cite{Forward2002, BenCharrada2016, Starke2009, DeGraaf2014}.
% Many other studies have proposed tools that automatically identify text 
% relevant to certain types of tasks and kinds of artifacts~\cite{Chaparro2017, Robillard2015, Xu2017}. 
Researchers have also contributed with valuable corpora
containing text annotated as relevant to particular tasks and artifacts~\cite{nadi2019Rep, Rastkar2010}
as well as automatic approaches for extracting such text~\cite{Chaparro2017, Robillard2015, Xu2017}, 
but investigating if  the properties 
of the text in certain artifacts apply to different kinds of artifacts
is beyond their scope~\cite{hutchinson2021, bird2009}.



To address gaps left by these studies on the relevance of natural language text
and to provide a foundation for this work of dissertation,
we present a formative study that \textit{characterizes} task-relevant text 
found in different kinds of artifacts. 
We examine the text that twenty developers deemed relevant in the  artifacts  associated with six software tasks to investigate 
rules that can guide us to relevant text~\cite{Kintsch1978a}.
Analysis of the task-relevant text  in bug
reports, API documents and \acf{qa} websites 
inspected as part of this study 
 show consistency in the meaning, or \textit{semantics}, of the
 text, suggesting that 
semantics might assist in the automatic identification of
task-relevant text for the different types of artifacts a developer 
might use when performing a task~\cite{Meyer2019, Li2013}.



% by




Approaches that interpret the meaning of the text have been successfully used for a variety of development activities,
such as for finding who should fix a bug~\cite{yang2016}, searching for comprehensive code examples~\cite{silva2019}, or assessing the quality of information available in bug reports~\cite{chaparro2019}.
Nonetheless, few studies~\cite{Ponzanelli2017, Liu2018Unakite} have investigated if and how accurately such approaches identify
text likely 
useful to a developer's task across the different types of natural language artifacts available online.


To determine whether
techniques that build upon semantic approaches 
apply to our domain problem,
the second part of this thesis describes 
the investigation of a design space
of six possible techniques that incorporate the semantics of words~\cite{Mikolov2013, Devlin2018Bert}
and sentences~\cite{fillmore1976frame, marques2021}
to automatically identify text likely relevant to a developer's task.
Assessment of these techniques reveals that semantic-based techniques
achieve recall comparable to a state-of-the-art technique~\cite{Xu2017}, but without the need for artifact-specific data,
and that some of our techniques also perform equivalently well across
multiple artifact types, what strengthens the claim that semantic-based  
techniques are more generalizable.


% Assessment of these techniques considers the text that each of them identifies
%  over different types of artifacts
% associated with fifty Android development tasks 
% for which human annotators identified task-relevant text.




Provided that we have found consistency in the text considered relevant within the natural language text of artifacts pertinent to a task and that semantic-based techniques can automatically identify such text; in the last part of this thesis, we examine the impact of tools that use semantic-based techniques to assist developers in 
completing a task. 



We present a controlled experiment where participants had to 
perform two Python programming tasks when assisted (or not) by a tool that automatically identifies task-relevant text 
in a curated set of artifacts associated with the tasks considered. 
With this experiment, we compare the correctness of the solutions of each task 
performed by participants with and without tool assistance
as well as the perceived usefulness of the text automatically identified and shown by the tool. 
Results indicate that participants found the text automatically identified
useful in two out of the three tasks of the experiment
and that tool support also led to more correct solutions 
in one of the tasks in the experiment. These results provide
initial evidence on the role of semantic-based tools 
for supporting a developer's discovery of task-relevant information
across different natural language artifacts.


% \art{should I have a concluding phrase here?}



% This experiment strengthen software engineering research on the 
% role of semantic-based tools for supporting
% automatic text identification~\cite{nadi2020, Xu2017,Lotufo2012}
% and we conclude this thesis by discussing implications of our findings and 
% directions for future work.




