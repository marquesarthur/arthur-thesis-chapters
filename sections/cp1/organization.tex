\section{Structure of the Thesis}
\label{cp1:organization}


In Chapter~\ref{ch:related-work}, we describe background information 
and previous approaches that seek to assist developers in 
finding useful information in natural language artifacts. 
The chapter details 
how developers forage information, existing approaches and tools 
that assist in the automatic
identification of text, and how these tools fit under the 
umbrella of studies that seek to 
improve a developer's work.


Chapter~\ref{ch:characterizing} presents our empirical study to characterize task-relevant text.
We provide details on the tasks and artifacts that we have selected for this study
and then we present our findings on the text considered relevant, 
how we observe consistency on the semantics of the task-relevant text,
and what are common challenges faced by developers trying to locate task-relevant text.


Chapter~\ref{ch:android-corpus} describes the groundwork 
for producing the corpus (\acs{DS-android}) that we use to evaluate the techniques 
detailed in the chapter that follows. It describes the selection of tasks, 
and 
artifacts pertinent to each task,
as well as how three human annotators identified relevant text in each of the artifacts gathered.



Chapter~\ref{ch:identifying} details the semantic-based techniques we investigate for automatically 
identifying task-relevant text.
The first two techniques that the chapter presents 
use word embeddings to identify likely relevant text via semantic similarity
and via a neural network.
A third sentence-level technique filters (or not) 
the output of the word-level techniques according to frame semantics analysis.
We combine these techniques for a total of six possible techniques, 
assessing 
the text that they automatically identify for the tasks and artifact types
available in the \acs{DS-android} corpus,
where we find that the neural network and the frame semantics techniques
are the most promising ones for automatically identifying 
task-relevant text. 




Chapter~\ref{ch:assisting} details our empirical experiment investigating 
whether \acs{tool}---a tool embedding a semantic-based technique---assists a software developer in locating information
that helps them complete Python programming tasks. We begin by detailing experimental procedures
and then, we report results from the experiment.


In Chapter~\ref{ch:discussion}, we discuss challenges and decisions 
made throughout our work, implications from our findings,
and potential future work.


Chapter~\ref{ch:summary} concludes this work by reflecting on the contributions in the thesis. 
