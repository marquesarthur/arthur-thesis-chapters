\section{Structure of the Thesis}
\label{cp1:organization}

In Chapter~\ref{ch:related-work}, we present an overview of state of the art. The chapter details 
how developers forage information and existing approaches and tools 
that assist in different phases of this process.


Chapter~\ref{ch:characterizing} presents our empirical study to characterize task-relevant text.
We provide details on the tasks and artifacts that we have selected for this study (Section~\ref{})
and then, we present our findings on the text considered relevant (Section~\ref{}).
Based on post-experiment interview analysis, we also discuss  
on the common themes related to locating task-relevant text (Section~\ref{}).


Chapter~\ref{ch:android-corpus} describes the groundwork 
for producing the corpus (\acs{DS-android}) that we use to evaluate the semantic-based techniques 
detailed in Chapter~\ref{ch:identifying}.
The first two techniques that the chapter presents 
use word embeddings to identify likely relevant text via semantic similarity
and via a deep neural network.
A third sentence-level technique filters (or not) 
the output of the word-level techniques according to a sentence's meaning, as captured by frame semantics (Section~\ref{cp5:approaches}).
We combine these techniques for a total of six possible techniques, comparing them by showing
the text that they automatically identify for the tasks and artifact types
available in the \acs{DS-android} corpus  (Section~\ref{cp5:evaluation}).




In chapter~\ref{ch:assisting}, we take the best performing technique identified in this empirical assessment and 
we investigate 
whether such technique assists a software developer in locating information
that helps them complete Python programming tasks. We begin by detailing experimental procedures (Section~\ref{cp6:experiment})
and then, we report results from the experiment (Section~\ref{cp6:results}).


In Chapter~\ref{ch:discussion}, we discuss challenges and decisions 
made throughout our work 
as well as implications from our findings. 
Chapter~\ref{ch:summary} concludes this work by reflecting on the contributions in the thesis
and by outlining  potential future work. 
\art{Will review where I will have future work}