\section{Structure of the Thesis}
\label{cp1:organization}

In Chapter~\ref{ch:related-work}, we present an overview of state of the art. The chapter details 
how developers forage information and existing approaches and tools 
that assist in different phases of this process.


Chapter~\ref{ch:characterizing} presents our empirical study to characterize task-relevant text.
We provide details on the tasks and artifacts that we have selected for this study (Section~\ref{})
and then, we present our findings on the text considered relevant (Section~\ref{}).
It also discusses common factors related to locating task-relevant text (Section~\ref{}).


Chapter~\ref{ch:android-corpus} describes the groundwork 
for producing the corpus (\acs{DS-android}) that we use to evaluate the semantic-based techniques 
detailed in the chapter that follows. It describes the selection of tasks (Section~\ref{cp4:corpus-tasks}) and 
artifacts pertinent to each task (Section~\ref{cp4:corpus-artifacts})
as well as how three human annotators identified relevant text in each of the artifacts gathered (Section~\ref{cp4:corpus-relevant-text}).


Chapter~\ref{ch:identifying} details the semantic-based techniques we investigate for automatically 
identifying task-relavant text.
The first two techniques that the chapter presents 
use word embeddings to identify likely relevant text via semantic similarity
and via a deep neural network.
A third sentence-level technique filters (or not) 
the output of the word-level techniques according to frame semantics analysis (Section~\ref{cp5:approaches}).
We combine these techniques for a total of six possible techniques, comparing them by showing
the text that they automatically identify for the tasks and artifact types
available in the \acs{DS-android} corpus  (Section~\ref{cp5:evaluation}).




In chapter~\ref{ch:assisting}, 
we investigate 
whether a tool embedding a semantic-based technique assists a software developer in locating information
that helps them complete Python programming tasks. We begin by detailing experimental procedures (Section~\ref{cp6:experiment})
and then, we report results from the experiment (Section~\ref{cp6:results}).


In Chapter~\ref{ch:discussion}, we discuss challenges and decisions 
made throughout our work 
as well as implications from our findings. 
Chapter~\ref{ch:summary} concludes this work by reflecting on the contributions in the thesis
and by outlining  potential future work. 
\art{Will review where I will have future work}