\setcounter{chapter}{3}
\setcounter{rq}{1}


\chapter{Android Task Corpus}
\label{ch:android-corpus}



% ART: move to an early chapter
% To produce a solution for a task, a developer engages in a variety of
% information-seeking activities.  One of these activities involves
% the identification of  text in artifacts \gm{relevant to helping solve the task}.

Developing techniques that can automatically identify text
relevant to a task requires a means of evaluating whether
a proposed technique works well. In this context,
``working well'' refers to whether the text identified
by an automatic technique is similar to text considered
by humans as being relevant to a task at hand. To support
the evaluation of proposed techniques, a corpus
are needed that include tasks to be performed on
systems, artifacts representing different kinds of information
that would be useful to humans performing the tasks
and an identification of text in those artifacts that
is helpful to the humans.

As we have described earlier in this thesis, while there
exist techniques tuned to identifying relevant text in
specific artifact types, there are not techniques able to
identify text across a range of artifact types. Thus,
the corpa that exist consist of tasks with only single,
or a small number, of types of artifacts. As a result,
there was a need to create a corpa that included
multiple artifacts types for a task. This chapter describes the
development of a corpus that overcomes the limitations of
existing corpa, bringing together tasks from an existing
well-known domain, Android development, with relevant text
for solving that task from multiple artifact types.


Figure~\ref{fig:corpus-creation-pipeline}
summarizes the process we use to create the corpus, which we call \acs{DS-android}.
We randomly sample a set of 50 Android development tasks from two common
task sources, GitHub\footnote{\url{https://github.com/}} and Stack Overflow\footnote{\url{https://stackoverflow.com/}}.
For each of these tasks, we use the Google search engine to find potential artifacts that likely contain relevant
information for that task in top-ranked Android development websites. 
We then gather task-relevant sentences from these artifacts
from experienced developers.
We detail each of these steps in turn in 
Sections~\ref{cp4:corpus-tasks},~\ref{cp4:corpus-artifacts} and ~\ref{cp4:corpus-relevant-text}.
We provide a summary of the final created corpus in Section~\ref{cp4:corpus-summary}.



We publicly share the corpus to help future research in the field~\cite{supplementary_material}.



\begin{figure}
    \centering
    \includegraphics[width=1.05\textwidth]{cp4/corpus-creation-pipeline}
    \caption{Summary of procedures for corpus creation}
    \label{fig:corpus-creation-pipeline}
\end{figure}

% In \circled{1}, we obtain a set of software tasks. For each task, pertinent artifacts are fetched in \circled{2}. In \circled{3}, text relevant to the task is identified by experts 
% \art{this seems redundant}


\section{Software Tasks}
\label{cp4:corpus-tasks}

We start corpus creation by identifying software tasks 
 which a developer will likely benefit from 
the use of additional information to complete.
We scope task selection to \textit{Android development} 
because the 
Android \acf{SDK} evolves constantly due to 
functionality, security and performance-related improvements~\cite{Li2018android, Mateus2020}.
These improvements impact its development community, requiring them to often  seek information regarding changes in the SDK~\cite{linares2014, bavota2014b, mcdonnell2013}.
For example, over 35,000 developers have used Q\&A forums to discuss tasks covering 87\% of the classes in the Android API~\cite{parnin2012}.


Two common places where Android task can be found are:


\begin{itemize}
    \item the description of an issue
    (e.g., a bug or feature request) reported in an issue tracking system; or in
    \item a post in a community forum, development mailing lists, and others.
\end{itemize}

Several studies have used issue tracking systems and software development communities for software tasks~\cite{Arya2019, baltes2019, nadi2020, Xu2017}
and, following the lead
of these studies, we select GitHub issues and \acf{SO} posts on Android development as 
the two sources for the tasks in our corpus.



\subsubsection{GitHub tasks}

To select tasks from GitHub, we are guided by studies that use 
stars~\cite{borges2016, borges2018}
as a proxy for a project's popularity~\cite{Ferreira2016, Xavier2020}.
We selected 14 projects,
ranging from mail clients\footnote{\url{https://github.com/k9mail/k-9}}
to development frameworks\footnote{\url{https://github.com/libgdx/libgdx}},
by filtering the list of top-starred projects in GitHub to those with the \textit{Java} and \textit{Android} tags.
We then randomly selected 25 distinct issues  originating from these starred projects as the GitHub tasks of our corpus.
While selecting issues, we \rev{balanced the number of issues selected per project} (average of 1.78 issues per project)
\rev{and we ensured that all the issues were resolved}. We also took care to 
check that they had at least one follow-up comment and that the issue title did not contain certain words, e.g., {\small \texttt{test}} or {\small  \texttt{ignore}},
as these words indicate issues  created automatically by scripts or bots---a common pitfall that researchers must be aware of when mining GitHub~\cite{kalliamvakou2014}.


Figure~\ref{fig:lock-screen-task} shows an example of a GitHub task in our corpus.
Although the expected behaviour is that the app controls should be visible even with the screen locked,  a user reports that the app screen is missing.
A developer addressing this issue might need to review the Android lock screen documentation~\cite{apiLockTask}
or refer to examples of applications that use the Android lock screen~\cite{mediumLockApp}.
For the remainder of this chapter, we use the lock screen task as a running example.


\begin{figure}
    \centering
    \includegraphics[width=\textwidth]{cp4/lock-screen-task}
    \caption{Sample GitHub task from our corpus}
    \label{fig:lock-screen-task}
\end{figure}



\subsubsection{Stack Overflow tasks}

%Provided that we restrict our corpus to tasks related to %the Android development domain,


We consider Stack Overflow posts as software tasks because to answer a post,
a developer often needs to provide references
supporting their answer~\cite{yazdaninia2021}.
Finding these references in a timely manner and writing the key information that helps a user understand 
 the provided solution encompasses many of the activities found in a developer's daily work, e.g., work-related browsing, coding, debugging, and reading/writing documentation~\cite{Meyer2017}.
For example, Figure~\ref{fig:webview-task} depicts a task where a developer describes her struggles using the Android WebView component~\cite{apiWebView}.
To answer this question, a developer will not only provide a code snippet, but also explain key points of the Android WebView API
and how they were used in the solution provided to the task, 
as presented in Figure~\ref{fig:webview-task-answer}.


\begin{figure}
    \centering
    \includegraphics[width=0.98\textwidth]{cp4/webview-task}
    \caption{Sample Stack Overflow question}
    \label{fig:webview-task}
\end{figure}



\begin{figure}
    \centering
    \includegraphics[width=0.98\textwidth]{cp4/webview-task-answer}
    \caption{Excerpt of a Stack Overflow answer}
    \label{fig:webview-task-answer}
\end{figure}

We randomly select 25 Stack Overflow posts from a curated list about Android development~\cite{baltes2020}.
This list was built by Baltes et al. 
using the Stack Overflow dump published on June 5, 2018~\cite{baltes2019-rep, SOTorrent2019}
and it contains 209,536 unique posts with the \textit{Java} and \textit{Android} tags.



% When selecting tasks in GitHub and Stack Overflow, a major challenge arises due to the sheer amount of data available.
% Baltes et al.~\cite{baltes2019} argues that even a cursory inspection of a sample set
% of Stack Overflow posts shows clear differences in a post's content or structure due to aspects such as programming languages, frameworks, associated technologies, and others.
% \gm{-It feels like there is a missing sentence about how differences relate
% to the 'sheer amount of data' and why the differences matter.}

% To ensure the tasks in the corpus we produce
% circumvent \gm{-circumvent has a negative connotation - is there a way
% to say this positively?} the heterogeneity of data on GitHub and Stack Overflow, we scope task selection to the \textit{Android} development domain. This decision
% restricts task selection to a single programming language (\textit{Java})
% while still enabling investigation of a domain that has been
% widely discussed by practitioners and researchers alike.



\section{Artifact Selection}
\label{cp4:corpus-artifacts}


When selecting artifacts pertinent to a task in our corpus, we seek to simulate everyday practices on how developers search the Web~\cite{rao2020, Xia2017}.
We formulate a query for each task and use a Web search engine to retrieve artifacts that are pertinent to that task, as described below.


\subsubsection{Artifact sources}

% \gm{Aren't most of these artifact
% types chosen because there are existing
% techniques? Have you considered doing
% this more empirically - report on the
% kinds of artifact types and select
% most common?}


The artifacts sought to find useful information or knowledge for completing a task
 depend on the type of task a developer performs.
For example, when using a new framework or library, a developer refers to official API documentation~\cite{Li2013,robillard2011field} while, for debugging or error diagnostic tasks, community-based sources are preferred~\cite{Li2013,Breu2010}.
Despite such variability, researchers have observed that 
technical blogs, API documentation, and community forums are sources
commonly used by developer to forage information that assists task completion~\cite{Li2013, josyula2018}.



We use this knowledge to restrict artifact selection to well-known and studied artifact types within these sources~\cite{Starke2009,Kevic2014, Li2013}, namely Android and Java SE API documentation, GitHub issues, Stack Overflow answers, and Web tutorials or blog posts on Java and Android development.





\subsubsection{Query formulation}


%  \gm{Seems like
% motivation that is in next paragraph
% should come first.}


Coming up with proper search terms is a critical step of any search~\cite{Haiduc2013}
and, ideally, we should be able to formulate a query with terms able to retrieve the most pertinent artifacts for a software task.
However, studies have shown that developers perform poorly in identifying good search terms~\cite{Starke2009,Kevic2014} and thus, using a task's title
as an educated approximation to terms that a developer might use is a common procedure adopted by other studies in the field (e.g.,~\cite{Xu2017} or ~\cite{Silva2019}).
Hence, we use a task's title (i.e., SO question or GitHub issue title) as the seed to search for pertinent artifacts.



\subsubsection{Search results}


We use \texttt{googlesearch} API~\cite{googlesearch} to get the results of a query. 
We fetch a maximum of 5 resources per artifact source --- a limitation necessary due to throttling or even blocking mechanisms in the APIs used to get the content of each source considered.
\gm{Isn't the real reason because you
don't want more?}
\art{I think it's both. I don't want more and there's throttling/IP blocking}
\gm{-the throttling argument is weak. You could just wait more time to search. I think this
is about the human evaluation capability so suggest you rely on that part of the argument.}


When selecting results, we exclude any entry that does not appear in the Amazon Alexa~\cite{alexa} Web traffic for Java and Android development in the period from April 2020 to March 2021. 
While applying this filter can significantly decrease the number of artifacts per task, it ensures that results are indeed related to software development. 
For instance, for a task discussing ``\textit{left and right-hand swap}'' 
filtering avoided fetching resources about  \textit{stock swap} operations.
Table~\ref{tbl:googlesearch-example-git} shows one search result per artifact source for the GitHub task introduced in Section~\ref{cp4:corpus-tasks}.





\begin{table}[H]
\centering    
\begin{footnotesize}
\begin{threeparttable}
\rowcolors{2}{}{lightgray}    
\begin{tabular}{l|l}

\hline

\multicolumn{2}{c}{\textit{No lock screen controls ever}}  \\

\hline
\hline

\multirow{1}{*}{API documentation}
& \href{https://developer.android.com/work/dpc/dedicated-devices/lock-task-mode}{Lock task mode - Android Developers} \\
% & Recents Screen - Android Developers \\

\multirow{1}{*}{Github issues}
& \href{https://github.com/AntennaPod/AntennaPod/issues/4448}{Lock screen controls disappear on Android 11 } \\
% & Bug: No lock screen image and controls \\


\multirow{1}{*}{StackOverflow answers}
& \href{https://stackoverflow.com/questions/24652078}{Media Control on Lock Screen like Google Play Music in android?} 
\\
% & How to disable home button in Android like lock screen apps do? \\



\multirow{1}{*}{Miscellaneous}
& \href{https://tinyurl.com/lock-task}{Create A React Native App - Which works on Lock Screen (Android) } \\

\hline


\end{tabular}
\end{threeparttable}
\end{footnotesize}
\caption{Sample of artifacts obtained for a Github task~\cite{git3578} }
\label{tbl:googlesearch-example-git}
\end{table}



\subsubsection{Artifact's content}

Last, we need to extract the natural language text within an artifact so that 
techniques that automate the identification of text relevant to a task can be built 
using our corpus.  This step requires processing an artifact's content 
into a sequence of individual sentences.
Given a search result \texttt{URL}, we use \texttt{BeautifulSoup}~\cite{beautifulsoup4},
\texttt{StackAPI}~\cite{StackAPI} and \texttt{PyGithub}~\cite{PyGithub}
to fetch the artifacts' content
and the Stanford CoreNLP toolkit~\cite{CoreNLP} for the identification of 
individual sentences,
which are common procedures for processing the artifact types found in our corpus~\cite{Arya2019, nadi2020}.










\clearpage


\section{Relevant text detection}
\label{cp4:corpus-relevant-text}




Next, we need text in a set of artifacts that could provide information that assists a developer solving her task.
In our corpus, this text represents \textit{golden data} that one can use to design and evaluate  
automatic tools that assist developers in the identification of  information useful to their tasks. 




To produce golden data, we follow the footsteps studies that 
ask human annotators to
mark the text that they deem useful and that provide information that could assist task completion~\cite{nadi2020, Robillard2015, marques2020}.
Overall, 
the \acs{DS-android} corpus consists of  
12,401 unique sentences
originating from artifacts associated to 50 Android development tasks and 
marked by three annotators as being relevant (or not)
to a task, as the following sections further detail. 


% \begin{itemize}
%     \item reliable golden data in a small yet diverse set of tasks and artifact types, including miscellaneous sources, which can provide new opportunities for the design of techniques that automatically detect text relevant to a task across different artifact types; and over
%     \item automatically produced golden data in a large number of
%     task and well-known artifact types where the relevant text detected by automatic approaches can serves as
%     baseline for comparison.
% \end{itemize}



\subsection{Tasks and artifacts}




We restrict the manual identification of text relevant to a task to a random subset of 
50  out of the 300 initially gathered tasks (Section~\ref{cp4:corpus-tasks}).
This decision was motivated by the fact that 
creating golden data for the entirety of our tasks 
would require asking human evaluators to inspect thousands of artifacts and more than 260,000 sentences, which would be a costly and time consuming activity. 



For each one of the tasks in this set (i.e., 25 GitHub tasks and 25 Stack Overflow tasks), we randomly selected 
one API document, a GitHub issue discussion, one Stack Overflow answer, as well as two  miscellaneous artifacts for a maximum of 5 artifacts per task for inspection.
Overall, 

comprises the inspection of 
12,401 unique sentences with an average of 63.59 sentences ($\pm 66.28$) per artifact 


\gm{There is no evaluation, so use annotators not
evaluators}
\art{updated}

We asked human annotators to read the content of these
artifacts and 
to mark sentences that they deemed useful and that provide information that assisted task completion---instructions similar to the ones used for the creation of the 
data in the \acs{DS-synthetic} corpus~\cite{marques2020}.
Since individuals might use different criteria to
assess relevance~\cite{Barry1994, Barry1998, Freund2015},
there is a risk that
the text selected by annotators does not overlap~\cite{Freund2013, Freund2015}.
Due to this reason, golden data in \acs{DS-android-small} consists of any sentence marked by annotators. 

% Section~\ref{cp4:threats} further details threats that might arise from this decision.



% Additionally, the text selected by a single annotator may still be crucial for task completion~\cite{marques2020}.


\subsubsection{Annotators}
\textcolor{white}{force ident} % this is just for the chapter outline

--- We recruited \red{n} graduate students with professional programming experience to produce \textit{golden} data for our tasks sample. \vspace{3mm}


\subsubsection{Annotation procedures}

\gm{Is this goal consistent with what the
original techniques sought to do as well?} \art{rephrased to be consistent with the original techniques}

\gm{Doesn't each annotator get each of the 10
tasks so there is no randomly assigned task?}
\art{Sorry. I missed this paragraph}

Our intention is that golden data reflect text that instruct developers to perform important actions to accomplish their task~\cite{Robillard2015, Lotufo2012}.
To produce such data, annotators had task descriptions and links to artifacts pertinent to the respective task at their disposal. We asked annotators to write a short plan (250 words max~\cite{Rastkar2010}) with instructions that a developer could follow to successfully complete the task. 
The purpose of the plan was to ensure that annotators built enough context about the task.
While perusing artifacts, annotators also had to manually highlight sentences that they deemed useful and that provided information that assisted task completion. 


The annotation process was facilitated by an in-house tool---in the form of a Web browser plugin shown in Figure~\ref{fig:corpus-annotation-tool}. In the figure, the top-right corner panel shows the browser extension. Annotators could start an annotation session and click the highlight button.
This would instrument the HTML of a page and identify each sentence in a paragraph. The tool allowed annotators to hove over individual sentences and select them as relevant (text in orange) by clicking on the hovered text. For example, the figure depicts that an annotator selected  the sentence
``\textit{Call {\small \texttt{ActivityOptions.setLockTaskEnabled()}} ... when starting the activity}'' as relevant for the lock mode task.


\begin{figure}
    \centering
    \includegraphics[width=\textwidth]{cp4/annotation-tool}
    \caption{Annotation tool and relevant sentences marked by an annotator}
    \label{fig:corpus-annotation-tool}
\end{figure}


\subsubsection{Results}

--- Provide summary of size of \acs{DS-android-small}. 

--- Descriptive statistics for marked sentences, similar to ICPC paper

--- State how the manually produced data can be used for the evaluation newly developed
techniques






\subsubsection{Metrics}

To compute an approach's accuracy, we use standard \textit{precision} and \textit{recall} metrics~\cite{Manning2009IR} and the text marked by the human annotators in \acs{DS-android-small}.
Since the text marked by them varies, we compute metrics when the manually produced golden data consists of the text marked by one, two, or the three annotators (i.e., $n=1, 2,$ or $3$).

\gm{I am not quite following ratio here...}
\art{I removed references to ratio and started using accuracy}



To ease interpreting precision and recall, Table~\ref{tbl:type-I-II-errors} show all possible evaluation outcomes. The \textit{relevant} and \textit{not-relevant} columns represent the text 
marked (or not) by the annotators. Rows represent the text automatically identified by an automatic approach.


\begin{table}[H]
\centering    
\begin{scriptsize}
\begin{threeparttable}
\begin{tabular}{l|l|l}

\hline

\textbf{}
& \textbf{Relevant}    
& \textbf{Not-relevant} \\

\hline
\hline

\textbf{Identified as relevant} & true positive ($TP$) & false positive ($FP$) \\
\hline
\textbf{Identified as Not-relevant} & false negative ($FN$) & true negative ($TN$) \\
\hline

\end{tabular}
\end{threeparttable}
\end{scriptsize}
\caption{Result outcomes}
\label{tbl:type-I-II-errors}
\end{table}

    



For a given task $t$ and artifact $a$, $precision_n$ is the fraction of the sentences identified that are marked as relevant by $n$ annotators over the total number of sentences identified, as shown in Equation~\ref{eq:cp4:precision}. For example, \textit{precision(t, API documentation)\textsubscript{2}} computes the number of sentences identified by \acs{Krec} for the task $t$ when 
golden data consist of sentences marked by two or more annotators.


\begin{equation}
\label{eq:cp4:precision}    
    Precision(t, a)_n = \frac{TP}{TP + FP}
\end{equation}


Recall ($recall_n$) represents how many of all sentences marked by at least $n$ annotators are identified by a technique (Equation~\ref{eq:cp4:recall}).


\begin{equation}
\label{eq:cp4:recall}        
    Recall(t, a)_n = \frac{TP}{TP + FN}
\end{equation}

% \vspace{3mm}




% \subsection{Results}
% \textcolor{white}{force ident} % this is just for the chapter outline


% --- Discuss results.\footnote{\red{think which summary tables should I have in the thesis body and which I can move to Appendices}} \vspace{3mm}



% --- Precision~\ref{tbl:ds-small-results-precision}  \vspace{3mm}


% --- Recall~\ref{tbl:ds-small-results-recall} \vspace{3mm}

% --- Likely explanation for the results obtained.

% % When interpreting results, we favor precision instead of recall.
% % A false positives may contribute to a developer abandoning reading of an artifact that would otherwise provide crucial information for her task~\cite{Rastkar2010}.




% \begin{table}[H]
\centering    
\begin{scriptsize}
\begin{threeparttable}
\begin{tabular}{lcccccc}

\hline


\multirow{2.5}{*}{Technique}
& \multicolumn{2}{c}{\textit{$Precision_{n=3}$}}
& \multicolumn{2}{c}{\textit{$Precision_{n=2}$}}
& \multicolumn{2}{c}{\textit{$Precision_{n=1}$}}
\\ \cmidrule(l){2-3} \cmidrule(l){4-5} \cmidrule(l){6-7} 


& \textit{mean}
& \textit{std}
& \textit{mean}
& \textit{std}
& \textit{mean}
& \textit{std}
\\


\hline
\hline

\acs{AnsBot} 
& 0.5 & 0.5 % = 3
& 0.5 & 0.5 % = 2
& 0.5 & 0.5 % = 1
\\

\acs{Krec} 
& 0.5 & 0.5 % = 3
& 0.5 & 0.5 % = 2
& 0.5 & 0.5 % = 1
\\

\acs{Hurried} 
& 0.5 & 0.5 % = 3
& 0.5 & 0.5 % = 2
& 0.5 & 0.5 % = 1
\\

\hline

\end{tabular}
\end{threeparttable}
\end{scriptsize}
\caption{Precision of each technique for the tasks of \acs{DS-android-small}}
\label{tbl:ds-small-results-precision}
\end{table}

    

% \begin{table}[H]
% \centering    
% \begin{scriptsize}
% \begin{threeparttable}
% \begin{tabular}{lcccccccccccc}

% \hline


% \multirow{2.5}{*}{Technique}
% & \multicolumn{10}{c}{\textit{Tasks}} 
% & \multicolumn{2}{c}{\textit{Precision}}
% \\  \cmidrule(l){2-11} \cmidrule(l){12-13} 



% &
% \textit{T1} & \textit{T2} & \textit{T3} & \textit{T4} & \textit{T5}
% & \textit{T6} & \textit{T7} & \textit{T8} & \textit{T9} & \textit{T10}
% & \textit{mean}
% & \textit{std}
% \\


% \hline
% \hline

% \acs{AnsBot} 
% & 0.5 & 0.5 & 0.5 & 0.5 & 0.5
% & 0.5 & 0.5 & 0.5 & 0.5 & 0.5
% & 0.5 % mean
% & 0.5 % std
% \\

% \acs{Krec} 
% & 0.5 & 0.5 & 0.5 & 0.5 & 0.5
% & 0.5 & 0.5 & 0.5 & 0.5 & 0.5
% & 0.5 % mean
% & 0.5 % std
% \\

% \acs{Hurried} 
% & 0.5 & 0.5 & 0.5 & 0.5 & 0.5
% & 0.5 & 0.5 & 0.5 & 0.5 & 0.5
% & 0.5 % mean
% & 0.5 % std
% \\

% \hline

% \end{tabular}
% \end{threeparttable}
% \end{scriptsize}
% \caption{Precision of each technique for the tasks of \acs{DS-android-small}}
% \label{tbl:ds-small-results-precision}
% \end{table}

    

% \begin{table}[H]
\centering    
\begin{scriptsize}
\begin{threeparttable}
\begin{tabular}{lcccccccccccc}

\hline


\multirow{2.5}{*}{Technique}
& \multicolumn{10}{c}{\textit{Tasks}} 
& \multicolumn{2}{c}{\textit{Recall}}
\\  \cmidrule(l){2-11} \cmidrule(l){12-13} 



&
\textit{T1} & \textit{T2} & \textit{T3} & \textit{T4} & \textit{T5}
& \textit{T6} & \textit{T7} & \textit{T8} & \textit{T9} & \textit{T10}
& \textit{mean}
& \textit{std}
\\


\hline
\hline

\acs{AnsBot} 
& 0.5 & 0.5 & 0.5 & 0.5 & 0.5
& 0.5 & 0.5 & 0.5 & 0.5 & 0.5
& 0.5 % mean
& 0.5 % std
\\

\acs{Krec} 
& 0.5 & 0.5 & 0.5 & 0.5 & 0.5
& 0.5 & 0.5 & 0.5 & 0.5 & 0.5
& 0.5 % mean
& 0.5 % std
\\

\acs{Hurried} 
& 0.5 & 0.5 & 0.5 & 0.5 & 0.5
& 0.5 & 0.5 & 0.5 & 0.5 & 0.5
& 0.5 % mean
& 0.5 % std
\\

\hline

\end{tabular}
\end{threeparttable}
\end{scriptsize}
\caption{Recall of each technique for the tasks of \acs{DS-android-small}}
\label{tbl:ds-small-results-recall}
\end{table}

    


% \subsection{Threats}
% \label{cp4:corpus-threats}

% --- \red{Argue that this is not a replication study but rather establishing thresholds} \vspace{3mm}

% --- Discuss threats \vspace{3mm}






\section{Summary}
\label{cp4:corpus-summary}

In this chapter, we introduced the need for corpora 
for the development of 
automatic techniques able to identify relevant text
to solve a task in artifacts
pertinent to the task.
Since  no such corpora
existed, we detailed 
a set of structured procedures for its creation. 
The \acs{DS-android} corpus that we constructed consists of  
12,401 unique sentences
originating from artifacts associated with 50 software tasks
drawn from GitHub issues and Stack Overflow posts about Android development. 
We found that 
three annotators with professional experience indicated that,
out of these 12,401 unique sentences, 
1,393 of them were relevant to a particular task and that multiple annotators agreed of the relevance of 29\% of the sentences,
which lead us to provide recommendations on how our data can be used for evaluation purposes.
Ultimately, we expect that the \acs{DS-android} corpus
provides a foundation for studies that explore relationships between software tasks and text found across different types of artifacts that a developer might seek information on and that are pertinent to the developer's task.
 






% \acs{DS-android-small}

% \acs{DS-android-large}