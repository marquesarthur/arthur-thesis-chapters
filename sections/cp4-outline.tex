\setcounter{chapter}{3}
\setcounter{rq}{1}


\chapter{Android Task Corpus}
\label{ch:android-corpus}



One of the challenges for characterizing task-relevant information as well as for proposing automatic techniques to automatically identify task-relevant text is 
the lack of large corpora containing 
software tasks and associated artifacts originating from heterogeneous sources.
To fill this gap, this chapter contributes with the Android tasks corpus (\textbf{\acs{DS-android}}), a dataset with 300 tasks comprising Android development and originating from StackOverflow questions 
as well as from GitHub issues on a variety of Android Open-Source Systems. 



This chapter main contributions are \textit{(i)} the procedures for the corpus creation, and; \textit{(ii)} the corpus itself. A well-structured corpus lays the foundation for several studies that explore relationships between software tasks, natural language artifacts, and text within these artifacts. 




\section{Motivation}
\textcolor{white}{force ident} % this is just for the chapter outline


When performing a software task, a software developer does not restrict her work 
solely to coding~\cite{Meyer2017}.
Rather, there are many activities that a developer undertakes to complete a software task
as when they:



\begin{itemize}
\item refer to API documentation or Q\&A websites for API usage purposes~\cite{umarji2008archetypal, Singer1998, robillard2011field};
\item exchange emails discussing a possible reusable library to be incorporated into their implementation~\cite{umarji2008archetypal, Bacchelli2012}; or
\item confirm a system's behaviour referring to past discussions in community forums or in the system's documentation~\cite{Arya2019, Lotufo2012, Singer1998}.
\end{itemize}
    

A common aspect to these activities is the need to work with unstructured textual data, which comprises 80\% of the overall information created and used in enterprises~\cite{Bavota2014, holzinger2013}.
Due to such prevalence, there has been a large body of studies utilizing various techniques to extract
information from this text that can be embedded in
tools for software developers~\cite{Bavota2014, Xu2017, Robillard2015, Lotufo2012}. For example, Xu et al. propose to mine relevant text from Stack Overflow
to generate answers to developers' technical question~\cite{Xu2017}
while FRAPT identifies key paragraphs explaining API elements in code tutorials~\cite{Jiang2017}.
As other examples, Lotufo et al. proposed a technique to identify sentences a developer would first read when inspecting bug reports~\cite{Lotufo2012} while Nadi and Treude investigated sentences that help a developer decide whether a Stack Overflow post is relevant to her task~\cite{nadi2020}




Although these studies provide significant contributions, the fact that information to answer a question a developer has is usually located across many artifacts~\cite{Rastkar2013t} is often overlooked.
Hence existing techniques operate in an artifact-centric basis
what can be insufficient to provide to developers all the information needed 
to completely and correctly accomplish a software task.



Since several studies suggest that developers use heterogeneous sources to put together the information needed for task completion~\cite{josyula2018, Li2013, rao2020} we seek to design techniques or approaches that can scale to cover different types of artifacts.
As a first step towards this goal, we require a corpus with heterogenous sources that a developer 
might use to locate information relevant to her task.









\clearpage

\section{Methodology}
\textcolor{white}{force ident} % this is just for the chapter outline

--- Provide an overview of the corpus creation procedures 


\subsection{Software Tasks}
\textcolor{white}{force ident} % this is just for the chapter outline

--- We consider a software task as a piece of work undertaken by a developer that often has to be finished within a certain time~\cite{2004merriam}. 
Two common places a software task can be found are:

\begin{itemize}
    \item the description of a bug or feature request reported in a bug tracking systems; or in
    \item a post in a community forum, development mailing lists, and others.
\end{itemize}

\vspace{3mm}

--- Given this definition, we select tasks from StackOverflow and GitHub to build our corpus

------ We scope task selection to the \textit{Android} development domain such that we can select common tasks across the two sources \vspace{3mm}


\subsubsection{StackOverflow tasks}
\textcolor{white}{force ident} % this is just for the chapter outline

--- Detail that StackOverflow tasks were selected from Baltes et al. dataset~\cite{baltes2019-rep}

------ Give example of a StackOverflow task \vspace{5mm}

\subsubsection{GitHub tasks}
\textcolor{white}{force ident} % this is just for the chapter outline

--- Because there is no GitHub dataset promptly available, we selected GitHub issues from top starred Android projects on GitHub 

------ Projects were selected among Open Source Systems that had the \textit{Java} and \textit{Android} tags. 

------ We randomly sampled 15 issues from a total of 10 distinct projects

------ Describe how we avoid common pitfalls  when mining GitHub~\cite{kalliamvakou2014}

------ Give example of a GitHub task 

\subsection{Artifact Selection}
\label{cp4:corpus-artifacts}
\textcolor{white}{force ident} % this is just for the chapter outline


--- Our artifact selection approach seeks to simulate everyday practices on how developers search the Web~\cite{rao2020, Xia2017} \vspace{3mm}


\subsubsection{Artifact sources}
\textcolor{white}{force ident} % this is just for the chapter outline

--- As there are many different sources of artifacts, we restrict artifact selection to well known and studied sources~\cite{Starke2009,Kevic2014, Li2013}:


\begin{itemize}
    \item Android and Java SE API documentation;
    \item Github issues;
    \item StackOverflow answers; and
    \item Web tutorials or blog posts from java and android development.
\end{itemize}


\vspace{3mm}


\subsubsection{Query formulation}
\textcolor{white}{force ident} % this is just for the chapter outline

--- To that end, we consider a task's title (i.e., SO question or GitHub issue title) as the seed used to search for pertinent artifacts in a search engine

------ This is a common procedure adopted by many studies in the field (e.g.,~\cite{Xu2017} or ~\cite{Silva2019}). \vspace{3mm}


--- For these sources, we use the \texttt{googlesearch} API~\cite{googlesearch} to perform Web searches


\subsubsection{Artifact selection}
\textcolor{white}{force ident} % this is just for the chapter outline

--- We fetch a maximum of 5 resources per artifact source --- a limitation necessary due to throttling or even blocking mechanisms in the APIs used the fetch data \vspace{3mm}

--- We exclude results that do not appear in Alexa~\cite{alexa} Java and Android categories in the period from 2020/04/01
to 2021/03/01





\subsection{Corpus Summary}
\textcolor{white}{force ident} % this is just for the chapter outline


--- Summary of contributions for the chapter \vspace{3mm}

------ A corpus containing 300 software tasks originating from StackOverflow and Github and associated artifacts containing potentially relevant information to correctly completing the task \vspace{3mm}

------ Methodology for the corpus creation as well as a replication package with the scripts used to create the Android task corpus \vspace{3mm}


--- Provide descriptive summary of the corpus

------ 300 tasks, 150 from SO and 150 from Git

------ Approximately 2,500 artifacts

------ Almost 260,000 sentences



% \acs{AnsBot} 

% \acs{Krec} 

% \acs{Hurried} 

% \acs{DS-synthetic} 

% 

% \acs{DS-android-small}

% \acs{DS-android-large}