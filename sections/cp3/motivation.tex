\section{Motivation}
\label{cp3:method}

% To produce a solution for a task, a developer engages in a variety of information-seeking activities which are the focus of our experiment.
% Particularly, we focus on the text that a developer deems relevant to working on a software task,
% which we define as ``problem reports and feature request descriptions often recorded in an issue tracking system such as Bugzilla''~\cite{Cubranic2005}.


In Chapter~\ref{ch:related-work} we described limitations 
of existing approaches for identifying task-relevant 
text across different types of artifacts. 
To investigate how to relax these limitations, we consider three questions:


\begin{enumerate}[label=\textit{RQ\arabic*},leftmargin=*]

    \item \textit{How much agreement is there between participants about the text
    relevant to a task?} With this question, we seek to understand
    the amount of information sought for task completion
    and whether participants see the same text as
    relevant to a task.

    \item \textit{What are common cues to the relevancy of text to a task?}
    With this question, we seek to determine if the rules governing how natural language information
    is constructed can guide us to information relevant to a task~\cite{Kintsch1978a}.
    We consider whether there
    are patterns in the  syntactic structure and in meaning of text identified as relevant.

    \item \textit{How do developers determine if text is relevant to a task?}
    With this question, we seek to identify common themes in developers' identification of task-relevant text. For instance, what are common challenges or factors that affect a developers' judgment on the relevancy of the text to a task?

\end{enumerate}



The first research question seeks to characterize text
in documents relevant to a task.
The second research question refers to the text itself.
It analyzes possible syntactic or semantic
predictive cues that may originate from commonalities in any of the text that developers deem relevant to the tasks we present them.
The last research question considers qualitative aspects that might provide
further insights to how developers identify relevant text.

