
\subsection{Threats to Validity}
\label{cp3:threats}


In this section, we discuss threats from our experimental design 
and from our
examination of the relevant text (HUs) produced by participants who 
inspected natural language artifacts from six information seeking tasks. 



The design of the six tasks in our experiment represents a threat to
\textit{construct validity}.  To mitigate this threat, we used results from studies discussing the search
behaviour of developers~\cite{umarji2008archetypal, Li2013, Xia2017} as
input to the design of our tasks.
For each task, our goal was to provide a set of artifacts so that a
participant could gather enough knowledge to complete the task, which
we confirmed through pilot studies.  The provision of artifacts was also
necessary to enable a systematic analysis of relevance through a
controlled experiment. We mitigated threats on artifact
selection by asking external researchers to provide their own list of
candidate artifacts, which was similar to our own list.



The \textit{external validity} of the experimental results is affected
by the participants, the tasks and the artifacts.  Considering the
subjective nature of relevance, the knowledge and particularities of
our participants may not generalize to other software developers.
Furthermore, the structure and linguistic characteristics of the
artifacts considered may not extend to other kinds of artifacts or the
same kind of artifacts in other domains.  We tried to mitigate this
threat by selecting different kinds of artifacts such as API
documentation, bug reports, and Q\&A pages while 
ensuring that each kind was present in at least
50\% of the tasks. 
However, our results
are bound to the domains and artifacts we evaluated and may not generalize.
Future work should confirm our findings on a wider range of artifacts and tasks.


There are also threats to the validity of our
\textit{conclusions}.
Experimental procedures encouraged participants to work on the tasks using their normal workflow.
However, at least one participant (\textit{P15}) indicated that they would normally
revisit a document as different information needs arise as part of a task.
As a result, our study may miss relevant information.
In addition, the text participants highlighted may not have always aligned with the text that contained information useful for completing the task.
To investigate this alignment further, we would have had to create an oracle containing the relevant text.
Unfortunately, creating the oracle is challenging because it would need to encompass different perceptions of relevance~\cite{Pirolli1999,saracevic1975,Nenkova2004} and usefulness~\cite{Freund2013, Freund2015}.
Thus, to avoid the subjective and imprecise process of judging alignment, we simply consider any text highlighted by participants as containing information useful to completing the tasks.



While the size of our corpus might also affect our conclusions,
we emphasize that the HUs from mid and top-tiers in our data comprise 250 sentences, 
which is similar to the amount of data studied previously. For instance, the McGill~\cite{Petrosyan2015}
and Android~\cite{Jiang2016b} datasets contain 238 and 141 relevant text fragments, respectively.


To extract the meaning of text through parsed frames, 
we used a general frame semantics database rather than one
specific to software engineering.  This led to frames that were
\textit{out-of-context}, causing us to adapt either the frame's
name or its description.  
Frames adapted to software engineering (e.g., \textit{being returned}
and \textit{fields}) represent a small fraction of all the extracted
frames. As a result, we argue that all the sentences containing them still have
similar semantic meanings and our conclusions are not
affected by these out-of-context frames. However, as we discuss in Chapter~\ref{ch:discussion}, 
having specific
software engineering frames could improve our results.
