\setcounter{chapter}{0}


\chapter{Identifying Task-Relevant Text}
\label{ch:identifying}


The sheer amount of information in natural language software artifacts may prevent a developer from comprehensively locating all the information that is needed to complete their task correctly and completely~\cite{Murphy2005}.
For instance,
finding information that assists a developer solve a 
task often require filtering
the content of artifacts pertinent to a task 
to less than a 20\% of an artifact's text.


To help developers in this activity, 
this chapter describes the design of techniques
that automatically identify text relevant
to a task in artifacts
pertinent to the task.
The two approaches that we explore use 
word-level and sentence-level
properties
to determine if the text is relevant to a task.
These properties can be embedded into a tool that 
assists developers
in finding the small parcel of text that might contribute with 
information to solve a task amidst large amounts of irrelevant text.
For that, we start by presenting background information in Section~\ref{cp5:background}.
Section~\ref{cp5:approaches} describes the techniques
we explore to detect relevant text. In turn, 
Section~\ref{cp5:evaluation} presents evaluation results.
We summarize key findings 
and venues for future work in Section~\ref{cp5:summary}.

